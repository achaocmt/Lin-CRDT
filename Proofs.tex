%!TEX root = draft.tex
\section{Proving \CRDTLin{}}
\label{sec:proofs}

\textblue{
I think this section should be about proving linearizability. Here we introduce:
\begin{itemize}
\item a semantics for CRDTs
\item a general methodology based on simulation relation (explain that it can also be seen as invariant checking)
\item the two proof strategies for building linearizations, as particular instances of that simulation relation
\end{itemize}}

TODO INTRODUCE THIS SECTION

\subsection{Proof Methodology}\label{ssec:proof-methodology}

Our approach for proving \CRDTLin{} (w.r.t. $\Spec{}$) consists in
instrumenting the object semantics with an auxiliary variable
$\alinord$ recording a linearization of the current history (recorded
in the object's global configurations) following the approach
of~\cite{VafeiadisHHS06,more citations}.

As explained in Section~\ref{sec:overview} there are two actions in the
semantics that can modify the state of replicas:
\begin{inparaenum}
\item the execution of a new client request which is shown as the
  \lstinline|atSource| procedure in the data type implementation, or
\item the execution of the downstream of an operation originating in a
  different replica shown in the \lstinline|downStream| procedure of
  the implementation.
\end{inparaenum}
%
While the first case (\lstinline|atSource|) represents the arrival of
a new operation, the latter (\lstinline|downStream|) represents the
propagation of effects of preexisting operations.
%
Since the sequence $\alinord$ is a global order over the operations,
it needs only be updated when a \emph{new} operation arrives,
therefore only the execution of an \lstinline|atSource| portion of an
operation engenders the extension of $\alinord$\footnote{For readers
  familiars with proving classical linearizability for concurrent
  objects, this is akin to assuming that \lstinline|atSource| is a
  ``fixed'' linearization point.
  However, differently from the classical case, the operation may not
  be necessarily placed at the end of the current linearization.}
by adding the current operation at some specific position. To guarantee that the
linearization is consistent with the visibility relation, the operation must be inserted in
the linearization after all the operations which are visible at the replica executing \lstinline|atSource|.
%%
%\gpnote*{Clarify?}
%{
%This guarantees that the linearization $\alinord$ is consistent with
%the visibility relation.
%}
When the object contains query-update operations, the operations are first rewritten according to a query-update rewriting $\gamma$ as in \autoref{definition:distributed linearizability} before being placed in the linearization.
Then, the objective is to prove that $\alinord$ being an
\crdtlinearization{} of the current history (w.r.t. $\Spec{}$) is an
invariant of the object. This invariant is strengthened with
additional requirements in order to make it inductive.
%
In the case of the objects we investigated, the inductive invariant is
defined as the conjunction of two properties:
\begin{itemize}
%\setlength{\itemsep}{0.5pt}
\item[-] $\mathsf{ReplicaStates}$: requiring that for each replica
  $\arep$ with local configuration $(\alabelset,\astate)$, the state
  $\sigma$ is obtained by applying the downstreams of the operations
  in $\alabelset$ (that have been delivered to $\arep$) in the order
  defined by $\alinord$, and
\item[-] $\mathsf{\CRDTLinshort{}}$: requiring that the sequence
  $\alinord$ is an \crdtlinearization{} of the current history (w.r.t.
  $\Spec{}$).
% is \crdtlinearizable{} w.r.t \Spec{} and $\mathit{lin}$ is a
% linearization. Here $\alabelset$ is the set of labels in domain of
% $\downstreams$.
\end{itemize}
To prove that this is an inductive invariant we rely on additional
assertions describing the effect of each downstream (we provide
examples in the following subsections), and also, a proof that:
\begin{itemize}
%\setlength{\itemsep}{0.5pt}
\item[-] $\mathsf{Refinement}$: each downstream produced by an update $\alabel$ and respectively, each query $\alabel$, is simulated by the application of $\alabel$ in the specification $\Spec$.
\end{itemize}
The proof of $\mathsf{Refinement}$ relies on a \emph{refinement
  mapping}~\cite{earlierLamport,AbadiL91} between replica states and states of the
specification, denoted by $\refmap$. The proof goals require that for
any two replica states $\sigma$ and $\sigma'$,
\begin{enumerate}
        \item if $\sigma'$ is obtained from $\sigma$ by applying a
          downstream $\delta$ produced by an update $\alabel$, then
          \mbox{$\refmap(\sigma)\xRightarrow{\alabel}\refmap(\sigma')$},
          where $\xRightarrow{\alabel}$ is the transition function of
          $\Spec$, and
        \item if a query $\alabel$ is applied on a state $\sigma$ or
          it is introduced by a rewriting of a query-update that
          executes \lstinline|atSource| on a state $\sigma$, then
          $\refmap(\sigma)\xRightarrow{\alabel}\refmap(\sigma)$.
\end{enumerate}
\gpnote*{Explain or remove}{
Actually, for the CRDT objects presented in Section~\ref{subsec:time order of execution as linearization}, we consider a stronger version of $\mathsf{Refinement}$ which requires that the state $\sigma$ and the downstream $\delta$ satisfy some particular condition (in the context of the first proof goal).
}

\gpnote[nomargin, inline]{Forward reference to exec-order, and
  ts-order.}
We have applied this methodology to a range of CRDT objects described in~\cite{ShapiroPBZ11}, and we have identified two instances which differ in the way in which the linearization $\alinord$ is extended (when executing \lstinline|atSource|). For the first class of objects that includes the OR-Set in Listing~\ref{lst:or-set}, the linearization $\alinord$ is defined by the order in which the \lstinline|atSource| procedures are executed, i.e., every new operation executing \lstinline|atSource| is appended to the current linearization (\S~\ref{subsec:time order of execution as linearization}). We say that such objects admit \emph{execution-order linearizations}.
The second class of objects that includes the RGA object in Listing~\ref{lst:rga}, supports \CRDTLinshort{} proofs in which the linearization is built based on timestamps, e.g., $\alinord$ is consistent with the order between the timestamps generated by operations (\S~\ref{subsec:time-stamp order as linearizabtion}). We say that such objects admit \emph{timestamp-order linearizations}.

\gpnote[nomargin, inline]{I agree with the content. Some rewriting is needed.}

%
%For some CRDT implementations, when proving its correctness, we can use the total-order of executions as linearization. For such CRDT implementations, our proof approach is as follows:
%
%
%Given a object $\aobj$ and a specification \Spec{}, we construct a predicate $P(\mathit{config},\mathit{lin})$ where $\mathit{config} = (\gstates, \avisord, \downstreams)$ is a configuration of $\llbracket \aobj \rrbracket_{\mathit{op}}$, and $\mathit{lin}$ is a sequence used as linearization. We require $P(\mathit{config},\mathit{lin})$ to be a conjunction of the following statements:
%
%\begin{itemize}
%\setlength{\itemsep}{0.5pt}
%\item[-] Condition $C_1$ (linearizability): The history $(\alabelset, \avisord)$ is \crdtlinearizable{} w.r.t \Spec{} and $\mathit{lin}$ is a linearization. Here $\alabelset$ is the set of labels in domain of $\downstreams$.
%
%\item[-] Condition $C_2$ (downstream): A statement about downstream for each operation label in $\mathit{config}$.
%
%\item[-] Condition $C_3$ (sequential explanation): For each replica $\arep$, we have that $\gstates(\arep).\astate = \mathit{apply}(\mathit{lin},\gstates(\arep).\alabelset)$.
%\end{itemize}
%
%Here the function $\mathit{apply}(\mathit{lin},S)$ returns a local state obtained by applying downstream of operations in $S$ according to total order of $\mathit{lin}$.

\subsection{Execution-Order Linearizations}
\label{subsec:time order of execution as linearization}

%Then, we need to prove that $P$ is a simulation relation. Moreover, we prove that we can always obtain a new linearization by putting a new operation after the tail of a old linearization. We call such relation a execution-order simulation relation.
%
%\begin{itemize}
%\setlength{\itemsep}{0.5pt}
%\item[-] $P(\mathit{config}_0,\epsilon)$ holds, where $\mathit{config}_0$ is the initial configuration.
%
%\item[-] If $P(\mathit{config},\mathit{lin})$ holds and $\mathit{config} {\xrightarrow{\alabel}} \mathit{config}'$, then $P(\mathit{config}', \mathit{lin} \cdot \alabel)$ holds.
%
%\item[-] If $P(\mathit{config},\mathit{lin})$ holds and $\mathit{config} {\xrightarrow{}} \mathit{config}'$, then $P(\mathit{config}',\mathit{lin})$ holds.
%\end{itemize}
%
%It is easy to see that, execution-order simulation relation implies a \crdtlinearizable{} relation proof with the execution as linearization.

In this section, we describe a first instantiation of the methodology described above using as example the OR-Set object in Listing~\ref{lst:or-set}. The proof that this object satisfies \crdtlin{} w.r.t. the specification defined in Example~\ref{} relies on several annotations which are given on the lines starting with {\tt //@} in Listing~\ref{lst:or-set} and that we describe hereafter.

%a two-vocabulary formula, the primed, resp., unprimed, variables denoting values after, resp., before, the execution of the procedure.

The annotations included in the \lstinline|atSource| procedures describe the construction of the linearization,
%We describe updates of the linearization using \lstinline|let| constructs that append a set of labels.
which is is extended by appending operations, possibly rewritten using the query-update rewriting $\gamma$ defined in Example~\ref{}. %, when they execute the \lstinline|atSource| procedure.
The annotations of the downstreams specify their effect, e.g., relating their effect with the set of operations visible to the replica where this operation has originated (when the downstream has been computed). They are written as two-vocabulary formulas, the primed, resp., unprimed, variables denoting values of variables after, resp., before, the execution of the downstream.
For instance, the downstream corresponding to a {\tt remove} operation removes the set of element-id pairs added by {\tt add} operations visible at the replica $\arep$ where the {\tt remove} originated, and which were not ``canceled'' by other {\tt remove}s visible at $\arep$ ({\tt remove}s that saw those {\tt add} operations).

In the following, we show that $\mathsf{ReplicaStates}$ is an inductive invariant. Since every operation is appended to the linearization when it executes \lstinline|atSource| it clearly follows, in the linearization order, all operations visible to it. Then, by the {\textred{causal delivery}} assumption, the order in which downstreams are applied at a given replica is also consistent with the visibility order. The main difficulty in showing that $\mathsf{ReplicaStates}$ is an inductive invariant is the situation where two downstreams correspond to two ``concurrent'' operations (not related by visibility) and their linearization order is different from the order in which they are applied at a given replica. However, as already assumed in some definitions of CRDT objects~\cite{}, such downstreams commute, i.e., applying them in one order or another leads to the same state. The fact that they commute is an easy consequence of the downstream annotations: two {\tt add} or two {\tt remove} downstreams commute obviously because they both add or both remove element-id pairs, while an {\tt add} and a {\tt remove} downstream commute when they are concurrent because in this case, the element-id pairs removed by {\tt remove} are different from the pair added by the {\tt add} (since the {\tt add} is not visible to {\tt remove}).

The proof of $\mathsf{Refinement}$ is quite straightforward. We consider a refinement mapping $\refmap$ defined as the identity function. For instance, applying the downstream produced by $\alabelshort[{\tt add}]{a,k}$~\footnote{In the proof of $\mathsf{Refinement}$, we implicitly assume that whenever an operation label $\alabel$ is rewritten to another label $\alabel'$, e.g. $\alabelshort[{\tt add}]{a}$ is rewritten to $\alabelshort[{\tt add}]{a,k}$, then the downstream produced by the original operation $\alabel$ must be simulated by the operation $\alabel'$ in the specification. Similarly, for query-updates $\alabel$ rewritten to $(\alabel_1,\alabel_2)$, the downstream produced by $\alabel$ must be simulated by the operation $\alabel_2$ in the specification.}   or the $\alabelshort[{\tt add}]{a,k}$ operation of the specification ??? has exactly the same effect. The same holds for the downstream produced by $\alabelshort[{\tt remove}]{a,R}$. Concerning query operations, there are two cases: (1) whenever the query-update $\alabelshort[{\tt remove}]{a}$ executes \lstinline|atSource| on a state $\sigma$, then the query $\alabellong[{\tt readIds}]{a}{R}{}$ (introduced by the query-update rewriting) should be enabled in state $\refmap(\sigma)=\sigma$, which clearly holds because the computation of $R$ in \lstinline|atSource| and as the result of $\alabelshort[{\tt readIds}]{a}$ in the specification state $\sigma$ are exactly the same, and (2) applying the query $\alabellongind[{\tt read}]{}{A}{}$ on the replica state $\sigma$ should result in the same return value as applying the same query in the context of the specification on the same state $\sigma$, which again holds trivially.

Finally, we describe the proof of the fact that $\mathsf{\CRDTLinshort{}}$ is an inductive invariant. As already mentioned, appending operations to the linearization when they execute \lstinline|atSource| clearly implies that $\alinord$ is consistent with the visibility. Next, the projection of $\alinord$ on the updates is obviously admitted by the specification (the updates are always enabled from the point of view of the specification).
%The only thing to show is that the element-id pairs removed by a {\tt remove} have been added by {\tt adds} which precede it in the linearization, and not already removed by other {\tt remove}s that precede it. This is a direct consequence of the fact that the linearization order is consistent with the visibility relation, and the invariant $\mathsf{ReplicaStates}$, which implies that every element-id pair in a the state of a replica $\arep$ has been added by an {\tt add} visible to $\arep$ and not removed by other {\tt remove}s visible to $\arep$. Then,
We also have to argue that for each query $\alabel_{\mathsf{qr}}\in\{\alabellongind[{\tt readIds}]{a}{R}{},\alabellongind[{\tt read}]{}{A}{}\}$, the sequence $\alinord'\cdot \alabel_{\mathsf{qr}}$ where $\alinord'$ is the projection of $\alinord$ on the set of updates
visible to $\alabel_{\mathsf{qr}}$ is admitted by the specification. First, by $\mathsf{ReplicaStates}$, the state $\sigma$ of the replica where $\alabel_{\mathsf{qr}}$ is applied is obtained by applying the downstreams of the operations visible to $\alabel_{\mathsf{qr}}$ in the linearization order. Then, by $\mathsf{Refinement}$, every downstream is simulated by the corresponding operation in the context of the specification. This implies that $\refmap(\sigma_0)\xRightarrow{\alinord'}\refmap(\sigma)$, where $\sigma_0$ is the initial replica state. The query $\alabel_{\mathsf{qr}}$ is also simulated by the same operation in the context of the specification, which implies that $\refmap(\sigma)\xRightarrow{\alabel_{\mathsf{qr}}}\refmap(\sigma)$. These two facts imply that $\refmap(\sigma_0)\xRightarrow{\alinord'\cdot \alabel_{\mathsf{qr}}}\refmap(\sigma)$ which means that $\alinord'\cdot \alabel_{\mathsf{qr}}$ is admitted by the specification.

TODO STATE WHICH OTHER OBJECTS SUPPORT THE SAME KIND OF PROOF.

\subsection{Timestamp-Order Linearizations}
\label{subsec:time-stamp order as linearizabtion}

CRDT objects such as RGA, that use timestamps for conflict resolution, may not admit execution-order linearizations. \fxwarning[nomargin, inline]{Example showing why execution-order linearizations are not valid for RGA.}
To prove that they are \crdtlinearizable{}, we consider an instantiation of the proof methodology described in Section~\ref{ssec:proof-methodology} where the linearization is also consistent with the order of the timestamps generated by the operations.
We describe this instantiation using the RGA object in Listing~\ref{lst:rga} as an example.

The construction of the linearization ensures that the operations that generate a timestamp, i.e., the invocations of ${\tt addAfter}$, are ordered in the linearization according to the timestamps. More precisely, any two operations $\alabelshort[{\tt addAfter}]{a,b}$ and $\alabelshort[{\tt addAfter}]{c,d}$ that generate two timestamps $\ats_{\tt b}$ and $\ats_{\tt d}$, respectively, are ordered $\alabelshort[{\tt addAfter}]{a,b}$ before $\alabelshort[{\tt addAfter}]{c,d}$ if $\ats_{\tt b}<\ats_{\tt d}$ and $\alabelshort[{\tt addAfter}]{c,d}$ before $\alabelshort[{\tt addAfter}]{a,b}$ if $\ats_{\tt d}<\ats_{\tt b}$. Moreover, for operations $\alabel$ that don't generate a timestamp, i.e., invocations of ${\tt remove}$ and ${\tt read}$, we consider a ``virtual'' timestamp which is defined as the \emph{maximal} timestamp of an operation visible to $\alabel$ (or $\bot$ if no operation is visible to $\alabel$), and then require that the linearization is consistent with the order between both ``real''~\footnote{That is, timestamps generated by the operation itself.} and ''virtual'' timestamps. \fxwarning[nomargin, inline]{Example showing a history and how the linearization evolves}

The annotations defining the linearization are included in the \lstinline|atSource| procedures. The function {\tt insert}($\alinord$,$\alabel$,$\ats$) inserts the label $\alabel$ in the linearization $\alinord$ just after the last operation $\alabel'$ in $\alinord$ which has a timestamp (real or virtual) smaller than or equal to $\ats$. The annotations describing the effect of downstreams are quite straightforward in this case, they are logical interpretations of the program statements.

The proof of $\mathsf{ReplicaStates}$ is quite similar to the case of OR-Set. Since the order between the timestamps generated by ${\tt addAfter}$ operations is consistent with the visibility relation (see Section~\ref{ssec:semantics}), it easily follows that the linearization order $\alinord$ is also consistent with the visibility relation. Then, as in the case of OR-Set, it remains to prove that any two downstreams that correspond to two ``concurrent'' operations (not related by visibility) commute. This easily follows from the fact that the downstreams apply set union which is obviously commutative. Also, note that by the {\textred{causal delivery}} assumption and the preconditions of ${\tt addAfter}$ and ${\tt remove}$, it cannot happen that an $\alabelshort[{\tt addAfter}]{a,b}$ operation adding ${\tt b}$ after ${\tt a}$ is concurrent with an operation that adds ${\tt a}$ to the list, i.e., $\alabelshort[{\tt addAfter}]{c,a}$, for some ${\tt c}$, or that an $\alabelshort[{\tt addAfter}]{a,b}$ operation adding ${\tt b}$ is concurrent with an operation $\alabelshort[{\tt remove}]{b}$ that removes ${\tt b}$. This ensures that reordering concurrent downstreams doesn't lead to ``invalid'' replica states where for instance, the timestamp tree ${\tt N}$ contains nodes which are not reachable from the root (which would happen if $\alabelshort[{\tt addAfter}]{a,b}$ is delivered before the operation $\alabelshort[{\tt addAfter}]{c,a}$ adding ${\tt a}$), or where the tombstone set ${\tt Tomb}$ contains elements which are not in the tree (which would happen if $\alabelshort[{\tt remove}]{b}$ is delivered before $\alabelshort[{\tt addAfter}]{a,b}$).

{\color {red} CW: Each abstract state is in the form $\abstate = (a_1,flag_1) \cdot \ldots \cdot (a_n,flag_n)$.} Concerning the proof of $\mathsf{Refinement}$, we consider a refinement mapping $\refmap$ which relates a replica state with the sequence of elements given by the function ${\tt traverse}$ used in ${\tt read}$ operations, i.e., $\refmap(({\tt N},{\tt Tomb}))={\tt traverse(N, Tomb)}$. The fact that downstreams of ${\tt remove}$ operations, and ${\tt read}$ queries are simulated by the corresponding operations of the specification (defined in ???) is straightforward. Concerning downstreams of $\alabellongind[{\tt addAfter}]{a,b}{\bot}{\ats_b}{}$ operations, we show that they are simulated by the corresponding specification operation $\alabelshort[{\tt addAfter}]{a,b}$ only when the timestamp $\ats_{\tt b}$ is strictly greater than all the timestamps stored in the replica state where it applies. This is sufficient because, by $\mathsf{ReplicaStates}$, every replica state is obtained by applying downstreams according to the linearization of their corresponding operations, and the linearization order is consistent with the timestamp order. Thus, let $({\tt N},{\tt Tomb})$ be a replica state such that $\ats_{\tt a} < \ats_{\tt b}$ for every $\ats_{\tt a}$ such that $({\tt c},\ats_{\tt a},{\tt a})\in {\tt N}$ for some ${\tt c}$. The effect of the downstream $\effector$ corresponding to $\alabellongind[{\tt addAfter}]{a,b}{\bot}{\ats_b}{}$ is to add ${\tt b}$ as a child of ${\tt a}$. Then, applying ${\tt traverse}$ on the new state will result in a sequence where ${\tt b}$ is placed just after ${\tt a}$ because it has the biggest timestamp among the children of ${\tt a}$ (and all the nodes in the tree ${\tt N}$). This corresponds exactly to the sequence obtained by applying the operation $\alabelshort[{\tt addAfter}]{a,b}$ in the context of the specification.

To prove that $\mathsf{\CRDTLinshort{}}$ is an inductive invariant, we use almost the same arguments as in the case of OR-Set. The two notable differences are that in this case, the specification doesn't admit any sequence of updates and also, that downstreams of ${\tt addAfter}$ operations are simulated by the corresponding specification operations only under a certain condition related to timestamps. Concerning the first point, the specification requires that any $\alabelshort[{\tt addAfter}]{a,b}$ or $\alabelshort[{\tt remove}]{a}$ operation is preceded by an operation adding ${\tt a}$. However, since $\alinord$ is consistent with the visibility relation, and the preconditions of the RGA $\alabelshort[{\tt addAfter}]{a,b}$ and $\alabelshort[{\tt remove}]{a}$ operations ensure that an operation $\alabelshort[{\tt addAfter}]{c,a}$, for some ${\tt c}$, is visible when applying them at the origin replica, it follows that the projection of $\alinord$ on updates is admitted by the specification. Second, we have to show that the side-condition added to the simulation of ${\tt addAfter}$ downstreams still guarantees that any sequence $\mathsf{dseq}$ of downstream applications consistent with the linearization order (considering only such sequences is possible because of the $\mathsf{ReplicaStates}$ invariant) can be simulated by the specification. This follows from the fact that the linearization order is consistent with the timestamps generated by operations, which implies that any ${\tt addAfter}$ downstream adding an element ${\tt b}$ with timestamp $\ats_{\tt b}$ is ordered in $\mathsf{dseq}$ after all ${\tt addAfter}$ downstreams adding elements with timestamps smaller than $\ats_{\tt b}$.

TODO STATE WHICH OTHER OBJECTS SUPPORT THE SAME KIND OF PROOF.

%%% Local Variables:
%%% mode: latex
%%% TeX-master: "draft"
%%% End:
