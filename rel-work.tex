%!TEX root = draft.tex
\section{Related Work}
\label{sec:rel-work}

\noindent
{\bf CRDT Correctness Criteria.}
CRDTs are hard to specify since concurrent interactions between
replicas must be considered to provide accurate explanations for the
conflict resolution and convergence properties they provide.
%
\citet{BurckhardtGYZ14, Burckhardt14} provide one of the first formal
frameworks where CRDTs and other weakly consistent replicated systems
can be specified.
%
The specification language they provide uses event structures where
operations are structured using partial orders (e.g. visibility), and
the validity of the operations is characterized using a set of
axioms on these partial orders.
% characterize the validity of invalidity
%of behaviours of operations.
%
This specification style bears some similarities with the literature
of weakly consistent shared memory systems, e.g.,~\cite{AlglaveMT14}.
%
While the specification language
of~\cite{BurckhardtGYZ14,Burckhardt14} appears to be complete, this
comes at the cost of complexity of the specifications.
%
The specification of an operation in this framework is given as a
function whose inputs are the operation name and its arguments
together with a \emph{context} for the operation.
%
A context is a graph of operations equipped with two partial orders:
visibility and arbitration (similar to our visibility and linearization, respectively,
except that the visibility is existentially quantified and not extracted from the execution).
%
%Arbitration is a total order over the operations whose purpose is to
%represent the deterministic conflict resolution of the algorithms.
%
The validity of an operation is given by a set of axioms relating the
visibility and arbitration orders with the result value of the
operation.
%
Therefore, to specify the behavior of a CRDT in this model one is
forced to think about the context graph and the axioms that constrain
them.
%
One of the goals of \CRDTLinshort{} is to simplify specifications and
free the user from reasoning about complicated graphs resulting from
concurrent operations.

\smallskip
\noindent
{\bf Sequential Specifications for CRDTs}
We are not the first to observe that CRDT specifications can be
simplified by considering sequential histories.
%
Recent works~\cite{PerrinMJ14, JagadeesanR18} made similar
observations.
%
\citet{PerrinMJ14} provides Update Consistency, a criterion which to
the best of our knowledge is the first to consider a linearization of
updates and reason about the validity of queries using sub-sequences
of this linearization.
%
Unlike \CRDTLinshort{}, to show that a history is Update Consistent,
the visibility relation of the specification history is existentially
quantified, whereas in our definition the visibility relation is
directly extracted from the implementation trace.
%
Hence, \CRDTLinshort{} preserves in the specification histories the
causality of execution histories, whereas Update Consistency does not.
%
While their definition is similar to ours, they do not prove the
correctness of CRDT implementations or reason about compositionality.

% %
% \gpnote[nomargin, inline]{They don't provide proofs of algorithms.
%   There isn't even code. They claim that OR-Set is not update
%   consistent w.r.t. a Set spec.}

\citet{JagadeesanR18} provide a correctness criterion for CRDTs under
Eventual Consistency called SEC.
%
SEC shares some similarities with~\cite{PerrinMJ14} but the
formalization is quite different.
There are some important differences between SEC and \CRDTLinshort{}:
\begin{inparaenum}
\item Firstly, \CRDTLinshort{} -- as well as Update Consistency --
  have a global total order for updates. This is is not the case for
  SEC whose definition is quite complex.
\item
  \label{it-dependencies} Secondly, CRDT specifications in SEC are
  assumed to be parameterized by a \emph{dependency} relation at the
  level of the type's API.
  %
  Then, SEC assumes that all independent operations commute and
  disregards the order between them even when issued by the same client.
  %
  It is unclear how such a specification could adequately capture
  systems enforcing (any or all) the session guarantees
  of~\cite{TerryDPSTW94}.
  %
  \CRDTLinshort{} strives strive to preserve the guarantees of the
  underlying system network.
  %
  This is important since we assume that certain
  operations that appear independent at the API level could be
  made data or control dependent by the client code.
  %
  This is the case in our query-update relabelings where queries
  provide arguments for the subsequent update.
\item While SEC is also proved to be compositional, since operations
  from different objects are necessarily independent (the dependency
  relation is given at the level of the API of an object, hence no
  dependencies between objects) together with
  point~(\ref{it-dependencies}) above means that a history of two
  different SEC objects is trivially SEC since the order between
  operations of different objects is ignored.
  For instance, the
  history in \figureautorefname~\ref{fig:negative_composition} is SEC because
  the operations of the different OR-Set objects are independent and the
  order in which they were executed by the replicas $\arep_1$ or $\arep_2$
  is not accounted for.
%   interleaving
%  of specifications of two different objects is such that any two
%  operations on the different objects are independent.
%  %
%  Hence, the composition of specifications of two objects accepts any
%  interleaving, for example any interleaving of the per-object linearizations
%  of~\autoref{fig:negative_composition} would be accepted.
  %
  We find this definition of composition counter-intuitive, but more
  problematically, the composition of specifications cannot capture
  causality relations between the different objects, a common
  pattern when writing distributed applications (e.g. for referential
  integrity in a key-value store).
  %
  In \CRDTLinshort{} the composition of a set of objects respects the
  causality of the client code as illustrated by the failure to combine
  some per-object linearizations in
  \figureautorefname~\ref{fig:negative_composition}.\footnote{Note that
    there exist per-object linearizations for this history which can be
    merged into a global linearization
    (see~\sectionautorefname~\ref{sec:compositionality}).}
  %
%  We show however in~\autoref{sec:compositionality} conditions on the
 % CRDTs under which the composition can always be linearized.
  % \gpnote{Is this clear?}
\end{inparaenum}

\smallskip
\noindent
{\bf Verification of CRDTs}
There are several works that approach the problem of verifying that a
CRDT implementation is correct w.r.t. a specification.
%
Here we mention but a few.
%
In~\cite{BurckhardtGYZ14, AttiyaBGMYZ16, Burckhardt14} along with the
formal specification, proofs of correctness of implementations are
given for several CRDTs.
%
Of interest is the notion of Replication Aware Simulation introduced
in~\cite{BurckhardtGYZ14}.
%
Compared to these simulations, our proofs remain at a much more
abstract level -- and are therefore simpler -- since our
specifications are simpler.
%
We acknowledge however that their method is more general than ours
which only applies to CRDTs satisfying the \CRDTLinshort{} criterion.

\citet{ZellerBP14} and~\citet{GomesKMB17} provide frameworks for the
verification of CRDTs in Isabelle/HOL\@.
%
Both of these works introduce a methodology to specify and prove the
correctness of CRDTs.
%
The main difference with these works is that they do not provide a
consistency criterion, and their proofs are similar to the simulations
of~\cite{BurckhardtGYZ14}, albeit using a different -- but similar
-- specification language based on partial orders of operations.

\section{Conclusion}
\label{sec:conclusion}

We have presented \CRDTLin{}, a correctness criterion for CRDTs
inspired by linearizability which simplifies the specification of the
data types by resorting to sequential reasoning for the
specifications.
%
We provide proof methodologies to show \CRDTLinshort{} for a number of
well documented CRDTs.
%
Moreover we prove that under certain conditions these proofs guarantee
compositionality of \CRDTLinshort{}, hence enabling modular reasoning
for systems with multiple CRDT objects.
%

While in this work we have concentrated on the proof of
operation-based CRDTs (c.f. our system model in
\sectionautorefname~\ref{sec:overview}), we have initial results
showing that the same techniques apply to state-based CRDTs
(see~\cite{ShapiroPBZ11} for a definition).
%
However, due to space reasons this extension is not included in this
paper.
%
On a similar vein, while we have assumed Causal Consistency for this
paper, there is no reason why our results should not apply for weaker
system models.
%
We will explore this in future work.


%%% Local Variables:
%%% mode: latex
%%% TeX-master: "draft"
%%% End:
