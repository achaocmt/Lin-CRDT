\section{Related Work}
\label{sec:rel-work}

\paragraph{CRDT Correctness Criteria}
CRDTs are hard to specify since concurrent interactions between
replicas must be considered to provide accurate explanations for the
conflict resolution and convergence properties they provide.
%
\citet{BurckhardtGYZ14, Burckhardt14} provide the one of the first formal
frameworks where CRDTs and other weakly consistent replicated systems
can be specified.
%
The specification language they provide uses event structures where
operations are structured using partial orders (e.g. visibility), and
axioms on these partial orders characterize the validity of invalidity
of behaviours of operations.
%
This specification style bears some similarities with the literature
of weakly consistent shared memory systems~\cite{AlglaveMT14}.
%
While the specification language
of~\cite{BurckhardtGYZ14,Burckhardt14} appears to be complete, this
comes at the cost of complexity of the specifications.
%
The specification of an operation in this framework is given as a
function whose inputs are the operation name and its arguments
together with a \emph{context} for the operation.
%
A context is a graph of operations equipped with two partial orders:
visibility (similar to our visibility) and arbitration.
%
Arbitration is a total order over the operations whose purpose is to
represent the deterministic conflict resolution of the algorithms.
%
The validity of an operation is given by a set of axioms relating the
visibility and arbitration orders with the result value of the
operation.
%
Therefore, to specify the behavior of a CRDT in this model one is
forced to think about the context graph and the axioms that constraint
them.
%
One of the goals of \CRDTLinshort{} is to simplify specifications and
free the user from reasoning about complicated graphs resulting from
concurrent operations.

\paragraph{Linearizability for CRDTs}
We are not the firsts to observe that CRDT specifications can be
simplified by considering sequential histories.
%
Two recent works~\cite{PerrinMJ14, JagadeesanR18} made similar
observations.
%
\citet{PerrinMJ14} provides a criterion called Update Consistency,
which to the best of our knowledge is the first to consider a
linearization of updates and queries as resulting from sub-sequences
of this linearization.
%
Unlike \CRDTLinshort{}, to show that a history is Update Consistent,
the visibility relation of the specification history is existentially
quantified, whereas in our definition the visibility relation is
directly extracted from the implementation trace.
%
Hence, \CRDTLinshort{} preserves in the specification histories the
causality of execution histories, whereas Update Consistency does not.
%
While their definition is similar to ours, they do not prove the
correctness of CRDT implementations as we do.

% %
% \gpnote[nomargin, inline]{They don't provide proofs of algorithms.
%   There isn't even code. They claim that OR-Set is not update
%   consistent w.r.t. a Set spec.}

\citet{JagadeesanR18} provide a correctness criterion for CRDTs under
Eventual Consistency which they call SEC.
%
SEC shares some similarities with~\cite{PerrinMJ14} but the
formalization is quite different.
There are some important differences between SEC and \CRDTLinshort{}:
\begin{inparaenum}
\item while \CRDTLinshort{} has three simple requirements, the
  definition SEC is far from simple.
  %
  In particular, \CRDTLinshort{} as well as Update Consistency have a
  global total order for updates, which is not the case for
  SEC.\footnote{At least this is not apparent from the definitions, nor
    stated in the paper.}
\item
  \label{it-dependencies}
  A second important difference is that in for the SEC framework CRDT
  specifications are assumed to be parameterized by a given dependency
  relation at the level of the type API.
  %
  Then, their definition assumes that all independent operations
  commute (regardless whether these are issued by the same client or
  not).
  %
  It is unclear how such a specification could be adequately capture
  systems enforcing (some of the) session guarantees
  of~\cite{TerryDPSTW94}.
  %
  With \CRDTLinshort{} we strive to preserve the guarantees enforced by the
  underlying system network.
  %
  This is particularly important since we assume that certain
  operations that appear to be independent w.r.t. the API could be
  data or control dependent when considering the client code.
  %
  This is certainly the case when we relabel query-update operations
  since the query usually provides arguments for the subsequent
  update.
\item A last important difference concerns their compositionality
  result.
  %
  While their compositionality result appears similar to ours, the
  fact that operations stemming from different objects are necessarily
  independent (the dependency relation is given at the level of the API
  of an object, hence no dependencies between objects) in conjunction
  with the point~(\ref{it-dependencies}) above means that the interleaving
  of specifications of two different objects is such that any two
  operations on the different objects are independent.
  %
  Hence, the composition of specifications of two objects accepts any
  interleaving, for example any interleaving of the linearizations
  of~\autoref{fig:negative_composition} would be accepted.
  %
  We find this definition of composition counter-intuitive, but more
  problematically, the composition of specifications cannot capture
  causality relations between the different objects, a very common
  pattern when writing distributed applications (e.g. for referential
  integrity in a key-value store).
  %
  Our result is stronger since the composition of the specification of
  two objects will respect the causality resulting from the client code as
  illustrated by the failure to linearize the interleavings
  in~\autoref{fig:negative_composition}.\footnote{Note that there
    are other linearizations for which this composition is indeed
    linearizable (see~\sectionautorefname~\ref{sec:compositionality}).}
  %
  We show however in~\autoref{sec:compositionality} conditions on the
  CRDTs under which the composition can always be linearized.
  % \gpnote{Is this clear?}
\end{inparaenum}

\paragraph{Verification of CRDTs}
There are several works that approach the problem of verifying that a
CRDT implementation is correct w.r.t. a specification.
%
Here we mention but a few.
%
In~\cite{BurckhardtGYZ14, AttiyaBGMYZ16, Burckhardt14} along with the
formal specification, proofs of correctness of implementations are
given for several CRDTs.
%
Of interest is the notion of Replication Aware Simulation introduced
in~\cite{BurckhardtGYZ14}.
%
Compared to these simulations, our proofs remain at a much more
abstract level -- and are therefore simpler -- since our
specifications are simpler.
%
We acknowledge however that their method is more general than ours
which only applies to CRDTs satisfying the \CRDTLinshort{} criterion.

\citet{ZellerBP14} and~\citet{GomesKMB17} provide frameworks for the
verification of CRDTs in Isabelle/HOL\@.
%
Both of these works introduce a methodology to specify and prove the
correctness of CRDTs.
%
The main difference with these works is that they do not provide a
consistency criterion, and their proofs are similar to the simulations
of~\cite{BurckhardtGYZ14}, albeit using a different -- but similar
-- specification language based on partial orders of operations.

%%% Local Variables:
%%% mode: latex
%%% TeX-master: "draft"
%%% End:
