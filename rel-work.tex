\section{Related Work}
\label{sec:rel-work}

\paragraph{CRDT Correctness Criteria}
CRDTs are hard to specify since concurrent interactions between
replicas must be considered to provide accurate explanations for the
conflict resolution and convergence properties they provide.
%
\citet{BurckhardtGYZ14, Burckhardt14} provide the one of the first formal
frameworks where CRDTs and other weakly consistent replicated systems
can be specified.
%
The specification language they provide uses event structures where
operations are structured using partial orders (e.g. visibility), and
axioms on these partial orders characterize the validity of invalidity
of behaviours of operations.
%
This specification style bares some similarities with the literature
of weakly consistent shared memory systems~\cite{AlglaveMT14}.
%
While the specification language
of~\cite{BurckhardtGYZ14,Burckhardt14} appears to be complete, this
comes at the cost of complexity of the specifications.
%
To specify the behavior of a CRDT in this model one is forced to think
about the many partial orders and axioms that constraint them.
%
One of the goals of \CRDTLinshort{} is to simplify specifications and
free the user from reasoning about complicated graphs resulting from
concurrent operations.

\paragraph{Linearizability for CRDTs}
We are not the firsts to observe that CRDT specifications can be
simplified by considering sequential histories.
%
Two recent works~\cite{PerrinMJ14, JagadeesanR18} made similar
observations.
%
\citet{PerrinMJ14} provides a criterion called Update Consistency,
which to the best of our knowledge is the first to consider a
linearization of updates and queries as resulting from sub-sequences
of this linearization.
%
Unlike \CRDTLinshort{}, to show that a history is Update Consistent,
the visibility relation of the specification history is existentially
quantified, whereas in our definition the visibility relation is
directly extracted from the implementation trace.
%
Hence, \CRDTLinshort{} preserves in the specification histories the
causality of execution histories, whereas Update Consistency does not.
%
While their definition is similar to ours, they do not prove the
correctness of CRDT implementations as we do.

% %
% \gpnote[nomargin, inline]{They don't provide proofs of algorithms.
%   There isn't even code. They claim that OR-Set is not update
%   consistent w.r.t. a Set spec.}

\citet{JagadeesanR18} provide a correctness criterion for CRDTs under
Eventual Consistency which the call SEC.
%
SEC shares some similarities with~\cite{PerrinMJ14} but the
formalization is quite different.
There are some important differences between~\cite{JagadeesanR18} and
CRDTLinshort{}:
\begin{inparaenum}
\item while our definition of \CRDTLin{} has three simple
  requirements, the definition SEC is far from simple.
  %
  In particular, \CRDTLin{} as well as Update Consistency have a
  global total order for updates, which is not the case for
  SEC.\footnote{At least this is not apparent from the definitions nor
    stated in the paper.}
\item
  \label{it-dependencies}
  A second important difference is in that in their framework CRDT
  specifications are assumed to be parameterized by a given dependency
  relation between operations of the data type.
  %
  Then, their definition assumes that all independent operations
  commute (regardless whether these are issued by the same client or
  not).
  %
  It is unclear that such a specification could be adequate for
  systems enforcing (some of their) session guarantees
  of~\cite{TerryDPSTW94}.
  %
  With \CRDTLin{} we strive to preserve the guarantees enforced by the
  underlying network of the system.
  %
  This is particularly important since we assume that certain
  operations that appear to be independent w.r.t. the API could be
  data or control dependent when considering the client code.
  %
  This is certainly the case when we relabel query-update operations
  since the query usually provides arguments for the subsequent
  update.
\item Another important difference concerns their compositionality
  result.
  %
  While their compositionality result which appears to be similar to
  ours, the fact that operations stemming from different objects are
  necessarily independent (the dependency relation is given at the level
  of the API of an object, hence one cannot give one between objects) in
  conjunction with the point~ref{it-dependencies} means that the
  interleaving of specifications of two different objects makes the
  operations of the objects independent.
  %
  Hence, the composition of specifications of two objects accepts any
  interleaving of the example in~\autoref{fig:negative_composition}
  which we find at least counter-intuitive.
  %
  Our result is strictly stronger, since the composition of the
  specification of two objects will respect causality stemming from
  the client code as illustrated by the failure to linearize the
  interleavings of~\autoref{fig:negative_composition}.
  %
  We show however in~\autoref{sec:compositionality} conditions on the
  CRDTs under which the composition can always be linearized.
  \gpnote{Is this clear?}
\end{inparaenum}

\paragraph{Verification of CRDTs}
There are several works that approach the problem of verifying that a
CRDT implementation is correct w.r.t. a certain specification.
%
Here we mention but a few.
%
In~\cite{BurckhardtGYZ14, AttiyaBGMYZ16, Burckhardt14} along with the
formal specification proofs of correctness of implementations are
given for several CRDTs.
%
Of interest is the notion of Replication Aware Simulation introduced
in~\cite{BurckhardtGYZ14}.
%
Compared to these simulations, our proofs remain at a much more
abstract level -- and are therefore simpler -- since our
specifications are simpler.
%
We acknowledge however that their method is more general than ours
which only applies to CRDTs satisfying the \CRDTLinshort{} criterion.

\citet{ZellerBP14} and~\citet{GomesKMB17} provide frameworks for the
verification of CRDTs in Isabelle/HOL\@.
%
Both of these works introduce a methodology to specify and prove the
correctness of CRDTs.
%
The main difference with these works is that they do not provide a
consistency criterion, and their proofs are similar to the simulations
of~\cite{BurckhardtGYZ14}, albeit using a different -- but similar
-- specification language based on partial orders of operations.

%%% Local Variables:
%%% mode: latex
%%% TeX-master: "draft"
%%% End:
