%!TEX root = draft.tex
\section{Overview}
\label{sec:overview}

\begin{figure}[t]
  % \centering
\begin{lstlisting}[caption={Pseudo-code of the Replicated Growable
Array (RGA) CRDT (adapted from~\cite{ShapiroPBZ11})},
captionpos=b,label={lst:rga}]
  payload Ti-Tree N, Set Tomb
  initial N = @|$\emptyset$|@, Tomb = @|$\emptyset$|@

  addAfter(a,b) :
    atSource :
      precondition : a = @|$\circ$|@ or (a != @|$\circ$|@ and (a,_,_) @|$\in$|@ N and a @|$\not\in$|@ Tomb)
      let ts@|$_{\mathtt{b}}$|@ = getTimestamp()
    downStream(a, ts@|$_{\mathtt{b}}$|@, b) :
      precondition: a = @|$\circ$|@ or (a != @|$\circ$|@ and (a, ts@|$_{\mathtt{a}}$|@,_) @|$\in$|@ N)
      N = N @|$\cup$|@ {(a, ts@|$_{\mathtt{b}}$|@, b)}

  remove(a) :
    atSource :
      precondition : (a,_,_) @|$\in$|@ N and a @|$\notin$|@ Tomb
    downStream(a) :
      precondition : (a,_,_) @|$\in$|@ N
      Tomb = Tomb @|$\cup$|@ {a}

  read() :
    return traverse(N, Tomb)
\end{lstlisting}
\end{figure}

\gpnote*{TODO}{
  Section sketch:
  \begin{itemize}
  \item Informal explanation of the system model. Multiple replicas,
    clients connect to any replica. Operations are propagated lazily.
  \item Explanation of the program model. We explain the code of RGA
    (\lstinline|atSource|, \lstinline|downStream|). Conflict
    resolution. Commutativity.
  \item We give a first intuition about the \CRDTLin{} of RGA. We
    might explain why standard Linearizability is not good enough.
  \item We present OR-Set. We explain that the linearization of OR-Set
    is more complicated without getting too tangled into de details.
    OR-Set is important because we need to explain that some
    operations will see a sub-sequence of the global linearization
    (in contrast to RGA above).
  \item If needed we can talk about the compositionality problem.
  \end{itemize}
}

\paragraph{System Model.}

We assume that the system is comprised of multiple nodes in a network.
In this work we will be concerned with the implementation of CRDTs,
and we will usually concentrate our discussion to the behaviors
allowed for \emph{an instance} of the data type.
We will generically call such an instance an \emph{object}.
As mentioned in~\autoref{sec:introduction} we assume that objects are
replicated among the participating nodes of the system -- which we
shall call replicas.

The execution model for an object is as follows:
\begin{itemize}
\item clients, which are programs issuing calls to the object, connect
  to any one replica (node of the system holding a copy of the object)
  and performs the operation in that replica, we shall call such a
  replica the origin or \emph{source},
\item executing an operation is done in two phases. Assuming that the
  operation requires reading and updating the state of the object, the
  state of the object in the source replica is read first (we shall sometimes
  refer to this part of the operation as the generator
  following~\cite{ShapiroPBZ11}), and then, if the state needs to be changed as part of the operation -- e.g.
  an \lstinline|addRight| operation of RGA -- an update is
  generated which shall be executed in all the replicas holding copies
  of the object (we shall refer to the update as the effector).
  We assume throughout this paper that effectors are executed immediate
  in the source replica,
\item finally, as the system progresses, the effector of the operation
  will be delivered to each of the replicas holding a copy of the
  object.
\end{itemize}
This model is depicted in~\autoref{fig:system-model}.
\fxwarning[nomargin, inline]{We need the picture}.

\paragraph{CRDT implementations}

Following the description above~\cite{ShapiroPBZ11} presents the code
for a number of CRDT implementations.
%
Here we consider the code of the RGA algorithm presented
in~\autoref{lst:rga}.
%
Let us first consider the structure of the data type implementation:
\begin{itemize}
\item the keyword \lstinline|payload| introduces the state that is
  used to represent the object. This is akin to the fields of a class
  file in an object oriented language such as Java. In the specific
  case of RGA we hate a variable \lstinline|N| of type
  \lstinline|Ti-Tree| (to be discussed later), and a variable
  \lstinline|Tomb| of type \lstinline|Set|.
\item After that we find the definitions of the three operations we
  discussed in the introduction: \lstinline|addAfter|,
  \lstinline|remove| and \lstinline|read|.
\item The effectful operations \lstinline|addAfter| and
  \lstinline|remove| have two labels marked in red:
  \lstinline|atSource| and \lstinline|downStream|.
  These represent the code to be executed as the generator and
  effector respectively. Hence, the code under the label
  \lstinline|atSource| is executed only at the source replica and it
  generates the arguments that the code under \lstinline|downStream|
  will execute in each of the replicas.
\item We can also notice that under the labels there are
  \lstinline|precondition| annotations indicating facts that are
  assumed about the state of the object upon execution of either of
  the generator or effector of the operation.
\end{itemize}
Reconsidering~\autoref{fig:system-model} we can then say that the
source of the arrows in each replica represent the execution of an
\lstinline|atSource| jointly with the \lstinline|downStream| of the
operation at the source replica.
%
The sink of the arrows represents the delivery and execution of the
generator of the operation in a replica other than the source.

\paragraph{RGA CRDT implementation}
\gpwarning[nomargin, inline]{
We explain the \lstinline|Ti-Tree| and the tombstone of RGA. We give
an example of conflict resolution: - two \lstinline|addAfter(a, _)|
operations, and perhaps a - \lstinline|remove(_)| and
\lstinline|addAfter(_, _)| operation. We explain the preconditions,
their relation to causality, etc.
}

As it is common to many CRDT implementations, in RGA replicas will use
a timestamp mechanism to keep track of the causality between updates
to the list, effectively capturing when two updates are concurrent,
and moreover, they will keep the information relating the causal order
in which elements are added to the list.
%
Provided with this causality information, the timestamps will be used
to resolve conflicts in a deterministic way.
%
More concretely, each replica will keep what we shall name a
\emph{Timestamp Tree} (\lstinline|Ti-Tree|) containing in every tree
node a pair with the element added to the list (for instance the
character \lstinline|b|), and a timestamp associated to it
(\lstinline|t|$_{\mathtt{b}}$) which will be used to resolve
conflicts.
%
We will encode the tree as a set containing triples (representing
nodes) of the form (\lstinline|a|, \lstinline|ts|$_{b}$, \lstinline|b|)
representing the fact that there is an element \lstinline|b| in the
tree with timestamp \lstinline|ts|$_{\mathtt{b}}$ and whose parent is the item
\lstinline|a| also present in the tree.
%
The tree-ness property will be ensured by construction.


If we look at the \lstinline|atSource| portion of the
\lstinline|addAfter(a,b)| method we can see that precondition requires
the \lstinline|a| to exist in the tree before the insertion of
\lstinline|b| after it.
%
We remark at this point that the data structure is initialized with a
preexisting initial element $\circ$.
%
The generator then samples a timestamp \lstinline|t|$_{\mathtt{b}}$
for \lstinline|b| which is assumed to be larger than any
timestamp presently occurring in the \lstinline|Ti-Tree|
\lstinline|N|.
\gpnote{Add note about uniqueness of TSs (rid).}
%
Looking at the \lstinline|downStream| portion of
\lstinline|addAfter(a,b)| we see that the effect of the operation is
to add the triple \lstinline|(a, ts|$_{\mathtt{b}}$\lstinline|, b)|
representing the fact that the element \lstinline|b| is a child of the
item \lstinline|a| in the tree.
%
Then, the tree structure is representative of the causality of the
tree.
%
Notice that a client of the object will only ever attempt to add an
element after another element which has already seen (mandated by the
\lstinline|addAfter| operation of the API).
%
Hence, the parent node of any node is causally related before it.
%
Similarly, nodes that are not related to each other on any path of
the tree (eg. siblings) are not causally related.
%
An example of such a tree is shown in~\ref{fig:rga-tree}.
\gpwarning*{Example}{Ask Chao for a concrete example.}
%

Considering a \lstinline|Ti-tree| constructed in this way, we can
obtain a list by traversing the tree in pre-order fashion, with the
proviso that siblings are ordered according to their timestamps. 
%
\autoref{fig:rga-tree} shows the tree that results from the given
tree. 

We have so far ignored the \lstinline|remove| operation.
%
Consider the case where a client issues an \lstinline|addAfter(a, b)|
on a replica whereas another client issues a \lstinline|remove(a)|
operation in another replica. 
%
If the effector of \lstinline|remove(a)| reaches every replica after
the effector of \lstinline|addAfter(a, b)| there is no problem since
the semantics is obvious (the element \lstinline|a| is removed after
the element \lstinline|b| has been added). 
%
However, if the operations reach some replica in the opposite order
(recall that these are concurrent operations) we have a problem, since
the precondition of the effector of \lstinline|addAfter(a, b)|
requires that the element \lstinline|a| be present in the
\lstinline|Ti-tree| of the replica.

To avoid this kind of conflict, thus rendering these operations
commutative (c.f. CRDT), RGA does not really erase elements from the
\lstinline|Ti-tree|.
%
Instead, an additional data structure called a tombstone is used to
keep track of elements that have been conceptually erased and should
not be considered when reading the data structure with the
\lstinline|read| operation. 
%
In our case the tombstone is simply a set \lstinline|Tomb| of
elements. 
%
With this explanation the code of the \lstinline|remove| operation
in~\autoref{lst:rga} should be self-explanatory. 

Finally, the implementation of \lstinline|read| performs the pre-order
traversal as explained before, where all the elements in the tombstone
\lstinline|Tomb| are ignored from the output list. 



\ce{Make the distinction between an operation being \emph{originated} at some replica, and whose downstream is \emph{executed} at some replica}

\gpwarning[nomargin, inline]{I think it would be useful to show a
  graphical example of linearizations here. Something like the
  examples in the HW paper would do.}

\emph{Optimistic replication  algorithms} are a type of distributed algorithms where each client contains a copy of data structure; a client operations takes effect instantly at its replica without any synchronization, and then broadcast to other replicas and got applied. Convergent or Commutative Replicated Data Types (CRDTs) is a typical kinds of optimistic replication algorithms. In this section, we will introduce CRDT algorithms and their formation.\footnote{To be exact, there are two kinds of CRDT algorithms: state-based and operation based. The implementations we discussed in this section is operation-based. The state-based CRDT can be similarly formalized and verified, and we discuss state-based CRDT in the discussion section of this paper.}

In CRDT, each query operation works locally, while each update operation will inform other replica about its update. We takes a algorithm, replicated growable array (RGA), as an example of operation-based CRDT and it is shown below.

\renewcommand{\algorithmcfname}{CRDT Implementation}

\subsection{Example: Replicated Growable Array (RGA)}
\label{sec:rga}

Let us now consider as an example the implementation of the
Replicated Growable Array (RGA) CRDT, which can be used to represent
lists.
In fact, this CRDT is generally used to implement distributed
concurrent text editing~\cite{AttiyaBGMYZ16}.

As we mentioned before, the important characteristic of CRDT
implementations is that conflicts resulting form concurrent updates in
overlapping sections of the list must be resolved deterministically in
all replicas that have copies of the list.
%
To do so, each replica must keep enough metadata to
\begin{inparaenum}
\item identify when two operations are concurrent and conflicting, and
\item establish a deterministic resulting list when conflicts are detected.
\end{inparaenum}
%
As it is common to many CRDT implementations, replicas will use a
timestamp mechanism to keep track of the causality between updates to
the list, effectively capturing when two updates are concurrent, and
moreover, they will keep the information relating the causal order in
which elements are added to the list.
%
Provided with this causality information, the timestamps will be used
to resolve conflicts in a deterministic way.
%
More concretely, each replica will keep what we shall name a
\emph{Timestamp Tree} (Ti-Tree) containing in every node a pair with the
element added to the list (for instance a line in a text editing
application), and a timestamp which will be used to resolve conflicts.
The tree structure is representative of the causality of the tree.
Two nodes that are related by ancestry in the tree are causally
related, and nodes that are not related to each other on any path of
the tree (eg. siblings) are not causally related.
%
An example of such a tree is shown in~\ref{fig:rga-tree}.
\gpwarning*{Example}{Ask Chao for a concrete example.}
%

\autoref{lst:rga} shows the pseudo-code of an implementation of RGA
taken from~\cite{}.
%
Firstly, observe that the API of RGA.
%
The first method, \lstinline|addAfter(a, b)|, adds the element
\lstinline|b| after the element \lstinline|a|, where the latter is
assumed to already be included in the list.
%
The second method, \lstinline|remove(a)| simply removes the element
\lstinline|a| which again is assumed to be present in the list.
%
And finally, the methods read simply returns the contents of the whole
list at present.

Let us now consider the pseudo-code of~\autoref{lst:rga}.
%
The keyword \lstinline|payload| represents the data that is stored in
each replica to represent the list.
%
In our case we have a Ti-Tree \lstinline|N| and a set of elements
\lstinline|Tomb| which represents elements that have been removed.
%
These elements are not immediately removed from the tree \lstinline|N|
because their presence might necessary to resolve conflicts later on.
%
However, they are marked in \lstinline|Tomb| as being erased.
%
Thus, the set is said to be a \emph{tombstone} and hence its name.
%
The keyword \lstinline|initial| indicates the initial values for the
payload at the creation of the CRDT object.

We then find the implementation of the method \lstinline|addAfter|,
which updates the contents of the list.
%
Each effectful method has an immediate effect in the replica where it
is executed, which we call the source (or origin), and a delayed
effect on all the other replicas that keep a copy of the object.
%
The effect of a method in the source replica is described under the
label \lstinline|atSource|, and the effect of the method in the other
replicas is marked under the label \lstinline|downStream|.
%
Both of these effects have a precondition, which is simply an
annotation stating which are the assumptions required for their
correct execution.
%
In the case of the source replica, and \lstinline|addAfter| operation
requires that the item be added after the default special initial
object element $\circ$, or otherwise after an element that was
already present in the Ti-Tree of the list.
%
Moreover, it assumes that the element was not previously removed.

\gpnote{Continue from here}



Keyword $\mathit{payload}$ indicate the local state of a replica, and
keyword $\mathit{initial}$ specifies the initial value of local state.
Each update operation is executed with two phases: Its first phase, marked $\mathit{atSource}$, is local to the current replica and does not modify the local state. It is enabled if its (optional) pre-condition, marked $\mathit{pre}$, is true currently in local state. It generates the information to be delivered, which is the argument of $\mathit{downstream}$. Its second phase, marked $\mathit{downstream}$, executed immediate after the current replica, and asynchronously at other replica when they receive the message of this operation. It is enabled if its (optional) pre-condition is true. The local state will be modified in this phase.
In RGA algorithm, a replica store the list as a timestamp insertion tree (TI-tree) $N$, and stores the deleted items in tombstone $\mathit{Tomb}$. A TI-tree $N$ is a set of tuples $(a,t,p)$, where $a$ is a item, $t$ is its unique time-stamp, and $p$ is the time-stamp of its ``parent'' node. Each time-stamp is a tuple $(c,r)$ with $c \in \mathbb{N}$ and $r \in \mathbb{R}$. A total order $<_{\mathit{ts}}$ between time-stamps is defined, such that $(c_1,r_1) <_{\mathit{ts}} (c_2,r_2)$, if $c_1 < c_2 \vee (c_1 = c_2 \wedge r_1 <_r r_2)$, where $<_r$ is a total-order over $\mathbb{R}$. There is a pre-existed item $\circ$ of TI-tree with time stamp $(0,r_0)$, which are considered as the root of the tree. Each element of $N$ should have unique item and time stamp, and the elements of $N$ are required to form a tree by following the parent field. The tombstone $\mathit{Tomb}$ is a set of items and records items been removed from the list.

Method $\mathit{add}(a,b)$ intends to add item $a$ into the list at a position immediately after that of a existing item $b$. Method $\mathit{rem}(a)$ removes $a$ from the list. Method $\mathit{read}$ returns the current list content. When the current replica does $\mathit{add}(a,b)$, it generate a tuple $(a,ts_a,ts_b)$ and put it into $N$. Here $ts_b$ is the time-stamp of $b$, and $ts_a$ is a new time-stamp that is larger than any time stamp in $N$. When the current replica does $\mathit{rem}(a)$, it put $a$ into tombstone. When the current replica does $\mathit{read}$, it uses function $\mathit{trans}(N,\mathit{Tomb})$ to return the list seen by the current replica, which is a sequences obtained by traversing $N$ in prefix order (children are visited in decreasing time-stamp order) and keeping only items that are not in $\mathit{Tomb}$.

\begin{lstlisting}[caption={Pseudo-code of the or-set CRDT},
captionpos=b,label={lst:or-set}]
  payload Set S
  initial S = @|$\emptyset$|@
  initial seq = @|$\epsilon$|@

  add(a) :
    atSource :
      let k = getUniqueIdentifier()
      //@ let seq@|$'$|@ = seq@|$\,\cdot\,\alabelshort[add]{a,k}$|@
    downStream(a, k) :
      S = S @|$\cup$|@ {(a, k)}
      //@ S@|$'$|@ = S @|$\cup$|@ {(a, k)}

  remove(a) :
    atSource :
      let R = @|$\{$|@ (a,k): (a,k) @|$\in$|@ S @|$\}$|@
      //@ let seq@|$'$|@ = seq@|$\,\cdot\,\alabellongind[readIds]{a}{R}{}\,\cdot\,\alabelshort[remove]{a,R}$|@
    downStream(R) :
      S = S @|$\setminus$|@ R
      //@ R = @|$\{ (a,k): \exists\ \alabel = \alabellongind[add]{a,k}{\bot}{*}.\ (\alabel, \alabelshort[remove]{a,R}) \in \avisord$|@
                       @|$\land\,\forall\ \alabel' = \alabellongind[remove]{a,*}{\bot}{*}.\ \{(\alabel,\alabel'),(\alabel',\alabelshort[remove]{a,R})\}\not\subseteq \avisord\}$|@
      //@ S@|$'$|@ = S @|$\setminus$|@ R

  read() :
    let A = {a : @|$\exists$|@ k. (a,k) @|$\in$|@ S}
    //@ let seq@|$'$|@ = seq@|$\,\cdot\,\alabellongind[read]{}{A}{}$|@
    return A
\end{lstlisting}
%      precondition : %@|$\exists$|@ k. (a,k) @|$\in$|@ S
%let @|$\alabel = \alabellongind[remove]{a,R}{\bot}{i}$|@
%      //@ @|$\alpha(S) \xrightarrow{\alabelshort[add]{a,k}} \alpha(S')$|@
%       //@ @|$\alpha(S) \xrightarrow{\alabellongind[readIds]{a}{R}{}} \alpha(S)$|@
%       //@ @|$\alpha(S) \xrightarrow{\alabelshort[remove]{a,R}} \alpha(S')$|@
%     //@ @|$\alpha(S) \xrightarrow{\alabellongind[read]{}{A}{}} \alpha(S)$|@

In the downstream of a $\alabellongNoret[\mathit{add}]{\argv}$
operation, an tuple $(\argv,\ats)$ will be added to the local state;
while in the downstream of a $\alabellongNoret[\mathit{add}]{\argv}$
operation, a set $S_1$ will be removed from the local state. We call
such tuple $(\argv,\ats)$ or set $S_1$ the content of the downstream.
The following is the condition $C_2$ for or-set implementatio.

%%% Local Variables:
%%% mode: latex
%%% TeX-master: "draft"
%%% End:
