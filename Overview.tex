%!TEX root = draft.tex
\section{Overview}

\textblue{
Here we introduce:
\begin{itemize}
\item a first example of a CRDT implementation: RGA
\item the notion of history with an example
\item explain what does it mean that the history is linearizable (show by example that standard linearizability is not enough): we have to talk about specifications (that they are sequential), about linearizing the visibility, the relationship between the visibility and the linearization (say that this linearization is not really the arbitration order in Burkhardt's: the arbitration is for solving conflicts (usually defined by timestamps), our linearization is for explaining the behavior)
\item a second example: OR-set, talk about the rewriting of labels
\item talk about compositionality: that it doesn't hold for the original definition, that we define two strengthenings, and how they compose
\end{itemize}}

\ce{Make the distinction between an operation being \emph{originated} at some replica, and whose downstream is \emph{executed} at some replica}

$\mathit{Optimistic \ replication \ algorithms}$ is a type of distributed algorithms where each client contains a copy of data structure; a client operations takes effect instantly at its replica without any synchronization, and then broadcast to other replicas and got applied. Convergent or Commutative Replicated Data Types (CRDTs) is a typical kinds of optimistic replication algorithms. In this section, we will introduce CRDT algorithms and their formation \footnote{To be exact, there are two kinds of CRDT algorithms: state-based and operation based. The implementations we discussed in this section is operation-based. The state-based CRDT can be similarly formalized and verified, and we discuss state-based CRDT in the discussion section of this paper.}.

In CRDT, each query operation works locally, while each update operation will inform other replica about its update. We takes a algorithm, replicated growable array (RGA), as an example of operation-based CRDT and it is shown below.


\begin{lstlisting}[caption={Pseudo-code of the Replicated Growable
    Array (RGA) CRDT (adapted from~\cite{ShapiroPBZ11})}, captionpos=b,label={lst:rga}]
  payload Ti-Tree N, Set Tomb
  initial N = @|$\emptyset$|@, Tomb = @|$\emptyset$|@

  addAfter(a,b) :
    atSource :
      precondition : b = @|$\circ$|@ or (b != @|$\circ$|@ and (b,_,_) @|$\in$|@ N and b @|$\not\in$|@ Tomb)
      let ts@|$_{\mathtt{a}}$|@ = (N == @|$\emptyset$|@)?(1,myID()):(max{c | (_,(_,c),_) @|$\in$|@ N},myID())
      let ts@|$_{\mathtt{b}}$|@ = (b == @|$\circ$|@)?(0,r@|$_{0}$|@):(timestamp of b in N)
    downStream(a, ts@|$_{\mathtt{a}}$|@, ts@|$_{\mathtt{b}}$|@) :
      precondition: b = @|$\circ$|@ or (b != @|$\circ$|@ and (b, ts@|$_{\mathtt{b}}$|@,_) @|$\in$|@ N)
      N = N ts@|$\cup$|@ {(a, ts@|$_{\mathtt{a}}$|@, ts@|$_{\mathtt{b}}$|@)}

  remove(a) :
    atSource :
      precondition : a != @|$\emptyset$|@ and (a,_,_) @|$\in$|@ N and a @|$\notin$|@ Tomb
    downStream(a) :
      precondition : a != @|$\emptyset$|@ and (a,_,_) @|$\in$|@ N
      Tomb = Tomb @|$\cup$|@ {a}

  read() :
    return traverse(N, Tomb)
\end{lstlisting}


\renewcommand{\algorithmcfname}{CRDT Implementation}
\noindent
%\begin{minipage}{.5\textwidth}
\noindent% \begin{algorithm}[H]
% $\mathit{payload}$ TI-tree N, set $\mathit{Tomb}$; \\
% $\mathit{initial}$ $\emptyset$,$\emptyset$; \\

% $add(a,b)$ \\
% \ \ $\mathit{atSource}$: \\
% \ \ \ \ $\mathit{pre}$: \ $b = \circ \vee$ %( b \neq \circ \wedge (b,\_,\_) \in N \wedge b \notin \mathit{Tomb})$ \\

% % \ \ \ \ \If {$N = \emptyset$}
% % { \ \ \ \ let \ $ts_a$ = (myID(),1); \\ }
% % \ \ \ \ \Else
% % {\ \ \ \ let \ $ts_a$ = (myID(),$\mathit{max}\{ c' \vert (\_,(\_,c'),\_) \in N \} +1$); \\ }

% % \ \ \ \ \%If {$b = \circ$}
% %    { \ \ \ \ let \ $ts_b$ = (0,0); \\ }
% %\ \ \ \ \Else
% %    { \ \ \ \ let \ $ts_b$ be time-stamp of $b$ in $N$; \\ }

% \ \ \ \ let \ $ts_a$ = ($N = \emptyset$) ? (1,$\mathit{myID}$()) ! ($\mathit{max}\{ c' \vert (\_,(\_,c'),\_) \in N \} +1$,$\mathit{myID}$()); \\
% \ \ \ \ let \ $ts_b$ = ($b = \circ$) ? (0,$r_0$) ! (the time-stamp of $b$ in $N$); \\

% \ \ $\mathit{downstream}(a,ts_a,ts_b)$: \\
% \ \ \ \ $\mathit{pre}$: \ $b = \circ \vee ( b \neq \circ \wedge (b,ts_b,\_) \in N)$ \\

% \ \ \ \ $N = N \cup \{ (a,ts_a,ts_b) \}$.


% $rem(a)$ \\
% \ \ $\mathit{atSource}$: \\
% \ \ \ \ $\mathit{pre}$: \ $a \neq \circ \wedge (a,\_,\_) \in N \wedge a \notin \mathit{Tomb}$ \\

% \ \ $\mathit{downstream}(a)$: $\mathit{pre}$ \ $a \neq \circ \wedge (a,\_,\_) \in N)$

% \ \ \ \ $\mathit{Tomb} = \mathit{Tomb} \cup \{ a \}$.

% $read()$ \\
% \ \ \ \ \KwRet $\mathit{trans}(N,\mathit{Tomb})$; \\

% \caption{RGA}
% \label{Method-rga}
% \end{algorithm}

\subsection{Example: Replicated Growable Array (RGA)}
\label{sec:rga}

Let us now consider as an example the implementation of the
Replicated Growable Array (RGA) CRDT, which can be used to represent
lists.
In fact, this CRDT is generally used to implement distributed
concurrent text editing~\cite{AttiyaBGMYZ16}.

As we mentioned before, the important characteristic of CRDT
implementations is that conflicts resulting form concurrent updates in
overlapping sections of the list must be resolved deterministically in
all replicas that have copies of the list.
%
To do so, each replica must keep enough metadata to
\begin{inparaenum}
\item identify when two operations are concurrent and conflicting, and
\item establish a deterministic resulting list when conflicts are detected.
\end{inparaenum}
%
As it is common to many CRDT implementations, replicas will use a
timestamp mechanism to keep track of the causality between updates to
the list, effectively capturing when two updates are concurrent, and
moreover, they will keep the information relating the causal order in
which elements are added to the list.
%
Provided with this causality information, the timestamps will be used
to resolve conflicts in a deterministic way.
%
More concretely, each replica will keep what we shall name a
\emph{Timestamp Tree} (Ti-Tree) containing in every node a pair with the
element added to the list (for instance a line in a text editing
application), and a timestamp which will be used to resolve conflicts.
The tree structure is representative of the causality of the tree.
Two nodes that are related by ancestry in the tree are causally
related, and nodes that are not related to each other on any path of
the tree (eg. siblings) are not causally related.
%
An example of such a tree is shown in~\ref{fig:rga-tree}.
\gpwarning*{Example}{Ask Chao for a concrete example.}
%

\autoref{lst:rga} shows the pseudo-code of an implementation of RGA
taken from~\cite{}.
%
Firstly, observe that the API of RGA.
%
The first method, \lstinline|addAfter(a, b)|, adds the element
\lstinline|b| after the element \lstinline|a|, where the latter is
assumed to already be included in the list.
%
The second method, \lstinline|remove(a)| simply removes the element
\lstinline|a| which again is assumed to be present in the list.
%
And finally, the methods read simply returns the contents of the whole
list at present.

Let us now consider the pseudo-code of~\autoref{lst:rga}. 
%
The keyword \lstinline|payload| represents the data that is stored in
each replica to represent the list.
%
In our case we have a Ti-Tree \lstinline|N| and a set of elements
\lstinline|Tomb| which represents elements that have been removed.
%
These elements are not immediately removed from the tree \lstinline|N|
because their presence might necessary to resolve conflicts later on.
%
However, they are marked in \lstinline|Tomb| as being erased.
%
Thus, the set is said to be a \emph{tombstone} and hence its name.
%
The keyword \lstinline|initial| indicates the initial values for the
payload at the creation of the CRDT object.

We then find the implementation of the method \lstinline|addAfter|,
which updates the contents of the list.
%
Each effectful method has an immediate effect in the replica where it
is executed, which we call the source (or origin), and a delayed
effect on all the other replicas that keep a copy of the object.
%
The effect of a method in the source replica is described under the
label \lstinline|atSource|, and the effect of the method in the other
replicas is marked under the label \lstinline|downStream|.
%
Both of these effects have a precondition, which is simply an
annotation stating which are the assumptions required for their
correct execution.
%
In the case of the source replica, and \lstinline|addAfter| operation
requires that the item be added after the default special initial
object element $\circ$, or otherwise after an element that was
already present in the Ti-Tree of the list.
%
Moreover, it assumes that the element was not previously removed.

\gpnote{Continue from here}



Keyword $\mathit{payload}$ indicate the local state of a replica, and
keyword $\mathit{initial}$ specifies the initial value of local state.
Each update operation is executed with two phases: Its first phase, marked $\mathit{atSource}$, is local to the current replica and does not modify the local state. It is enabled if its (optional) pre-condition, marked $\mathit{pre}$, is true currently in local state. It generates the information to be delivered, which is the argument of $\mathit{downstream}$. Its second phase, marked $\mathit{downstream}$, executed immediate after the current replica, and asynchronously at other replica when they receive the message of this operation. It is enabled if its (optional) pre-condition is true. The local state will be modified in this phase.
In RGA algorithm, a replica store the list as a timestamp insertion tree (TI-tree) $N$, and stores the deleted items in tombstone $\mathit{Tomb}$. A TI-tree $N$ is a set of tuples $(a,t,p)$, where $a$ is a item, $t$ is its unique time-stamp, and $p$ is the time-stamp of its ``parent'' node. Each time-stamp is a tuple $(c,r)$ with $c \in \mathbb{N}$ and $r \in \mathbb{R}$. A total order $<_{\mathit{ts}}$ between time-stamps is defined, such that $(c_1,r_1) <_{\mathit{ts}} (c_2,r_2)$, if $c_1 < c_2 \vee (c_1 = c_2 \wedge r_1 <_r r_2)$, where $<_r$ is a total-order over $\mathbb{R}$. There is a pre-existed item $\circ$ of TI-tree with time stamp $(0,r_0)$, which are considered as the root of the tree. Each element of $N$ should have unique item and time stamp, and the elements of $N$ are required to form a tree by following the parent field. The tombstone $\mathit{Tomb}$ is a set of items and records items been removed from the list.

Method $\mathit{add}(a,b)$ intends to add item $a$ into the list at a position immediately after that of a existing item $b$. Method $\mathit{rem}(a)$ removes $a$ from the list. Method $\mathit{read}$ returns the current list content. When the current replica does $\mathit{add}(a,b)$, it generate a tuple $(a,ts_a,ts_b)$ and put it into $N$. Here $ts_b$ is the time-stamp of $b$, and $ts_a$ is a new time-stamp that is larger than any time stamp in $N$. When the current replica does $\mathit{rem}(a)$, it put $a$ into tombstone. When the current replica does $\mathit{read}$, it uses function $\mathit{trans}(N,\mathit{Tomb})$ to return the list seen by the current replica, which is a sequences obtained by traversing $N$ in prefix order (children are visited in decreasing time-stamp order) and keeping only items that are not in $\mathit{Tomb}$.

\begin{lstlisting}[caption={Pseudo-code of the or-set CRDT}, captionpos=b,label={lst:or-set}]
  payload Set S
  initial S = @|$\emptyset$|@
  @{ initial seq = @|$\epsilon$|@ }

  add(a) :
    atSource :
      let k = getUniqueIdentifier()
      //@ let seq@|$'$|@ = seq@|$\,\cdot\,\alabelshort[add]{a,k}$|@ 
    downStream(a, k) :
      S = S @|$\cup$|@ {(a, k)}
      //@ S@|$'$|@ = S @|$\cup$|@ {(a, k)} 
      //@ @|$\alpha(S) \xrightarrow{\alabelshort[add]{a,k}} \alpha(S')$|@

  remove(a) :
    atSource :
      let R = @|$\{$|@ (a,k): (a,k) @|$\in$|@ S @|$\}$|@
      //@ let seq@|$'$|@ = seq@|$\,\cdot\,\alabellongind[readIds]{a}{R}{}\,\cdot\,\alabelshort[remove]{a,R}$|@ 
      //@ @|$\alpha(S) \xrightarrow{\alabellongind[readIds]{a}{R}{}} \alpha(S)$|@
    downStream(R) :
      S = S @|$\setminus$|@ R
      //@ R = @|$\{ (a,k): \exists\ \alabel = \alabellongind[add]{a,k}{\bot}{*}.\ (\alabel, \downstreams^{-1}({\tt this})) \in \avisord$|@ 
      	               @|$\land\,\forall\ \alabel' = \alabellongind[remove]{a,*}{\bot}{*}.\ \{(\alabel,\alabel'),(\alabel',\downstreams^{-1}({\tt this}))\}\not\subseteq \avisord\}$|@ 
      //@ S@|$'$|@ = S @|$\setminus$|@ R
      //@ @|$\alpha(S) \xrightarrow{\alabelshort[remove]{a,R}} \alpha(S')$|@

  read() :
    let A = {a : @|$\exists$|@ k. (a,k) @|$\in$|@ S}
    //@ let seq@|$'$|@ = seq@|$\,\cdot\,\alabellongind[read]{}{A}{}$|@ 
    //@ @|$\alpha(S) \xrightarrow{\alabellongind[read]{}{A}{}} \alpha(S)$|@
    return A
\end{lstlisting}
%      precondition : %@|$\exists$|@ k. (a,k) @|$\in$|@ S
%let @|$\alabel = \alabellongind[remove]{a,R}{\bot}{i}$|@


In the downstream of a $\alabellongNoret[\mathit{add}]{\argv}$
operation, an tuple $(\argv,\ats)$ will be added to the local state;
while in the downstream of a $\alabellongNoret[\mathit{add}]{\argv}$
operation, a set $S_1$ will be removed from the local state. We call
such tuple $(\argv,\ats)$ or set $S_1$ the content of the downstream.
The following is the condition $C_2$ for or-set implementation. 

%%% Local Variables:
%%% mode: latex
%%% TeX-master: "draft"
%%% End:
