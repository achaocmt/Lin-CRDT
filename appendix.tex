\section{Definitions of Section \ref{sec:CRDT implementations}}
\label{sec:appendix definitions of section CRDT implementations}







\subsection{OR-set Implementation and Formation}
\label{subsec:or-set implementation and formation}

The or-set implementation is shown below. Here function $\mathit{myRep}()$ returns the current replica identifier.

\renewcommand{\algorithmcfname}{CRDT Implementation}
\noindent
%\begin{minipage}{.5\textwidth}
\noindent\begin{algorithm}[H]
$\mathit{payload}$ set $S$; $\mathit{maxTS}$\\
$\mathit{initial}$ $\emptyset$; $(0,\mathit{myRep}())$\\

$add(a)$ \\
\ \ $\mathit{atSource}$: \\
\ \ \ \ assume $\mathit{maxTS} = (c,r')$; \\
\ \ \ \ let $\mathit{ts}' =(c+1,\mathit{myRep}())$; \\

\ \ $\mathit{downstream}((a,\mathit{ts}'))$: \\
\ \ \ \ $S = S \cup \{ (a,\mathit{ts}') \}$; \\
\ \ \ \ $\mathit{maxTS} = \mathit{max} \{ \mathit{maxTS},\mathit{ts}' \}$;


$rem(a)$ \\
\ \ $\mathit{atSource}$: \\
\ \ \ \ $\mathit{pre}$: \ $\exists \mathit{ts}', (a,\mathit{ts}') \in S$ \\
\ \ \ \ let $S_1 = \{ (a,\mathit{ts}') \vert (a,\mathit{ts}') \in S \}$; \\

\ \ $\mathit{downstream}(S_1)$: \\
\ \ \ \ $S = S \setminus S_1$.

$read()$ \\
\ \ \ \ \KwRet $\{ a \vert \exists \mathit{ts}, (a,\mathit{ts}) \in S \}$; \\

\caption{OR-set}
\label{Method-or-set}
\end{algorithm}


The formation of or-set is as follows: $I(r) = (\Sigma, \sigma_0, \mathit{Msg}, \mathit{do},\mathit{receive})$, where

\begin{itemize}
\setlength{\itemsep}{0.5pt}
\item[-] $\Sigma = \{ (S,\mathit{ts}) \vert$, $S$ is a set, each item of $S$ is of the form $(a',\mathit{ts}')$ with $a' \in D$ and $\mathit{ts}' \in \mathbb{N} \times \mathbb{R}.$ $\mathit{ts} \in \mathbb{N} \times \mathbb{R} \}$. $\Sigma_0 = (\emptyset,(0,\mathit{myRep}()))$.

\item[-] Each message content in $\mathit{Msg}$ is either in $D \times \mathbb{N} \times \mathbb{R}$, or a subset of $D \times \mathbb{N} \times \mathbb{R}$.

\item[-] $\mathit{do}((S,(c,r')),\mathit{add},a) = ((S \cup \{ (a, (c+1,r)) \}, (c+1,r)),(a,(c+1,r)))$.

\item[-] If $\exists \mathit{ts}', (a,\mathit{ts}') \in S$, then $\mathit{do}((S,\mathit{ts}),\mathit{rem},a) = ((S \setminus S_1,\mathit{ts}), S_1)$, where $S_1 = \{ (a,\mathit{ts}'') \in S \}$.

\item[-] $\mathit{do}((S,\mathit{ts}),\mathit{read}) = ((S,\mathit{ts}),\{ a \vert \exists \mathit{ts}', (a,\mathit{ts})' \in S \})$.

\item[-] $\mathit{receive}((S,\mathit{ts}),(a,\mathit{ts}')) = (S \cup \{ (a,\mathit{ts}') \}, \mathit{max}( \mathit{ts},\mathit{ts}' ))$,

\item[-] $\mathit{receive}((S,\mathit{ts}),S_1) = (S \setminus S_1,\mathit{ts})$,
\end{itemize}





\section{Proofs of Section \ref{sec:proving distributed linearizability}}
\label{sec:appendix proofs of section proving distributed linearizability}



\subsection{Proof in Subsection \ref{subsec:t0 specification and t1 specification}}
\label{subsec:proof in subsection subsec t0 specification and t1 specification}





\subsection{Proof of OR-set Implementation}
\label{subsec:appendix proofs of or-set implementation}

The following lemma states a property that can be generated from $P(\mathit{config},h,\mathit{lin},\mathit{map})$ for or-set.

\begin{lemma}
\label{lemma:a property that can be obtained from P for or-set}
If $P(\mathit{config},h,\mathit{lin},\mathit{map})$ holds for or-set, then each $\mathit{add}$ operation generate a new unique time-stamp. Moreover, for each replica $r'$,

\begin{itemize}
    \setlength{\itemsep}{0.5pt}
    \item[-] $R(r').S = \{ (b,\mathit{ts}') \vert b \in D, \exists o' = (\mathit{add}(b),\_,\_,\mathit{ts}'), o' \in \mathit{vd}(h,\mathit{del},r'), \forall o'' = (\mathit{rem}(b),\_,\_,\_), o'' \in \mathit{vd}(h,\mathit{config},r') \Rightarrow (o',o'') \notin h.\mathit{vis} \}$.

    \item[-] $R(r').\mathit{maxTS} = (0,r')$ if $\mathit{vd}(h,\mathit{config},r') = \emptyset$; otherwise, $R(r').\mathit{maxTS}$ is the maximal time stamp of $\mathit{add}$ operations of $\mathit{vd}(h,\mathit{config},r')$.
    \end{itemize}
\end{lemma}

\begin {proof}
By $C_4$, it is easy to see that each $\mathit{add}$ operation generate a new unique time-stamp by induction. The property of $R(r')$ can be also easily proved by induction, since the visibility relation is transitive. $\qed$
\end {proof}


The following lemma states that our $P(\mathit{config},h,\mathit{lin},\mathit{map})$ is an invariant of or-set.

\begin{lemma}
\label{lemma:P is an invariant of or-set}
$P(\mathit{config},h,\mathit{lin},\mathit{map})$ is an invariant of or-set.
\end{lemma}

\begin {proof}

Let us prove that $P$ is a simulation relation. It is obvious that $P(\mathit{config}_0,\epsilon,\emptyset,\emptyset)$ holds.

Assume $P((R,T,\mathit{MsgHB},\mathit{MsgDel}),h,\mathit{lin},\mathit{map})$ holds. Here we do not give the detailed value of $\mathit{MsgHB}'$ and $\mathit{MsgDel}'$, since it can be obtained from the definition of $\llbracket \mathit{obj} \rrbracket_{\mathit{op}}$.

\begin{itemize}
\setlength{\itemsep}{0.5pt}
\item[-] If $(R,T,\mathit{MsgHB},\mathit{MsgDel}) {\xrightarrow{\mathit{do}(\mathit{add},a,r,\mathit{mid})}} (R',T',\mathit{MsgHB}',\mathit{MsgDel}')$: Then,

    \begin{itemize}
    \setlength{\itemsep}{0.5pt}
    \item[-] $R' = R[ r: (R(r).S \cup \{ (a,\mathit{ts}) \},\mathit{ts}) ]$ and $T' = T \cup \{ (\mathit{mid},(a,\mathit{ts}),r) \}$. Here $\mathit{ts} = ( \mathit{max} \{ c \vert (\_,(c,\_)) \in R(r).S \} +1,r)$.

    \item[-] Let $h' = h \otimes i$, where $i$ is the identifier of the newly-generated $\mathit{add}$ operation.

    \item[-] Let $\mathit{lin}' = \mathit{lin} \cdot (\mathit{add}(a),i,\mathit{vd}(h,\mathit{config},r))$.

    \item[-] Let $\mathit{map}' = \mathit{map} \cup \{ (\mathit{mid},i) \}$.
    \end{itemize}

    It is easy to see that $h'$ is still distributed linearizable and $\mathit{lin}'$ is its linearization. We need to prove that $R'(r) = \mathit{apply}(\mathit{lin}',\mathit{vd}(h',\mathit{del}',r))$ and $C_4$ still holds for message $\mathit{mid}$.

    We already know that $R(r) = \mathit{apply}(\mathit{lin},\mathit{vd}(h,\mathit{del},r))$. %Based on $C_4$, it is not hard to prove that,

    %\begin{itemize}
    %\setlength{\itemsep}{0.5pt}
    %\item[-] $\mathit{Prop}_1$: Each $\mathit{add}$ operation generate a new unique time-stamp.

    %\item[-] $\mathit{Prop}_2$: for each replica $r'$, $R(r') = \{ (b,\mathit{ts}') \vert b \in D, \exists o' = (\mathit{add}(b),\_,\_,\mathit{ts}'), o' \in \mathit{vd}(h,\mathit{config},r'), \forall o'' = (\mathit{rem}(b),\_,\_,\_), o'' \in \mathit{vd}(h,\mathit{config},r') \Rightarrow (o',o'') \notin h.\mathit{vis} \}$.
    %\end{itemize}

    By Lemma \ref{lemma:a property that can be obtained from P for or-set}, it is not hard to see that $C_4$ still holds for message $\mathit{mid}$. From construction of $R'(r)$, Lemma \ref{lemma:a property that can be obtained from P for or-set} and $C_4$ holds for message $\mathit{mid}$, we can see that $R'(r) = \mathit{apply}(\mathit{lin}',\mathit{vd}(h',\mathit{del}',r))$.%, and $\mathit{Prop}_1$ and $\mathit{Prop}_2$ also hold for $(\mathit{config}',h',\mathit{lin}',\mathit{map}')$.


\item[-] If $(R,T,\mathit{MsgHB},\mathit{MsgDel}) {\xrightarrow{\mathit{do}(\mathit{rem},a,r,\mathit{mid})}} (R',T',\mathit{MsgHB}',\mathit{MsgDel}')$: Then,

    \begin{itemize}
    \setlength{\itemsep}{0.5pt}
    \item[-] $R' = R[ r: (R(r).S \setminus \{ (a,\mathit{ts}) \in R(r).S \},R(r).\mathit{maxTS}) ]$ and $T' = T \cup \{ (\mathit{mid},\{ (a,\mathit{ts}) \in R(r) \},r) \}$.

    \item[-] Let $h' = h \otimes i$, where $i$ is the identifier of the newly-generated $\mathit{rem}$ operation.

    \item[-] Let $\mathit{lin}' = \mathit{lin} \cdot (\mathit{add}(a),i,\mathit{vd}(h,\mathit{config},r))$.

    \item[-] Let $\mathit{map}' = \mathit{map} \cup \{ (\mathit{mid},i) \}$.
    \end{itemize}

    It is easy to see that $h'$ is still distributed linearizable and $\mathit{lin}'$ is its linearization. We need to prove that $R'(r) = \mathit{apply}(\mathit{lin}',\mathit{vd}(h',\mathit{del}',r))$ and $C_4$ still holds for message $\mathit{mid}$.

    By Lemma \ref{lemma:a property that can be obtained from P for or-set}, it is not hard to see that $C_4$ still holds for message $\mathit{mid}$. From construction of $R'(r)$, Lemma \ref{lemma:a property that can be obtained from P for or-set} and $C_4$ holds for message $\mathit{mid}$, we can see that $R'(r) = \mathit{apply}(\mathit{lin}',\mathit{vd}(h',\mathit{del}',r))$.%, and $\mathit{Prop}_1$ and $\mathit{Prop}_2$ also hold for $(\mathit{config}',h',\mathit{lin}',\mathit{map}')$.


\item[-] If $(R,T,\mathit{MsgHB},\mathit{MsgDel}) {\xrightarrow{\mathit{do}(\mathit{read},S_1,r)}} (R',T',\mathit{MsgHB}',\mathit{MsgDel}')$: Then,

    \begin{itemize}
    \setlength{\itemsep}{0.5pt}
    \item[-] $R' = R$ and $T' = T$.

    \item[-] Let $h' = h \otimes i$, where $i$ is the identifier of the newly-generated $\mathit{read}$ operation.

    \item[-] Let $\mathit{lin}' = \mathit{lin} \cdot (\mathit{read} \Rightarrow S_1,i,\mathit{vd}(h,\mathit{config},r))$.

    \item[-] Let $\mathit{map}'$.
    \end{itemize}

    We need to prove that $h'$ is distributed linearizable and $\mathit{lin}'$ is a linearization. Assume that in $\mathit{OR}$-$\mathit{set}_s$, $\mathit{state}_0 {\xrightarrow{\mathit{lin}}} \mathit{state}$ and $\mathit{state} {\xrightarrow{ (\mathit{read} \Rightarrow S_2, i, \mathit{vd}(h,\mathit{config},r) ) }} \mathit{state}$. Then by definition of $\mathit{OR}$-$\mathit{set}_s$, we can see that, $a \in S_2$, if there exists $(\mathit{add}(a),j,\_) \in \mathit{lin}'$, and for each $(\mathit{rem}(a),\_,S_2) \in \mathit{lin}'$, we have $j \notin S_2$. Lemma \ref{lemma:a property that can be obtained from P for or-set}, we can see that $S_1 = S_2$, and since $h$ is distributed linearizable and $\mathit{lin}$ is a linearization of $h$, we can see $h'$ is distributed linearizable and $\mathit{lin}'$ is a linearization.

\item[-] If $(R,T,\mathit{MsgHB},\mathit{MsgDel}) {\xrightarrow{\mathit{receive}(\mathit{mid},r)}} (R',T',\mathit{MsgHB}',\mathit{MsgDel}')$, where $(\mathit{mid},(a,\mathit{ts}),r') \in T$: Then,

    \begin{itemize}
    \setlength{\itemsep}{0.5pt}
    \item[-] $R' = R[ r: (R(r).S \cup \{ (a,\mathit{ts}) \},\mathit{max} \{ R(r).\mathit{maxTS},\mathit{ts} \} ) ]$ and $T' = T$.

    \item[-] Let $h' = h$.

    \item[-] Let $\mathit{lin}' = \mathit{lin}$.

    \item[-] Let $\mathit{map}' = \mathit{map}$.
    \end{itemize}

    We need to prove that $R'(r) = \mathit{apply}(\mathit{lin}',\mathit{vd}(h',\mathit{del}',r))$.

    We already know that $R(r) = \mathit{apply}(\mathit{lin},\mathit{vd}(h,\mathit{del},r))$. Since $R'(r)$ is obtained from $R(r)$ by applying message $\mathit{mid}$, and $\mathit{apply}(\mathit{lin}',\mathit{vd}(h',\mathit{del}',r))$ is obtained from $\mathit{apply}(\mathit{lin},\mathit{vd}(h,\mathit{del},r))$ by applying message $\mathit{mid}$. Therefore, $R'(r) = \mathit{apply}(\mathit{lin}',\mathit{vd}(h',\mathit{del}',r))$.

\item[-] If $(R,T,\mathit{MsgHB},\mathit{MsgDel}) {\xrightarrow{\mathit{receive}(\mathit{mid},r)}} (R',T',\mathit{MsgHB}',\mathit{MsgDel}')$, where $(\mathit{mid},S_1,r') \in T$: Then,

    \begin{itemize}
    \setlength{\itemsep}{0.5pt}
    \item[-] $R' = R[ r: (R(r).S \setminus S_1, R(r).\mathit{maxTS}) ]$ and $T' = T$.

    \item[-] Let $h' = h$.

    \item[-] Let $\mathit{lin}' = \mathit{lin}$.

    \item[-] Let $\mathit{map}' = \mathit{map}$.
    \end{itemize}

    We need to prove that $R'(r) = \mathit{apply}(\mathit{lin}',\mathit{vd}(h',\mathit{del}',r))$.

    We already know that $R(r) = \mathit{apply}(\mathit{lin},\mathit{vd}(h,\mathit{del},r))$. Since $R'(r)$ is obtained from $R(r)$ by applying message $\mathit{mid}$, and $\mathit{apply}(\mathit{lin}',\mathit{vd}(h',\mathit{del}',r))$ is obtained from $\mathit{apply}(\mathit{lin},\mathit{vd}(h,\mathit{del},r))$ by applying message $\mathit{mid}$. Therefore, $R'(r) = \mathit{apply}(\mathit{lin}',\mathit{vd}(h',\mathit{del}',r))$.
\end{itemize}

This completes the proof of this lemma. $\qed$
\end {proof}




\subsection{Proof of RGA}
\label{subsec:appendix proofs of rga}

The following lemma states a property that can be generated from $P(\mathit{config},h,\mathit{lin},\mathit{map})$ for RGA.

\begin{lemma}
\label{lemma:a property that can be obtained from P for rga}
If $P(\mathit{config},h,\mathit{lin},\mathit{map})$ holds, then each $\mathit{add}$ operation generate a new unique time-stamp. Moreover, for each replica $r'$,

\begin{itemize}
    \setlength{\itemsep}{0.5pt}
    \item[-] $R(r').N = \{ (a,\mathit{ts}_a,\mathit{ts}_b) \vert \exists o' = (\mathit{add}(\_,\_),i,\_,\_), \mathit{map}(i) = (a,\mathit{ts}_a,\mathit{ts}_b), o' \in \mathit{vd}(h,\mathit{config},r') \}$.

    \item[-] $R(r').\mathit{Tomb} = \{ a \vert \exists o = (\mathit{rem}(a),i,\_,\_), \mathit{map}(i) \in \mathit{vd}(h,\mathit{config},r') \}$.
    \end{itemize}
\end{lemma}

\begin {proof}
By $C_4$, it is easy to see that each $\mathit{add}$ operation generate a new unique time-stamp by induction. The property of $R(r')$ can be also easily proved by induction, since the visibility relation is transitive. $\qed$
\end {proof}


The following lemma states that our $P(\mathit{config},h,\mathit{lin},\mathit{map})$ is an invariant of rga.

\begin{lemma}
\label{lemma:P is an invariant of rga}
$P(\mathit{config},h,\mathit{lin},\mathit{map})$ is an invariant of rga.
\end{lemma}

\begin {proof}

Let us prove that $P$ is a simulation relation. It is obvious that $P(\mathit{config}_0,\epsilon,\emptyset,\emptyset)$ holds.

Assume $P((R,T,\mathit{MsgHB},\mathit{MsgDel}),h,\mathit{lin},\mathit{map})$ holds. Here we do not give the detailed value of $\mathit{MsgHB}'$ and $\mathit{MsgDel}'$, since it can be obtained from the definition of $\llbracket \mathit{obj} \rrbracket_{\mathit{op}}$.

\begin{itemize}
\setlength{\itemsep}{0.5pt}
\item[-] If $(R,T,\mathit{MsgHB},\mathit{MsgDel}) {\xrightarrow{\mathit{do}(\mathit{add},a,b,r,\mathit{mid})}} (R',T',\mathit{MsgHB}',\mathit{MsgDel}')$: Then,

    \begin{itemize}
    \setlength{\itemsep}{0.5pt}
    \item[-] $R' = R[ r: (R(r).N \cup \{ (a,\mathit{ts}_a,\mathit{ts}_b) \}, R(r).\mathit{Tomb}) ]$ and $T' = T \cup \{ (\mathit{mid},(a,\mathit{ts}_a,\mathit{ts}_b),r) \}$. Here $\mathit{ts}_a = ( \mathit{max} \{ c \vert (\_,(c,\_),\_) \in R(r).N \vee (\_,\_,(c,\_)) \in R(r).N \} +1,r)$, and $\mathit{ts}_b$ is the time-stamp of $b$ in $R(r).N$.

    \item[-] Let $h' = h \otimes i$, where $i$ is the identifier of the newly-generated $\mathit{add}$ action.

    \item[-] $\mathit{lin}'$ is obtained from $\mathit{lin}$ by inserting $(\mathit{add}(a,b),i,\mathit{vd}(h,\mathit{config},r))$ after the last operation with time-stamp less or equal than $\mathit{ts}_a$.

    \item[-] Let $\mathit{map}' = \mathit{map} \cup \{ (\mathit{mid},i) \}$.
    \end{itemize}

    It is easy to see that $h'$ is still distributed linearizable and $\mathit{lin}'$ is its linearization. We need to prove that $R'(r) = \mathit{apply}(\mathit{lin}',\mathit{vd}(h',\mathit{del}',r))$ and $C_4$ still holds for message $\mathit{mid}$.

    We already know that $R(r) = \mathit{apply}(\mathit{lin},\mathit{vd}(h,\mathit{del},r))$.

    By Lemma \ref{lemma:a property that can be obtained from P for rga}, it is not hard to see that $C_4$ still holds for message $\mathit{mid}$. From the fact that $\mathit{ts}_a$ is unique, the fact that there is no $\mathit{rem}(a)$ in $h$, the construction of $R'(r)$, Lemma \ref{lemma:a property that can be obtained from P for rga} and $C_4$ holds for message $\mathit{mid}$, we can see that $R'(r) = \mathit{apply}(\mathit{lin}',\mathit{vd}(h',\mathit{del}',r))$.


\item[-] If $(R,T,\mathit{MsgHB},\mathit{MsgDel}) {\xrightarrow{\mathit{do}(\mathit{rem},a,r,\mathit{mid})}} (R',T',\mathit{MsgHB}',\mathit{MsgDel}')$: Then,

    \begin{itemize}
    \setlength{\itemsep}{0.5pt}
    \item[-] $R' = R[ r: (R(r).N,R(r).\mathit{Tomb} \cup \{ a \} ) ]$ and $T' = T \cup \{ (\mathit{mid},\{ a \},r) \}$.

    \item[-] Let $h' = h \otimes i$, where $i$ is the identifier of the newly-generated $\mathit{rem}$ operation.

    \item[-] $\mathit{lin}'$ is obtained from $\mathit{lin}$ by inserting $(\mathit{rem}(a),i,\mathit{vd}(h,\mathit{config},r))$ after the last operation with time-stamp less or equal than the time-stamp of operation $i$.

    \item[-] Let $\mathit{map}' = \mathit{map} \cup \{ (\mathit{mid},i) \}$.
    \end{itemize}

    By Lemma \ref{lemma:a property that can be obtained from P for rga}, it is easy to see that $\mathit{lin} \uparrow_{\mathit{vd}(h,\mathit{config},r)}$ contains a $\mathit{add}(a,\_)$ operation $o$ and $(o,i) \in h'.\mathit{vis}$. By Lemma \ref{lemma:a property that can be obtained from P for rga}, it is easy to see that $\mathit{lin} \uparrow_{\mathit{vd}(h,\mathit{config},r)}$ does not contain $\mathit{rem}(a)$. Since $i$ does not visible to any operation in $\mathit{vd}(h,\mathit{config},r)$, we can see that $h'$ is still distributed linearizable and $\mathit{lin}'$ is its linearization.

    We need to prove that $R'(r) = \mathit{apply}(\mathit{lin}',\mathit{vd}(h',\mathit{del}',r))$ and $C_4$ still holds for message $\mathit{mid}$.

    It is obvious that $C_4$ holds for message $\mathit{mid}$. By Lemma \ref{lemma:a property that can be obtained from P for rga}, the construction of $R'(r)$, and $C_4$ holds for message $\mathit{mid}$, we can see that $R'(r) = \mathit{apply}(\mathit{lin}',\mathit{vd}(h',\mathit{del}',r))$.


\item[-] If $(R,T,\mathit{MsgHB},\mathit{MsgDel}) {\xrightarrow{\mathit{do}(\mathit{read},l,r)}} (R',T',\mathit{MsgHB}',\mathit{MsgDel}')$: Then,

    \begin{itemize}
    \setlength{\itemsep}{0.5pt}
    \item[-] $R' = R$ and $T' = T$.

    \item[-] Let $h' = h \otimes i$, where $i$ is the identifier of the newly-generated $\mathit{read}$ operation.

    \item[-] $\mathit{lin}'$ is obtained from $\mathit{lin}$ by inserting $(\mathit{rem}(a),i,\mathit{vd}(h,\mathit{config},r))$ after the last operation with time-stamp less or equal than the time-stamp of operation $i$.

    \item[-] Let $\mathit{map}' = \mathit{map}$.
    \end{itemize}

    We need to prove that $h'$ is distributed linearizable and $\mathit{lin}'$ is a linearization. Assume that in $\mathit{list}_s^{\mathit{af}}$, $\mathit{state}_0 {\xrightarrow{\mathit{lin}}} \mathit{state}$ and $\mathit{state} {\xrightarrow{ (\mathit{read} \Rightarrow l_1, i, \mathit{vd}(h,\mathit{config},r) ) }} \mathit{state}$.

    By Lemma \ref{lemma:a property that can be obtained from P for rga} and RGA implementation, we can see that $l$ and $l_1$ has the same items.

    Given items $a,b$, assume that $a$ is before $b$ in $l$, then, there are two possibilities,

    \begin{itemize}
    \setlength{\itemsep}{0.5pt}
    \item[-] $a$ is a ancestor of $b$ in $R(r).N$,

    \item[-] there exists items $c_1,c_2,c_3$, such that in $R(r).N$, $c_2$ and $c_3$ are sons of $c_1$, $c_2$ is a ancestor of $a$, $c_3$ is a ancestor of $b$, and the time-stamp of $c_2$ is larger than that of $c_3$.
    \end{itemize}

    If the first possibility holds, then there exists items $d_1,\ldots,d_k$, such that in $R(r).N$, $b$ is a son of $d_1$, $d_1$ is a son of $d_2$, $\ldots$, and $d_k$ is a son of $a$. It is easy to see that $(\mathit{add}(a,\_),\mathit{add}(d_k,a)),(\mathit{add}(d_k,a),\mathit{add}(d_{\mathit{k-1}},d_k)), \ldots, (\mathit{add}(d_1,d_2),\mathit{add}(b,d_1)) \in h.\mathit{vis}$. Since $\mathit{lin}$ is consistent with visibility relation, we know that in $\mathit{lin}$, $\mathit{add}(a,\_)$ is before $\mathit{add}(d_k,a)$, $\mathit{add}(d_k,a)$ is before $\mathit{add}(d_{\mathit{k-1}},d_k)$, $\ldots$, and $\mathit{add}(d_1,d_2)$ is before $\mathit{add}(b,d_1)$. According to $\mathit{list}_s^{\mathit{af}}$, it is easy to see that in $a$ is before $b$ in $l_1$.

    If the second possibility holds, then it is easy to see that $(\mathit{add}(c_2,c_1),\mathit{add}(a,c_2)),$ $(\mathit{add}(c_1,\_),\mathit{add}(c_2,c_1)), (\mathit{add}(c_3,c_1),\mathit{add}(b,c_3)),(\mathit{add}(c_1,\_),\mathit{add}(c_3,c_1)), \in h.\mathit{vis}$. Since $\mathit{lin}$ is consistent with visibility relation and time-stamp, we know that in $\mathit{lin}$, $\mathit{add}(c_3,c_1)$ is before $\mathit{add}(c_2,c_1)$, $\mathit{add}(c_2,c_1)$ is before $\mathit{add}(a,c_2)$, and $\mathit{add}(c_3,c_1)$ is before $\mathit{add}(b,c_3)$. According to $\mathit{list}_s^{\mathit{af}}$, it is easy to see that in $a$ is before $b$ in $l_1$.

    Therefore, $h'$ is distributed linearizable and $\mathit{lin}'$ is a linearization.

\item[-] If $(R,T,\mathit{MsgHB},\mathit{MsgDel}) {\xrightarrow{\mathit{receive}(\mathit{mid},r)}} (R',T',\mathit{MsgHB}',\mathit{MsgDel}')$, where $(\mathit{mid},(a,\mathit{ts}_a,\mathit{ts}_b),r') \in T$: Then,

    \begin{itemize}
    \setlength{\itemsep}{0.5pt}
    \item[-] $R' = R[ r: ( R(r).N \cup \{ (a,\mathit{ts}_a,\mathit{ts}_b) \}, R(r).\mathit{Tomb} ) ]$ and $T' = T$.

    \item[-] Let $h' = h$.

    \item[-] Let $\mathit{lin}' = \mathit{lin}$.

    \item[-] Let $\mathit{map}' = \mathit{map}$.
    \end{itemize}

    We need to prove that $R'(r) = \mathit{apply}(\mathit{lin}',\mathit{vd}(h',\mathit{del}',r))$.

    We already know that $R(r) = \mathit{apply}(\mathit{lin},\mathit{vd}(h,\mathit{del},r))$.

    We can see that $R'(r)$ is obtained from $R(r)$ by applying message $\mathit{mid}$, and $\mathit{apply}(\mathit{lin}',\mathit{vd}(h',\mathit{del}',r))$ is obtained from $\mathit{apply}(\mathit{lin},\mathit{vd}(h,\mathit{del},r))$ by additionally applying messages $\mathit{mid}$, but possibly in the middle of $\mathit{lin}'$. It is easy to see that $\mathit{map}(\mathit{mid})$ is a $\mathit{add}(a,\_)$ operation. By Lemma \ref{lemma:a property that can be obtained from P for rga}, we can see that there is no $\mathit{add}(a,\_)$ nor $\mathit{rem}(a)$ in $\mathit{vd}(h,\mathit{del},r)$. Thus, for each $\mathit{lin}''$ generated from $\mathit{lin}'$ by postponing message $\mathit{mid}$ to a later position, we can see that $\mathit{apply}(\mathit{lin}'',\mathit{vd}(h',\mathit{del}',r)) = \mathit{apply}(\mathit{lin}',\mathit{vd}(h',\mathit{del}',r))$.

    Therefore, $R'(r) = \mathit{apply}(\mathit{lin}',\mathit{vd}(h',\mathit{del}',r))$.

\item[-] If $(R,T,\mathit{MsgHB},\mathit{MsgDel}) {\xrightarrow{\mathit{receive}(\mathit{mid},r)}} (R',T',\mathit{MsgHB}',\mathit{MsgDel}')$, where $(\mathit{mid},a,r') \in T$: Then,

    \begin{itemize}
    \setlength{\itemsep}{0.5pt}
    \item[-] $R' = R[ r: (R(r).N,R(r).\mathit{Tomb} \cup \{ a \}) ]$ and $T' = T$.

    \item[-] Let $h' = h$.

    \item[-] Let $\mathit{lin}' = \mathit{lin}$.

    \item[-] Let $\mathit{map}' = \mathit{map}$.
    \end{itemize}

    We need to prove that $R'(r) = \mathit{apply}(\mathit{lin}',\mathit{vd}(h',\mathit{del}',r))$.

    We already know that $R(r) = \mathit{apply}(\mathit{lin},\mathit{vd}(h,\mathit{del},r))$.

    We can see that $R'(r)$ is obtained from $R(r)$ by applying message $\mathit{mid}$, and $\mathit{apply}(\mathit{lin}',\mathit{vd}(h',\mathit{del}',r))$ is obtained from $\mathit{apply}(\mathit{lin},\mathit{vd}(h,\mathit{del},r))$ by additionally applying messages $\mathit{mid}$, but possibly in the middle of $\mathit{lin}'$.

    It is easy to see that, for each $\mathit{lin}''$ generated from $\mathit{lin}'$ by postponing message $\mathit{mid}$ to a later position, we have $\mathit{apply}(\mathit{lin}'',\mathit{vd}(h',\mathit{del}',r)) = \mathit{apply}(\mathit{lin}',\mathit{vd}(h',\mathit{del}',r))$.

    Therefore, $R'(r) = \mathit{apply}(\mathit{lin}',\mathit{vd}(h',\mathit{del}',r))$.
\end{itemize}

This completes the proof of this lemma. $\qed$
\end {proof}










\section{Proofs of Section \ref{sec:compositionality of distributed linearizability}}
\label{sec:appendix proofs of section compositionality of distributed linearizability}





















\section{For State-based CRDT}
\label{sec:for state-based CRDT}

\begin{example}[List with add-between interface]
\label{definition:sequential specification of list with add-after interface}
Such kind of list is similar as list with add-after interface. One difference is the $\mathit{add}$ method: $\mathit{add}(b,a,c)$ inserts item $b$ into the list at some nondeterministic position between position of $a$ and position of $c$. The other difference is that, we assume that the initial value of list is $(\circ_1,\mathit{true}) \cdot (\circ_2,\mathit{true})$ and these two nodes can not be removed. The sequential specification $\mathit{list}_s^{\mathit{ab}}$ of list is given as follows: Here $\mathit{ab}$ represents add-between. When the context is clear, in $\mathit{read}$ operation, we will omit $\circ_1$ and $\circ_2$.
\begin{itemize}
\setlength{\itemsep}{0.5pt}
\item[-] $\{ \mathit{state} = (a_1,f_1) \cdot \ldots \cdot (a_n,f_n) \wedge k < m < l \wedge b \notin \{ a_1, \ldots, a_n \} \}$ $add(b,a_k,a_l)$ $\{ \mathit{state} = (a_1,f_1) \cdot \ldots \cdot (a_m,f_m) \cdot (b,\mathit{true}) \cdot (a_{m+1},f_{m+1}) \cdot \ldots \cdot (a_n,f_n) \}$. Here the chosen of $m$ is deterministic.
\item[-] $\{ \mathit{state} = (a_1,f_1) \cdot \ldots \cdot (a_n,f_n) \wedge S = \{ a \vert (a,\mathit{true}) \in \mathit{state} \} \wedge l = a_1 \cdot \ldots \cdot a_n \uparrow_{S} \}$ $(read() \Rightarrow l)$ $\{ \mathit{state} = (a_1,f_1) \cdot \ldots \cdot (a_n,f_n) \}$.
\end{itemize}
\end{example}










Given a object $\mathit{obj}$ of a state-based CRDT with $\Sigma$ be the set of local states, we define its semantics as a set of executions generated from an LTS $\llbracket \mathit{obj} \rrbracket_s = (\mathit{Config},\mathit{config}_0,\Sigma',\rightarrow)$ as in \figurename~\ref{fig:the semantics of a state-based CRDT object}.

\begin{figure}[ht]
$\mathit{RState} = \mathbb{R} \rightarrow \Sigma$

$\mathit{TState} = \mathbb{MID} \times \mathbb{MSG} \times \mathbb{R}$.

$\mathit{Config} = \mathit{RState} \times \mathit{TState}$, $\mathit{config}_0 \in \mathit{Config}$.

$\Sigma' = \mathit{do}(\mathbb{M} \times \mathbb{D} \times \mathbb{D} \times \mathbb{R}) \cup \mathit{send}(\mathbb{MID} \times \mathbb{R}) \cup \mathit{receive}(\mathbb{MID} \times \mathbb{R})$

\[
\begin{array}{l c}
\bigfrac{ R(r) = \sigma, r.\mathit{do}(\sigma,m,a) = (\sigma',b) }
{ (R,T) {\xrightarrow{\mathit{do}(m,a,b,r)}} (R[r:\sigma'],T) }
\end{array}
\]


\[
\begin{array}{l c}
\bigfrac{ R(r) = \sigma, \mathit{unique}(\mathit{mid}) }
{ (R,T) {\xrightarrow{\mathit{send}(\mathit{mid},r)}} (R,T \cup \{ (\mathit{mid},\sigma,r) \}) }
\end{array}
\]


\[
\begin{array}{l c}
\bigfrac{ R(r) = \sigma, r.\mathit{receive}(\sigma,\sigma') = \sigma'',(\mathit{mid},\sigma',r') \in T, r \neq r'}
{ (R,T) {\xrightarrow{\mathit{receive}(\mathit{mid},r)}} (R[r:\sigma''],T) }
\end{array}
\]
\caption{The definition of semantics of $\llbracket \mathit{obj} \rrbracket_s$}
\label{fig:the semantics of a state-based CRDT object}
\end{figure}

A configuration $(R,T)$ is a snapshot of distributed system and contains two parts: $R$ gives the local state of each replica, and $T$ gives the set of messages that has been generated. Let $\mathbb{MID}$ be the set of message identifiers of message content. A message is a tuple $(\mathit{mid},\mathit{msg},r)$, where $\mathit{mid} \in \mathbb{MID}$ is the identifier, $\mathit{msg} \in \mathbb{MSG}$ is the message content, and $r$ is the original replica of message. $\mathit{config}_0$ is the initial configuration, which maps each replica into the initial local state, and has no message inside. Since $\mathit{obj}$ is a state-based CRDT, each message content is chosen from $\Sigma$.

Each element of $\Sigma'$ is called an action. $\rightarrow \in \mathit{Config} \times \Sigma' \times \mathit{Config}$ is the transition relation and describe a single step of distributed systems. The first rule in \figurename~\ref{fig:the semantics of a state-based CRDT object} describes replica $r$ performs a operation $m(a) \Rightarrow b$ and works locally. The second rule describes that a replica $r$ may nondeterministically decide to send a message with its local state as message content. Here $\mathit{unique}$ is a function that ensures $\mathit{mid}$ be a fresh message identifier. The third rule describes delivery of a message to a replica $r$ other than its origin replica $r'$.

A sequence $l$ of actions is an execution of $\llbracket \mathit{obj} \rrbracket_s = (\mathit{Config},\mathit{config}_0,\Sigma',\rightarrow)$, if there exists $(R,T) \in \mathit{Config}$, such that $\mathit{config}_0 {\xrightarrow{ l }} (R,T)$. The semantics of $\mathit{obj}$ is defined as the set of executions of $\llbracket \mathit{obj} \rrbracket_s$. Given an execution, when the context is clear, we can associate a unique operation identifier to each action. Or we can say, it is safe to assume each action is in the form of either $\mathit{do}(i,m,a,b,r)$, or $\mathit{send}(i,\mathit{mid},r)$, or $\mathit{receive}(i,\mathit{mid},r)$, where $i \in \mathbb{OID}$ is a unique operation identifier.








Given an execution $l = \alpha_1 \cdot \ldots \cdot \alpha_k$ of $\llbracket \mathit{obj} \rrbracket_s$ of state-based CRDT $\mathit{obj}$, we can obtain a corresponding history $\mathit{history}(l) = (\mathit{Op},\mathit{ro},\mathit{vis})$, such that

\begin{itemize}
\setlength{\itemsep}{0.5pt}
\item[-] Each operation in $\mathit{Op}$ is a tuple $(\ell,i,\mathit{obj})$, such that $i$ is the operation identifier of a $\mathit{do}(m,a,b,r)$ action of $l$.

\item[-] $(o_1,o_2) \in \mathit{ro}$, if they are of same replica, and the index of $o_1$ in $h$ is before that of $o_2$.

\item[-] Let us defines a delivery relation $\mathit{del} \subseteq \mathbb{OP} \times \mathbb{OP}$ as follows: $(o_1,o_2) \in \mathit{del}$, if: $o_1$ and $o_2$ are of different replicas, there exists a $\mathit{send}(\mathit{mid},r)$ action and a $\mathit{receive}(\mathit{mid},r')$ action, $o_1$ and $\mathit{send}(\mathit{mid},r)$ happen on a same replica and $o_1$ happens earlier, $\mathit{receive}(\mathit{mid},r)$ and $o_2$ happen on a same replica and $\mathit{receive}(\mathit{mid},r)$ happens earlier.

\item[-] $\mathit{vis} = (\mathit{ro}+\mathit{del})^*$.
\end{itemize}

Intuitively, each local state can be considered as the consequence of all updates it receives. Since state-based CRDT sends the modified local state as message, the visibility relation is then the transitive closure of replica order and message delivery relation. Let $\mathit{history}(\llbracket \mathit{obj} \rrbracket_s)$ be the set of histories of all executions of $\llbracket \mathit{obj} \rrbracket_s$.






\subsection{Proof Strategy of State-based CRDT}
\label{subsec:proof strategy of operation-based CRDT}

Given a state-based CRDT object $\mathit{obj}$ and a sequential specification $\mathit{spec}$, we need to construct a invariant $\mathit{inv}(\mathit{config},h,\mathit{lin},\mathit{del},\mathit{map})$, where

\begin{itemize}
\setlength{\itemsep}{0.5pt}
\item[-] $\mathit{config}$ is a configuration of $\llbracket \mathit{obj} \rrbracket_s$.

\item[-] $h$ is a history.

\item[-] $h$ is distributed linearizable w.r.t $\mathit{spec}$ and $\mathit{lin}$ is a linearization.

\item[-] $\mathit{del} \subseteq \mathbb{MID} \times \mathbb{R}$ is the message delivery relation.

\item[-] $\mathit{map} \subseteq \mathbb{MID} \times 2^{\mathbb{OID}}$ maps each message $\mathit{mid}$ to a set $S_1$ of operations. Intuitively, $S_1$ is the set of operations whose information are contained in $\mathit{mid}$.
\end{itemize}

$\mathit{inv}(\mathit{config},h,\mathit{lin},\mathit{del},\mathit{map})$ needs to satisfy the following properties:

\begin{itemize}
\setlength{\itemsep}{0.5pt}
\item[-] The visibility of $h$ is transitive.

\item[-] $\mathit{del}$ preserves causal delivery: If $(o_1,o_2) \in \mathit{vis}$ and $(o_2,r) \in \mathit{del}$, then $(o_1,r) \in \mathit{del}$.

\item[-] $\mathit{map}$ preserves causal delivery: Given $o_1,o_3 \in \mathit{map}(\mathit{mid})$, if $\exists o_2$, such that $(o_1,o_2),(o_2,o_3) \in \mathit{vis}$, then $o_2 \in \mathit{map}(\mathit{mid})$.

\item[-] $\mathit{inv}$ holds initially: $\mathit{inv}(\mathit{config}_0,\epsilon,\emptyset,\emptyset,\emptyset)$ holds, where $\mathit{config}_0$ is the initial configuration of $\llbracket \mathit{obj} \rrbracket_s$.

\item[-] $\mathit{inv}$ is a transition invariant:

    \begin{itemize}
    \setlength{\itemsep}{0.5pt}
    \item[-] If $\mathit{inv}(\mathit{config},h,\mathit{lin},\mathit{del},\mathit{map})$ holds and $\mathit{config} {\xrightarrow{\mathit{do}(m,a,b,r)}} \mathit{config}'$, then $\mathit{inv}(\mathit{config}', h \otimes i, \mathit{lin} \cdot i,\mathit{del},\mathit{map})$ holds. Note that here we always put a new operation in the last of linearization.

        Here $i$ is the identifier of the newly-generated $\mathit{do}$ action. Given $h = (\mathit{Op},\mathit{ro},\mathit{vis})$, then, $h \otimes i = (\mathit{Op}',\mathit{ro}',\mathit{vis}')$, where $\mathit{Op}' = \mathit{Op} \cup \{ (m(a) \Rightarrow b,i,\mathit{obj}) \}$, $\mathit{ro}' = \mathit{ro} \cup \{ (j,i) \vert j \in \mathit{Op}, j$ is of replica $r \}$, and $\mathit{vis}' = (\mathit{vis} \cup \{ (j,i) \vert j \in \mathit{Op},(j,r) \in \mathit{del} \} \cup \{ (j,i) \vert j \in \mathit{Op}, j$ is of replica $r \})^*$.

    \item[-] If $\mathit{inv}(\mathit{config},h,\mathit{lin},\mathit{del},\mathit{map})$ holds and $\mathit{config} {\xrightarrow{\mathit{send}(\mathit{mid},r)}} \mathit{config}'$, then $\mathit{inv}(\mathit{config}',h,\mathit{lin},\mathit{del},\mathit{map}')$ holds, where $\mathit{map}' = \mathit{map} \cup (\mathit{mid}, \mathit{vd}(h,\mathit{del},r))$.


    \item[-] If $\mathit{inv}(\mathit{config},h,\mathit{lin},\mathit{del},\mathit{map})$ holds and $\mathit{config} {\xrightarrow{\mathit{receive}(\mathit{mid},r)}} \mathit{config}'$, then $\mathit{inv}(\mathit{config}',h,\mathit{lin},\mathit{del}',\mathit{map})$ holds, where $\mathit{del}' = \mathit{del} \cup \{ (i,r) \vert i \in \mathit{map}(\mathit{mid}) \}$.
    \end{itemize}
\end{itemize}

Here $\mathit{vd}(h,\mathit{del},r) = \{ i \vert (i,j) \in h.\mathit{vis}, j$ is of replica $r \} \cup \{ i \vert (i,r) \in \mathit{del} \}$ is the set of operations that are either to some operation of replica $r$, or has been delivered into replica $r$. An invariant $\mathit{inv}$ satisfies above properties is called invariant of state-based CRDT. The following lemma states that the existence of such invariant implies distributed linearizability.

\begin{lemma}
\label{lemma:invariant of state-based CRDT implies distributed linearizability}
If there exists a invariant $\mathit{inv}$ of state-based CRDT for object $\mathit{obj}$ and sequential specification $\mathit{spec}$, then each history of $\mathit{history}(\llbracket \mathit{obj} \rrbracket_s)$ is distributed linearizable w.r.t $\mathit{spec}$.
\end{lemma}

\begin {proof}
Given an execution $l=\alpha_1 \cdot \ldots \cdot \alpha_n$, let $\mathit{config}_0 {\xrightarrow{\alpha_1}} \mathit{config}_1 \ldots {\xrightarrow{\alpha_n}} \mathit{config}_n$ be the transitions from initial configuration. We need to prove that, for each $1 \leq k \leq n$, we have $\mathit{inv}(\mathit{config}_k,h_k,\mathit{lin}_k,\mathit{del}_k,\mathit{map}_k)$ holds, where $h_k$ is the history of execution $l_k = \alpha_1 \cdot \ldots \cdot \alpha_k$, $\mathit{lin}_k$ is the linearization of $h_k$, $\mathit{del}_k$ records message delivery relation of $l_k$, and $\mathit{map}_k$ records the operations contained in each message in $l_k$.

Since $\mathit{inv}$ holds initially and is a transition invariant, it is easy to prove above requirements by induction on execution. This completes the proof of this lemma. $\qed$
\end {proof}


For many state-based CRDT implementations, $\mathit{inv}((R,T),h,\mathit{lin},\mathit{del},\mathit{map}) = C_1 \wedge C_2$, where

\begin{itemize}
\setlength{\itemsep}{0.5pt}
%\item[-] For each update operation $o$ of $h$, define $\mathit{ds}(o)$ which is a local state. %be the local state of replica $r$ at the time point immediately after $o$ is launched. Here $r$ is the replica of $o$.

\item[-] $C_1: \forall (\mathit{mid},\mathit{msg},\_) \in T$, $\mathit{msg} = \mathit{apply}(\mathit{lin},\mathit{map}(\mathit{mid}))$.

\item[-] $C_2: \forall r$, $R(r) = \mathit{apply}(\mathit{lin},\mathit{vd}(h,\mathit{del},r))$.
\end{itemize}

The function $\mathit{apply}(\mathit{lin},S)$ returns a local state by applying ``virtual messages'' of operations in $S$ according to total order $\mathit{lin}$. Here for each update operation $o$ of $h$, we need to define a local state $\mathit{ds}(o)$, which is the ``virtual messages'' of $o$. Note that state-based CRDT send message randomly, instead of each message for a update operation. This is the reason why we need to manually generate virtual message for each update operation.

To give $\mathit{inv}$, it only remains to give the virtual messages. The virtual message of state-based PN-counter and state-based multi-value register as follows. The proof of them being invariants of state-based CRDT is given in Appendix \ref{subsec:appendix proof of state-based PN-counter} and Appendix \ref{subsec:appendix proof of state-based multi-value register}, respectively.

\begin{example}[virtual messages of state-based PN-counter]
\label{example:virtual messagess of state-based PN-counter}

For each update operation $o$, $\mathit{ds}(o) = (P,N)$, where

\begin{itemize}
\setlength{\itemsep}{0.5pt}
\item[-] $\forall r'$, $P[r'] = \vert \{ o' \vert o'$ is a $\mathit{inc}$ operation of replica $r'$, $o' = o \vee (o',o) \in h.\mathit{vis} \} \vert$.

\item[-] $\forall r'$, $N[r'] = \vert \{ o' \vert o'$ is a $\mathit{dec}$ operation of replica $r'$, $o' = o \vee (o',o) \in h.\mathit{vis} \} \vert$.
\end{itemize}
\end{example}

\begin{example}[virtual messages of state-based Multi-value Register]
\label{example:virtual messages of state-based multi-value register}

For each update operation $o = (\mathit{write}(a),\_,\_)$ of replica $r$, $\mathit{ds}(o) = (a,V)$, where

\begin{itemize}
\setlength{\itemsep}{0.5pt}
\item[-] $\forall r'$, $V[r'] = \vert \{ o' \vert o'$ is a $\mathit{write}$ operation of replica $r'$, $o' = o \vee (o',o) \in h.\mathit{vis} \} \vert$.
\end{itemize}
\end{example}















\subsection{Proof of State-based PN-counter}
\label{subsec:appendix proof of state-based PN-counter}

The following lemma states that each visibility-closed set is a union of operations visible to a set of operations. Its proof is obvious and omitted here.

\begin{lemma}
\label{lemma:a transitive-closed set is a union of visibility of several sets}
Given a set $\mathit{Op}$ of operations and a transitive and acyclic visibility relation $\mathit{vis} \subseteq \mathit{Op} \times \mathit{Op}$, if given a set $S \subseteq \mathit{Op}$, if $S$ satisfies that $\forall o_1,o_2 \in S, o_2 \in S \wedge (o_1,o_2) \in \mathit{vis} \Rightarrow o_1 \in S$, then there exists a set $O \subseteq \mathit{Op}$, such that $S = \cup_{o \in O} \mathit{vis}^{-1}(o)$.
\end{lemma}

The following lemma states that given two operations $o_1,o_2$, for each replica $r$, either the set of operations of replica $r$ visible to $o_1$ is a subset of that of $o_2$, or the set of operations of replica $r$ visible to $o_2$ is a subset of that of $o_1$. Its proof is obvious and omitted here.

\begin{lemma}
\label{lemma:the view of a replica of one operation is contained in another operaiton, or vice versa}
Assume that $\mathit{inv}((R,T),h,\mathit{lin},\mathit{del},\mathit{map})$ holds. Let $S_o^r = \{ o' \vert (o',o) \in \mathit{vis}, o'$ is of replica $r \}$. Then for each operations $o_1$ and $o_2$, and for each replica $r$, $S_{\mathit{o1}}^r \subseteq S_{\mathit{o2}}^r \vee S_{\mathit{o2}}^r \subseteq S_{\mathit{o1}}^r$.
\end{lemma}


Recall that $\mathit{inv} = C_1 \wedge C_2$ with the virtual messages defined as follows: For each update operation $o$, $\mathit{ds}(o) = (P,N)$, where

\begin{itemize}
\setlength{\itemsep}{0.5pt}
\item[-] $\forall r'$, $P[r'] = \vert \{ o' \vert o'$ is a $\mathit{inc}$ operation of replica $r'$, $o' = o \vee (o',o) \in h.\mathit{vis} \} \vert$.

\item[-] $\forall r'$, $N[r'] = \vert \{ o' \vert o'$ is a $\mathit{dec}$ operation of replica $r'$, $o' = o \vee (o',o) \in h.\mathit{vis} \} \vert$.
\end{itemize}

The following lemma states that $\mathit{inv}$ is an invariant of state-based PN-counter.

\begin{lemma}
\label{lemma:inv is an invariant of state-based CRDT for state-based PN-counter}
$\mathit{inv}$ is an invariant of state-based PN-counter.
\end{lemma}

\begin {proof}

It is obvious that $\mathit{inv}(\mathit{config}_0,\epsilon,\emptyset,\emptyset,\emptyset)$ holds.

Let us prove that $\mathit{inv}$ is a transition invariant: assume $\mathit{inv}((R,T),h,\mathit{lin},\mathit{del},\mathit{map})$ holds,

\begin{itemize}
\setlength{\itemsep}{0.5pt}
\item[-] If $(R,T) {\xrightarrow{\mathit{do}(\mathit{inc},r)}} (R',T')$: Then,

    \begin{itemize}
    \setlength{\itemsep}{0.5pt}
    \item[-] It is easy to see that $R' = R[ r: ( R(r).P[r: R(r).P(r)+1 ], R(r).N ) ]$ and $T' = T$.

    \item[-] Let $h' = h \otimes i$, where $i$ is the identifier of the newly-generated $\mathit{inc}$ action.

    \item[-] Let $\mathit{lin}' = \mathit{lin} \cdot (\mathit{inc},i,\mathit{obj})$.

    \item[-] Let $\mathit{del}' = \mathit{del}$ and $\mathit{map}' = \mathit{map}$.
    \end{itemize}

    It is easy to see that $\mathit{lin}'$ is a linearization of $h'$. It is obvious that all other properties hold, except for $C_2$ for replica $r$. Therefore, let us prove that $R'(r) = \mathit{apply}(\mathit{lin}',\mathit{vd}(h',\mathit{del}',r))$.

    Since $R(r) = \mathit{apply}(\mathit{lin},\mathit{vd}(h,\mathit{del},r))$ and $\mathit{lin}' = \mathit{lin} \cdot (\mathit{inc},i,\mathit{obj})$, we know that $\mathit{apply}(\mathit{lin}',\mathit{vd}(h',\mathit{del}',r)) = \mathit{merge}(R(r),\mathit{ds}(i))$. Therefore, we need to prove that $R'(r) = \mathit{merge}(R(r),\mathit{ds}(i))$.

    Since $\mathit{vd}(h,\mathit{del},r)$ satisfies that, $\forall o_1,o_2 \in \mathit{vd}(h,\mathit{del},r), o_2 \in \mathit{vd}(h,\mathit{del},r) \wedge (o_1,o_2) \in \mathit{vis} \Rightarrow o_1 \in \mathit{vd}(h,\mathit{del},r)$, by Lemma \ref{lemma:a transitive-closed set is a union of visibility of several sets}, we know that there exists a set $O$, such that $\mathit{vd}(h,\mathit{del},r) = \cup_{o \in O} \mathit{vis}^{-1}(o)$. By Lemma \ref{lemma:the view of a replica of one operation is contained in another operaiton, or vice versa} and the construction of $\mathit{ds}$, we can see that $R(r) = (P',N')$, where for each replica $r'$, $P'[r'] = \vert \{ j \in \mathit{vd}(h,\mathit{del},r) \uparrow_{\mathit{inc}}$ and $j$ is of replica $r \} \vert$ and $N'[r'] = \vert \{ j \in \mathit{vd}(h,\mathit{del},r) \uparrow_{\mathit{dec}}$ and $j$ is of replica $r \} \vert$.

    We already know that $\mathit{ds}(i) = (P'',N'')$, where for each replica $r'$, $P''[r'] = \vert \{ j \in \mathit{vd}(h',\mathit{del}',r) \uparrow_{\mathit{inc}}$ and $j$ is of replica $r \} \vert$ and $N''[r'] = \vert \{ j \in \mathit{vd}(h',\mathit{del}',r) \uparrow_{\mathit{dec}}$ and $j$ is of replica $r \} \vert$. Then, it is obvious that $\mathit{merge}(R(r),\mathit{ds}(i)) = \mathit{ds}(i)$. It is also easy to see that $\mathit{ds}(i) = (R(r).P[r: R(r).P(r)+1], R(r).N) = R'(r)$. Therefore, $R'(r) = \mathit{merge}(R(r),\mathit{ds}(i))$.

\item[-] If $(R,T) {\xrightarrow{\mathit{do}(\mathit{dec},r)}} (R',T')$: Similarly as that of $(R,T) {\xrightarrow{\mathit{do}(\mathit{inc},r)}} (R',T')$.

\item[-] If $(R,T) {\xrightarrow{\mathit{do}(\mathit{read},k,r)}} (R',T')$: Then,

    \begin{itemize}
    \setlength{\itemsep}{0.5pt}
    \item[-] It is obvious that $R' = R$ and $T' = T$.

    \item[-] Let $h' = h \otimes i$, where $i$ is the identifier of the newly-generated $\mathit{read}$ action.

    \item[-] Let $\mathit{lin}' = \mathit{lin} \cdot ((\mathit{read}() \Rightarrow k,i,\mathit{obj}), \mathit{vd}(h,\mathit{del},r) )$.

    \item[-] Let $\mathit{del}' = \mathit{del}$ and $\mathit{map}' = \mathit{map}$.
    \end{itemize}

    It is easy to see that all other properties hold, except for $h'$ being distributed linearizable w.r.t $\mathit{spec}$ with $\mathit{lin}'$ the linearization. Let us prove that $h'$ is distributed linearizable w.r.t $\mathit{spec}$ and $\mathit{lin}'$ is a linearization. It is easy to see that only operation $i$ need to be checked.

    It is easy to see that $\mathit{vd}(h,\mathit{del},r) = \mathit{vis}^{-1}(i)$. Similarly as the case of $(R,T) {\xrightarrow{\mathit{do}(\mathit{inc},r)}} (R',T')$, we can prove that $R(r) = (P',N')$, where for each replica $r'$, $P'[r'] = \vert \{ j \in \mathit{vd}(h,\mathit{del},r) \uparrow_{\mathit{inc}}$ and $j$ is of replica $r \} \vert = \vert \{ j \in \mathit{vis}^{-1}(i) \uparrow_{\mathit{inc}}$ and $j$ is of replica $r \} \vert$ and $N'[r'] = \vert \{ j \in \mathit{vd}(h,\mathit{del},r) \uparrow_{\mathit{dec}}$ and $j$ is of replica $r \} \vert = \vert \{ j \in \mathit{vis}^{-1}(i) \uparrow_{\mathit{dec}}$ and $j$ is of replica $r \} \vert$. Since $k = \Sigma_{r'} P��[r'] - \Sigma_{r'} N'[r']$, $k$ is obtained by minus the number of all visible $\mathit{dec}$ of $i$ from the number of all visible $\mathit{inc}$ of $i$. Therefore, we can see that $((\mathit{read}() \Rightarrow k,i,\mathit{obj}), \mathit{vd}(h,\mathit{del},r) )$ of $\mathit{lin}'$ is ``correct''. Then, $h'$ is distributed linearizable w.r.t $\mathit{spec}$ and $\mathit{lin}'$ is a linearization.

\item[-] If $(R,T) {\xrightarrow{\mathit{send}(\mathit{mid},r)}} (R',T')$: Then,

    \begin{itemize}
    \setlength{\itemsep}{0.5pt}
    \item[-] It is obvious that $R' = R$. Let $T' = T \cup \{ (\mathit{mid},R(r),r) \}$.

    \item[-] Let $h' = h$.

    \item[-] Let $\mathit{lin}' = \mathit{lin}$.

    \item[-] Let $\mathit{del}' = \mathit{del}$.

    \item[-] Let $\mathit{map}' = \mathit{map} \cup \{ (\mathit{mid},\mathit{vd}(h,\mathit{del},r)) \}$.
    \end{itemize}

    It is easy to see that all other properties hold, except for checking $C_1$ for $\mathit{mid}$. This holds obviously since the message content of message $\mathit{mid}$ is $R(r)$, and we already know that $R(r) = \mathit{apply}(\mathit{lin},\mathit{vd}(h,\mathit{del},r)) = \mathit{apply}(\mathit{lin},\mathit{map}(\mathit{mid}))$.

\item[-] If $(R,T) {\xrightarrow{\mathit{receive}(\mathit{mid},r)}} (R',T')$: Then,

    \begin{itemize}
    \setlength{\itemsep}{0.5pt}
    \item[-] Let $R' = R[ r: \mathit{merge}(R(r),\mathit{msg})]$ where $(\mathit{mid},\mathit{msg},\_) \in T$. It is obvious that $T' = T$.

    \item[-] Let $h' = h$.

    \item[-] Let $\mathit{lin}' = \mathit{lin}$.

    \item[-] Let $\mathit{del}' = \mathit{del} \cup \{ (i,r) \vert i \in \mathit{map}(\mathit{mid}) \}$.

    \item[-] Let $\mathit{map}' = \mathit{map}$.
    \end{itemize}

    It is easy to see that all other properties hold, except for $C_2$ for replica $r$. Therefore, let us prove that $R'(r) = \mathit{apply}(\mathit{lin}',\mathit{vd}(h',\mathit{del}',r))$.

    We already know that $R'(r) = \mathit{merge}(R(r), \mathit{msg})$, $R(r) = \mathit{apply}(\mathit{lin},\mathit{vd}(h,\mathit{del},r))$ and $\mathit{msg} = \mathit{apply}(\mathit{lin},\mathit{map}(\mathit{mid}))$. It is easy to see that $\mathit{vd}(h',\mathit{del}',r) = \mathit{vd}(h,\mathit{del},r) \cup \mathit{map}(\mathit{mid})$. It is easy to prove that, applying messages in any order lead to the same consequence. Therefore, we have $\mathit{merge}(R(r), \mathit{msg}) = \mathit{apply}(\mathit{lin}',\mathit{vd}(h,\mathit{del},r) \cup \mathit{map}(\mathit{mid}))$. Then, we have $R'(r) = \mathit{apply}(\mathit{lin}',\mathit{vd}(h',\mathit{del}',r))$.
\end{itemize}

This completes the proof of this lemma. $\qed$
\end {proof}




\subsection{Proof of State-based Multi-value Register}
\label{subsec:appendix proof of state-based multi-value register}

Recall that $\mathit{inv} = C_1 \wedge C_2$ with the virtual messages defined as follows: For each update operation $o$, $\mathit{ds}(o) = (a,V)$, where

\begin{itemize}
\setlength{\itemsep}{0.5pt}
\item[-] $\forall r'$, $V[r'] = \vert \{ o' \vert o'$ is a $\mathit{write}$ operation of replica $r'$, $o' = o \vee (o',o) \in h.\mathit{vis} \} \vert$.
\end{itemize}

The following lemma states that $\mathit{inv}$ is an invariant of state-based multi-value register.

\begin{lemma}
\label{lemma:inv is an invariant of state-based CRDT for state-based multi-value register}
$\mathit{inv}$ is an invariant of state-based multi-value register.
\end{lemma}

\begin {proof}

It is obvious that $\mathit{inv}(\mathit{config}_0,\epsilon,\emptyset,\emptyset,\emptyset)$ holds.

Let us prove that $\mathit{inv}$ is a transition invariant: assume $\mathit{inv}((R,T),h,\mathit{lin},\mathit{del},\mathit{map})$ holds,

\begin{itemize}
\setlength{\itemsep}{0.5pt}
\item[-] If $(R,T) {\xrightarrow{\mathit{do}(\mathit{write},a,r)}} (R',T')$: Then,

    \begin{itemize}
    \setlength{\itemsep}{0.5pt}
    \item[-] $R' = R[ r: \{ (a,V') \} ], R(r).N)]$ and $T' = T$. Here $\forall r' \neq r, V'[r'] = \mathit{max} \{ V_1(r��) \vert (\_,V_1) \in R(r) \}$, and $V'[r] = \mathit{max} \{ V_1(r) \vert (\_,V_1) \in R(r) \} + 1$.

    \item[-] Let $h' = h \otimes i$, where $i$ is the identifier of the newly-generated $\mathit{inc}$ action.

    \item[-] Let $\mathit{lin}' = \mathit{lin} \cdot (\mathit{inc},i,\mathit{vis}^{-1}(i))$.

    \item[-] Let $\mathit{del}' = \mathit{del}$ and $\mathit{map}' = \mathit{map}$.
    \end{itemize}

    It is easy to see that $\mathit{lin}'$ is a linearization of $h'$. It is obvious that all other properties hold, except for $C_2$ for replica $r$. Therefore, let us prove that $R'(r) = \mathit{apply}(\mathit{lin}',\mathit{vd}(h',\mathit{del}',r))$.

    It is easy to see that $\mathit{vd}(h',\mathit{del}',r) = h'.\mathit{vis}^{-1}(i)$. And then, we need to prove that $(a,V') = \mathit{apply}(\mathit{lin}',h'.\mathit{vis}^{-1}(i))$.

    Recall that $R(r) = \mathit{apply}(\mathit{lin},\mathit{vd}(h,\mathit{del},r))$, from Lemma \ref{lemma:a transitive-closed set is a union of visibility of several sets}, we know that there exists set $O$, such that $\mathit{vd}(h,\mathit{del},r) = \cup_{o \in O} \mathit{vis}^{-1}(o)$. We can prove that, for each $o = \mathit{write}(b)$, $\mathit{apply}(\mathit{lin},\mathit{vis}^{-1}(o)) = (b,V_b)$, where $\forall r' \neq r, V_b[r'] = \vert \{ o' \vert o' \in \mathit{vis}^{-1}(o), o'$ is of replica $r' \} \vert$, and $V_b[r] = \vert \{ o' \vert o' \in \mathit{vis}^{-1}(o), o'$ is of replica $r' \} \vert + 1$.

    It is not hard to prove that the order of merging virtual message is not important, and a virtual message can be applied multiple times. By Lemma \ref{lemma:the view of a replica of one operation is contained in another operaiton, or vice versa}, we can see that $\mathit{apply}(\mathit{lin},\mathit{vd}(h,\mathit{del},r))$ is obtained by merging $\{ o \in O \vert \mathit{apply}(\mathit{lin},\mathit{vis}^{-1}(o)) \}$. Therefore, we can see that $\mathit{apply}(\mathit{lin}',h'.\mathit{vis}^{-1}(i)) = \mathit{apply}(\mathit{lin}',\mathit{vd}(h',\mathit{del}',r))$ is obtained by merging $\{ o \in O \vert \mathit{apply}(\mathit{lin},\mathit{vis}^{-1}(o)) \} \cup \{ \mathit{ds}(i) \}$. By Lemma \ref{lemma:the view of a replica of one operation is contained in another operaiton, or vice versa}, it is not hard to see that $\mathit{apply}(\mathit{lin}',h'.\mathit{vis}^{-1}(i)) = \mathit{ds}(i)$.

    Then, we need to prove that $(a,V') = \mathit{ds}(i)$. This holds since $R(r) = \mathit{apply}(\mathit{lin},\mathit{vd}(h,\mathit{del},r))$ is obtained by merging $\{ o \in O \vert \mathit{apply}(\mathit{lin},\mathit{vis}^{-1}(o)) \}$, Lemma \ref{lemma:the view of a replica of one operation is contained in another operaiton, or vice versa}, and the value of $V'$.

\item[-] If $(R,T) {\xrightarrow{\mathit{do}(\mathit{read},S,r)}} (R',T')$: Then,

    \begin{itemize}
    \setlength{\itemsep}{0.5pt}
    \item[-] It is obvious that $R' = R$ and $T' = T$.

    \item[-] Let $h' = h \otimes i$, where $i$ is the identifier of the newly-generated $\mathit{read}$ action.

    \item[-] Let $\mathit{lin}' = \mathit{lin} \cdot (\mathit{read}() \Rightarrow S,i,\mathit{vd}(h,\mathit{del},r) )$.

    \item[-] Let $\mathit{del}' = \mathit{del}$ and $\mathit{map}' = \mathit{map}$.
    \end{itemize}

    It is easy to see that all other properties hold, except for $h'$ being distributed linearizable w.r.t $\mathit{spec}$ with $\mathit{lin}'$ the linearization. Let us prove that $h'$ is distributed linearizable w.r.t $\mathit{spec}$ and $\mathit{lin}'$ is a linearization. It is easy to see that only operation $i$ need to be checked.

    It is easy to see that $\mathit{vd}(h,\mathit{del},r) = h'.\mathit{vis}^{-1}(i)$. Similarly as the case of $(R,T) {\xrightarrow{\mathit{do}(\mathit{write},a,r)}} (R',T')$, we can prove that there exists a set $O$, such that $R(r) = \mathit{apply}(\mathit{lin},\mathit{vd}(h,\mathit{del},r))$ is obtained by merging $\{ o \in O \vert \mathit{apply}(\mathit{lin},\mathit{vis}^{-1}(o)) \}$.

    By the definition of merging, it is same to assume that $O = \mathit{max}_{\mathit{vis}} \mathit{vd}(h,\mathit{del},r)$. Assume that for each operation $o = \mathit{write}(a) \in O$, $\mathit{apply}(\mathit{lin},\mathit{vis}^{-1}(o))) = (a,V_o)$. Then it is not hard to see that $R(r) = \{ (a,V_o) \vert o = \mathit{write}(a) \in O \}$. Therefore, $S = \{ a \vert o = \mathit{write}(a) \in \mathit{vis}^{-1}(i), \forall o' = \mathit{write}(\_) \in \mathit{vis}^{-1}(i), (o,o') \notin \mathit{vis} \}$. According to sequential specification $\mathit{spec}$, $(\mathit{read} \Rightarrow S,i,\mathit{obj})$ of $\mathit{lin}'$ is ``correct''. Then, $h'$ is distributed linearizable w.r.t $\mathit{spec}$ and $\mathit{lin}'$ is a linearization.

\item[-] If $(R,T) {\xrightarrow{\mathit{send}(\mathit{mid},r)}} (R',T')$: Then,

    \begin{itemize}
    \setlength{\itemsep}{0.5pt}
    \item[-] It is obvious that $R' = R$. Let $T' = T \cup \{ (\mathit{mid},R(r),r) \}$.

    \item[-] Let $h' = h$.

    \item[-] Let $\mathit{lin}' = \mathit{lin}$.

    \item[-] Let $\mathit{del}' = \mathit{del}$.

    \item[-] Let $\mathit{map}' = \mathit{map} \cup \{ (\mathit{mid},\mathit{vd}(h,\mathit{del},r)) \}$.
    \end{itemize}

    It is easy to see that all other properties hold, except for checking $C_1$ for $\mathit{mid}$. This holds obviously since the message content of message $\mathit{mid}$ is $R(r)$, and we already know that $R(r) = \mathit{apply}(\mathit{lin},\mathit{vd}(h,\mathit{del},r)) = \mathit{apply}(\mathit{lin},\mathit{map}(\mathit{mid}))$.

\item[-] If $(R,T) {\xrightarrow{\mathit{receive}(\mathit{mid},r)}} (R',T')$: Then,

    \begin{itemize}
    \setlength{\itemsep}{0.5pt}
    \item[-] Let $R' = R[ r: \mathit{merge}(R(r),\mathit{msg})]$ where $(\mathit{mid},\mathit{msg},\_) \in T$. It is obvious that $T' = T$.

    \item[-] Let $h' = h$.

    \item[-] Let $\mathit{lin}' = \mathit{lin}$.

    \item[-] Let $\mathit{del}' = \mathit{del} \cup \{ (i,r) \vert i \in \mathit{map}(\mathit{mid}) \}$.

    \item[-] Let $\mathit{map}' = \mathit{map}$.
    \end{itemize}

    It is easy to see that all other properties hold, except for $C_2$ for replica $r$. Therefore, let us prove that $R'(r) = \mathit{apply}(\mathit{lin}',\mathit{vd}(h',\mathit{del}',r))$.

    We already know that $R'(r) = \mathit{merge}(R(r), \mathit{msg})$, $R(r) = \mathit{apply}(\mathit{lin},\mathit{vd}(h,\mathit{del},r))$ and $\mathit{msg} = \mathit{apply}(\mathit{lin},\mathit{map}(\mathit{mid}))$. It is easy to see that $\mathit{vd}(h',\mathit{del}',r) = \mathit{vd}(h,\mathit{del},r) \cup \mathit{map}(\mathit{mid})$. It is easy to prove that, applying messages in any order lead to the same consequence. Therefore, we have $\mathit{merge}(R(r), \mathit{msg}) = \mathit{apply}(\mathit{lin}',\mathit{vd}(h,\mathit{del},r) \cup \mathit{map}(\mathit{mid}))$. Then, we have $R'(r) = \mathit{apply}(\mathit{lin}',\mathit{vd}(h',\mathit{del}',r))$.
\end{itemize}

This completes the proof of this lemma. $\qed$
\end {proof}

