\section{\crdtimp{}}
\label{sec:crdt implementation}



\subsection{Multi-Value Register Implementation}
\label{subsec:multi-value register implementation}

\cite{ShapiroPBZ11} shows how to obtain a state-based \crdtimp{} from a operation-based \crdtimp{}, and we draw it in Listing~\ref{lst:operation-based emulation of state-based object}. To do an operation $f(a)$, we compute the state-based update and perform merge method in downstream. Here the precondition of downstream is empty because merge is always enabled.


\begin{minipage}[t]{1.0\linewidth}
\begin{lstlisting}[frame=top,caption={operation-based emulation of state-based object},
captionpos=b,label={lst:operation-based emulation of state-based object}]
  payload S ( the state-based payload )
  initial initial payload of S

  update method f(a)
    atSource :
      precondition : precondition of f(a)
      let s = atSource of f(a) in state-based
    downStream(s) :
      S = merge(S,s)
\end{lstlisting}
\end{minipage}

\cite{ShapiroPBZ11} gives a state-based multi-value register implementation. As discussed above, we give its operation-based version in Listing~\ref{lst:operation-based multi-value register}. Here $myRep()$ is a function that returns current replica identifier, and $reps()$ is a function that returns the number of replicas in distributed system. This implementation assumes that the number of replicas are fixed. A payload $S$ is a set of $(a,V)$ pairs, where $a$ is a value and $V$ is a vector called version vector. %Annotation1 is an annotation for the current payload, and annotation2 is an annotation for downstream of $write(a)$.
Given vector clock $V$ and $V'$, we say that $V > V'$, if for each replica $\arep$, we have $V[\arep] > V'[\arep]$. Annotation1 is an annotation for downstream of $write(a)$.


\begin{minipage}[t]{1.0\linewidth}
\begin{lstlisting}[frame=top,caption={Pseudo-code of operation-based multi-value register},
captionpos=b,label={lst:operation-based multi-value register}]
  payload Set S
  initial S = @|$\emptyset$|@
  initial seq = @|$\epsilon$|@

  write(a) :
    atSource :
      let g = myRep()
      let @|$\mathcal{V}$|@ = @|$\{ V \vert \exists x, (x,V) \in S \}$|@
      let @|$V'$|@ = @|$[ max_{V \in \mathcal{V}} V[j]]_{j \neq g}$|@
      let @|$V'[g]$|@ = @|$max_{V \in \mathcal{V}} V[g]$|@ + 1
      //@ let seq@|$'$|@ = seq@|$\,\cdot\,\alabellongind[readIds]{}{S}{}\,\cdot\,\alabelshort[write]{a,V',S}$|@
    downStream(a, V@|$'$|@) :
      let A = @|$\{ (a_1,V_1) \in S \vert \neg V' > V_1 \}$|@
      let B = @|$\{ (a,V') \}$|@, if @|$\forall (a_1,V_1) \in S, \neg V_1 > V'$|@. Otherwise, let B = @|$\emptyset$|@
      S = A @|$\cup$|@ B
      //@ Annotation1 : @|$\forall \arep, V'[\arep]$|@ = @|$\vert \{ \alabel = \alabellongind[write]{\_,\_}{\bot}{*}, \alabel$|@ happens on replica @|$\arep,  (\alabel,\alabelshort[write]{a,V'}) \in \avisord \vee \alabel = \alabelshort[write]{a,V'} \} \vert$|@

  read() :
    let @|$S_1$|@  = {a : @|$\exists$|@ V. (a,V) @|$\in$|@ S}
    //@ let seq@|$'$|@ = seq@|$\,\cdot\,\alabellongind[read]{}{S_1}{}$|@
    return @|$S_1$|@
\end{lstlisting}
\end{minipage}

%//@ Annotation1 : S =  @|$\{ (a,V) \vert \exists \alabel = \alabellongind[write]{a,V}{\bot}{*}, \alabel$|@ is maximal w.r.t @|$\avisord$|@ among write operations applied in current replica @|$\}$|@





\subsection{The Wooki Algorithm}
\label{subsec:the Wooki algorithm}

The Wooki algorithm of \ref{the paper of Wooki} is given in Listing~\ref{lst:wooki algorithm}. Note that here $integrateIns$ is a recursive method used by $addBetween$ method.

In local of each replica, Wooki algorithm stores the list as a sequence of W-characters. A W-character $w$ is a tuple $<id,v,degree,flag>$, where $id$ is the identifier of $w$; $v$ is the value of $w$; $degree$ is the degree of $w$; $flag \in \{ \mathit{true},\mathit{false} \}$ is the flag of $w$ and indicates whether $w$ is ``visible'' in list. A identifier $id$ of W-character is a tuple $(ctr,\arep)$, where $ctr \in \mathbb{N}$. We use $degree(w)$ to denote the degree of $w$.

A W-string is an ordered sequence of W-characters $w_b \cdot w_1 \cdot \ldots \cdot w_n \cdot w_e$, where $w_b$ and $w_e$ are special W-characters that mark the beginning and the ending of the sequence. The values of $w_b$ and $w_e$ are $\circ_b$ and $\circ_e$, respectively. The degree of $w_b$ and $w_e$ are $0$. We define the following function for a W-string $str$:

\begin{itemize}
\setlength{\itemsep}{0.5pt}
\item[-] $\vert str \vert$ returns the length of $str$,

\item[-] $str[p]$ returns the W-character at position $p$ in $str$. Her we assume that the first element of $str$ is at position 0.

\item[-] $pos(str,w)$ returns the position of W-character $w$ in $S$.

\item[-] $insert(str,w,p)$ inserts W-character $w$ into $str$ at position $p$.

\item[-] $subseq(str,w_1,w_2)$ returns the part of $str$ between the W-characters $w_1$ and $w_2$ (excluding $w_1$ and $w_2$).

\item[-] $contains(str,a)$ returns true if there exists a W-character in $str$ with value $a$.

\item[-] $values(str)$ returns the sequence of visible (with $\mathit{true}$ flag) values of $str$.

\item[-] $getWchar(str,a)$ returns the W-character with value $a$ in $str$.

\item[-] $changeFlag(str,pos,f)$ changes the flag of $str[pos]$ into $f$.
\end{itemize}

%Note that only $values(str)$ distinguish whether a W-character is with flag $\mathit{true}$ or with flag $\mathit{false}$.

A total order $<_{id}$ is given for identifiers of W-characters for conflict resolution. Given two identifiers $(ctr_1,\arep_1)$ and $(ctr_2,\arep_2)$, we have $(ctr_1,\arep_1) <{id} (ctr_2,\arep_2)$, if $\arep_1 < \arep_2 \vee (\arep_1 = \arep_2 \wedge ctr_1 < ctr_2)$. Given a W-string $str$ and two W-characters $a,b$ of $str$, we write $a <_{str} b$ to indicate that $pos(str,a) < pos(str,b)$.

\begin{minipage}[t]{1.0\linewidth}
\begin{lstlisting}[frame=top,caption={Pseudo-code of Wooki algorithm},
captionpos=b,label={lst:wooki algorithm}]
  payload int @|$clock$|@, W-string @|$string_s$|@
  initial @|$clock$|@ = 0, @|$string_s$|@ = @|$\epsilon$|@
  initial seq = @|$\epsilon$|@

  addBetween(b,a,c) :
    atSource :
      precondition :  @|$contains(string_s,b) \wedge contains(string_s,c) \wedge pos(string_s,c) - pos(string_s,b) = 1\wedge \neg contains(string_s,a)$|@
      let g = myRep()
      let @|$c_p$|@ = @|$getWchar(string_s,b)$|@
      let @|$c_n$|@ = @|$getWchar(string_s,c)$|@
      @|$clock$|@ = @|$clock$|@ + 1
      //@ let seq@|$'$|@ = seq@|$\,\cdot\,\alabellongind[addBetween]{a,b,c}{}{}$|@
    downStream((w,@|$c_p$|@,@|$c_n$|@)) : with @|$w = ((clock,g), a, max(degree(c_p),degree(c_n)) +1, \mathit{true})$|@
      integrateIns(@|$w,c_p,c_n$|@)

  remove(a) :
    atSource :
      precondition : @|$contains(string_s,a)$|@
      let w = @|$getWchar(string_s,a)$|@
      //@ let seq@|$'$|@ = seq@|$\,\cdot\,\alabellongind[remove]{a}{}{}$|@
    downStream(w) :
      let p = @|$pos(string_s,w)$|@
      @|$changeFlag(string_s,p,\mathit{false})$|@

  read() :
    let s = @|$values(string_s)$|@
    //@ let seq@|$'$|@ = seq@|$\,\cdot\,\alabellongind[read]{}{s}{}$|@
    return s

  integrateIns(@|$c,c_p,c_n$|@)
    let @|$S$|@ = @|$string_s$|@
    let @|$S'$|@ = @|$subseq(S,c_p,c_n)$|@
    if @|$S' = \epsilon$|@
      then  @|$insert(S,c,pos(S,c_n))$|@
    else
      Let i = 0
      Let @|$d_{min}$|@ be the minimal degree of W-characters in @|$S'$|@
      Let F be the projection of @|$S'$|@ into W-characters with degree @|$d_{min}$|@
      while (@|$i < \vert F \vert -1 \wedge F[i] <_{id} c$|@) do
        i = i+1
      integrateIns(@|$c,F[i-1],F[i]$|@)
\end{lstlisting}
\end{minipage}

The payload of each replica is a integer value $clock$ used to generate identifier, and a W-string $string_s$.

To do $addBetween(b,a,c)$, we first ensure that $b$ and $c$ are adjacent in $string_s$ and $a$ is not in $string_s$. Then, we generate a W-character $w$ for value $a$, and calls method $integrateIns(w,c_p,c_n)$ to put $w$ between $c_p$ and $c_n$, which are the W-characters of $b$ and $c$ in $string_s$, respectively.

$integrateIns(c,c_p,c_n)$ is a recursive method and works as follows: If there are no W-character between $c_p$ and $c_n$ (for example, in the current replica), then $w$ is put after $c_p$. Else, Wooki selects a sequence $F$ of W-characters, such that each W-character of $F$ is between $c_p$ and $c_n$, and has minimal degree. It can be proved that W-characters in $F$ are sorted by the $<_{id}$ order. Then, we choose the position of $c$, and recursive call $integrateIns(c,F[i-1],F[i])$. We can see that, the minimal degree of the sub-sequence of $S$ between $F[i-1]$ and $F[i]$ is larger than that of the sub-sequence of $S$ between $c_p$ and $c_n$.

To do $remove(a)$, we just set the flag of W-character of $a$ in $string_s$ to be $\mathit{false}$. To do $read()$, we return $values(string_s)$.







\section{Sequential Specifications}
\label{sec:sequential specifications}

%In this section we give several sequential specifications.


\subsection{The Sequential Specification of Counter}
\label{subsec:the sequential specification of counter}

The sequential specification $\mathit{counter}_s$ of counter is so that $\abstates = \mathbb{Z}$, that is the state will be an integer, and the transitions are given as follows:
\begin{itemize}
\setlength{\itemsep}{0.5pt}
\item[-] $k \xRightarrow{\alabelshort[{\tt inc}]{}} k+1$
\item[-] $k \xRightarrow{\alabelshort[{\tt dec}]{}} k-1$
\item[-] $k \xRightarrow{\alabellong[{\tt read}]{}{k}{}} k$
\end{itemize}

Here $\alabelshort[{\tt inc}]{}$ increase the counter by $1$, $\alabelshort[{\tt dec}]{}$ decrease the counter by $1$, and $\alabellong[{\tt read}]{}{k}{}$ returns the value of the counter. 


\subsection{The Sequential Specification of Multi-Value Register}
\label{subsec:the sequential specification of multi-value register}

The query-update rewriting of multi-value register is as follows: $\gamma( \alabelshort[{\tt write}]{a}) = ( \alabellong[{\tt readIds}]{}{S}{}, \alabelshort[{\tt write}]{a,id,S})$.


Each abstract state $\abstate$ is a set of tuples $(a,id)$, where $a$ is a data and $id$ is a identifier. The sequential specification $\mathit{mvreg}_s$ of multi-value register is given with the transitions as follows:

\begin{itemize}
\setlength{\itemsep}{0.5pt}
\item[-] $\abstate \xRightarrow{\alabellong[{\tt readIds}]{}{\abstate}{}} \abstate$

\item[-] $\big(\abstate\ |\ id \ is \ unique \big) \xRightarrow{ \alabelshort[{\tt write}]{a,id,S} }  \abstate \setminus S \cup \{ (a,id) \}$

\item[-] $\abstate \xRightarrow{\alabellong[{\tt read}]{}{ \{ a \vert \exists id, (a,id) \in \abstate \} }{}} \abstate$
\end{itemize}

Here $\alabellong[{\tt readIds}]{}{S}{}$ returns the abstract state, $\alabelshort[{\tt write}]{a,id,S}$ removes $S$ from the abstract state and puts $\{ (a,id) \}$ into the abstract state, and $\alabellong[{\tt read}]{}{S'}{}$ returns the value of multi-value register.



\subsection{The Sequential Specification of List with Add-Between Interface}
\label{subsec:the sequential specification of list with add-between interface}

Each abstract state $\abstate$ is a sequence of tuples $(a,flag)$, where $a$ is a data and $flag \in \{ \mathit{true},\mathit{false} \}$ is flag. $(a,\mathit{true})$ means that $a$ is in the list, while $(a,\mathit{false})$ means that $a$ has once been added into the list and then removed. The sequential specification $\mathit{listBet}_s$ of list with add-between interface is given with the transitions as follows:

\begin{itemize}
\setlength{\itemsep}{0.5pt}
\item[-] $\big(\abstate\ |\ \abstate = (a_1,flag_1)\cdot \ldots \cdot (a_n,flag_n) \wedge 1 \leq i \leq k < j \leq n \wedge (a,\_) \notin \abstate \big) \xRightarrow{ \alabelshort[{\tt addBetween}]{a_i,a,a_j} } (a_1,flag_1)\cdot \ldots (a_k,flag_k) \cdot (a,\mathit{true}) \cdot (a_{k+1},flag_{k+1}) \cdot \ldots \cdot (a_n,flag_n)$

\item[-] $\big(\abstate\ |\ \abstate = (a_1,flag_1)\cdot \ldots \cdot (a_n,flag_n) \wedge 1 \leq k \leq n \big) \xRightarrow{ \alabelshort[{\tt remove}]{a_k} } (a_1,flag_1)\cdot \ldots (a_k,\mathit{false}) \cdot \ldots \cdot (a_n,flag_n)$

\item[-] $(a_1,flag_1)\cdot \ldots \cdot (a_n,flag_n) \xRightarrow{\alabellong[{\tt read}]{}{s}{}} (a_1,flag_1)\cdot \ldots \cdot (a_n,flag_n)$, here $s$ is the projection of $a_1 \cdot \ldots \cdot a_n$ into $\{ a \vert a$ is with flag $\mathit{true} \}$
\end{itemize}

$\alabelshort[{\tt addBetween}]{a_i,a,a_j}$ puts $(a,\mathit{true})$ into a random position between $(a_i,flag_i)$ and $(a_j,flag_j)$, and we assume that each value is put into list at most once. $\alabelshort[{\tt remove}]{a_k}$ removes $a_k$ from the list by setting its flag into $\mathit{false}$. $\alabellong[{\tt read}]{}{s}{}$ returns the content of list.

Assume that the initial value of list is $(\circ_b,\mathit{true}) \cdot (\circ_e,\mathit{true})$, and $\circ_b$ and $\circ_e$ are never removed. When the context is clear, in $\ensuremath{\tt read}$ operation, we will omit $\circ_b$ and $\circ_e$ in return value.




\subsection{The Sequential Specification of OR-Set}
\label{subsec:the sequential specification of or-set}

The query-update rewriting of OR-set is as follows: $\gamma( \alabelshort[{\tt remove}]{a} ) = ( \alabellong[{\tt readIds}]{a}{S}{}, \alabelshort[{\tt remove}]{a,S})$.



Each abstract state $\abstate$ is a set of tuples $(a,id)$, where $a$ is a data and $id$ is a identifier. The sequential specification $\mathit{ORset}_s$ of OR-set is given with the transitions as follows:

\begin{itemize}
\setlength{\itemsep}{0.5pt}
\item[-] $\abstate \xRightarrow{\alabellong[{\tt readIds}]{a}{ \{ (a,id) \vert (a,id) \in \abstate \} }{} } \abstate$

\item[-] $\abstate \xRightarrow{ \alabelshort[{\tt remove}]{a,S} }  \abstate \setminus S$

\item[-] $\big(\abstate\ |\ id \ is \ unique \big) \xRightarrow{ \alabelshort[{\tt add}]{a,id} }  \abstate \cup \{ (a,id) \}$

\item[-] $\abstate \xRightarrow{\alabellong[{\tt read}]{a}{ \{ a \vert \exists id, (a,id) \in \abstate \} }{}} \abstate$
\end{itemize}

Here $\alabellong[{\tt readIds}]{a}{S}{}$ returns the set of pairs with data $a$, $\alabelshort[{\tt remove}]{a,S}$ removes $S$ from the abstract state, $\alabelshort[{\tt add}]{a,id}$ puts $\{ (a,id) \}$ into the abstract state, and $\alabellong[{\tt read}]{}{S'}{}$ returns the value of the or-set.




\subsection{The Sequential Specification of List with Add-After Interface}
\label{subsec:the sequential specification of list with add-after interface}

Each abstract state $\abstate$ is a sequence of tuples $(a,flag)$, where $a$ is a data and $flag \in \{ \mathit{true},\mathit{false} \}$ is flag. $(a,\mathit{true})$ means that $a$ is in the list, while $(a,\mathit{false})$ means that $a$ has once been added into the list and then removed. The sequential specification $\mathit{listAft}_s$ of list with add-after interface is given with the transitions as follows:

\begin{itemize}
\setlength{\itemsep}{0.5pt}
\item[-] $\big(\abstate\ |\ \abstate = (a_1,flag_1)\cdot \ldots \cdot (a_n,flag_n) \wedge 1 \leq k < n \wedge (a,\_) \notin \abstate \big) \xRightarrow{ \alabelshort[{\tt addAfter}]{a_k,a} } (a_1,flag_1)\cdot \ldots (a_k,flag_k) \cdot (a,\mathit{true}) \cdot (a_{k+1},flag_{k+1}) \cdot \ldots \cdot (a_n,flag_n)$

\item[-] $\big(\abstate\ |\ \abstate = (a_1,flag_1)\cdot \ldots \cdot (a_n,flag_n) \wedge 1 \leq k \leq n \big) \xRightarrow{ \alabelshort[{\tt remove}]{a_k} } (a_1,flag_1)\cdot \ldots (a_k,\mathit{false}) \cdot \ldots \cdot (a_n,flag_n)$

\item[-] $(a_1,flag_1)\cdot \ldots \cdot (a_n,flag_n) \xRightarrow{\alabellong[{\tt read}]{}{s}{}} (a_1,flag_1)\cdot \ldots \cdot (a_n,flag_n)$, here $s$ is the projection of $a_1 \cdot \ldots \cdot a_n$ into $\{ a \vert a$ is with flag $\mathit{true} \}$
\end{itemize}

$\alabelshort[{\tt addAfter}]{a_k,a}$ puts $(a,\mathit{true})$ immediately after $(a_k,flag_k)$, and we assume that each value is put into list at most once. $\alabelshort[{\tt remove}]{a_k}$ removes $a_k$ from the list by setting its flag into $\mathit{false}$. $\alabellong[{\tt read}]{}{s}{}$ returns the content of list.

Assume that the initial value of list is $(\circ,\mathit{true})$, and $\circ_b$ and $\circ$ is never removed. When the context is clear, in $\ensuremath{\tt read}$ operation, we will omit $\circ$ in return value.








\section{Proofs of \crdtimp}
\label{sec:appendix proofs of crdt implementations}



\subsection{Proof of Operation-Based Multi-Value Register}
\label{subsec:proof of operation-based multi-value register}

Given two sequences $l_1,l_2$  such that $l_2$ is a permutation of $l_1$, let $\mathit{diff}(l_1,l_2) = \{ (a,b) \vert$ the order of $a$ and $b$ in $l_1$ is different from that of $l_2 \}$. Given a sequence $l$ and two elements $a$ and $b$ of $l$, let $\mathit{swap}(l,a,b)$ be a sequence obtained from $l$ by swapping $a$ and $b$. The following lemma states that, given two specification sequences $(\alabelset_1, \aseqord_1)$ and $(\alabelset, \aseqord_2)$ that are generated from a same history and both consistent with visibility relation, we can obtain $\aseqord_2$ from $\aseqord_1$ by several time of swapping adjacent pair of concurrent operations.

\begin{lemma}
\label{lemma:given two sequence consistent with visibility order, one can be obtained from the other}
Given a history $(\alabelset,\avisord)$ and two two specification sequences $(\alabelset_1, \aseqord_1)$ and $(\alabelset, \aseqord_2)$ that are both consistent with $\avisord$. If $\aseqord_1 \neq \aseqord_2$, then we can obtain $\aseqord_2$ from $\aseqord_1$ by several time of swapping adjacent pair of concurrent operations.
\end{lemma}

\begin {proof}

First, we need to prove that, if $\mathit{diff}(\aseqord_1,\aseqord_2) \neq \emptyset$, then, there exists $(\alabel_1,\alabel_2) \in \mathit{diff}(\aseqord_1,\aseqord_2)$, such that $l_1$ and $l_2$ are concurrent, and $l_1$ and $l_2$ are adjacent in $\aseqord_1$.

We prove this by contradiction. Assume $\mathit{diff}(\aseqord_1,\aseqord_2) \neq \emptyset$, and for each $(\alabel_1,\alabel_2) \in \mathit{diff}(\aseqord_1,\aseqord_2)$, we have that either $\alabel_1$ and $\alabel_2$ are not concurrent, or $\alabel_1$ and $\alabel_2$ are not adjacent in $\aseqord_1$.

Since $\mathit{diff}(\aseqord_1,\aseqord_2) \neq \emptyset$, let $(\alabel,\alabel')$ be a element of $\mathit{diff}(\aseqord_1,\aseqord_2)$, and the distance of $\alabel$ and $\alabel'$ is minimal in $\{$ the distance between $\alabel_1$ and $\alabel_2 \vert (\alabel_1,\alabel_2) \in \mathit{diff}(\aseqord_1,\aseqord_2) \}$. Let us prove that $\alabel$ and $\alabel'$ are adjacent by contradiction: If there exists $\alabel''$ between $\alabel$ and $\alabel'$. Assume that in $\aseqord_1$, $\alabel$ is before $\alabel''$, and $\alabel''$ is before $\alabel'$. By assumption, the order between $\alabel$ and $\alabel''$, and between $\alabel''$ and $\alabel'$ is the same in $\aseqord_1$ and in $\aseqord_2$. This implies that $\alabel$ is still before $\alabel'$ in $\aseqord_2$, which contradicts the fact that $(\alabel,\alabel') \in \mathit{diff}(\aseqord_1,\aseqord_2)$.

Since $\alabel$ and $\alabel'$ are adjacent and $(\alabel,\alabel') \in \mathit{diff}(\aseqord_1,\aseqord_2)$, by assumption we know that $\alabel$ and $\alabel'$ are not concurrent. Or we can say, $(\alabel,\alabel') \in \avisord \vee (\alabel',\alabel) \in \avisord$. This contradicts that both $\aseqord_1$ and $\aseqord_2$ are consistent with visibility relation. This completes the proof of the first part.

Since $\aseqord_1 \neq \aseqord_2$, we have $\mathit{diff}(\aseqord_1,\aseqord_2) \neq \emptyset$, and then, as discussed above, there exists $(\alabel,\alabel') \in \mathit{diff}(\aseqord_1,\aseqord_2)$, such that $\alabel$ and $\alabel'$ are concurrent, and $\alabel$ and $\alabel'$ are adjacent in $\aseqord_1$. Let $\aseqord_3 = \mathit{swap}(\aseqord_1,\alabel,\alabel')$. It is easy to see that $\mathit{diff}(\aseqord_1,\aseqord_2) > \mathit{diff}(\aseqord_3,\aseqord_2)$. Therefore, by several times of above process, we finally obtain $\aseqord_2$ from $\aseqord_1$ by swapping pairs of operations. This completes the proof of this lemma. $\qed$
\end {proof}


Then, let us prove that the operation-based multi-value register implementation is \crdtlinearizable{} w.r.t $\mathit{mvreg}_s$.

\begin{lemma}
\label{lemma:multi-value register is correct}
The operation-based multi-value register implementation is \crdtlinearizable{} w.r.t $\mathit{mvreg}_s$
\end{lemma}


\begin {proof}

Let us give two facts:

\begin{itemize}
\setlength{\itemsep}{0.5pt}
\item[-] $fact1$: Let $(a_1,V_1)$ and $(a_2,V_2)$ be the downstream for $\alabel_1$ and $\alabel_2$, respectively, and assume that $(\alabel_1,\alabel_2) \in \avisord$. Then, $V_1 < V_2$.
\item[-] $fact2$: Let $(a_1,V_1)$ and $(a_2,V_2)$ be the downstream for $\alabel_1$ and $\alabel_2$, respectively, and assume that $\alabel_1$ and $\alabel_2$ are concurrent. Then, $\neg (V_1 < V_2 \vee V_2 < V_1)$.

%\item[-] $fact3$: Let $S$ be the payload of a replica. Then, $S$ =  $\{ (a,V) \vert \exists \alabel = \alabellongind[write]{a,V}{\bot}{*}, \alabel$ is maximal w.r.t $\avisord$ among write operations applied in current replica $\}$.
\end{itemize}


\noindent Proof of $fact1$: Assume $\alabel_1$ happens on replica $\arep$. By the {\textred{causal delivery}} assumption, we know that for each replica $\arep' \neq \arep$, $\alabel_2$ see more or equal number of operations happens on replica $\arep'$ than that of $\alabel_1$, and $\alabel_2$ see more number of operations happens on replica $\arep$ than that of $\alabel_1$. By Annotation1, we know that $\forall \arep' \neq \arep$, $V_1[\arep'] \leq V_2[\arep']$, and $V_1[\arep] < V_2[\arep]$. Therefore, $V_1 < V_2$.

\noindent Proof of $fact2$: Let us prove that $\neg (V_1 < V_2 \vee V_2 < V_1)$ by contradiction. It is obvious that $\alabel_1$ and $\alabel_2$ happens on different replicas. Assume that $V_1 < V_2$, and assume that $\alabel_1$ happens on replica $\arep_1$. Since $V_1 < V_2$, we know that $V_1[\arep_1] \leq V_2[\arep_1]$. By the {\textred{causal delivery}} assumption and Annotation1, this means $(\alabel_1,\alabel_2) \in \avisord$, contradicts the assumption that $\alabel_1$ and $\alabel_2$ are concurrent. Similarly, we can see that $\neg (V_2 < V_1)$. Therefore, $\neg (V_1 < V_2 \vee V_2 < V_1)$.

Let us propose Annotation2, which is an annotation of payload and obviously holds in the initial global configuration.

\begin{itemize}
\setlength{\itemsep}{0.5pt}
\item[-] Annotation2: Let $S$ be the payload of a replica. Then, $S$ =  $\{ (a,V) \vert \exists \alabel = \alabellongind[write]{a,V}{\bot}{*}, \alabel$ is maximal w.r.t $\avisord$ among write operations applied in current replica $\}$.
\end{itemize}

Our proof of the lemma proceed as follows:

\begin{itemize}
\setlength{\itemsep}{0.5pt}
\item[-] We need to prove that $\mathsf{ReplicaStates}$ is an inductive invariant.

Since every operation is appended to the linearization when it executes {\tt atSource} it clearly follows, the linearization order is consistent with visibility order. Then, by the {\textred{causal delivery}} assumption, the order in which downstreams are applied at a given replica is also consistent with the visibility order. Let $\aseqord_1$ be the projection of linearization order into labels applied in a replica $\arep$, and $\aseqord_2$ be the order of labels applied in replica $\arep$. By Lemma \ref{lemma:given two sequence consistent with visibility order, one can be obtained from the other}, $\aseqord_2$ can be obtained from $\aseqord_1$ by several time of swapping adjacent pair of concurrent operations.

Let us prove that applying downstream of such pair of operations commute, and we only need to consider the case of two concurrent $write$ labels. Let $(a_1,V_1)$ and $(a_2,V_2)$ be the downstream of labels $\alabellongind[write]{a_1,V_1}{\bot}{*}$ and $\alabellongind[write]{a_2,V_2}{\bot}{*}$, respectively. Given a payload $S$, assume we obtained $S'$ from $S$ by applying $(a_1,V_1)$ and then applying $(a_2,V_2)$, and assume we obtained $S''$ from $S$ by applying $(a_2,V_2)$ and then applying $(a_1,V_1)$. In the process of obtaining $S'$ or $S''$ from $S$, we add $(a_1,V_1)$ and $(a_2,V_2)$ into $S$ and remove the following tuple $(a_3,V_3) \in S \cup \{(a_1,V_1),(a_2,V_2)\}$: either $V_3 < V_1$, or $V_3 < V_2$, or $(a_3,V_3) = (a_1,V_1) \wedge V_1 < V_2$, or $(a_3,V_3) = (a_2,V_2) \wedge V_2 < V_1$. Since we already know that $\alabellongind[write]{a_1,V_1}{\bot}{*}$ and $\alabellongind[write]{a_2,V_2}{\bot}{*}$ are concurrent, by  $fact_2$, we know that $\neg (V_1 < V_2 \vee V_2 < V_1)$. Therefore, $S' = S''$.



\item[-] Let we prove that the Annotation1 and Annotation2 is an inductive invariant.

We prove by induction on executions. Obvious they hold in $\aglobalstate_0$. Assume they hold along the execution $\aglobalstate_0 \xrightarrow{}^* \aglobalstate$ and there is a new transition $\aglobalstate \xrightarrow{} \aglobalstate'$. We need to prove that they still hold in $\aglobalstate'$. We only need to consider $write$ action or downstream:

    \begin{itemize}
    \setlength{\itemsep}{0.5pt}
    \item[-] For case of a $\alabellongind[write]{a,V'}{\bot}{*}$ action of replica $\arep$: Let $S$ and $S'$ be the payload of replica $\arep$ of $\aglobalstate$ and $\aglobalstate'$, respectively. Obviously $S' = \{ (a,V') \}$.

        Since $\alabellongind[write]{a,V'}{\bot}{*}$ is larger than any labels in $S$ w.r.t the visibility relation, Annotation2 still holds in $\aglobalstate'$.

        By Annotation2 we know what labels are contained in $S$, by Annotation1 we know the content of these labels. By the {\textred{causal delivery}} assumption we know that if a label $\alabel$ is visible to a label $\alabel'$ of $S$, then $\alabel$ must be already applied in replica $\arep$. let $\mathcal{V} = \{ V \vert (\_,V) \in S \}$ be the set of vector clocks of $S$. Therefore, for each replica $\arep' \neq \arep$, $max_{V \in \mathcal{V}} V[\arep']$ is the number of operations happen on replica $\arep'$ and has been applied in replica $\arep$ during $\aglobalstate_0 \xrightarrow{}^* \aglobalstate$, and $max_{V \in \mathcal{V}} V[\arep]$ is the number of operations happen on replica $\arep$ during $\aglobalstate_0 \xrightarrow{}^* \aglobalstate$. We can see that, for each replica $\arep' \neq \arep$, $V'[\arep'] = max_{V \in \mathcal{V}} V[\arep']$, and $V'[\arep] = max_{V \in \mathcal{V}} V[\arep] +1$. Therefore, Annotation1 still holds in $\aglobalstate'$.

    \item[-] For case of applying downstream $(a,V')$: We only need to consider Annotation2. Let $S$ and $S'$ be the payload of replica $\arep$ of $\aglobalstate$ and $\aglobalstate'$, respectively.

        By the {\textred{causal delivery}} assumption, if $\alabellongind[write]{a,V'}{\bot}{*}$ is visible to a operation $\alabel$, then $\alabel$ does not applied in $\aglobalstate'$ yet. By Annotation2, $fact1$ and $fact2$, we know that, $\forall (b,V) \in S$, we have $\neg(V > V')$. Therefore, we have $S' = S \setminus \{ (b,V) \vert (b,V) \in S \wedge V < V' \} \cup \{ (a,V') \}$. By $fact1$ and $fact2$, each element in $\{ (b,V) \vert (b,V) \in S \wedge V < V' \}$ is visible to $\alabellongind[write]{a,V'}{\bot}{*}$, and they are not in $S'$. Therefore, Annotation2 still holds in $\aglobalstate'$.
    \end{itemize}

\item[-] Let us prove that $\mathsf{Refinement}$ holds. We consider a refinement mapping $\refmap$ defined as the identity.

    \begin{itemize}
    \setlength{\itemsep}{0.5pt}
    \item[-] For $(a,V')$ of downstream and $write(a,V',S_1)$ in sequential specification:

    Assume we obtain payload $S'$ from $S$ by doing downstream of $(a,V')$, in sequential specification have $\abstate \xrightarrow{write(a,V',S_1)} \abstate'$, and $\refmap(S) = \abstate$, or we can say, $S = \abstate$. We need to prove that $S' = \abstate'$.

    By the {\textred{causal delivery}} assumption, if $(\alabellongind[write]{a,V'}{\bot}{*},\alabel) \in \avisord$, then the downstream of $\alabel$ is not applied yet in the replica of $S$. By Annotation2, $fact1$ and $fact2$, we can see that, $\forall (b,V) \in S$, $\neg(V > V')$. Therefore, according to the implementaiton, we can see that $S' = S \setminus S_2 \cup \{ (a,V') \}$, where $S_2 = \{ (b,V) \vert (b,V) \in S \wedge V < V' \}$.

    According to Annotation2, we can see that, $S_1 = \{ (b,V) \vert \exists \alabel = \alabellongind[write]{b,V}{\bot}{*}, (b,V)$ is the downstream of $\alabel, \alabel$ is maximal among $write$ operations visible to $\alabellongind[write]{a,V'}{\bot}{*} \}$. We can see that $\abstate' = \abstate \setminus S_1 \cup \{ (a,V') \}$.

    Let us prove $S' = \abstate'$ by contradiction.

        \begin{itemize}
        \setlength{\itemsep}{0.5pt}
        \item[-] If there exists item $(c,V'')$ in $\abstate'$ but not in $S'$: we can see that $(c,V'') \in S$, $(c,V'') \notin S_1$, and $(c,V'') \in S_2$.

        Since $(c,V'') \notin S_1$, we know that there exists a $write$ operation $\alabel$, such that $(\alabellongind[write]{c,V''}{\bot}{*},\alabel),(\alabel,\alabellongind[write]{a,V'}{\bot}{*}) \in \avisord$. Since $(c,V'') \in S$, we can see that the downstream $\alabel$ is not applied yet in the replica of $S$, while in $S'$, the downstream of $\alabellongind[write]{a,V'}{\bot}{*}$ is applied. This violates the {\textred{causal delivery}} assumption.

        \item[-] If there exists item $(c,V'')$ in $S'$ but not in $S\abstate'$: we can see that $(c,V'') \in S$, $(c,V'') \notin S_2$, and $(c,V'') \in S_1$.

        Since $(c,V'') \notin S_2$, we know that $\neg(V'' < V)$. Since $(c,V'') \in S_1$, we know that $(\alabellongind[write]{c,V''}{\bot}{*},\alabellongind[write]{a,V'}{\bot}{*}) \in \avisord$. This contradicts $fact1$ and $fact2$.
        \end{itemize}

    Therefore, we know that $S' = \abstate'$, and the case of $(a,V')$ and $write(a,V',S_1)$ holds.

    \item[-] When the query-update $\alabelshort[write]{a}$ executes {\tt atSource} on a state $S$, then the query $\alabellong[readIds]{}{R}{}$ (introduced by the query-update rewriting) should be enabled in state $\refmap(S)=\abstate$, which clearly holds because the computation of $R$ in {\tt atSource} returns $S$, and the result of $\alabelshort[readIds]{}$ in the specification state $\abstate = S$ also returns $S$.

    \item[-] Applying the query $\alabelshort[read]{}$ on the payload $S$ should result in the same return value as applying the same query in the context of the specification on the same state $\abstate = \refmap(S)$, which again holds trivially.
    \end{itemize}

\item[-] Finally, we describe the proof of the fact that $\mathsf{\CRDTLinshort{}}$ is an inductive invariant. As already mentioned, appending operations to the linearization when they execute {\tt atSource} clearly implies that $\aseqord$ is consistent with the visibility. Next, the projection of $\aseqord$ on the updates is obviously admitted by the specification (the updates are always enabled from the point of view of the specification).
We also have to argue that for each query $\alabel_{\mathsf{qr}}\in\{\alabellongind[readIds]{}{R}{},\alabellongind[read]{}{A}{}\}$, the sequence $\aseqord'\cdot \alabel_{\mathsf{qr}}$ where $\aseqord'$ is the projection of $\aseqord$ on the set of updates
visible to $\alabel_{\mathsf{qr}}$ is admitted by the specification. First, by $\mathsf{ReplicaStates}$, the state $\sigma$ of the replica where $\alabel_{\mathsf{qr}}$ is applied is obtained by applying the downstreams of the operations visible to $\alabel_{\mathsf{qr}}$ in the linearization order. Then, by $\mathsf{Refinement}$, every downstream is simulated by the corresponding operation in the context of the specification. This implies that $\refmap(\sigma_0)\xRightarrow{\aseqord'}\refmap(\sigma)$, where $\sigma_0$ is the initial replica state. The query $\alabel_{\mathsf{qr}}$ is also simulated by the same operation in the context of the specification, which implies that $\refmap(\sigma)\xRightarrow{\alabel_{\mathsf{qr}}}\refmap(\sigma)$. These two facts imply that $\refmap(\sigma_0)\xRightarrow{\aseqord'\cdot \alabel_{\mathsf{qr}}}\refmap(\sigma)$ which means that $\aseqord'\cdot \alabel_{\mathsf{qr}}$ is admitted by the specification.
\end{itemize}

This completes the proof of this lemma. $\qed$
\end {proof}
















\section{\crdtlin{} with Non-Deterministic Sequential Specifications and Its Proof}
\label{sec:appendix RA-linearizability with non-deterministic sequential specifications and its proof}



\subsection{\crdtlin{} with Non-Deterministic Sequential Specifications}
\label{subsec:RA-linearizability with non-deterministic sequential specifications}

Recall that a sequential specification \Spec{} is deterministic, if for every label, the transition from a given initial state can produce at most one final state. Otherwise, we say that \Spec{} is non-deterministic.

Similarly as in Section \ref{subsec:definition of distributed linearizability}, let us provide the definition of \crdtlin{}. For presentation reasons, we first consider the case where all the labels in the history are either queries or updates.

\begin{definition}
\label{definition:ralinearizability1 with non-deterministic specifications}
A history $h = (\alabelset,\avisord)$ with $\alabelset\subseteq \queries\cup\updates$ is \crdtlinearizable{} w.r.t. a non-deterministic sequential specification \Spec{}, if there exists a specification sequence $(\alabelset, \aseqord) \in \Spec{}$, called the \emph{\crdtlinearization{}} of $h$, where we remark that the set of labels are identical, such that
\begin{enumerate}[(i)]
\item \aseqord{} is consistent with  \avisord{}, that is: $(\avisord \cup \aseqord)^{+}$ is acyclic,

\item the projection of $\aseqord$ to \emph{updates} is admitted by $\Spec$, i.e. $\aseqord\!\downarrow_{\updates} \in \Spec$,

\item assume $\abstate_0$ is the initial abstract value of \Spec. For each sub-sequence $s$ of $\aseqord\!\downarrow_{\updates}$ (including $s = \aseqord\!\downarrow_{\updates}$), we fix a abstract state $\abstate_s$ which is obtained by $\abstate_0 \xrightarrow{ s }^* \abstate_s$, whenever such transitions are defined.

    We require that, given two sub-sequences $s_1,s_2$ of $\aseqord\!\downarrow_{\updates}$, if $s_2$ contains more operation than $s_1$, and both $\abstate_{s_1}$ and $\abstate_{s_2}$ are defined, then, $\abstate_{s_1} \xrightarrow{ s_2 - s_1 }^* \abstate_{s_2}$, where $s_2 - s_1$ denotes the sequences obtained from $s_2$ by removing elements of $s_1$.


\item for each query $\alabel_{\mathsf{qr}}\in \alabelset$, let $s = \avisord^{-1}(\alabel_{\mathsf{qr}})\cap \updates$. Then, we require that $\abstate_s \xrightarrow{ \alabel_{\mathsf{qr}} }^* \abstate_s$.
\end{enumerate}
In this case we say that $(\alabelset, \aseqord)$ is an \emph{\crdtlinearization{}} of $h$ w.r.t. $\Spec{}$.
\end{definition}

Since the sequential specification is non-deterministic, we fix the consequence $\abstate_s$ in sequential specification when applying update operation sequence $s$ from the initial state $\abstate_0$ of sequential specification. We also require that, when $s_1$ is a sub-sequence of $s_2$, the corresponding abstract of $s_1$ is a ``sub-state'' of that of $s_2$.

The case when histories include query-updates is similarly dealt with as Definition \ref{definition:distributed linearizability}. We do it by rewriting of the original history where each query-update is decomposed into a label representing the query part and another label representing the update part.

\begin{definition}[\CRDTLin{} with Non-Deterministic Sequential Specifications]
\label{definition:distributed linearizability with non-deterministic sequential specifications}
A history $h =(\alabelset,\avisord)$ is \crdtlinearizable{} w.r.t. a non-deterministic sequential specification \Spec{}, if there exists a query-update rewriting $\gamma$ such that $\gamma(h)$ is \crdtlinearizable{} w.r.t. \Spec{}.
\end{definition}

A set $H$ of histories is called \crdtlinearizable{} w.r.t a non-deterministic sequential specification $\Spec$ when each history $h\in H$ is \crdtlinearizable{} w.r.t. $\Spec$. A data type implementation is \crdtlinearizable{} w.r.t. a non-deterministic sequential specification $\Spec$ if for any object $\aobj$ of the data type, the set $\histories(\aobj)$ is linearizable w.r.t. $\Spec$.

Let us begin to consider convergence. Given a \crdtlinearizable{} history with two replicas $r_1,r_2$ see the same set of operations.

According to Definition \ref{definition:ralinearizability1 with non-deterministic specifications} and Definition \ref{definition:distributed linearizability with non-deterministic sequential specifications}, we have already fix the non-deterministic choice. Therefore, it is obvious that we have convergence, as formalized in the following lemma.

\begin{lemma}
\label{lemma:distributed linarizability implies convergence for non-deterministic sequential specifications}
If a history $h$ is \crdtlinearizable{} w.r.t. a non-deterministic sequential specification \Spec, then $h$ is convergent.
\end{lemma}






\subsection{Proof Methodology for \crdtlin{} with Non-Deterministic Sequential Specifications}
\label{subsec:proof methodology for RA-linearizability with non-deterministic sequential specifications}

Our proof methodology for \crdtlin{} with non-deterministic sequential specifications are nearly the same as that for deterministic sequential specifications.

We need to instrument the object semantics with an auxiliary variable $\aseqord$ recording a linearization of the current history. The linearization $\aseqord$ is updated every time an operation executes the {\tt atSource} portion of a method, and should guarantee that the linearization $\aseqord$ is consistent with the visibility relation. We need to prove that $\mathsf{ReplicaStates}$ and $\mathsf{\CRDTLinshort{}}$ are inductive invariant. We also rely on additional assertions describing the effect of each downstream and then, prove that $\mathsf{Refinement}$ holds.

The only difference is that, since the definition of linearization in Definition \ref{definition:ralinearizability1 with non-deterministic specifications} is more complex than that of Definition \ref{definition:ralinearizability1}, we need to prove more things: We need to prove that, given the global sequence of update operations $s$ and a sub-sequence $s_1$ of $s$ whose transitions has definition in sequential specifications, in sequential specification, the abstract state obtained by executing $s_1$ is a ``sub-state'' of that of $s$.

Similar as in the proof of or-set, for Wooki, the linearization $\aseqord$ is defined by the order in which the {\tt atSource} procedures are executed, i.e., every new operation executing {\tt atSource} is added at the end of the current linearization.





\subsection{Proof of Wooki}
\label{subsec:proof of Wooki}


Given a a W-string $s$ and two W-characters $c_1,c_2$, such that $c_1$ and $c_2$ are with degree $i$,  $c_1 <_s c_2$, and there are no W-character with degree $i$ between $c_1$ and $c_2$, we say that $c_1$ and $c_2$ are degree-$i$-adjacent in $s$. The following lemma confirms an intuition of Wooki algorithm: in recursive method $integrateIns$, the W-characters of W-string $F$ are sorted by $<_{id}$ order.

\begin{lemma}
\label{lemma:in F of Wooki, W-characters are ordered in id-order}
Given a W-string $s$ and two W-characters $c_1,c_2$, which are degree-$i$-adjacent. Then, for $W$-characters $c_3$ and $c_4$ with degree $i+1$ and $c_1 <_s \{ c_3,c_4 \} <_s c_2$, we have that, $c_3 <_s c_4$, if and only if $c_3 <_{id} c_4$.
\end{lemma}

\begin {proof}
Let us prove this property by induction.

It is obvious that this property holds initially. Let us prove the induction part by contradiction. Assume this property hold for $s$, and does not hold after we do $addBetween(a,b,c)$ to $s$. Let $c_a$ be the W-character of $a$. Let $s'$ be the W-string obtained from $s$ by doing $addBetween(a,b,c)$. Then, we consider all the possibilities:

\begin{itemize}
\setlength{\itemsep}{0.5pt}
\item[-] Case 1: There exists degree-$j$-adjacent $c_1$ and $c_2$ of $s$, and there exists W-character $c_3$, such that $c_a$ and $c_3$ are of degree $j+1$, $c_1 <_{s'} \{ c_a,c_3 \} <_{s'} c_2$, and it is not the case that $c_3 <_{s'} c_a$ if and only if $c_3 <_{id} c_a$: According to Wooki algorithm, since $c_1$ and $c_2$ are degree-$j$-adjacent and $c_1 <_{s'} c_a <_{s'} c_2$, in some time of recursive call of $integrateIns$ of $addBetween(a,b,c)$, we must call $integrateIns(c_a,c_1,c_2)$. By induction assumption we know that in $s$, all W-characters $S$ of degree $j+1$ between $c_1$ and $c_2$ (including $c_3$) are sorted by $<_{id}$. By Wooki algorithm, $c_a$ will be sorted with $S$. This implies that $c_3 <_{s'} c_a$ if and only if $c_3 <_{id} c_a$, contradicts the assumption that it is not the case that $c_3 <_{s'} c_a$ if and only if $c_3 <_{id} c_a$.

\item[-] Case 2: There exists degree-$j$-adjacent $c_1$ and $c_2$ of $s$, and there exists W-character $c_3,c_4$, such that (1) $c_a$ is of degree $j$, $c_1 <_{s'} c_a <_{s'} c_2$, (2) $c_3$ and $c_4$ are of degree $j+1$, and we have $c_1 <_{s'} \{c_3,c_4\} <_{s'} c_a$, and (3) it is not the case that $c_3 <_{s'} c_4$ if and only if $c_3 <_{id} c_4$: It is obvious that $c_1 <_s \{ c_3,c_4 \} <_s c_2$, and since $c_3$ and $c_4$ are degree-$j$-adjacent in $s$, by induction assumption we know that $c_3 <_s c_4$ if and only if $c_3 <_{id} c_4$. This obviously also holds in $s'$, and contradicts the assumption that it is not the case that $c_3 <_{s'} c_4$ if and only if $c_3 <_{id} c_4$.
\end{itemize}

This completes the proof of this lemma. $\qed$
\end {proof}


The following lemma states that, in Wooki algorithm, commute two $addBetween$ operations lead to a same consequence.

\begin{lemma}
\label{lemma:in Wooki algorithm, commute two addBetween operations lead to a same consequence}
a
\end{lemma}


















\section{Proofs of Section \ref{sec:compositionality of distributed linearizability}}
\label{sec:appendix proofs of section compositionality of distributed linearizability}





\subsection{Proofs of Lemma \ref{lemma:several t0-specifications}}
\label{subsec:appendix proofs of Lemma several t0-specifications}

A specification $\mathit{spec}$ is called t0-specification, if given a history $h$ that is distributed linearizable w.r.t $\mathit{spec}$, then any sequence that is consistent with visibility relation is a linearization of $h$.

Given two sequences $l_1,l_2$, let $\mathit{diff}(l_1,l_2) = \{ (o_1,o_2) \vert$ the order of $o_1$ and $o_2$ in $l_1$ is different from that of $l_2 \}$. Given a sequence $l$ and two elements $o_1$ an $o_2$ of $l$, let $\mathit{swap}(l,o_1,o_2)$ be a sequence obtained from $l$ by swapping $o_1$ and $o_2$.

The following lemma states that $\mathit{OR}$-$\mathit{set}_s$ is a t0-specification.

\begin{lemma}
\label{lemma:or-set is a t0-specification}
$\mathit{OR}$-$\mathit{set}_s$ is a t0-specification.
\end{lemma}

\begin {proof}
Given a distributed linearizable history $h$ and assume that $\mathit{lin}$ is a linearization. It is obvious that $\mathit{lin}$ is consistent with visibility relation. We need to prove that, each such sequence $\mathit{lin}'$ described below is also a linearization of $h$

\begin{itemize}
\setlength{\itemsep}{0.5pt}
\item[-] $\mathit{lin}'$ contains the same set of elements as that of $\mathit{lin}$.

\item[-] $\mathit{lin}'$ is consistent with visibility relation.
\end{itemize}

We prove this by showing that each such $\mathit{lin}'$ can be obtained from $\mathit{lin}$ by several times of swapping a pair of adjacent elements. Our proof requires the following two properties:

\begin{itemize}
\setlength{\itemsep}{0.5pt}
\item[-] The first property is: Given a linarization $\mathit{lin}$ and a sequence $\mathit{lin}'$ consistent with visibility relation of $h$, if $\mathit{diff}(\mathit{lin},\mathit{lin}') \neq \emptyset$, there exists $(o_1,o_2) \in \mathit{diff}(\mathit{lin},\mathit{lin}')$, such that $o_1$ and $o_2$ are concurrent, and $o_1$ and $o_2$ are adjacent in $\mathit{lin}$.

    We prove this by contradiction. Assume $\mathit{diff}(\mathit{lin},\mathit{lin}') \neq \emptyset$, and for each $(o_1,o_2) \in \mathit{diff}(\mathit{lin},\mathit{lin}')$, we have that either $o_1$ and $o_2$ are not concurrent, or $o_1$ and $o_2$ are not adjacent in $\mathit{lin}$.

    Since $\mathit{diff}(\mathit{lin},\mathit{lin}') \neq \emptyset$, let $(o,o')$ be a element of $\mathit{diff}(\mathit{lin},\mathit{lin}')$, and the distance of $o_1$ and $o_2$ is minimal in $\{$ the distance between $o_1$ and $o_2 \vert (o_1,o_2) \in \mathit{diff}(\mathit{lin},\mathit{lin}') \}$. Let us prove that $o$ and $o'$ are adjacent by contradiction: If there exists $o''$ between $o$ and $o'$. Assume that in $\mathit{lin}$, $o$ is before $o''$, and $o''$ is before $o'$. By assumption, the order between $o$ and $o''$, and between $o''$ and $o'$ is the same in $\mathit{lin}$ and in $\mathit{lin}'$. This implies that $o$ is still before $o'$ in $\mathit{lin}'$, which contradicts the fact that $(o,o') \in \mathit{diff}(\mathit{lin},\mathit{lin}')$.

    Since $o$ and $o'$ are adjacent and $(o,o') \in \mathit{diff}(\mathit{lin},\mathit{lin}')$, by assumption we know that $o$ and $o'$ are not concurrent. Or we can say, $(o,o') \in \mathit{vis} \vee \mathit{o',o} \in \mathit{vis}$. This contradicts that both $\mathit{lin}$ and $\mathit{lin}'$ are consistent with visibility relation. This completes the proof of the first step.

\item[-] The second property is: Given a linearization $\mathit{lin}$ and $o_1,o_2 \in \mathit{lin}$, such that $o_1$ and $o_2$ are concurrent and adjacent in $\mathit{lin}$, then, $l = \mathit{swap}(\mathit{lin},o_1,o_2)$ is also a linearization.

    Let $o_1 = (\ell_1,\mathit{id}_1,S_1)$ and $o_2 = (\ell_2,\mathit{id}_2,S_2)$. Since $o_1$ and $o_2$ are concurrent, we know that $\mathit{id}_1 \notin S_2 \wedge \mathit{id}_2 \notin S_1$. Assume $\mathit{lin} = l_1 \cdot o_1 \cdot o_2 \cdot l_2$. Assume in the abstract state of $\mathit{OR}$-$\mathit{set}_s$, we have $\sigma_0 {\xrightarrow{l_1}} \sigma_1 {\xrightarrow{o_1}} \sigma_2 {\xrightarrow{o_2}} \sigma_3 {\xrightarrow{l_2}} \sigma_4$, where $\sigma_0$ is the initial state of $\mathit{OR}$-$\mathit{set}_s$. Then, we need to prove that, there exists $\sigma'_2$, such that $\sigma_1 {\xrightarrow{o_2}} \sigma'_2 {\xrightarrow{o_1}} \sigma_3$. We prove this by consider all the possible cases:

    \begin{itemize}
    \setlength{\itemsep}{0.5pt}
    \item[-] If $o_1 = (\mathit{add}(a_1),\mathit{id}_1,S_1)$ and $o_2 = (\mathit{add}(a_2),\mathit{id}_2,S_2)$: We can see that $\sigma_2$ is obtained from $\sigma_1$ by inserting $(a_1,\mathit{id}_1,\mathit{true})$, and $\sigma_3$ is obtained from $\sigma_2$ by inserting $(a_2,\mathit{id}_2,\mathit{true})$. Let $\sigma'_2$ be obtained from $\sigma_1$ by inserting $(a_2,\mathit{id}_2,\mathit{true})$. Then, it is easy to see that $\sigma_1 {\xrightarrow{o_2}} \sigma'_2 {\xrightarrow{o_1}} \sigma_3$.

    \item[-] If $o_1 = (\mathit{add}(a_1),\mathit{id}_1,S_1)$ and $o_2 = (\mathit{rem}(a_2),\mathit{id}_2,S_2)$: We can see that $\sigma_2$ is obtained from $\sigma_1$ by inserting $(a_1,\mathit{id}_1,\mathit{true})$, and $\sigma_3$ is obtained from $\sigma_2$ by marking $a_2$ with identifiers of $S_2$ into $\mathit{false}$. Let $\sigma'_2$ be obtained from $\sigma_1$ by marking $a_2$ with identifiers of $S_2$ into $\mathit{false}$. Since $\mathit{id_1} \notin S_2$, we can see that $\sigma_1 {\xrightarrow{o_2}} \sigma'_2 {\xrightarrow{o_1}} \sigma_3$.

    \item[-] If $o_1 = (\mathit{add}(a_1),\mathit{id}_1,S_1)$ and $o_2 = (\mathit{read}() \Rightarrow l_2,\mathit{id}_2,S_2)$: Let $\sigma'_2 = \sigma_1$. Since $\mathit{id}_1 \notin S_2$, it is easy to see that $\sigma_1 {\xrightarrow{o_2}} \sigma'_2 {\xrightarrow{o_1}} \sigma_3$.

    \item[-] If $o_1 = (\mathit{rem}(a_1),\mathit{id}_1,S_1)$ and $o_2 = (\mathit{add}(a_2),\mathit{id}_2,S_2)$: We can see that $\sigma_2$ is obtained from $\sigma_1$ by marking $a_1$ with identifiers of $S_1$ into $\mathit{false}$, and $\sigma_3$ is obtained from $\sigma_2$ by inserting $(a_2,\mathit{id}_2,\mathit{true})$. Let $\sigma'_2$ be obtained from $\sigma_1$ by inserting $(a_2,\mathit{id}_2,\mathit{true})$. Since $\mathit{id}_2 \notin S_1$, we can see that $\sigma_1 {\xrightarrow{o_2}} \sigma'_2 {\xrightarrow{o_1}} \sigma_3$.

    \item[-] If $o_1 = (\mathit{rem}(a_1),\mathit{id}_1,S_1)$ and $o_2 = (\mathit{rem}(a_2),\mathit{id}_2,S_2)$: We can see that $\sigma_2$ is obtained from $\sigma_1$ by marking $a_1$ with identifiers of $S_1$ into $\mathit{false}$, and $\sigma_3$ is obtained from $\sigma_2$ by marking $a_2$ with identifiers of $S_2$ into $\mathit{false}$. Let $\sigma'_2$ be obtained from $\sigma_1$ by marking $a_2$ with identifiers of $S_2$ into $\mathit{false}$. Then, it is easy to see that $\sigma_1 {\xrightarrow{o_2}} \sigma'_2 {\xrightarrow{o_1}} \sigma_3$.

    \item[-] If $o_1 = (\mathit{rem}(a_1),\mathit{id}_1,S_1)$ and $o_2 = (\mathit{read}() \Rightarrow l_2,\mathit{id}_2,S_2)$: Let $\sigma'_2 = \sigma_1$. Since $\mathit{id}_1 \notin S_2$, it is easy to see that $\sigma_1 {\xrightarrow{o_2}} \sigma'_2 {\xrightarrow{o_1}} \sigma_3$.

    \item[-] If $o_1 = (\mathit{read}() \Rightarrow l_1,\mathit{id}_1,S_1)$ and $o_2 = (\mathit{add}(a_1),\mathit{id}_2,S_2)$: Let $\sigma'_2$ be obtained from $\sigma_1$ by inserting $(a_1,\mathit{id}_1,\mathit{true})$. Since $\mathit{id}_2 \notin S_1$, it is easy to see that $\sigma_1 {\xrightarrow{o_2}} \sigma'_2 {\xrightarrow{o_1}} \sigma_3$.

    \item[-] If $o_1 = (\mathit{read}() \Rightarrow l_1,\mathit{id}_1,S_1)$ and $o_2 = (\mathit{rem}(a_1),\mathit{id}_2,S_2)$: Let $\sigma'_2$ be obtained from $\sigma_1$ by marking $a_2$ with identifiers of $S_2$ into $\mathit{false}$. Since $\mathit{id}_2 \notin S_1$, it is easy to see that $\sigma_1 {\xrightarrow{o_2}} \sigma'_2 {\xrightarrow{o_1}} \sigma_3$.

    \item[-] If $o_1 = (\mathit{read}() \Rightarrow l_1,\mathit{id}_1,S_1)$ and $o_2 = (\mathit{read}() \Rightarrow l_2,\mathit{id}_2,S_2)$: Let $\sigma'_2 = \sigma_1$. Then, it is easy to see that $\sigma_1 {\xrightarrow{o_2}} \sigma'_2 {\xrightarrow{o_1}} \sigma_3$.
    \end{itemize}
\end{itemize}

Based on these two steps, given a linearization $\mathit{lin}$ and a sequence $\mathit{lin}' \neq \mathit{lin}$ which is consistent with visibility relation: We have $\mathit{diff}(\mathit{lin},\mathit{lin}') \neq \emptyset$. Based on the first above property, there exists $(o_1,o_2) \in \mathit{diff}(\mathit{lin},\mathit{lin}')$, such that $o_1$ and $o_2$ are concurrent, and $o_1$ and $o_2$ are adjacent in $\mathit{lin}$. Based on the second above property, $\mathit{lin}'' = \mathit{swap}(\mathit{lin},o_1,o_2)$ is also a linearization. Moreover, it is easy to see that $\mathit{diff}(\mathit{lin},\mathit{lin}') > \mathit{diff}(\mathit{lin}'',\mathit{lin}')$. Therefore, by several times of above process, we finally obtain $\mathit{lin}'$ from $\mathit{lin}$ by swapping pairs of operations, and prove that $\mathit{lin}'$ is also a linearization. This completes the proof of this lemma. $\qed$
\end {proof}



The following lemma states that $\mathit{set}_s$ is a t0-specification.

\begin{lemma}
\label{lemma:set is a t0-specification}
$\mathit{set}_s$ is a t0-specification.
\end{lemma}

\begin {proof}

We prove this lemma similarly as that of Lemma \ref{lemma:or-set is a t0-specification}. We need to prove that, given a linearization $\mathit{lin}$ and $o_1,o_2 \in \mathit{lin}$, such that $o_1$ and $o_2$ are concurrent and adjacent in $\mathit{lin}$, then, $l = \mathit{swap}(\mathit{lin},o_1,o_2)$ is also a linearization.

Let $o_1 = (\ell_1,\mathit{id}_1,S_1)$ and $o_2 = (\ell_2,\mathit{id}_2,S_2)$. Since $o_1$ and $o_2$ are concurrent, we know that $\mathit{id}_1 \notin S_2 \wedge \mathit{id}_2 \notin S_1$. Assume $\mathit{lin} = l_1 \cdot o_1 \cdot o_2 \cdot l_2$. Assume in the abstract state of $\mathit{set}_s$, we have $\sigma_0 {\xrightarrow{l_1}} \sigma_1 {\xrightarrow{o_1}} \sigma_2 {\xrightarrow{o_2}} \sigma_3 {\xrightarrow{l_2}} \sigma_4$, where $\sigma_0$ is the initial state of $\mathit{set}_s$. Then, we need to prove that, there exists $\sigma'_2$, such that $\sigma_1 {\xrightarrow{o_2}} \sigma'_2 {\xrightarrow{o_1}} \sigma_3$. We prove this by consider all the possible cases:

\begin{itemize}
\setlength{\itemsep}{0.5pt}
\item[-] If $o_1 = (\mathit{add}(a_1),\mathit{id}_1,S_1)$ and $o_2 = (\mathit{add}(a_2),\mathit{id}_2,S_2)$: We can see that, if $(a_1,\_) \in \sigma_1$, then $\sigma_2 = \sigma_1$; else, $\sigma_2$ is obtained from $\sigma_1$ by inserting $(a_1,\mathit{true})$. We can also see that, if $(a_2,\_) \in \sigma_2$, then $\sigma_3 = \sigma_2$; else, $\sigma_3$ is obtained from $\sigma_2$ by inserting $(a_2,\mathit{true})$. Let $\sigma'_2$ be: if $(a_2,\_) \in \sigma_1$, then $\sigma'_2 = \sigma_1$; else, $\sigma'_2$ is obtained from $\sigma_1$ by inserting $(a_2,\mathit{true})$. Then, it is easy to see that $\sigma_1 {\xrightarrow{o_2}} \sigma'_2 {\xrightarrow{o_1}} \sigma_3$.

\item[-] If $o_1 = (\mathit{add}(a_1),\mathit{id}_1,S_1)$ and $o_2 = (\mathit{rem}(a_2),\mathit{id}_2,S_2)$: Let $\sigma'_2$ be: if $(a_2,\mathit{false}) \in \sigma_1$, then $\sigma'_2 = \sigma_1$; else, $\sigma'_2$ is obtained from $\sigma_1$ by marking $a_2$ into $\mathit{false}$. Since $\mathit{vis}^{-1}(o_2) \cdot o_2 \in \mathit{set}_s$, we know that $(a_2,\_) \in \sigma_1$. Then, it is easy to see that $\sigma_1 {\xrightarrow{o_2}} \sigma'_2 {\xrightarrow{o_1}} \sigma_3$.

\item[-] If $o_1 = (\mathit{add}(a_1),\mathit{id}_1,S_1)$ and $o_2 = (\mathit{read}() \Rightarrow l_2,\mathit{id}_2,S_2)$: Let $\sigma'_2 = \sigma_1$. Since $\mathit{id}_1 \notin S_2$, it is easy to see that $\sigma_1 {\xrightarrow{o_2}} \sigma'_2 {\xrightarrow{o_1}} \sigma_3$.

\item[-] If $o_1 = (\mathit{rem}(a_1),\mathit{id}_1,S_1)$ and $o_2 = (\mathit{add}(a_2),\mathit{id}_2,S_2)$: Let $\sigma'_2$ be: if $(a_2,\_) \in \sigma_1$, then $\sigma'_2 = \sigma_1$; else, $\sigma'_2$ is obtained from $\sigma_1$ by inserting $(a_2,\mathit{true})$. Since $\mathit{vis}^{-1}(o_1) \cdot o_1 \in \mathit{set}_s$, we know that $(a_1,\_) \in \sigma_1$. Then, it is easy to see that $\sigma_1 {\xrightarrow{o_2}} \sigma'_2 {\xrightarrow{o_1}} \sigma_3$.

\item[-] If $o_1 = (\mathit{rem}(a_1),\mathit{id}_1,S_1)$ and $o_2 = (\mathit{rem}(a_2),\mathit{id}_2,S_2)$: Let $\sigma'_2$ be: if $(a_2,\mathit{false}) \in \sigma_1$, then $\sigma'_2 = \sigma_1$; else, $\sigma'_2$ is obtained from $\sigma_1$ by marking $a_2$ into $\mathit{false}$. Then, it is easy to see that $\sigma_1 {\xrightarrow{o_2}} \sigma'_2 {\xrightarrow{o_1}} \sigma_3$.

\item[-] If $o_1 = (\mathit{rem}(a_1),\mathit{id}_1,S_1)$ and $o_2 = (\mathit{read}() \Rightarrow l_2,\mathit{id}_2,S_2)$: Let $\sigma'_2 = \sigma_1$. Since $\mathit{id}_1 \notin S_2$, it is easy to see that $\sigma_1 {\xrightarrow{o_2}} \sigma'_2 {\xrightarrow{o_1}} \sigma_3$.

\item[-] If $o_1 = (\mathit{read}() \Rightarrow l_1,\mathit{id}_1,S_1)$ and $o_2 = (\mathit{add}(a_1),\mathit{id}_2,S_2)$: Let $\sigma'_2$ be: if $(a_2,\_) \in \sigma_1$, then $\sigma'_2 = \sigma_1$; else, $\sigma'_2$ is obtained from $\sigma_1$ by inserting $(a_2,\mathit{true})$. Since $\mathit{id}_2 \notin S_1$, it is easy to see that $\sigma_1 {\xrightarrow{o_2}} \sigma'_2 {\xrightarrow{o_1}} \sigma_3$.

\item[-] If $o_1 = (\mathit{read}() \Rightarrow l_1,\mathit{id}_1,S_1)$ and $o_2 = (\mathit{rem}(a_1),\mathit{id}_2,S_2)$: Let $\sigma'_2$ be: if $(a_2,\mathit{false}) \in \sigma_1$, then $\sigma'_2 = \sigma_1$; else, $\sigma'_2$ is obtained from $\sigma_1$ by marking $a_2$ into $\mathit{false}$. Since $\mathit{id}_2 \notin S_1$, it is easy to see that $\sigma_1 {\xrightarrow{o_2}} \sigma'_2 {\xrightarrow{o_1}} \sigma_3$.

\item[-] If $o_1 = (\mathit{read}() \Rightarrow l_1,\mathit{id}_1,S_1)$ and $o_2 = (\mathit{read}() \Rightarrow l_2,\mathit{id}_2,S_2)$: Let $\sigma'_2 = \sigma_1$. Then, it is easy to see that $\sigma_1 {\xrightarrow{o_2}} \sigma'_2 {\xrightarrow{o_1}} \sigma_3$.
\end{itemize}

This completes the proof of this lemma. $\qed$
\end {proof}




The following lemma states that $\mathit{counter}_s$ is a t0-specification.

\begin{lemma}
\label{lemma:counter is a t0-specification}
$\mathit{counter}_s$ is a t0-specification.
\end{lemma}


\begin {proof}

We prove this lemma similarly as that of Lemma \ref{lemma:or-set is a t0-specification}. We need to prove that, given a linearization $\mathit{lin}$ and $o_1,o_2 \in \mathit{lin}$, such that $o_1$ and $o_2$ are concurrent and adjacent in $\mathit{lin}$, then, $l = \mathit{swap}(\mathit{lin},o_1,o_2)$ is also a linearization.

Let $o_1 = (\ell_1,\mathit{id}_1,S_1)$ and $o_2 = (\ell_2,\mathit{id}_2,S_2)$. Since $o_1$ and $o_2$ are concurrent, we know that $\mathit{id}_1 \notin S_2 \wedge \mathit{id}_2 \notin S_1$. Assume $\mathit{lin} = l_1 \cdot o_1 \cdot o_2 \cdot l_2$. Assume in the abstract state of $\mathit{counter}_s$, we have $\sigma_0 {\xrightarrow{l_1}} \sigma_1 {\xrightarrow{o_1}} \sigma_2 {\xrightarrow{o_2}} \sigma_3 {\xrightarrow{l_2}} \sigma_4$, where $\sigma_0$ is the initial state of $\mathit{counter}_s$. Then, we need to prove that, there exists $\sigma'_2$, such that $\sigma_1 {\xrightarrow{o_2}} \sigma'_2 {\xrightarrow{o_1}} \sigma_3$. We prove this by consider all the possible cases:

\begin{itemize}
\setlength{\itemsep}{0.5pt}
\item[-] If $o_1 = (\mathit{inc},\mathit{id}_1,S_1)$ and $o_2 = (\mathit{inc},\mathit{id}_2,S_2)$: Assume that $\sigma_1 = k$, then $\sigma_2 = \mathit{k+1}$ and $\sigma_3 = \mathit{k+2}$. Let $\sigma'_2 = \mathit{k+1}$. Then, it is easy to see that $\sigma_1 {\xrightarrow{o_2}} \sigma'_2 {\xrightarrow{o_1}} \sigma_3$.

\item[-] If $o_1 = (\mathit{inc},\mathit{id}_1,S_1)$ and $o_2 = (\mathit{dec},\mathit{id}_2,S_2)$: Assume that $\sigma_1 = k$, and let $\sigma'_2 = \mathit{k-1}$. Then, it is easy to see that $\sigma_1 {\xrightarrow{o_2}} \sigma'_2 {\xrightarrow{o_1}} \sigma_3$.

\item[-] If $o_1 = (\mathit{inc},\mathit{id}_1,S_1)$ and $o_2 = (\mathit{read}() \Rightarrow k_2,\mathit{id}_2,S_2)$: Let $\sigma'_2 = \sigma_1$. Since $\mathit{id}_1 \notin S_2$, it is easy to see that $\sigma_1 {\xrightarrow{o_2}} \sigma'_2 {\xrightarrow{o_1}} \sigma_3$.

\item[-] If $o_1 = (\mathit{dec},\mathit{id}_1,S_1)$ and $o_2 = (\mathit{inc},\mathit{id}_2,S_2)$: Assume that $\sigma_1 = k$, and let $\sigma'_2 = \mathit{k+1}$. Then, it is easy to see that $\sigma_1 {\xrightarrow{o_2}} \sigma'_2 {\xrightarrow{o_1}} \sigma_3$.

\item[-] If $o_1 = (\mathit{dec},\mathit{id}_1,S_1)$ and $o_2 = (\mathit{dec},\mathit{id}_2,S_2)$: Assume that $\sigma_1 = k$, and let $\sigma'_2 = \mathit{k-1}$. Then, it is easy to see that $\sigma_1 {\xrightarrow{o_2}} \sigma'_2 {\xrightarrow{o_1}} \sigma_3$.

\item[-] If $o_1 = (\mathit{dec},\mathit{id}_1,S_1)$ and $o_2 = (\mathit{read}() \Rightarrow k_2,\mathit{id}_2,S_2)$: Let $\sigma'_2 = \sigma_1$. Since $\mathit{id}_1 \notin S_2$, it is easy to see that $\sigma_1 {\xrightarrow{o_2}} \sigma'_2 {\xrightarrow{o_1}} \sigma_3$.

\item[-] If $o_1 = (\mathit{read}() \Rightarrow k_1,\mathit{id}_1,S_1)$ and $o_2 = (\mathit{inc},\mathit{id}_2,S_2)$: Assume that $\sigma_1 = k$, and let $\sigma'_2 = \mathit{k+1}$. Since $\mathit{id}_2 \notin S_1$, it is easy to see that $\sigma_1 {\xrightarrow{o_2}} \sigma'_2 {\xrightarrow{o_1}} \sigma_3$.

\item[-] If $o_1 = (\mathit{read}() \Rightarrow k_1,\mathit{id}_1,S_1)$ and $o_2 = (\mathit{dec},\mathit{id}_2,S_2)$: Assume that $\sigma_1 = k$, and let $\sigma'_2 = \mathit{k-1}$. Since $\mathit{id}_2 \notin S_1$, it is easy to see that $\sigma_1 {\xrightarrow{o_2}} \sigma'_2 {\xrightarrow{o_1}} \sigma_3$.

\item[-] If $o_1 = (\mathit{read}() \Rightarrow k_1,\mathit{id}_1,S_1)$ and $o_2 = (\mathit{read}() \Rightarrow k_2,\mathit{id}_2,S_2)$: Let $\sigma'_2 = \sigma_1$. Then, it is easy to see that $\sigma_1 {\xrightarrow{o_2}} \sigma'_2 {\xrightarrow{o_1}} \sigma_3$.
\end{itemize}

This completes the proof of this lemma. $\qed$
\end {proof}


With Lemma \ref{lemma:or-set is a t0-specification}, Lemma \ref{lemma:set is a t0-specification} and Lemma \ref{lemma:counter is a t0-specification}, we can now prove Lemma \ref{lemma:several t0-specifications}.


\SeveralTZeroSpecifications*

\begin {proof}
This lemma holds obviously from Lemma \ref{lemma:or-set is a t0-specification}, Lemma \ref{lemma:set is a t0-specification} and Lemma \ref{lemma:counter is a t0-specification}. $\qed$
\end {proof}





\subsection{Proofs of Lemma \ref{lemma:several t1-specifications}}
\label{subsec:appendix proofs of Lemma several t1-specifications}


The following lemma states that $\mathit{list}_s^{\mathit{af}}$ is a t1-specification.

\begin{lemma}
\label{lemma:list-af is a t1-specification}
$\mathit{list}_s^{\mathit{af}}$ is a t1-specification.
\end{lemma}

\begin {proof}

Given a distributed linearizable history $h$ and a linearization $\mathit{lin}$ that is a strict time-stamp order candidate, we need to prove that, each strict time-stamp order candidate $\mathit{lin}'$ is a linearization.

We prove this by showing that each such $\mathit{lin}'$ can be obtained from $\mathit{lin}$ by several times of swapping a pair of adjacent elements. Our proof requires the following two properties:

\begin{itemize}
\setlength{\itemsep}{0.5pt}
\item[-] The first property is: Given a linarization $\mathit{lin}$ that is a strict time-stamp order candidate, and a strict time-stamp order candidate $\mathit{lin}'$. If $\mathit{diff}(\mathit{lin},\mathit{lin}') \neq \emptyset$, there exists $(o_1,o_2) \in \mathit{diff}(\mathit{lin},\mathit{lin}')$, such that $o_1$ and $o_2$ are concurrent, $o_1$ and $o_2$ are adjacent in $\mathit{lin}$, and the time-stamp of $o_1$ in $h$ equals that of $o_2$.

    We prove this by contradiction. Assume $\mathit{diff}(\mathit{lin},\mathit{lin}') \neq \emptyset$, and for each $(o_1,o_2) \in \mathit{diff}(\mathit{lin},\mathit{lin}')$, we have that either $o_1$ and $o_2$ are not concurrent, or $o_1$ and $o_2$ are not adjacent in $\mathit{lin}$, or the time-stamp of $o_1$ in $h$ is different from that of $o_2$.

    By the definition of strict time-stamp order candidate, it is easy to see that if $o_1$ and $o_2$ have different time-stamp, then their order is the same between $\mathit{lin}$ and $\mathit{lin}'$. Therefore, we know that the time-stamp of $o_1$ in $h$ equals that of $o_2$.

    Since $\mathit{diff}(\mathit{lin},\mathit{lin}') \neq \emptyset$, let $(o,o')$ be a element of $\mathit{diff}(\mathit{lin},\mathit{lin}')$, and the distance of $o_1$ and $o_2$ is minimal in $\{$ the distance between $o_1$ and $o_2 \vert (o_1,o_2) \in \mathit{diff}(\mathit{lin},\mathit{lin}') \}$. Let us prove that $o$ and $o'$ are adjacent by contradiction: If there exists $o''$ between $o$ and $o'$. Assume that in $\mathit{lin}$, $o$ is before $o''$, and $o''$ is before $o'$. By assumption, the order between $o$ and $o''$, and between $o''$ and $o'$ is the same in $\mathit{lin}$ and in $\mathit{lin}'$. This implies that $o$ is still before $o'$ in $\mathit{lin}'$, which contradicts the fact that $(o,o') \in \mathit{diff}(\mathit{lin},\mathit{lin}')$.

    Since $o$ and $o'$ are adjacent and $(o,o') \in \mathit{diff}(\mathit{lin},\mathit{lin}')$, by assumption we know that $o$ and $o'$ are not concurrent. Or we can say, $(o,o') \in \mathit{vis} \vee \mathit{o',o} \in \mathit{vis}$. This contradicts that both $\mathit{lin}$ and $\mathit{lin}'$ are consistent with visibility relation. This completes the proof of the first step.

\item[-] The second property is: Given a linearization $\mathit{lin}$ that is a strict time-stamp order candidate, and $o_1,o_2 \in \mathit{lin}$, such that $o_1$ and $o_2$ are concurrent and adjacent in $\mathit{lin}$, and $o_1$ and $o_2$ have the same time-stamp in $h$. Then, $l = \mathit{swap}(\mathit{lin},o_1,o_2)$ is also a linearization and is also a strict time-stamp order candidate. It is obvious that $l$ is still a strict time-stamp order candidate.

    Let $o_1 = (\ell_1,\mathit{id}_1,S_1)$ and $o_2 = (\ell_2,\mathit{id}_2,S_2)$. Since $o_1$ and $o_2$ are concurrent, we know that $\mathit{id}_1 \notin S_2 \wedge \mathit{id}_2 \notin S_1$. Assume $\mathit{lin} = l_1 \cdot o_1 \cdot o_2 \cdot l_2$. Assume in the abstract state of $\mathit{list}_s^{\mathit{af}}$, we have $\sigma_0 {\xrightarrow{l_1}} \sigma_1 {\xrightarrow{o_1}} \sigma_2 {\xrightarrow{o_2}} \sigma_3 {\xrightarrow{l_2}} \sigma_4$, where $\sigma_0$ is the initial state of $\mathit{OR}$-$\mathit{set}_s$. Then, we need to prove that, there exists $\sigma'_2$, such that $\sigma_1 {\xrightarrow{o_2}} \sigma'_2 {\xrightarrow{o_1}} \sigma_3$. We prove this by consider all the possible cases:

    \begin{itemize}
    \setlength{\itemsep}{0.5pt}
    \item[-] If $o_1 = (\mathit{add}(a_1,b_1),\mathit{id}_1,S_1)$ and $o_2 = (\_,\mathit{id}_2,S_2)$: This case is impossible. We can see that the time-stamp of $a$ is larger than operations in $S_1$, and thus, the time-stamp of $o_1$ is the time-stamp of $a$. Since $\mathit{id}_1 \notin S_2$, we know that the time-stamp of $o_2$ is different from that of $o_1$, contradicts the assumption that $o_1$ and $o_2$ have same time-stamp.

    \item[-] If $o_1 = (\_,\mathit{id}_1,S_1)$ and $o_2 = (\mathit{add}(a_2,b_2),\mathit{id}_2,S_2)$: Similarly, we can prove that this case is impossible.

    \item[-] If $o_1 = (\mathit{rem}(a_1),\mathit{id}_1,S_1)$ and $o_2 = (\mathit{rem}(a_2),\mathit{id}_2,S_2)$: Let $\sigma'_2$ be obtained from $\sigma_1$ by marking $a_2$ into $\mathit{false}$. Then, it is easy to see that $\sigma_1 {\xrightarrow{o_2}} \sigma'_2 {\xrightarrow{o_1}} \sigma_3$.

    \item[-] If $o_1 = (\mathit{rem}(a_1),\mathit{id}_1,S_1)$ and $o_2 = (\mathit{read}() \Rightarrow \mathit{list}_1,\mathit{id}_2,S_2)$: Let $\sigma'_2 = \sigma_1$. Since $\mathit{id}_1 \notin S_2$, it is easy to see that $\sigma_1 {\xrightarrow{o_2}} \sigma'_2 {\xrightarrow{o_1}} \sigma_3$.

    \item[-] If $o_1 = (\mathit{read}() \Rightarrow \mathit{list}_1,\mathit{id}_1,S_1)$ and $o_2 = (\mathit{read}() \Rightarrow \mathit{list}_2,\mathit{id}_2,S_2)$: Let $\sigma'_2 = \sigma_1$. Then, it is easy to see that $\sigma_1 {\xrightarrow{o_2}} \sigma'_2 {\xrightarrow{o_1}} \sigma_3$.
    \end{itemize}
\end{itemize}

Based on these two steps, given a linearization $\mathit{lin}$ that is a strict time-stamp order candidate, and a sequence $\mathit{lin}' \neq \mathit{lin}$ that is a strict time-stamp order candidate: We have $\mathit{diff}(\mathit{lin},\mathit{lin}') \neq \emptyset$. Based on the first above property, there exists $(o_1,o_2) \in \mathit{diff}(\mathit{lin},\mathit{lin}')$, such that $o_1$ and $o_2$ are concurrent, and $o_1$ and $o_2$ are adjacent in $\mathit{lin}$, and $o_1$ and $o_2$ have a same time-stamp. Based on the second above property, $\mathit{lin}'' = \mathit{swap}(\mathit{lin},o_1,o_2)$ is also a linearization, and is a strict time-stamp order candidate. Moreover, it is easy to see that $\mathit{diff}(\mathit{lin},\mathit{lin}') > \mathit{diff}(\mathit{lin}'',\mathit{lin}')$. Therefore, by several times of above process, we finally obtain $\mathit{lin}'$ from $\mathit{lin}$ by swapping pairs of operations, and prove that $\mathit{lin}'$ is also a linearization, and is a strict time-stamp order candidate. This completes the proof of this lemma. $\qed$
\end {proof}


The following lemma states that $\mathit{reg}_s$ is a t1-specification.

\begin{lemma}
\label{lemma:reg is a t1-specification}
$\mathit{reg}_s$ is a t1-specification.
\end{lemma}

\begin {proof}

We prove this lemma similarly as that of Lemma \ref{lemma:list-af is a t1-specification}. We need to prove that, given a linearization $\mathit{lin}$ that is a strict time-stamp order candidate, and $o_1,o_2 \in \mathit{lin}$, such that $o_1$ and $o_2$ are concurrent and adjacent in $\mathit{lin}$, and $o_1$ and $o_2$ have the same time-stamp in $h$. Then, $l = \mathit{swap}(\mathit{lin},o_1,o_2)$ is also a linearization and is also a strict time-stamp order candidate. It is obvious that $l$ is still a strict time-stamp order candidate.

Let $o_1 = (\ell_1,\mathit{id}_1,S_1)$ and $o_2 = (\ell_2,\mathit{id}_2,S_2)$. Since $o_1$ and $o_2$ are concurrent, we know that $\mathit{id}_1 \notin S_2 \wedge \mathit{id}_2 \notin S_1$. Assume $\mathit{lin} = l_1 \cdot o_1 \cdot o_2 \cdot l_2$. Assume in the abstract state of $\mathit{reg}_s$, we have $\sigma_0 {\xrightarrow{l_1}} \sigma_1 {\xrightarrow{o_1}} \sigma_2 {\xrightarrow{o_2}} \sigma_3 {\xrightarrow{l_2}} \sigma_4$, where $\sigma_0$ is the initial state of $\mathit{OR}$-$\mathit{set}_s$. Then, we need to prove that, there exists $\sigma'_2$, such that $\sigma_1 {\xrightarrow{o_2}} \sigma'_2 {\xrightarrow{o_1}} \sigma_3$. We prove this by consider all the possible cases:


\begin{itemize}
\setlength{\itemsep}{0.5pt}
\item[-] If $o_1 = (\mathit{write}(a_1),\mathit{id}_1,S_1)$ and $o_2 = (\_,\mathit{id}_2,S_2)$: This case is impossible. We can see that the time-stamp of $a$ is larger than operations in $S_1$, and thus, the time-stamp of $o_1$ is the time-stamp of $a$. Since $\mathit{id}_1 \notin S_2$, we know that the time-stamp of $o_2$ is different from that of $o_1$, contradicts the assumption that $o_1$ and $o_2$ have same time-stamp.

\item[-] If $o_1 = (\_,\mathit{id}_1,S_1)$ and $o_2 = (\mathit{write}(a_2),\mathit{id}_2,S_2)$: Similarly, we can prove that this case is impossible.

\item[-] If $o_1 = (\mathit{read}() \Rightarrow a_1,\mathit{id}_1,S_1)$ and $o_2 = (\mathit{read}() \Rightarrow a_2,\mathit{id}_2,S_2)$: Let $\sigma'_2 = \sigma_1$. Then, it is easy to see that $\sigma_1 {\xrightarrow{o_2}} \sigma'_2 {\xrightarrow{o_1}} \sigma_3$.
\end{itemize}
This completes the proof of this lemma. $\qed$
\end {proof}


With Lemma \ref{lemma:list-af is a t1-specification} and Lemma \ref{lemma:reg is a t1-specification}, we can now prove Lemma \ref{lemma:several t1-specifications}.

\SeveralTOneSpecifications*

\begin {proof}
This lemma holds obviously from Lemma \ref{lemma:list-af is a t1-specification} and Lemma \ref{lemma:reg is a t1-specification}. $\qed$
\end {proof}









\subsection{Proof of Lemma \ref{lemma:several t0-specifications can be composed}}
\label{subsec:appendix proofs of lemma several t0-specifications can be composed}

\composingTZero*
\begin {proof}
Assume that $h = (\mathit{Op},\mathit{ro},\mathit{vis})$. We need to prove that, if $h \uparrow_{\mathit{obj}}$ is distributed linearizable for each object $\mathit{obj}$ of $h$, then $h$ is distributed linearizable.

We construct a linearization $\mathit{lin}$ of $h$ in a process as follows:

\begin{itemize}
\setlength{\itemsep}{0.5pt}
\item[-] Initially a set $\mathit{Op}' = \mathit{Op}$ and $\mathit{lin} = \epsilon$.

\item[-] We begin a loop as follows: In each round of the loop, we choose an operation $o$ that is minimal w.r.t $\mathit{vis}$ in $\mathit{Op}'$, let $\mathit{Op}' = \mathit{Op}' \setminus \{ o \}$, and let $\mathit{lin} = \mathit{lin} \cdot o$.
\end{itemize}

If this process terminates with $\mathit{Op}' = \emptyset$: Then it is easy to see that $\mathit{lin}$ is consistent with $\mathit{vis}$, and thus, for each object $\mathit{obj}$, it is easy to see that $\mathit{lin} \uparrow_{\mathit{obj}}$ is consistent with $\mathit{vis} \uparrow_{\mathit{obj}}$. By the definition of t0-specifications, we know that, for each object $\mathit{obj}$, $\mathit{lin} \uparrow_{\mathit{obj}}$ is a linearization of $h \uparrow_{\mathit{obj}}$. Therefore, $h$ is distributed linearizable.

Let us prove that this process terminates with $\mathit{Op}' = \emptyset$ by contradiction: Assume this process terminates with $\mathit{Op}' \neq \emptyset$, then it is easy to see that $\mathit{vis}^*$ has cycle, which contradicts the assumption that $\mathit{vis}^*$ is acyclic. Therefore, this process terminates with $\mathit{Op}' = \emptyset$. $\qed$
\end {proof}





\subsection{Proof of Lemma \ref{lemma:several t0-specifications and one t1-specification can be composed}}
\label{subsec:appendix proofs of lemma several t0-specifications and one t1-specification can be composed}


\composingTZeroAndOneTOne*
\begin {proof}
Assume that $h = (\mathit{Op},\mathit{ro},\mathit{vis})$. Let $\mathit{obj}_1$ be the only object that uses t1-specification, and let $\mathit{objs}_0$ be the set of other objects. We need to prove that, if $h \uparrow_{\mathit{obj}}$ is distributed linearizable for each object $\mathit{obj}$ of $h$, then $h$ is distributed linearizable.

We construct a linearization $\mathit{lin}$ of $h$ in a process as follows:

\begin{itemize}
\setlength{\itemsep}{0.5pt}
\item[-] Initially a set $\mathit{Op}' = \mathit{Op}$ and $\mathit{lin} = \epsilon$.

\item[-] We begin a loop as follows: in each round of the loop, we choose an operation $o$ shown below, and then let $\mathit{Op}' = \mathit{Op}' \setminus \{ o \}$, and let $\mathit{lin} = \mathit{lin} \cdot o$.

    \begin{itemize}
    \setlength{\itemsep}{0.5pt}
    \item[-] either $o$ is of an operation of $\mathit{objs}_0$ and is minimal w.r.t $\mathit{vis}$ in $\mathit{Op}'$,

    \item[-] or $o$ is of an operation of $\mathit{obj}_1$, is minimal w.r.t $\mathit{vis}$ in $\mathit{Op}'$, and has the minimal time-stamp among operations of $\mathit{obj}_1$ in $\mathit{Op}'$.
    \end{itemize}
\end{itemize}

If this process terminates with $\mathit{Op}' = \emptyset$: Then it is easy to see that $\mathit{lin}$ is consistent with $\mathit{vis}$, and thus, for each object $\mathit{obj}$, it is easy to see that $\mathit{lin} \uparrow_{\mathit{obj}}$ is consistent with $\mathit{vis} \uparrow_{\mathit{obj}}$. It is also easy to see that for operation of $\mathit{obj}_1$, $\mathit{lin}$ is consistent with time-stamp. By the definition of t0-specifications, we know that, for each object $\mathit{obj} \in \mathit{objs}$, $\mathit{lin} \uparrow_{\mathit{obj}}$ is a linearization of $h \uparrow_{\mathit{obj}}$. By the definition of t1-specifications, we know that, $\mathit{lin} \uparrow_{\mathit{obj}_1}$ is a linearization of $h \uparrow_{\mathit{obj}_1}$. Therefore, $h$ is distributed linearizable.

Let us prove that this process terminates with $\mathit{Op}' = \emptyset$ by contradiction: Assume this process terminates with $\mathit{Op}' \neq \emptyset$. Let set $S_1 = \{ o' \vert o'$ is minimal w.r.t $\mathit{vis}$ in $\mathit{Op}'$ $\}$. Then, we can see that, for each operation $o \in S_1$, $o$ is of object $\mathit{obj}_1$, and $o$ does not have minimal time-stamps among operations of $\mathit{obj}_1$ in $\mathit{Op}'$. Let $o_0$ be the operation that is of object $\mathit{obj}_1$ and has minimal time-stamp among operations of $\mathit{obj}_1$ in $\mathit{Op}'$. It is obvious that $o_0 \notin S_1$. Therefore, there exists operations $o_1,\ldots,o_k$, such that $o_1 \in S_1$, $o_1$ is of object $\mathit{obj}_1$, $(o_1,o_2),\ldots,(o_k,o_0) \in \mathit{vis}$. Since the visibility is transitive, we have that $(o_1,o_0) \in \mathit{vis}$. We already know that the time-stamp of $o_0$ is less than that of $o_1$. This contradicts the assumption that time-stamp is consistent with visiblity. Therefore, this process terminates with $\mathit{Op}' = \emptyset$. $\qed$

%Let us prove that this process terminates with $\mathit{Op}' = \emptyset$ by contradiction: Assume this process terminates with $\mathit{Op}' \neq \emptyset$. Let set $S_1 = \{ o' \vert o'$ is minimal w.r.t $\mathit{vis}$ in $\mathit{Op}'$ $\}$. Then, we can see that, for each operation $o \in S_1$, $o$ is of object $\mathit{obj}_1$, and $o$ does not have minimal time-stamps among operations of $\mathit{obj}_1$ in $O'$. Thus, let $o_1 \in S_1$ be the operation that has minimal time-stamp among operations in $S_1$. We can see that there exists $o_2 \in O'$, such that $o_2$ is of object $\mathit{obj}_1$, $o_2 \notin S_1$, and the time-stamp of $o_2$ is less than that of $o_1$. Since $o_2 \notin S_1$, we can see that there exists operation $o_3 \in S_1$, such that $(o_3,o_2) \in \mathit{vis}$. We can see that the time-stamp of $o_1$ is less than that of $o_3$, and thus, the time-stamp of $o_2$ is less than that of $o_3$. Therefore, we found that the time-stamp is not consistent with visibility order for $o_2$ and $o_3$, which contradicts the assumption that time-stamp is consistent with visiblity. $\qed$
\end {proof}






\subsection{Proof of Lemma \ref{lemma:several t0-specifications and several t1-specification can be composed}}
\label{subsec:appendix proofs of lemma several t0-specifications and several t1-specification can be composed}


\composingTZeroAndTOne*
\begin {proof}
Assume that $h = (\mathit{Op},\mathit{ro},\mathit{vis})$. Let $\mathit{objs}_0$ be the set of objects that use t0-specifications in $h$, and let $\mathit{objs}_1$ be the set of objects that use t1-specifications in $h$. We need to prove that, if $h \uparrow_{\mathit{obj}}$ is distributed linearizable for each object $\mathit{obj}$ of $h$, then $h$ is distributed linearizable.

We construct a linearization $\mathit{lin}$ of $h$ in a process as follows:

\begin{itemize}
\setlength{\itemsep}{0.5pt}
\item[-] Initially a set $\mathit{Op}' = \mathit{Op}$ and $\mathit{lin} = \epsilon$.

\item[-] We begin a loop as follows: in each round of the loop, we choose an operation $o$ shown below, and then let $\mathit{Op}' = \mathit{Op}' \setminus \{ o \}$, and let $\mathit{lin} = \mathit{lin} \cdot o$.

    \begin{itemize}
    \setlength{\itemsep}{0.5pt}
    \item[-] either $o$ is of an operation of objects in $\mathit{objs}_0$ and is minimal w.r.t $\mathit{vis}$ in $\mathit{Op}'$,

    \item[-] or $o$ is of an operation of object $\mathit{obj}_1 \in \mathit{objs}_1$, is minimal w.r.t $\mathit{vis}$ in $\mathit{Op}'$, and has the minimal time-stamp among operations of $\mathit{obj}_1$ in $\mathit{Op}'$.
    \end{itemize}
\end{itemize}

If this process terminates with $\mathit{Op}' = \emptyset$: Then it is easy to see that $\mathit{lin}$ is consistent with $\mathit{vis}$, and thus, for each object $\mathit{obj}$, it is easy to see that $\mathit{lin} \uparrow_{\mathit{obj}}$ is consistent with $\mathit{vis} \uparrow_{\mathit{obj}}$. It is also easy to see that for each object $\mathit{ojb}_1 \in \mathit{objs}_1$, $\mathit{lin}$ is consistent with time-stamp of $\mathit{obj}_1$. By the definition of t0-specifications, we know that, for each object $\mathit{obj} \in \mathit{objs}_0$, $\mathit{lin} \uparrow_{\mathit{obj}}$ is a linearization of $h \uparrow_{\mathit{obj}}$. By the definition of t1-specifications, we know that, for each object $\mathit{obj}_1 \in \mathit{objs}_1$, $\mathit{lin} \uparrow_{\mathit{obj}_1}$ is a linearization of $h \uparrow_{\mathit{obj}_1}$. Therefore, $h$ is distributed linearizable.

Let us prove that this process terminates with $\mathit{Op}' = \emptyset$ by contradiction: Assume this process terminates with $\mathit{Op}' \neq \emptyset$. Let set $S_1 = \{ o' \vert o'$ is minimal w.r.t $\mathit{vis}$ in $\mathit{Op}'$ $\}$. Then, we can see that, for each operation $o \in S_1$, there exists a object $\mathit{obj}_1 \in \mathit{objs}_1$, such that $o$ is of $\mathit{obj}_1$, and $o$ does not have minimal time-stamps among operations of $\mathit{obj}_1$ in $\mathit{Op}'$.

Let $S_2 = \{ o \vert \exists \mathit{obj}_1 \in \mathit{objs}_1, o$ is of object $\mathit{obj}_1, o$ has minimal time-stamp among operations of $\mathit{obj}_1$ in $\mathit{Op}' \}$. It is easy to see that $\forall o \in S_2$, $o \notin S_1$.

Thus, it is easy to see that, for each operation $o' \in S_2$, there exists an operation $o \in S_1$ and operations $o'_1,\ldots,o'_k$, such that $(o,o'_1),(o'_1,o'_2),\ldots,(o'_k,o') \in \mathit{vis}$. Since the visibility relation is transitive, we have that $(o,o') \in \mathit{vis}$.

Let $S_3 = \{ (o,o') \vert o \in S_1, o' \in S_2, \exists o'_1,\ldots,o'_k, (o,o'_1),(o'_1,o'_2),\ldots,(o'_k,o') \in \mathit{vis} \}$. Let $S_4 = \{ (\mathit{obj},\mathit{obj}') \vert \exists (o,o') \in S_3$, $o$ is of object $\mathit{obj}$, $o'$ is of object $\mathit{obj}' \}$.

Let us prove that there is a cycle in $S_4$ by contradiction. Given $(\mathit{obj}_2,\mathit{obj}_1) \in S_4$, we know that there is a operation of object of $\mathit{obj}_2$ in $S_1$, and thus, there must exists a operation of object of $\mathit{obj}_2$ in $S_2$. By definition of $S_2$, it is easy to see that there exists $\mathit{obj}_3$, such that $(\mathit{obj}_3,\mathit{obj}_2) \in S_4$. Since $S_4$ has no cycle, we applying this process and finally terminate with $(\mathit{obj}_k,\mathit{obj}_{\mathit{k-1}}),\ldots,(\mathit{obj}_2,\mathit{obj}_1) \in S_4$ and could not found any $\mathit{obj}'$ to make $(\mathit{obj}',\mathit{obk}_k) \in S_4$. However, this implies that there is a operation of $\mathit{obj}_k$ that has minimal time-stamp among operations of $\mathit{obj}_k$ in $\mathit{Op}'$, and is in $S_1$. This contradicts our conclusion that $\forall o \in S_2$, $o \notin S_1$. Therefore, this is a cycle in $S_4$.

Let the cycle in $S_4$ be $(\mathit{obj}_1,\mathit{obj}_k),(\mathit{obj}_k,\mathit{obj}_{\mathit{k-1}}),\ldots,(\mathit{obj}_2,\mathit{obj}_1)$. Then, there exists operations $o^{0}_{\mathit{o1}}, o^{1}_{\mathit{o1}},\ldots, o^{0}_{\mathit{ok}}, o^{1}_{\mathit{ok}}$, such that

\begin{itemize}
\setlength{\itemsep}{0.5pt}
\item[-] $o^{0}_{\mathit{o1}}, o^{1}_{\mathit{o1}}$ is of object $\mathit{obj}_1$, $\ldots$, $o^{0}_{\mathit{ok}}, o^{1}_{\mathit{ok}}$ is of object $\mathit{obj}_k$.

\item[-] $(o^{1}_{\mathit{o1}},o^{0}_{\mathit{ok}}), (o^{1}_{\mathit{ok}},o^{0}_{\mathit{ok-1}})$, $\ldots$, $(o^{1}_{\mathit{o2}},o^{0}_{\mathit{o1}}) \in S_3$.
\end{itemize}

Thus, it is easy to see $(o^{1}_{\mathit{o1}},o^{0}_{\mathit{ok}}), (o^{1}_{\mathit{ok}},$ $o^{0}_{\mathit{ok-1}})$, $\ldots$, $(o^{1}_{\mathit{o2}},o^{0}_{\mathit{o1}}) \in \mathit{vis}$. By definition of $S_2$, we can see that $\mathit{ts}(o^{0}_{\mathit{o1}}) < \mathit{ts}(o^{1}_{\mathit{o1}}), \ldots, \mathit{ts}(o^{0}_{\mathit{ok}}) < \mathit{ts}(o^{1}_{\mathit{ok}})$. This contradicts the definition of causal-time-stamp. Therefore, this process terminates with $\mathit{Op}' = \emptyset$. $\qed$
\end {proof}










\section{For State-based CRDT}
\label{sec:for state-based CRDT}

\begin{example}[List with add-between interface]
\label{definition:sequential specification of list with add-after interface}
Such kind of list is similar as list with add-after interface. One difference is the $\mathit{add}$ method: $\mathit{add}(b,a,c)$ inserts item $b$ into the list at some nondeterministic position between position of $a$ and position of $c$. The other difference is that, we assume that the initial value of list is $(\circ_1,\mathit{true}) \cdot (\circ_2,\mathit{true})$ and these two nodes can not be removed. The sequential specification $\mathit{list}_s^{\mathit{ab}}$ of list is given as follows: Here $\mathit{ab}$ represents add-between. When the context is clear, in $\mathit{read}$ operation, we will omit $\circ_1$ and $\circ_2$.
\begin{itemize}
\setlength{\itemsep}{0.5pt}
\item[-] $\{ \mathit{state} = (a_1,f_1) \cdot \ldots \cdot (a_n,f_n) \wedge k < m < l \wedge b \notin \{ a_1, \ldots, a_n \} \}$ $add(b,a_k,a_l)$ $\{ \mathit{state} = (a_1,f_1) \cdot \ldots \cdot (a_m,f_m) \cdot (b,\mathit{true}) \cdot (a_{m+1},f_{m+1}) \cdot \ldots \cdot (a_n,f_n) \}$. Here the chosen of $m$ is deterministic.
\item[-] $\{ \mathit{state} = (a_1,f_1) \cdot \ldots \cdot (a_n,f_n) \wedge S = \{ a \vert (a,\mathit{true}) \in \mathit{state} \} \wedge l = a_1 \cdot \ldots \cdot a_n \uparrow_{S} \}$ $(read() \Rightarrow l)$ $\{ \mathit{state} = (a_1,f_1) \cdot \ldots \cdot (a_n,f_n) \}$.
\end{itemize}
\end{example}










Given a object $\mathit{obj}$ of a state-based CRDT with $\Sigma$ be the set of local states, we define its semantics as a set of executions generated from an LTS $\llbracket \mathit{obj} \rrbracket_s = (\mathit{Config},\mathit{config}_0,\Sigma',\rightarrow)$ as in \figurename~\ref{fig:the semantics of a state-based CRDT object}.

\begin{figure}[ht]
$\mathit{RState} = \mathbb{R} \rightarrow \Sigma$

$\mathit{TState} = \mathbb{MID} \times \mathbb{MSG} \times \mathbb{R}$.

$\mathit{Config} = \mathit{RState} \times \mathit{TState}$, $\mathit{config}_0 \in \mathit{Config}$.

$\Sigma' = \mathit{do}(\mathbb{M} \times \mathbb{D} \times \mathbb{D} \times \mathbb{R}) \cup \mathit{send}(\mathbb{MID} \times \mathbb{R}) \cup \mathit{receive}(\mathbb{MID} \times \mathbb{R})$

\[
\begin{array}{l c}
\bigfrac{ R(r) = \sigma, r.\mathit{do}(\sigma,m,a) = (\sigma',b) }
{ (R,T) {\xrightarrow{\mathit{do}(m,a,b,r)}} (R[r:\sigma'],T) }
\end{array}
\]


\[
\begin{array}{l c}
\bigfrac{ R(r) = \sigma, \mathit{unique}(\mathit{mid}) }
{ (R,T) {\xrightarrow{\mathit{send}(\mathit{mid},r)}} (R,T \cup \{ (\mathit{mid},\sigma,r) \}) }
\end{array}
\]


\[
\begin{array}{l c}
\bigfrac{ R(r) = \sigma, r.\mathit{receive}(\sigma,\sigma') = \sigma'',(\mathit{mid},\sigma',r') \in T, r \neq r'}
{ (R,T) {\xrightarrow{\mathit{receive}(\mathit{mid},r)}} (R[r:\sigma''],T) }
\end{array}
\]
\caption{The definition of semantics of $\llbracket \mathit{obj} \rrbracket_s$}
\label{fig:the semantics of a state-based CRDT object}
\end{figure}

A configuration $(R,T)$ is a snapshot of distributed system and contains two parts: $R$ gives the local state of each replica, and $T$ gives the set of messages that has been generated. Let $\mathbb{MID}$ be the set of message identifiers of message content. A message is a tuple $(\mathit{mid},\mathit{msg},r)$, where $\mathit{mid} \in \mathbb{MID}$ is the identifier, $\mathit{msg} \in \mathbb{MSG}$ is the message content, and $r$ is the original replica of message. $\mathit{config}_0$ is the initial configuration, which maps each replica into the initial local state, and has no message inside. Since $\mathit{obj}$ is a state-based CRDT, each message content is chosen from $\Sigma$.

Each element of $\Sigma'$ is called an action. $\rightarrow \in \mathit{Config} \times \Sigma' \times \mathit{Config}$ is the transition relation and describe a single step of distributed systems. The first rule in \figurename~\ref{fig:the semantics of a state-based CRDT object} describes replica $r$ performs a operation $m(a) \Rightarrow b$ and works locally. The second rule describes that a replica $r$ may nondeterministically decide to send a message with its local state as message content. Here $\mathit{unique}$ is a function that ensures $\mathit{mid}$ be a fresh message identifier. The third rule describes delivery of a message to a replica $r$ other than its origin replica $r'$.

A sequence $l$ of actions is an execution of $\llbracket \mathit{obj} \rrbracket_s = (\mathit{Config},\mathit{config}_0,\Sigma',\rightarrow)$, if there exists $(R,T) \in \mathit{Config}$, such that $\mathit{config}_0 {\xrightarrow{ l }} (R,T)$. The semantics of $\mathit{obj}$ is defined as the set of executions of $\llbracket \mathit{obj} \rrbracket_s$. Given an execution, when the context is clear, we can associate a unique operation identifier to each action. Or we can say, it is safe to assume each action is in the form of either $\mathit{do}(i,m,a,b,r)$, or $\mathit{send}(i,\mathit{mid},r)$, or $\mathit{receive}(i,\mathit{mid},r)$, where $i \in \mathbb{OID}$ is a unique operation identifier.








Given an execution $l = \alpha_1 \cdot \ldots \cdot \alpha_k$ of $\llbracket \mathit{obj} \rrbracket_s$ of state-based CRDT $\mathit{obj}$, we can obtain a corresponding history $\mathit{history}(l) = (\mathit{Op},\mathit{ro},\mathit{vis})$, such that

\begin{itemize}
\setlength{\itemsep}{0.5pt}
\item[-] Each operation in $\mathit{Op}$ is a tuple $(\ell,i,\mathit{obj})$, such that $i$ is the operation identifier of a $\mathit{do}(m,a,b,r)$ action of $l$.

\item[-] $(o_1,o_2) \in \mathit{ro}$, if they are of same replica, and the index of $o_1$ in $h$ is before that of $o_2$.

\item[-] Let us defines a delivery relation $\mathit{del} \subseteq \mathbb{OP} \times \mathbb{OP}$ as follows: $(o_1,o_2) \in \mathit{del}$, if: $o_1$ and $o_2$ are of different replicas, there exists a $\mathit{send}(\mathit{mid},r)$ action and a $\mathit{receive}(\mathit{mid},r')$ action, $o_1$ and $\mathit{send}(\mathit{mid},r)$ happen on a same replica and $o_1$ happens earlier, $\mathit{receive}(\mathit{mid},r)$ and $o_2$ happen on a same replica and $\mathit{receive}(\mathit{mid},r)$ happens earlier.

\item[-] $\mathit{vis} = (\mathit{ro}+\mathit{del})^*$.
\end{itemize}

Intuitively, each local state can be considered as the consequence of all updates it receives. Since state-based CRDT sends the modified local state as message, the visibility relation is then the transitive closure of replica order and message delivery relation. Let $\mathit{history}(\llbracket \mathit{obj} \rrbracket_s)$ be the set of histories of all executions of $\llbracket \mathit{obj} \rrbracket_s$.






\subsection{Proof Strategy of State-based CRDT}
\label{subsec:proof strategy of operation-based CRDT}

Given a state-based CRDT object $\mathit{obj}$ and a sequential specification $\mathit{spec}$, we need to construct a invariant $\mathit{inv}(\mathit{config},h,\mathit{lin},\mathit{del},\mathit{map})$, where

\begin{itemize}
\setlength{\itemsep}{0.5pt}
\item[-] $\mathit{config}$ is a configuration of $\llbracket \mathit{obj} \rrbracket_s$.

\item[-] $h$ is a history.

\item[-] $h$ is distributed linearizable w.r.t $\mathit{spec}$ and $\mathit{lin}$ is a linearization.

\item[-] $\mathit{del} \subseteq \mathbb{MID} \times \mathbb{R}$ is the message delivery relation.

\item[-] $\mathit{map} \subseteq \mathbb{MID} \times 2^{\mathbb{OID}}$ maps each message $\mathit{mid}$ to a set $S_1$ of operations. Intuitively, $S_1$ is the set of operations whose information are contained in $\mathit{mid}$.
\end{itemize}

$\mathit{inv}(\mathit{config},h,\mathit{lin},\mathit{del},\mathit{map})$ needs to satisfy the following properties:

\begin{itemize}
\setlength{\itemsep}{0.5pt}
\item[-] The visibility of $h$ is transitive.

\item[-] $\mathit{del}$ preserves causal delivery: If $(o_1,o_2) \in \mathit{vis}$ and $(o_2,r) \in \mathit{del}$, then $(o_1,r) \in \mathit{del}$.

\item[-] $\mathit{map}$ preserves causal delivery: Given $o_1,o_3 \in \mathit{map}(\mathit{mid})$, if $\exists o_2$, such that $(o_1,o_2),(o_2,o_3) \in \mathit{vis}$, then $o_2 \in \mathit{map}(\mathit{mid})$.

\item[-] $\mathit{inv}$ holds initially: $\mathit{inv}(\mathit{config}_0,\epsilon,\emptyset,\emptyset,\emptyset)$ holds, where $\mathit{config}_0$ is the initial configuration of $\llbracket \mathit{obj} \rrbracket_s$.

\item[-] $\mathit{inv}$ is a transition invariant:

    \begin{itemize}
    \setlength{\itemsep}{0.5pt}
    \item[-] If $\mathit{inv}(\mathit{config},h,\mathit{lin},\mathit{del},\mathit{map})$ holds and $\mathit{config} {\xrightarrow{\mathit{do}(m,a,b,r)}} \mathit{config}'$, then $\mathit{inv}(\mathit{config}', h \otimes i, \mathit{lin} \cdot i,\mathit{del},\mathit{map})$ holds. Note that here we always put a new operation in the last of linearization.

        Here $i$ is the identifier of the newly-generated $\mathit{do}$ action. Given $h = (\mathit{Op},\mathit{ro},\mathit{vis})$, then, $h \otimes i = (\mathit{Op}',\mathit{ro}',\mathit{vis}')$, where $\mathit{Op}' = \mathit{Op} \cup \{ (m(a) \Rightarrow b,i,\mathit{obj}) \}$, $\mathit{ro}' = \mathit{ro} \cup \{ (j,i) \vert j \in \mathit{Op}, j$ is of replica $r \}$, and $\mathit{vis}' = (\mathit{vis} \cup \{ (j,i) \vert j \in \mathit{Op},(j,r) \in \mathit{del} \} \cup \{ (j,i) \vert j \in \mathit{Op}, j$ is of replica $r \})^*$.

    \item[-] If $\mathit{inv}(\mathit{config},h,\mathit{lin},\mathit{del},\mathit{map})$ holds and $\mathit{config} {\xrightarrow{\mathit{send}(\mathit{mid},r)}} \mathit{config}'$, then $\mathit{inv}(\mathit{config}',h,\mathit{lin},\mathit{del},\mathit{map}')$ holds, where $\mathit{map}' = \mathit{map} \cup (\mathit{mid}, \mathit{vd}(h,\mathit{del},r))$.


    \item[-] If $\mathit{inv}(\mathit{config},h,\mathit{lin},\mathit{del},\mathit{map})$ holds and $\mathit{config} {\xrightarrow{\mathit{receive}(\mathit{mid},r)}} \mathit{config}'$, then $\mathit{inv}(\mathit{config}',h,\mathit{lin},\mathit{del}',\mathit{map})$ holds, where $\mathit{del}' = \mathit{del} \cup \{ (i,r) \vert i \in \mathit{map}(\mathit{mid}) \}$.
    \end{itemize}
\end{itemize}

Here $\mathit{vd}(h,\mathit{del},r) = \{ i \vert (i,j) \in h.\mathit{vis}, j$ is of replica $r \} \cup \{ i \vert (i,r) \in \mathit{del} \}$ is the set of operations that are either to some operation of replica $r$, or has been delivered into replica $r$. An invariant $\mathit{inv}$ satisfies above properties is called invariant of state-based CRDT. The following lemma states that the existence of such invariant implies distributed linearizability.

\begin{lemma}
\label{lemma:invariant of state-based CRDT implies distributed linearizability}
If there exists a invariant $\mathit{inv}$ of state-based CRDT for object $\mathit{obj}$ and sequential specification $\mathit{spec}$, then each history of $\mathit{history}(\llbracket \mathit{obj} \rrbracket_s)$ is distributed linearizable w.r.t $\mathit{spec}$.
\end{lemma}

\begin {proof}
Given an execution $l=\alpha_1 \cdot \ldots \cdot \alpha_n$, let $\mathit{config}_0 {\xrightarrow{\alpha_1}} \mathit{config}_1 \ldots {\xrightarrow{\alpha_n}} \mathit{config}_n$ be the transitions from initial configuration. We need to prove that, for each $1 \leq k \leq n$, we have $\mathit{inv}(\mathit{config}_k,h_k,\mathit{lin}_k,\mathit{del}_k,\mathit{map}_k)$ holds, where $h_k$ is the history of execution $l_k = \alpha_1 \cdot \ldots \cdot \alpha_k$, $\mathit{lin}_k$ is the linearization of $h_k$, $\mathit{del}_k$ records message delivery relation of $l_k$, and $\mathit{map}_k$ records the operations contained in each message in $l_k$.

Since $\mathit{inv}$ holds initially and is a transition invariant, it is easy to prove above requirements by induction on execution. This completes the proof of this lemma. $\qed$
\end {proof}


For many state-based CRDT implementations, $\mathit{inv}((R,T),h,\mathit{lin},\mathit{del},\mathit{map}) = C_1 \wedge C_2$, where

\begin{itemize}
\setlength{\itemsep}{0.5pt}
%\item[-] For each update operation $o$ of $h$, define $\mathit{ds}(o)$ which is a local state. %be the local state of replica $r$ at the time point immediately after $o$ is launched. Here $r$ is the replica of $o$.

\item[-] $C_1: \forall (\mathit{mid},\mathit{msg},\_) \in T$, $\mathit{msg} = \mathit{apply}(\mathit{lin},\mathit{map}(\mathit{mid}))$.

\item[-] $C_2: \forall r$, $R(r) = \mathit{apply}(\mathit{lin},\mathit{vd}(h,\mathit{del},r))$.
\end{itemize}

The function $\mathit{apply}(\mathit{lin},S)$ returns a local state by applying ``virtual messages'' of operations in $S$ according to total order $\mathit{lin}$. Here for each update operation $o$ of $h$, we need to define a local state $\mathit{ds}(o)$, which is the ``virtual messages'' of $o$. Note that state-based CRDT send message randomly, instead of each message for a update operation. This is the reason why we need to manually generate virtual message for each update operation.

To give $\mathit{inv}$, it only remains to give the virtual messages. The virtual message of state-based PN-counter and state-based multi-value register as follows. The proof of them being invariants of state-based CRDT is given in Appendix \ref{subsec:appendix proof of state-based PN-counter} and Appendix \ref{subsec:appendix proof of state-based multi-value register}, respectively.

\begin{example}[virtual messages of state-based PN-counter]
\label{example:virtual messagess of state-based PN-counter}

For each update operation $o$, $\mathit{ds}(o) = (P,N)$, where

\begin{itemize}
\setlength{\itemsep}{0.5pt}
\item[-] $\forall r'$, $P[r'] = \vert \{ o' \vert o'$ is a $\mathit{inc}$ operation of replica $r'$, $o' = o \vee (o',o) \in h.\mathit{vis} \} \vert$.

\item[-] $\forall r'$, $N[r'] = \vert \{ o' \vert o'$ is a $\mathit{dec}$ operation of replica $r'$, $o' = o \vee (o',o) \in h.\mathit{vis} \} \vert$.
\end{itemize}
\end{example}

\begin{example}[virtual messages of state-based Multi-value Register]
\label{example:virtual messages of state-based multi-value register}

For each update operation $o = (\mathit{write}(a),\_,\_)$ of replica $r$, $\mathit{ds}(o) = (a,V)$, where

\begin{itemize}
\setlength{\itemsep}{0.5pt}
\item[-] $\forall r'$, $V[r'] = \vert \{ o' \vert o'$ is a $\mathit{write}$ operation of replica $r'$, $o' = o \vee (o',o) \in h.\mathit{vis} \} \vert$.
\end{itemize}
\end{example}















\subsection{Proof of State-based PN-counter}
\label{subsec:appendix proof of state-based PN-counter}

The following lemma states that each visibility-closed set is a union of operations visible to a set of operations. Its proof is obvious and omitted here.

\begin{lemma}
\label{lemma:a transitive-closed set is a union of visibility of several sets}
Given a set $\mathit{Op}$ of operations and a transitive and acyclic visibility relation $\mathit{vis} \subseteq \mathit{Op} \times \mathit{Op}$, if given a set $S \subseteq \mathit{Op}$, if $S$ satisfies that $\forall o_1,o_2 \in S, o_2 \in S \wedge (o_1,o_2) \in \mathit{vis} \Rightarrow o_1 \in S$, then there exists a set $O \subseteq \mathit{Op}$, such that $S = \cup_{o \in O} \mathit{vis}^{-1}(o)$.
\end{lemma}

The following lemma states that given two operations $o_1,o_2$, for each replica $r$, either the set of operations of replica $r$ visible to $o_1$ is a subset of that of $o_2$, or the set of operations of replica $r$ visible to $o_2$ is a subset of that of $o_1$. Its proof is obvious and omitted here.

\begin{lemma}
\label{lemma:the view of a replica of one operation is contained in another operaiton, or vice versa}
Assume that $\mathit{inv}((R,T),h,\mathit{lin},\mathit{del},\mathit{map})$ holds. Let $S_o^r = \{ o' \vert (o',o) \in \mathit{vis}, o'$ is of replica $r \}$. Then for each operations $o_1$ and $o_2$, and for each replica $r$, $S_{\mathit{o1}}^r \subseteq S_{\mathit{o2}}^r \vee S_{\mathit{o2}}^r \subseteq S_{\mathit{o1}}^r$.
\end{lemma}


Recall that $\mathit{inv} = C_1 \wedge C_2$ with the virtual messages defined as follows: For each update operation $o$, $\mathit{ds}(o) = (P,N)$, where

\begin{itemize}
\setlength{\itemsep}{0.5pt}
\item[-] $\forall r'$, $P[r'] = \vert \{ o' \vert o'$ is a $\mathit{inc}$ operation of replica $r'$, $o' = o \vee (o',o) \in h.\mathit{vis} \} \vert$.

\item[-] $\forall r'$, $N[r'] = \vert \{ o' \vert o'$ is a $\mathit{dec}$ operation of replica $r'$, $o' = o \vee (o',o) \in h.\mathit{vis} \} \vert$.
\end{itemize}

The following lemma states that $\mathit{inv}$ is an invariant of state-based PN-counter.

\begin{lemma}
\label{lemma:inv is an invariant of state-based CRDT for state-based PN-counter}
$\mathit{inv}$ is an invariant of state-based PN-counter.
\end{lemma}

\begin {proof}

It is obvious that $\mathit{inv}(\mathit{config}_0,\epsilon,\emptyset,\emptyset,\emptyset)$ holds.

Let us prove that $\mathit{inv}$ is a transition invariant: assume $\mathit{inv}((R,T),h,\mathit{lin},\mathit{del},\mathit{map})$ holds,

\begin{itemize}
\setlength{\itemsep}{0.5pt}
\item[-] If $(R,T) {\xrightarrow{\mathit{do}(\mathit{inc},r)}} (R',T')$: Then,

    \begin{itemize}
    \setlength{\itemsep}{0.5pt}
    \item[-] It is easy to see that $R' = R[ r: ( R(r).P[r: R(r).P(r)+1 ], R(r).N ) ]$ and $T' = T$.

    \item[-] Let $h' = h \otimes i$, where $i$ is the identifier of the newly-generated $\mathit{inc}$ action.

    \item[-] Let $\mathit{lin}' = \mathit{lin} \cdot (\mathit{inc},i,\mathit{obj})$.

    \item[-] Let $\mathit{del}' = \mathit{del}$ and $\mathit{map}' = \mathit{map}$.
    \end{itemize}

    It is easy to see that $\mathit{lin}'$ is a linearization of $h'$. It is obvious that all other properties hold, except for $C_2$ for replica $r$. Therefore, let us prove that $R'(r) = \mathit{apply}(\mathit{lin}',\mathit{vd}(h',\mathit{del}',r))$.

    Since $R(r) = \mathit{apply}(\mathit{lin},\mathit{vd}(h,\mathit{del},r))$ and $\mathit{lin}' = \mathit{lin} \cdot (\mathit{inc},i,\mathit{obj})$, we know that $\mathit{apply}(\mathit{lin}',\mathit{vd}(h',\mathit{del}',r)) = \mathit{merge}(R(r),\mathit{ds}(i))$. Therefore, we need to prove that $R'(r) = \mathit{merge}(R(r),\mathit{ds}(i))$.

    Since $\mathit{vd}(h,\mathit{del},r)$ satisfies that, $\forall o_1,o_2 \in \mathit{vd}(h,\mathit{del},r), o_2 \in \mathit{vd}(h,\mathit{del},r) \wedge (o_1,o_2) \in \mathit{vis} \Rightarrow o_1 \in \mathit{vd}(h,\mathit{del},r)$, by Lemma \ref{lemma:a transitive-closed set is a union of visibility of several sets}, we know that there exists a set $O$, such that $\mathit{vd}(h,\mathit{del},r) = \cup_{o \in O} \mathit{vis}^{-1}(o)$. By Lemma \ref{lemma:the view of a replica of one operation is contained in another operaiton, or vice versa} and the construction of $\mathit{ds}$, we can see that $R(r) = (P',N')$, where for each replica $r'$, $P'[r'] = \vert \{ j \in \mathit{vd}(h,\mathit{del},r) \uparrow_{\mathit{inc}}$ and $j$ is of replica $r \} \vert$ and $N'[r'] = \vert \{ j \in \mathit{vd}(h,\mathit{del},r) \uparrow_{\mathit{dec}}$ and $j$ is of replica $r \} \vert$.

    We already know that $\mathit{ds}(i) = (P'',N'')$, where for each replica $r'$, $P''[r'] = \vert \{ j \in \mathit{vd}(h',\mathit{del}',r) \uparrow_{\mathit{inc}}$ and $j$ is of replica $r \} \vert$ and $N''[r'] = \vert \{ j \in \mathit{vd}(h',\mathit{del}',r) \uparrow_{\mathit{dec}}$ and $j$ is of replica $r \} \vert$. Then, it is obvious that $\mathit{merge}(R(r),\mathit{ds}(i)) = \mathit{ds}(i)$. It is also easy to see that $\mathit{ds}(i) = (R(r).P[r: R(r).P(r)+1], R(r).N) = R'(r)$. Therefore, $R'(r) = \mathit{merge}(R(r),\mathit{ds}(i))$.

\item[-] If $(R,T) {\xrightarrow{\mathit{do}(\mathit{dec},r)}} (R',T')$: Similarly as that of $(R,T) {\xrightarrow{\mathit{do}(\mathit{inc},r)}} (R',T')$.

\item[-] If $(R,T) {\xrightarrow{\mathit{do}(\mathit{read},k,r)}} (R',T')$: Then,

    \begin{itemize}
    \setlength{\itemsep}{0.5pt}
    \item[-] It is obvious that $R' = R$ and $T' = T$.

    \item[-] Let $h' = h \otimes i$, where $i$ is the identifier of the newly-generated $\mathit{read}$ action.

    \item[-] Let $\mathit{lin}' = \mathit{lin} \cdot ((\mathit{read}() \Rightarrow k,i,\mathit{obj}), \mathit{vd}(h,\mathit{del},r) )$.

    \item[-] Let $\mathit{del}' = \mathit{del}$ and $\mathit{map}' = \mathit{map}$.
    \end{itemize}

    It is easy to see that all other properties hold, except for $h'$ being distributed linearizable w.r.t $\mathit{spec}$ with $\mathit{lin}'$ the linearization. Let us prove that $h'$ is distributed linearizable w.r.t $\mathit{spec}$ and $\mathit{lin}'$ is a linearization. It is easy to see that only operation $i$ need to be checked.

    It is easy to see that $\mathit{vd}(h,\mathit{del},r) = \mathit{vis}^{-1}(i)$. Similarly as the case of $(R,T) {\xrightarrow{\mathit{do}(\mathit{inc},r)}} (R',T')$, we can prove that $R(r) = (P',N')$, where for each replica $r'$, $P'[r'] = \vert \{ j \in \mathit{vd}(h,\mathit{del},r) \uparrow_{\mathit{inc}}$ and $j$ is of replica $r \} \vert = \vert \{ j \in \mathit{vis}^{-1}(i) \uparrow_{\mathit{inc}}$ and $j$ is of replica $r \} \vert$ and $N'[r'] = \vert \{ j \in \mathit{vd}(h,\mathit{del},r) \uparrow_{\mathit{dec}}$ and $j$ is of replica $r \} \vert = \vert \{ j \in \mathit{vis}^{-1}(i) \uparrow_{\mathit{dec}}$ and $j$ is of replica $r \} \vert$. Since $k = \Sigma_{r'} P��[r'] - \Sigma_{r'} N'[r']$, $k$ is obtained by minus the number of all visible $\mathit{dec}$ of $i$ from the number of all visible $\mathit{inc}$ of $i$. Therefore, we can see that $((\mathit{read}() \Rightarrow k,i,\mathit{obj}), \mathit{vd}(h,\mathit{del},r) )$ of $\mathit{lin}'$ is ``correct''. Then, $h'$ is distributed linearizable w.r.t $\mathit{spec}$ and $\mathit{lin}'$ is a linearization.

\item[-] If $(R,T) {\xrightarrow{\mathit{send}(\mathit{mid},r)}} (R',T')$: Then,

    \begin{itemize}
    \setlength{\itemsep}{0.5pt}
    \item[-] It is obvious that $R' = R$. Let $T' = T \cup \{ (\mathit{mid},R(r),r) \}$.

    \item[-] Let $h' = h$.

    \item[-] Let $\mathit{lin}' = \mathit{lin}$.

    \item[-] Let $\mathit{del}' = \mathit{del}$.

    \item[-] Let $\mathit{map}' = \mathit{map} \cup \{ (\mathit{mid},\mathit{vd}(h,\mathit{del},r)) \}$.
    \end{itemize}

    It is easy to see that all other properties hold, except for checking $C_1$ for $\mathit{mid}$. This holds obviously since the message content of message $\mathit{mid}$ is $R(r)$, and we already know that $R(r) = \mathit{apply}(\mathit{lin},\mathit{vd}(h,\mathit{del},r)) = \mathit{apply}(\mathit{lin},\mathit{map}(\mathit{mid}))$.

\item[-] If $(R,T) {\xrightarrow{\mathit{receive}(\mathit{mid},r)}} (R',T')$: Then,

    \begin{itemize}
    \setlength{\itemsep}{0.5pt}
    \item[-] Let $R' = R[ r: \mathit{merge}(R(r),\mathit{msg})]$ where $(\mathit{mid},\mathit{msg},\_) \in T$. It is obvious that $T' = T$.

    \item[-] Let $h' = h$.

    \item[-] Let $\mathit{lin}' = \mathit{lin}$.

    \item[-] Let $\mathit{del}' = \mathit{del} \cup \{ (i,r) \vert i \in \mathit{map}(\mathit{mid}) \}$.

    \item[-] Let $\mathit{map}' = \mathit{map}$.
    \end{itemize}

    It is easy to see that all other properties hold, except for $C_2$ for replica $r$. Therefore, let us prove that $R'(r) = \mathit{apply}(\mathit{lin}',\mathit{vd}(h',\mathit{del}',r))$.

    We already know that $R'(r) = \mathit{merge}(R(r), \mathit{msg})$, $R(r) = \mathit{apply}(\mathit{lin},\mathit{vd}(h,\mathit{del},r))$ and $\mathit{msg} = \mathit{apply}(\mathit{lin},\mathit{map}(\mathit{mid}))$. It is easy to see that $\mathit{vd}(h',\mathit{del}',r) = \mathit{vd}(h,\mathit{del},r) \cup \mathit{map}(\mathit{mid})$. It is easy to prove that, applying messages in any order lead to the same consequence. Therefore, we have $\mathit{merge}(R(r), \mathit{msg}) = \mathit{apply}(\mathit{lin}',\mathit{vd}(h,\mathit{del},r) \cup \mathit{map}(\mathit{mid}))$. Then, we have $R'(r) = \mathit{apply}(\mathit{lin}',\mathit{vd}(h',\mathit{del}',r))$.
\end{itemize}

This completes the proof of this lemma. $\qed$
\end {proof}




\subsection{Proof of State-based Multi-value Register}
\label{subsec:appendix proof of state-based multi-value register}

Recall that $\mathit{inv} = C_1 \wedge C_2$ with the virtual messages defined as follows: For each update operation $o$, $\mathit{ds}(o) = (a,V)$, where

\begin{itemize}
\setlength{\itemsep}{0.5pt}
\item[-] $\forall r'$, $V[r'] = \vert \{ o' \vert o'$ is a $\mathit{write}$ operation of replica $r'$, $o' = o \vee (o',o) \in h.\mathit{vis} \} \vert$.
\end{itemize}

The following lemma states that $\mathit{inv}$ is an invariant of state-based multi-value register.

\begin{lemma}
\label{lemma:inv is an invariant of state-based CRDT for state-based multi-value register}
$\mathit{inv}$ is an invariant of state-based multi-value register.
\end{lemma}

\begin {proof}

It is obvious that $\mathit{inv}(\mathit{config}_0,\epsilon,\emptyset,\emptyset,\emptyset)$ holds.

Let us prove that $\mathit{inv}$ is a transition invariant: assume $\mathit{inv}((R,T),h,\mathit{lin},\mathit{del},\mathit{map})$ holds,

\begin{itemize}
\setlength{\itemsep}{0.5pt}
\item[-] If $(R,T) {\xrightarrow{\mathit{do}(\mathit{write},a,r)}} (R',T')$: Then,

    \begin{itemize}
    \setlength{\itemsep}{0.5pt}
    \item[-] $R' = R[ r: \{ (a,V') \} ], R(r).N)]$ and $T' = T$. Here $\forall r' \neq r, V'[r'] = \mathit{max} \{ V_1(r��) \vert (\_,V_1) \in R(r) \}$, and $V'[r] = \mathit{max} \{ V_1(r) \vert (\_,V_1) \in R(r) \} + 1$.

    \item[-] Let $h' = h \otimes i$, where $i$ is the identifier of the newly-generated $\mathit{inc}$ action.

    \item[-] Let $\mathit{lin}' = \mathit{lin} \cdot (\mathit{inc},i,\mathit{vis}^{-1}(i))$.

    \item[-] Let $\mathit{del}' = \mathit{del}$ and $\mathit{map}' = \mathit{map}$.
    \end{itemize}

    It is easy to see that $\mathit{lin}'$ is a linearization of $h'$. It is obvious that all other properties hold, except for $C_2$ for replica $r$. Therefore, let us prove that $R'(r) = \mathit{apply}(\mathit{lin}',\mathit{vd}(h',\mathit{del}',r))$.

    It is easy to see that $\mathit{vd}(h',\mathit{del}',r) = h'.\mathit{vis}^{-1}(i)$. And then, we need to prove that $(a,V') = \mathit{apply}(\mathit{lin}',h'.\mathit{vis}^{-1}(i))$.

    Recall that $R(r) = \mathit{apply}(\mathit{lin},\mathit{vd}(h,\mathit{del},r))$, from Lemma \ref{lemma:a transitive-closed set is a union of visibility of several sets}, we know that there exists set $O$, such that $\mathit{vd}(h,\mathit{del},r) = \cup_{o \in O} \mathit{vis}^{-1}(o)$. We can prove that, for each $o = \mathit{write}(b)$, $\mathit{apply}(\mathit{lin},\mathit{vis}^{-1}(o)) = (b,V_b)$, where $\forall r' \neq r, V_b[r'] = \vert \{ o' \vert o' \in \mathit{vis}^{-1}(o), o'$ is of replica $r' \} \vert$, and $V_b[r] = \vert \{ o' \vert o' \in \mathit{vis}^{-1}(o), o'$ is of replica $r' \} \vert + 1$.

    It is not hard to prove that the order of merging virtual message is not important, and a virtual message can be applied multiple times. By Lemma \ref{lemma:the view of a replica of one operation is contained in another operaiton, or vice versa}, we can see that $\mathit{apply}(\mathit{lin},\mathit{vd}(h,\mathit{del},r))$ is obtained by merging $\{ o \in O \vert \mathit{apply}(\mathit{lin},\mathit{vis}^{-1}(o)) \}$. Therefore, we can see that $\mathit{apply}(\mathit{lin}',h'.\mathit{vis}^{-1}(i)) = \mathit{apply}(\mathit{lin}',\mathit{vd}(h',\mathit{del}',r))$ is obtained by merging $\{ o \in O \vert \mathit{apply}(\mathit{lin},\mathit{vis}^{-1}(o)) \} \cup \{ \mathit{ds}(i) \}$. By Lemma \ref{lemma:the view of a replica of one operation is contained in another operaiton, or vice versa}, it is not hard to see that $\mathit{apply}(\mathit{lin}',h'.\mathit{vis}^{-1}(i)) = \mathit{ds}(i)$.

    Then, we need to prove that $(a,V') = \mathit{ds}(i)$. This holds since $R(r) = \mathit{apply}(\mathit{lin},\mathit{vd}(h,\mathit{del},r))$ is obtained by merging $\{ o \in O \vert \mathit{apply}(\mathit{lin},\mathit{vis}^{-1}(o)) \}$, Lemma \ref{lemma:the view of a replica of one operation is contained in another operaiton, or vice versa}, and the value of $V'$.

\item[-] If $(R,T) {\xrightarrow{\mathit{do}(\mathit{read},S,r)}} (R',T')$: Then,

    \begin{itemize}
    \setlength{\itemsep}{0.5pt}
    \item[-] It is obvious that $R' = R$ and $T' = T$.

    \item[-] Let $h' = h \otimes i$, where $i$ is the identifier of the newly-generated $\mathit{read}$ action.

    \item[-] Let $\mathit{lin}' = \mathit{lin} \cdot (\mathit{read}() \Rightarrow S,i,\mathit{vd}(h,\mathit{del},r) )$.

    \item[-] Let $\mathit{del}' = \mathit{del}$ and $\mathit{map}' = \mathit{map}$.
    \end{itemize}

    It is easy to see that all other properties hold, except for $h'$ being distributed linearizable w.r.t $\mathit{spec}$ with $\mathit{lin}'$ the linearization. Let us prove that $h'$ is distributed linearizable w.r.t $\mathit{spec}$ and $\mathit{lin}'$ is a linearization. It is easy to see that only operation $i$ need to be checked.

    It is easy to see that $\mathit{vd}(h,\mathit{del},r) = h'.\mathit{vis}^{-1}(i)$. Similarly as the case of $(R,T) {\xrightarrow{\mathit{do}(\mathit{write},a,r)}} (R',T')$, we can prove that there exists a set $O$, such that $R(r) = \mathit{apply}(\mathit{lin},\mathit{vd}(h,\mathit{del},r))$ is obtained by merging $\{ o \in O \vert \mathit{apply}(\mathit{lin},\mathit{vis}^{-1}(o)) \}$.

    By the definition of merging, it is same to assume that $O = \mathit{max}_{\mathit{vis}} \mathit{vd}(h,\mathit{del},r)$. Assume that for each operation $o = \mathit{write}(a) \in O$, $\mathit{apply}(\mathit{lin},\mathit{vis}^{-1}(o))) = (a,V_o)$. Then it is not hard to see that $R(r) = \{ (a,V_o) \vert o = \mathit{write}(a) \in O \}$. Therefore, $S = \{ a \vert o = \mathit{write}(a) \in \mathit{vis}^{-1}(i), \forall o' = \mathit{write}(\_) \in \mathit{vis}^{-1}(i), (o,o') \notin \mathit{vis} \}$. According to sequential specification $\mathit{spec}$, $(\mathit{read} \Rightarrow S,i,\mathit{obj})$ of $\mathit{lin}'$ is ``correct''. Then, $h'$ is distributed linearizable w.r.t $\mathit{spec}$ and $\mathit{lin}'$ is a linearization.

\item[-] If $(R,T) {\xrightarrow{\mathit{send}(\mathit{mid},r)}} (R',T')$: Then,

    \begin{itemize}
    \setlength{\itemsep}{0.5pt}
    \item[-] It is obvious that $R' = R$. Let $T' = T \cup \{ (\mathit{mid},R(r),r) \}$.

    \item[-] Let $h' = h$.

    \item[-] Let $\mathit{lin}' = \mathit{lin}$.

    \item[-] Let $\mathit{del}' = \mathit{del}$.

    \item[-] Let $\mathit{map}' = \mathit{map} \cup \{ (\mathit{mid},\mathit{vd}(h,\mathit{del},r)) \}$.
    \end{itemize}

    It is easy to see that all other properties hold, except for checking $C_1$ for $\mathit{mid}$. This holds obviously since the message content of message $\mathit{mid}$ is $R(r)$, and we already know that $R(r) = \mathit{apply}(\mathit{lin},\mathit{vd}(h,\mathit{del},r)) = \mathit{apply}(\mathit{lin},\mathit{map}(\mathit{mid}))$.

\item[-] If $(R,T) {\xrightarrow{\mathit{receive}(\mathit{mid},r)}} (R',T')$: Then,

    \begin{itemize}
    \setlength{\itemsep}{0.5pt}
    \item[-] Let $R' = R[ r: \mathit{merge}(R(r),\mathit{msg})]$ where $(\mathit{mid},\mathit{msg},\_) \in T$. It is obvious that $T' = T$.

    \item[-] Let $h' = h$.

    \item[-] Let $\mathit{lin}' = \mathit{lin}$.

    \item[-] Let $\mathit{del}' = \mathit{del} \cup \{ (i,r) \vert i \in \mathit{map}(\mathit{mid}) \}$.

    \item[-] Let $\mathit{map}' = \mathit{map}$.
    \end{itemize}

    It is easy to see that all other properties hold, except for $C_2$ for replica $r$. Therefore, let us prove that $R'(r) = \mathit{apply}(\mathit{lin}',\mathit{vd}(h',\mathit{del}',r))$.

    We already know that $R'(r) = \mathit{merge}(R(r), \mathit{msg})$, $R(r) = \mathit{apply}(\mathit{lin},\mathit{vd}(h,\mathit{del},r))$ and $\mathit{msg} = \mathit{apply}(\mathit{lin},\mathit{map}(\mathit{mid}))$. It is easy to see that $\mathit{vd}(h',\mathit{del}',r) = \mathit{vd}(h,\mathit{del},r) \cup \mathit{map}(\mathit{mid})$. It is easy to prove that, applying messages in any order lead to the same consequence. Therefore, we have $\mathit{merge}(R(r), \mathit{msg}) = \mathit{apply}(\mathit{lin}',\mathit{vd}(h,\mathit{del},r) \cup \mathit{map}(\mathit{mid}))$. Then, we have $R'(r) = \mathit{apply}(\mathit{lin}',\mathit{vd}(h',\mathit{del}',r))$.
\end{itemize}

This completes the proof of this lemma. $\qed$
\end {proof}
















































\forget{
\section{\crdtimp{}}
\label{sec:crdt implementation}



\subsection{Multi-Value Register Implementation}
\label{subsec:multi-value register implementation}

\cite{ShapiroPBZ11} shows how to obtain a state-based \crdtimp{} from a operation-based \crdtimp{}, and we draw it in Listing~\ref{lst:operation-based emulation of state-based object}. To do an operation $f(a)$, we compute the state-based update and perform merge method in downstream. Here the precondition of downstream is empty because merge is always enabled.


\begin{minipage}[t]{1.0\linewidth}
\begin{lstlisting}[frame=top,caption={operation-based emulation of state-based object},
captionpos=b,label={lst:operation-based emulation of state-based object}]
  payload S ( the state-based payload )
  initial initial payload of S

  update method f(a)
    atSource :
      precondition : precondition of f
      let s = atSource of f(a) in state-based
    downStream(s) :
      S = merge(S,s)
\end{lstlisting}
\end{minipage}

\cite{ShapiroPBZ11} gives a state-based multi-value register implementation. As discussed above, we give its operation-based version in Listing~\ref{lst:operation-based multi-value register}.


The following is a multi-value register implementation.

\begin{minipage}[t]{1.0\linewidth}
\begin{lstlisting}[frame=top,caption={Pseudo-code of the or-set CRDT},
captionpos=b,label={lst:operation-based multi-value register}]
  payload Set S
  initial S = @|$\emptyset$|@
  initial seq = @|$\epsilon$|@

  add(a) :
    atSource :
      let k = getUniqueIdentifier()
      //@ let seq@|$'$|@ = seq@|$\,\cdot\,\alabelshort[add]{a,k}$|@
    downStream(a, k) :
      S = S @|$\cup$|@ {(a, k)}
      //@ S@|$'$|@ = S @|$\cup$|@ {(a, k)}

  remove(a) :
    atSource :
      let R = @|$\{$|@ (a,k): (a,k) @|$\in$|@ S @|$\}$|@
      //@ let seq@|$'$|@ = seq@|$\,\cdot\,\alabellongind[readIds]{a}{R}{}\,\cdot\,\alabelshort[remove]{a,R}$|@
    downStream(R) :
      S = S @|$\setminus$|@ R
      //@ R = @|$\{ (a,k): \exists\ \alabel = \alabellongind[add]{a,k}{\bot}{*}.\ (\alabel, \alabelshort[remove]{a,R}) \in \avisord$|@
                       @|$\land\,\forall\ \alabel' = \alabellongind[remove]{a,*}{\bot}{*}.\ \{(\alabel,\alabel'),(\alabel',\alabelshort[remove]{a,R})\}\not\subseteq \avisord\}$|@
      //@ S@|$'$|@ = S @|$\setminus$|@ R

  read() :
    let A = {a : @|$\exists$|@ k. (a,k) @|$\in$|@ S}
    //@ let seq@|$'$|@ = seq@|$\,\cdot\,\alabellongind[read]{}{A}{}$|@
    return A
\end{lstlisting}
\end{minipage}









\subsection{OR-set Implementation and Formation}
\label{subsec:or-set implementation and formation}

The or-set implementation is shown below. Here function $\mathit{myRep}()$ returns the current replica identifier.

\renewcommand{\algorithmcfname}{CRDT Implementation}
\noindent
%\begin{minipage}{.5\textwidth}
\noindent\begin{algorithm}[H]
$\mathit{payload}$ set $S$; $\mathit{maxTS}$\\
$\mathit{initial}$ $\emptyset$; $(0,\mathit{myRep}())$\\

$add(a)$ \\
\ \ $\mathit{atSource}$: \\
\ \ \ \ assume $\mathit{maxTS} = (c,r')$; \\
\ \ \ \ let $\mathit{ts}' =(c+1,\mathit{myRep}())$; \\

\ \ $\mathit{downstream}((a,\mathit{ts}'))$: \\
\ \ \ \ $S = S \cup \{ (a,\mathit{ts}') \}$; \\
\ \ \ \ $\mathit{maxTS} = \mathit{max} \{ \mathit{maxTS},\mathit{ts}' \}$;


$rem(a)$ \\
\ \ $\mathit{atSource}$: \\
\ \ \ \ $\mathit{pre}$: \ $\exists \mathit{ts}', (a,\mathit{ts}') \in S$ \\
\ \ \ \ let $S_1 = \{ (a,\mathit{ts}') \vert (a,\mathit{ts}') \in S \}$; \\

\ \ $\mathit{downstream}(S_1)$: \\
\ \ \ \ $S = S \setminus S_1$.

$read()$ \\
\ \ \ \ \KwRet $\{ a \vert \exists \mathit{ts}, (a,\mathit{ts}) \in S \}$; \\

\caption{OR-set}
\label{Method-or-set}
\end{algorithm}


The formation of or-set is as follows: $I(r) = (\Sigma, \sigma_0, \mathit{Msg}, \mathit{do},\mathit{receive})$, where

\begin{itemize}
\setlength{\itemsep}{0.5pt}
\item[-] $\Sigma = \{ (S,\mathit{ts}) \vert$, $S$ is a set, each item of $S$ is of the form $(a',\mathit{ts}')$ with $a' \in D$ and $\mathit{ts}' \in \mathbb{N} \times \mathbb{R}.$ $\mathit{ts} \in \mathbb{N} \times \mathbb{R} \}$. $\Sigma_0 = (\emptyset,(0,\mathit{myRep}()))$.

\item[-] Each message content in $\mathit{Msg}$ is either in $D \times \mathbb{N} \times \mathbb{R}$, or a subset of $D \times \mathbb{N} \times \mathbb{R}$.

\item[-] $\mathit{do}((S,(c,r')),\mathit{add},a) = ((S \cup \{ (a, (c+1,r)) \}, (c+1,r)),(a,(c+1,r)))$.

\item[-] If $\exists \mathit{ts}', (a,\mathit{ts}') \in S$, then $\mathit{do}((S,\mathit{ts}),\mathit{rem},a) = ((S \setminus S_1,\mathit{ts}), S_1)$, where $S_1 = \{ (a,\mathit{ts}'') \in S \}$.

\item[-] $\mathit{do}((S,\mathit{ts}),\mathit{read}) = ((S,\mathit{ts}),\{ a \vert \exists \mathit{ts}', (a,\mathit{ts})' \in S \})$.

\item[-] $\mathit{receive}((S,\mathit{ts}),(a,\mathit{ts}')) = (S \cup \{ (a,\mathit{ts}') \}, \mathit{max}( \mathit{ts},\mathit{ts}' ))$,

\item[-] $\mathit{receive}((S,\mathit{ts}),S_1) = (S \setminus S_1,\mathit{ts})$,
\end{itemize}






\section{\Spec{}}
\label{sec:specification}


















\section{Proofs of Section \ref{sec:proving distributed linearizability}}
\label{sec:appendix proofs of section proving distributed linearizability}





\subsection{Proof of OR-set Implementation}
\label{subsec:appendix proofs of or-set implementation}

The following lemma states a property that can be generated from $P(\mathit{config},h,\mathit{lin},\mathit{map})$ for or-set.

\begin{lemma}
\label{lemma:a property that can be obtained from P for or-set}
If $P(\mathit{config},h,\mathit{lin},\mathit{map})$ holds for or-set, then each $\mathit{add}$ operation generate a new unique time-stamp. Moreover, for each replica $r'$,

\begin{itemize}
    \setlength{\itemsep}{0.5pt}
    \item[-] $R(r').S = \{ (b,\mathit{ts}') \vert b \in D, \exists o' = (\mathit{add}(b),\_,\_,\mathit{ts}'), o' \in \mathit{vd}(h,\mathit{del},r'), \forall o'' = (\mathit{rem}(b),\_,\_,\_), o'' \in \mathit{vd}(h,\mathit{config},r') \Rightarrow (o',o'') \notin h.\mathit{vis} \}$.

    \item[-] $R(r').\mathit{maxTS} = (0,r')$ if $\mathit{vd}(h,\mathit{config},r') = \emptyset$; otherwise, $R(r').\mathit{maxTS}$ is the maximal time stamp of $\mathit{add}$ operations of $\mathit{vd}(h,\mathit{config},r')$.
    \end{itemize}
\end{lemma}

\begin {proof}
By $C_4$, it is easy to see that each $\mathit{add}$ operation generate a new unique time-stamp by induction. The property of $R(r')$ can be also easily proved by induction, since the visibility relation is transitive. $\qed$
\end {proof}


The following lemma states that our $P(\mathit{config},h,\mathit{lin},\mathit{map})$ is an invariant of or-set.

\begin{lemma}
\label{lemma:P is an invariant of or-set}
$P(\mathit{config},h,\mathit{lin},\mathit{map})$ is an invariant of or-set.
\end{lemma}

\begin {proof}

Let us prove that $P$ is a simulation relation. It is obvious that $P(\mathit{config}_0,\epsilon,\emptyset,\emptyset)$ holds.

Assume $P((R,T,\mathit{MsgHB},\mathit{MsgDel}),h,\mathit{lin},\mathit{map})$ holds. Here we do not give the detailed value of $\mathit{MsgHB}'$ and $\mathit{MsgDel}'$, since it can be obtained from the definition of $\llbracket \mathit{obj} \rrbracket_{\mathit{op}}$.

\begin{itemize}
\setlength{\itemsep}{0.5pt}
\item[-] If $(R,T,\mathit{MsgHB},\mathit{MsgDel}) {\xrightarrow{\mathit{do}(\mathit{add},a,r,\mathit{mid})}} (R',T',\mathit{MsgHB}',\mathit{MsgDel}')$: Then,

    \begin{itemize}
    \setlength{\itemsep}{0.5pt}
    \item[-] $R' = R[ r: (R(r).S \cup \{ (a,\mathit{ts}) \},\mathit{ts}) ]$ and $T' = T \cup \{ (\mathit{mid},(a,\mathit{ts}),r) \}$. Here $\mathit{ts} = ( \mathit{max} \{ c \vert (\_,(c,\_)) \in R(r).S \} +1,r)$.

    \item[-] Let $h' = h \otimes i$, where $i$ is the identifier of the newly-generated $\mathit{add}$ operation.

    \item[-] Let $\mathit{lin}' = \mathit{lin} \cdot (\mathit{add}(a),i,\mathit{vd}(h,\mathit{config},r))$.

    \item[-] Let $\mathit{map}' = \mathit{map} \cup \{ (\mathit{mid},i) \}$.
    \end{itemize}

    It is easy to see that $h'$ is still distributed linearizable and $\mathit{lin}'$ is its linearization. We need to prove that $R'(r) = \mathit{apply}(\mathit{lin}',\mathit{vd}(h',\mathit{del}',r))$ and $C_4$ still holds for message $\mathit{mid}$.

    We already know that $R(r) = \mathit{apply}(\mathit{lin},\mathit{vd}(h,\mathit{del},r))$. %Based on $C_4$, it is not hard to prove that,

    %\begin{itemize}
    %\setlength{\itemsep}{0.5pt}
    %\item[-] $\mathit{Prop}_1$: Each $\mathit{add}$ operation generate a new unique time-stamp.

    %\item[-] $\mathit{Prop}_2$: for each replica $r'$, $R(r') = \{ (b,\mathit{ts}') \vert b \in D, \exists o' = (\mathit{add}(b),\_,\_,\mathit{ts}'), o' \in \mathit{vd}(h,\mathit{config},r'), \forall o'' = (\mathit{rem}(b),\_,\_,\_), o'' \in \mathit{vd}(h,\mathit{config},r') \Rightarrow (o',o'') \notin h.\mathit{vis} \}$.
    %\end{itemize}

    By Lemma \ref{lemma:a property that can be obtained from P for or-set}, it is not hard to see that $C_4$ still holds for message $\mathit{mid}$. From construction of $R'(r)$, Lemma \ref{lemma:a property that can be obtained from P for or-set} and $C_4$ holds for message $\mathit{mid}$, we can see that $R'(r) = \mathit{apply}(\mathit{lin}',\mathit{vd}(h',\mathit{del}',r))$.%, and $\mathit{Prop}_1$ and $\mathit{Prop}_2$ also hold for $(\mathit{config}',h',\mathit{lin}',\mathit{map}')$.


\item[-] If $(R,T,\mathit{MsgHB},\mathit{MsgDel}) {\xrightarrow{\mathit{do}(\mathit{rem},a,r,\mathit{mid})}} (R',T',\mathit{MsgHB}',\mathit{MsgDel}')$: Then,

    \begin{itemize}
    \setlength{\itemsep}{0.5pt}
    \item[-] $R' = R[ r: (R(r).S \setminus \{ (a,\mathit{ts}) \in R(r).S \},R(r).\mathit{maxTS}) ]$ and $T' = T \cup \{ (\mathit{mid},\{ (a,\mathit{ts}) \in R(r) \},r) \}$.

    \item[-] Let $h' = h \otimes i$, where $i$ is the identifier of the newly-generated $\mathit{rem}$ operation.

    \item[-] Let $\mathit{lin}' = \mathit{lin} \cdot (\mathit{add}(a),i,\mathit{vd}(h,\mathit{config},r))$.

    \item[-] Let $\mathit{map}' = \mathit{map} \cup \{ (\mathit{mid},i) \}$.
    \end{itemize}

    It is easy to see that $h'$ is still distributed linearizable and $\mathit{lin}'$ is its linearization. We need to prove that $R'(r) = \mathit{apply}(\mathit{lin}',\mathit{vd}(h',\mathit{del}',r))$ and $C_4$ still holds for message $\mathit{mid}$.

    By Lemma \ref{lemma:a property that can be obtained from P for or-set}, it is not hard to see that $C_4$ still holds for message $\mathit{mid}$. From construction of $R'(r)$, Lemma \ref{lemma:a property that can be obtained from P for or-set} and $C_4$ holds for message $\mathit{mid}$, we can see that $R'(r) = \mathit{apply}(\mathit{lin}',\mathit{vd}(h',\mathit{del}',r))$.%, and $\mathit{Prop}_1$ and $\mathit{Prop}_2$ also hold for $(\mathit{config}',h',\mathit{lin}',\mathit{map}')$.


\item[-] If $(R,T,\mathit{MsgHB},\mathit{MsgDel}) {\xrightarrow{\mathit{do}(\mathit{read},S_1,r)}} (R',T',\mathit{MsgHB}',\mathit{MsgDel}')$: Then,

    \begin{itemize}
    \setlength{\itemsep}{0.5pt}
    \item[-] $R' = R$ and $T' = T$.

    \item[-] Let $h' = h \otimes i$, where $i$ is the identifier of the newly-generated $\mathit{read}$ operation.

    \item[-] Let $\mathit{lin}' = \mathit{lin} \cdot (\mathit{read} \Rightarrow S_1,i,\mathit{vd}(h,\mathit{config},r))$.

    \item[-] Let $\mathit{map}'$.
    \end{itemize}

    We need to prove that $h'$ is distributed linearizable and $\mathit{lin}'$ is a linearization. Assume that in $\mathit{OR}$-$\mathit{set}_s$, $\mathit{state}_0 {\xrightarrow{\mathit{lin}}} \mathit{state}$ and $\mathit{state} {\xrightarrow{ (\mathit{read} \Rightarrow S_2, i, \mathit{vd}(h,\mathit{config},r) ) }} \mathit{state}$. Then by definition of $\mathit{OR}$-$\mathit{set}_s$, we can see that, $a \in S_2$, if there exists $(\mathit{add}(a),j,\_) \in \mathit{lin}'$, and for each $(\mathit{rem}(a),\_,S_2) \in \mathit{lin}'$, we have $j \notin S_2$. Lemma \ref{lemma:a property that can be obtained from P for or-set}, we can see that $S_1 = S_2$, and since $h$ is distributed linearizable and $\mathit{lin}$ is a linearization of $h$, we can see $h'$ is distributed linearizable and $\mathit{lin}'$ is a linearization.

\item[-] If $(R,T,\mathit{MsgHB},\mathit{MsgDel}) {\xrightarrow{\mathit{receive}(\mathit{mid},r)}} (R',T',\mathit{MsgHB}',\mathit{MsgDel}')$, where $(\mathit{mid},(a,\mathit{ts}),r') \in T$: Then,

    \begin{itemize}
    \setlength{\itemsep}{0.5pt}
    \item[-] $R' = R[ r: (R(r).S \cup \{ (a,\mathit{ts}) \},\mathit{max} \{ R(r).\mathit{maxTS},\mathit{ts} \} ) ]$ and $T' = T$.

    \item[-] Let $h' = h$.

    \item[-] Let $\mathit{lin}' = \mathit{lin}$.

    \item[-] Let $\mathit{map}' = \mathit{map}$.
    \end{itemize}

    We need to prove that $R'(r) = \mathit{apply}(\mathit{lin}',\mathit{vd}(h',\mathit{del}',r))$.

    We already know that $R(r) = \mathit{apply}(\mathit{lin},\mathit{vd}(h,\mathit{del},r))$. Since $R'(r)$ is obtained from $R(r)$ by applying message $\mathit{mid}$, and $\mathit{apply}(\mathit{lin}',\mathit{vd}(h',\mathit{del}',r))$ is obtained from $\mathit{apply}(\mathit{lin},\mathit{vd}(h,\mathit{del},r))$ by applying message $\mathit{mid}$. Therefore, $R'(r) = \mathit{apply}(\mathit{lin}',\mathit{vd}(h',\mathit{del}',r))$.

\item[-] If $(R,T,\mathit{MsgHB},\mathit{MsgDel}) {\xrightarrow{\mathit{receive}(\mathit{mid},r)}} (R',T',\mathit{MsgHB}',\mathit{MsgDel}')$, where $(\mathit{mid},S_1,r') \in T$: Then,

    \begin{itemize}
    \setlength{\itemsep}{0.5pt}
    \item[-] $R' = R[ r: (R(r).S \setminus S_1, R(r).\mathit{maxTS}) ]$ and $T' = T$.

    \item[-] Let $h' = h$.

    \item[-] Let $\mathit{lin}' = \mathit{lin}$.

    \item[-] Let $\mathit{map}' = \mathit{map}$.
    \end{itemize}

    We need to prove that $R'(r) = \mathit{apply}(\mathit{lin}',\mathit{vd}(h',\mathit{del}',r))$.

    We already know that $R(r) = \mathit{apply}(\mathit{lin},\mathit{vd}(h,\mathit{del},r))$. Since $R'(r)$ is obtained from $R(r)$ by applying message $\mathit{mid}$, and $\mathit{apply}(\mathit{lin}',\mathit{vd}(h',\mathit{del}',r))$ is obtained from $\mathit{apply}(\mathit{lin},\mathit{vd}(h,\mathit{del},r))$ by applying message $\mathit{mid}$. Therefore, $R'(r) = \mathit{apply}(\mathit{lin}',\mathit{vd}(h',\mathit{del}',r))$.
\end{itemize}

This completes the proof of this lemma. $\qed$
\end {proof}




\subsection{Proof of RGA}
\label{subsec:appendix proofs of rga}

The following lemma states a property that can be generated from $P(\mathit{config},h,\mathit{lin},\mathit{map})$ for RGA.

\begin{lemma}
\label{lemma:a property that can be obtained from P for rga}
If $P(\mathit{config},h,\mathit{lin},\mathit{map})$ holds, then each $\mathit{add}$ operation generate a new unique time-stamp. Moreover, for each replica $r'$,

\begin{itemize}
    \setlength{\itemsep}{0.5pt}
    \item[-] $R(r').N = \{ (a,\mathit{ts}_a,\mathit{ts}_b) \vert \exists o' = (\mathit{add}(\_,\_),i,\_,\_), \mathit{map}(i) = (a,\mathit{ts}_a,\mathit{ts}_b), o' \in \mathit{vd}(h,\mathit{config},r') \}$.

    \item[-] $R(r').\mathit{Tomb} = \{ a \vert \exists o = (\mathit{rem}(a),i,\_,\_), \mathit{map}(i) \in \mathit{vd}(h,\mathit{config},r') \}$.
    \end{itemize}
\end{lemma}

\begin {proof}
By $C_4$, it is easy to see that each $\mathit{add}$ operation generate a new unique time-stamp by induction. The property of $R(r')$ can be also easily proved by induction, since the visibility relation is transitive. $\qed$
\end {proof}


The following lemma states that our $P(\mathit{config},h,\mathit{lin},\mathit{map})$ is an invariant of rga.

\begin{lemma}
\label{lemma:P is an invariant of rga}
$P(\mathit{config},h,\mathit{lin},\mathit{map})$ is an invariant of rga.
\end{lemma}

\begin {proof}

Let us prove that $P$ is a simulation relation. It is obvious that $P(\mathit{config}_0,\epsilon,\emptyset,\emptyset)$ holds.

Assume $P((R,T,\mathit{MsgHB},\mathit{MsgDel}),h,\mathit{lin},\mathit{map})$ holds. Here we do not give the detailed value of $\mathit{MsgHB}'$ and $\mathit{MsgDel}'$, since it can be obtained from the definition of $\llbracket \mathit{obj} \rrbracket_{\mathit{op}}$.

\begin{itemize}
\setlength{\itemsep}{0.5pt}
\item[-] If $(R,T,\mathit{MsgHB},\mathit{MsgDel}) {\xrightarrow{\mathit{do}(\mathit{add},a,b,r,\mathit{mid})}} (R',T',\mathit{MsgHB}',\mathit{MsgDel}')$: Then,

    \begin{itemize}
    \setlength{\itemsep}{0.5pt}
    \item[-] $R' = R[ r: (R(r).N \cup \{ (a,\mathit{ts}_a,\mathit{ts}_b) \}, R(r).\mathit{Tomb}) ]$ and $T' = T \cup \{ (\mathit{mid},(a,\mathit{ts}_a,\mathit{ts}_b),r) \}$. Here $\mathit{ts}_a = ( \mathit{max} \{ c \vert (\_,(c,\_),\_) \in R(r).N \vee (\_,\_,(c,\_)) \in R(r).N \} +1,r)$, and $\mathit{ts}_b$ is the time-stamp of $b$ in $R(r).N$.

    \item[-] Let $h' = h \otimes i$, where $i$ is the identifier of the newly-generated $\mathit{add}$ action.

    \item[-] $\mathit{lin}'$ is obtained from $\mathit{lin}$ by inserting $(\mathit{add}(a,b),i,\mathit{vd}(h,\mathit{config},r))$ after the last operation with time-stamp less or equal than $\mathit{ts}_a$.

    \item[-] Let $\mathit{map}' = \mathit{map} \cup \{ (\mathit{mid},i) \}$.
    \end{itemize}

    It is easy to see that $h'$ is still distributed linearizable and $\mathit{lin}'$ is its linearization. We need to prove that $R'(r) = \mathit{apply}(\mathit{lin}',\mathit{vd}(h',\mathit{del}',r))$ and $C_4$ still holds for message $\mathit{mid}$.

    We already know that $R(r) = \mathit{apply}(\mathit{lin},\mathit{vd}(h,\mathit{del},r))$.

    By Lemma \ref{lemma:a property that can be obtained from P for rga}, it is not hard to see that $C_4$ still holds for message $\mathit{mid}$. From the fact that $\mathit{ts}_a$ is unique, the fact that there is no $\mathit{rem}(a)$ in $h$, the construction of $R'(r)$, Lemma \ref{lemma:a property that can be obtained from P for rga} and $C_4$ holds for message $\mathit{mid}$, we can see that $R'(r) = \mathit{apply}(\mathit{lin}',\mathit{vd}(h',\mathit{del}',r))$.


\item[-] If $(R,T,\mathit{MsgHB},\mathit{MsgDel}) {\xrightarrow{\mathit{do}(\mathit{rem},a,r,\mathit{mid})}} (R',T',\mathit{MsgHB}',\mathit{MsgDel}')$: Then,

    \begin{itemize}
    \setlength{\itemsep}{0.5pt}
    \item[-] $R' = R[ r: (R(r).N,R(r).\mathit{Tomb} \cup \{ a \} ) ]$ and $T' = T \cup \{ (\mathit{mid},\{ a \},r) \}$.

    \item[-] Let $h' = h \otimes i$, where $i$ is the identifier of the newly-generated $\mathit{rem}$ operation.

    \item[-] $\mathit{lin}'$ is obtained from $\mathit{lin}$ by inserting $(\mathit{rem}(a),i,\mathit{vd}(h,\mathit{config},r))$ after the last operation with time-stamp less or equal than the time-stamp of operation $i$.

    \item[-] Let $\mathit{map}' = \mathit{map} \cup \{ (\mathit{mid},i) \}$.
    \end{itemize}

    By Lemma \ref{lemma:a property that can be obtained from P for rga}, it is easy to see that $\mathit{lin} \uparrow_{\mathit{vd}(h,\mathit{config},r)}$ contains a $\mathit{add}(a,\_)$ operation $o$ and $(o,i) \in h'.\mathit{vis}$. By Lemma \ref{lemma:a property that can be obtained from P for rga}, it is easy to see that $\mathit{lin} \uparrow_{\mathit{vd}(h,\mathit{config},r)}$ does not contain $\mathit{rem}(a)$. Since $i$ does not visible to any operation in $\mathit{vd}(h,\mathit{config},r)$, we can see that $h'$ is still distributed linearizable and $\mathit{lin}'$ is its linearization.

    We need to prove that $R'(r) = \mathit{apply}(\mathit{lin}',\mathit{vd}(h',\mathit{del}',r))$ and $C_4$ still holds for message $\mathit{mid}$.

    It is obvious that $C_4$ holds for message $\mathit{mid}$. By Lemma \ref{lemma:a property that can be obtained from P for rga}, the construction of $R'(r)$, and $C_4$ holds for message $\mathit{mid}$, we can see that $R'(r) = \mathit{apply}(\mathit{lin}',\mathit{vd}(h',\mathit{del}',r))$.


\item[-] If $(R,T,\mathit{MsgHB},\mathit{MsgDel}) {\xrightarrow{\mathit{do}(\mathit{read},l,r)}} (R',T',\mathit{MsgHB}',\mathit{MsgDel}')$: Then,

    \begin{itemize}
    \setlength{\itemsep}{0.5pt}
    \item[-] $R' = R$ and $T' = T$.

    \item[-] Let $h' = h \otimes i$, where $i$ is the identifier of the newly-generated $\mathit{read}$ operation.

    \item[-] $\mathit{lin}'$ is obtained from $\mathit{lin}$ by inserting $(\mathit{rem}(a),i,\mathit{vd}(h,\mathit{config},r))$ after the last operation with time-stamp less or equal than the time-stamp of operation $i$.

    \item[-] Let $\mathit{map}' = \mathit{map}$.
    \end{itemize}

    We need to prove that $h'$ is distributed linearizable and $\mathit{lin}'$ is a linearization. Assume that in $\mathit{list}_s^{\mathit{af}}$, $\mathit{state}_0 {\xrightarrow{\mathit{lin}}} \mathit{state}$ and $\mathit{state} {\xrightarrow{ (\mathit{read} \Rightarrow l_1, i, \mathit{vd}(h,\mathit{config},r) ) }} \mathit{state}$.

    By Lemma \ref{lemma:a property that can be obtained from P for rga} and RGA implementation, we can see that $l$ and $l_1$ has the same items.

    Given items $a,b$, assume that $a$ is before $b$ in $l$, then, there are two possibilities,

    \begin{itemize}
    \setlength{\itemsep}{0.5pt}
    \item[-] $a$ is a ancestor of $b$ in $R(r).N$,

    \item[-] there exists items $c_1,c_2,c_3$, such that in $R(r).N$, $c_2$ and $c_3$ are sons of $c_1$, $c_2$ is a ancestor of $a$, $c_3$ is a ancestor of $b$, and the time-stamp of $c_2$ is larger than that of $c_3$.
    \end{itemize}

    If the first possibility holds, then there exists items $d_1,\ldots,d_k$, such that in $R(r).N$, $b$ is a son of $d_1$, $d_1$ is a son of $d_2$, $\ldots$, and $d_k$ is a son of $a$. It is easy to see that $(\mathit{add}(a,\_),\mathit{add}(d_k,a)),(\mathit{add}(d_k,a),\mathit{add}(d_{\mathit{k-1}},d_k)), \ldots, (\mathit{add}(d_1,d_2),\mathit{add}(b,d_1)) \in h.\mathit{vis}$. Since $\mathit{lin}$ is consistent with visibility relation, we know that in $\mathit{lin}$, $\mathit{add}(a,\_)$ is before $\mathit{add}(d_k,a)$, $\mathit{add}(d_k,a)$ is before $\mathit{add}(d_{\mathit{k-1}},d_k)$, $\ldots$, and $\mathit{add}(d_1,d_2)$ is before $\mathit{add}(b,d_1)$. According to $\mathit{list}_s^{\mathit{af}}$, it is easy to see that in $a$ is before $b$ in $l_1$.

    If the second possibility holds, then it is easy to see that $(\mathit{add}(c_2,c_1),\mathit{add}(a,c_2)),$ $(\mathit{add}(c_1,\_),\mathit{add}(c_2,c_1)), (\mathit{add}(c_3,c_1),\mathit{add}(b,c_3)),(\mathit{add}(c_1,\_),\mathit{add}(c_3,c_1)), \in h.\mathit{vis}$. Since $\mathit{lin}$ is consistent with visibility relation and time-stamp, we know that in $\mathit{lin}$, $\mathit{add}(c_3,c_1)$ is before $\mathit{add}(c_2,c_1)$, $\mathit{add}(c_2,c_1)$ is before $\mathit{add}(a,c_2)$, and $\mathit{add}(c_3,c_1)$ is before $\mathit{add}(b,c_3)$. According to $\mathit{list}_s^{\mathit{af}}$, it is easy to see that in $a$ is before $b$ in $l_1$.

    Therefore, $h'$ is distributed linearizable and $\mathit{lin}'$ is a linearization.

\item[-] If $(R,T,\mathit{MsgHB},\mathit{MsgDel}) {\xrightarrow{\mathit{receive}(\mathit{mid},r)}} (R',T',\mathit{MsgHB}',\mathit{MsgDel}')$, where $(\mathit{mid},(a,\mathit{ts}_a,\mathit{ts}_b),r') \in T$: Then,

    \begin{itemize}
    \setlength{\itemsep}{0.5pt}
    \item[-] $R' = R[ r: ( R(r).N \cup \{ (a,\mathit{ts}_a,\mathit{ts}_b) \}, R(r).\mathit{Tomb} ) ]$ and $T' = T$.

    \item[-] Let $h' = h$.

    \item[-] Let $\mathit{lin}' = \mathit{lin}$.

    \item[-] Let $\mathit{map}' = \mathit{map}$.
    \end{itemize}

    We need to prove that $R'(r) = \mathit{apply}(\mathit{lin}',\mathit{vd}(h',\mathit{del}',r))$.

    We already know that $R(r) = \mathit{apply}(\mathit{lin},\mathit{vd}(h,\mathit{del},r))$.

    We can see that $R'(r)$ is obtained from $R(r)$ by applying message $\mathit{mid}$, and $\mathit{apply}(\mathit{lin}',\mathit{vd}(h',\mathit{del}',r))$ is obtained from $\mathit{apply}(\mathit{lin},\mathit{vd}(h,\mathit{del},r))$ by additionally applying messages $\mathit{mid}$, but possibly in the middle of $\mathit{lin}'$. It is easy to see that $\mathit{map}(\mathit{mid})$ is a $\mathit{add}(a,\_)$ operation. By Lemma \ref{lemma:a property that can be obtained from P for rga}, we can see that there is no $\mathit{add}(a,\_)$ nor $\mathit{rem}(a)$ in $\mathit{vd}(h,\mathit{del},r)$. Thus, for each $\mathit{lin}''$ generated from $\mathit{lin}'$ by postponing message $\mathit{mid}$ to a later position, we can see that $\mathit{apply}(\mathit{lin}'',\mathit{vd}(h',\mathit{del}',r)) = \mathit{apply}(\mathit{lin}',\mathit{vd}(h',\mathit{del}',r))$.

    Therefore, $R'(r) = \mathit{apply}(\mathit{lin}',\mathit{vd}(h',\mathit{del}',r))$.

\item[-] If $(R,T,\mathit{MsgHB},\mathit{MsgDel}) {\xrightarrow{\mathit{receive}(\mathit{mid},r)}} (R',T',\mathit{MsgHB}',\mathit{MsgDel}')$, where $(\mathit{mid},a,r') \in T$: Then,

    \begin{itemize}
    \setlength{\itemsep}{0.5pt}
    \item[-] $R' = R[ r: (R(r).N,R(r).\mathit{Tomb} \cup \{ a \}) ]$ and $T' = T$.

    \item[-] Let $h' = h$.

    \item[-] Let $\mathit{lin}' = \mathit{lin}$.

    \item[-] Let $\mathit{map}' = \mathit{map}$.
    \end{itemize}

    We need to prove that $R'(r) = \mathit{apply}(\mathit{lin}',\mathit{vd}(h',\mathit{del}',r))$.

    We already know that $R(r) = \mathit{apply}(\mathit{lin},\mathit{vd}(h,\mathit{del},r))$.

    We can see that $R'(r)$ is obtained from $R(r)$ by applying message $\mathit{mid}$, and $\mathit{apply}(\mathit{lin}',\mathit{vd}(h',\mathit{del}',r))$ is obtained from $\mathit{apply}(\mathit{lin},\mathit{vd}(h,\mathit{del},r))$ by additionally applying messages $\mathit{mid}$, but possibly in the middle of $\mathit{lin}'$.

    It is easy to see that, for each $\mathit{lin}''$ generated from $\mathit{lin}'$ by postponing message $\mathit{mid}$ to a later position, we have $\mathit{apply}(\mathit{lin}'',\mathit{vd}(h',\mathit{del}',r)) = \mathit{apply}(\mathit{lin}',\mathit{vd}(h',\mathit{del}',r))$.

    Therefore, $R'(r) = \mathit{apply}(\mathit{lin}',\mathit{vd}(h',\mathit{del}',r))$.
\end{itemize}

This completes the proof of this lemma. $\qed$
\end {proof}










\section{Proofs of Section \ref{sec:compositionality of distributed linearizability}}
\label{sec:appendix proofs of section compositionality of distributed linearizability}





\subsection{Proofs of Lemma \ref{lemma:several t0-specifications}}
\label{subsec:appendix proofs of Lemma several t0-specifications}

A specification $\mathit{spec}$ is called t0-specification, if given a history $h$ that is distributed linearizable w.r.t $\mathit{spec}$, then any sequence that is consistent with visibility relation is a linearization of $h$.

Given two sequences $l_1,l_2$, let $\mathit{diff}(l_1,l_2) = \{ (o_1,o_2) \vert$ the order of $o_1$ and $o_2$ in $l_1$ is different from that of $l_2 \}$. Given a sequence $l$ and two elements $o_1$ an $o_2$ of $l$, let $\mathit{swap}(l,o_1,o_2)$ be a sequence obtained from $l$ by swapping $o_1$ and $o_2$.

The following lemma states that $\mathit{OR}$-$\mathit{set}_s$ is a t0-specification.

\begin{lemma}
\label{lemma:or-set is a t0-specification}
$\mathit{OR}$-$\mathit{set}_s$ is a t0-specification.
\end{lemma}

\begin {proof}
Given a distributed linearizable history $h$ and assume that $\mathit{lin}$ is a linearization. It is obvious that $\mathit{lin}$ is consistent with visibility relation. We need to prove that, each such sequence $\mathit{lin}'$ described below is also a linearization of $h$

\begin{itemize}
\setlength{\itemsep}{0.5pt}
\item[-] $\mathit{lin}'$ contains the same set of elements as that of $\mathit{lin}$.

\item[-] $\mathit{lin}'$ is consistent with visibility relation.
\end{itemize}

We prove this by showing that each such $\mathit{lin}'$ can be obtained from $\mathit{lin}$ by several times of swapping a pair of adjacent elements. Our proof requires the following two properties:

\begin{itemize}
\setlength{\itemsep}{0.5pt}
\item[-] The first property is: Given a linarization $\mathit{lin}$ and a sequence $\mathit{lin}'$ consistent with visibility relation of $h$, if $\mathit{diff}(\mathit{lin},\mathit{lin}') \neq \emptyset$, there exists $(o_1,o_2) \in \mathit{diff}(\mathit{lin},\mathit{lin}')$, such that $o_1$ and $o_2$ are concurrent, and $o_1$ and $o_2$ are adjacent in $\mathit{lin}$.

    We prove this by contradiction. Assume $\mathit{diff}(\mathit{lin},\mathit{lin}') \neq \emptyset$, and for each $(o_1,o_2) \in \mathit{diff}(\mathit{lin},\mathit{lin}')$, we have that either $o_1$ and $o_2$ are not concurrent, or $o_1$ and $o_2$ are not adjacent in $\mathit{lin}$.

    Since $\mathit{diff}(\mathit{lin},\mathit{lin}') \neq \emptyset$, let $(o,o')$ be a element of $\mathit{diff}(\mathit{lin},\mathit{lin}')$, and the distance of $o_1$ and $o_2$ is minimal in $\{$ the distance between $o_1$ and $o_2 \vert (o_1,o_2) \in \mathit{diff}(\mathit{lin},\mathit{lin}') \}$. Let us prove that $o$ and $o'$ are adjacent by contradiction: If there exists $o''$ between $o$ and $o'$. Assume that in $\mathit{lin}$, $o$ is before $o''$, and $o''$ is before $o'$. By assumption, the order between $o$ and $o''$, and between $o''$ and $o'$ is the same in $\mathit{lin}$ and in $\mathit{lin}'$. This implies that $o$ is still before $o'$ in $\mathit{lin}'$, which contradicts the fact that $(o,o') \in \mathit{diff}(\mathit{lin},\mathit{lin}')$.

    Since $o$ and $o'$ are adjacent and $(o,o') \in \mathit{diff}(\mathit{lin},\mathit{lin}')$, by assumption we know that $o$ and $o'$ are not concurrent. Or we can say, $(o,o') \in \mathit{vis} \vee \mathit{o',o} \in \mathit{vis}$. This contradicts that both $\mathit{lin}$ and $\mathit{lin}'$ are consistent with visibility relation. This completes the proof of the first step.

\item[-] The second property is: Given a linearization $\mathit{lin}$ and $o_1,o_2 \in \mathit{lin}$, such that $o_1$ and $o_2$ are concurrent and adjacent in $\mathit{lin}$, then, $l = \mathit{swap}(\mathit{lin},o_1,o_2)$ is also a linearization.

    Let $o_1 = (\ell_1,\mathit{id}_1,S_1)$ and $o_2 = (\ell_2,\mathit{id}_2,S_2)$. Since $o_1$ and $o_2$ are concurrent, we know that $\mathit{id}_1 \notin S_2 \wedge \mathit{id}_2 \notin S_1$. Assume $\mathit{lin} = l_1 \cdot o_1 \cdot o_2 \cdot l_2$. Assume in the abstract state of $\mathit{OR}$-$\mathit{set}_s$, we have $\sigma_0 {\xrightarrow{l_1}} \sigma_1 {\xrightarrow{o_1}} \sigma_2 {\xrightarrow{o_2}} \sigma_3 {\xrightarrow{l_2}} \sigma_4$, where $\sigma_0$ is the initial state of $\mathit{OR}$-$\mathit{set}_s$. Then, we need to prove that, there exists $\sigma'_2$, such that $\sigma_1 {\xrightarrow{o_2}} \sigma'_2 {\xrightarrow{o_1}} \sigma_3$. We prove this by consider all the possible cases:

    \begin{itemize}
    \setlength{\itemsep}{0.5pt}
    \item[-] If $o_1 = (\mathit{add}(a_1),\mathit{id}_1,S_1)$ and $o_2 = (\mathit{add}(a_2),\mathit{id}_2,S_2)$: We can see that $\sigma_2$ is obtained from $\sigma_1$ by inserting $(a_1,\mathit{id}_1,\mathit{true})$, and $\sigma_3$ is obtained from $\sigma_2$ by inserting $(a_2,\mathit{id}_2,\mathit{true})$. Let $\sigma'_2$ be obtained from $\sigma_1$ by inserting $(a_2,\mathit{id}_2,\mathit{true})$. Then, it is easy to see that $\sigma_1 {\xrightarrow{o_2}} \sigma'_2 {\xrightarrow{o_1}} \sigma_3$.

    \item[-] If $o_1 = (\mathit{add}(a_1),\mathit{id}_1,S_1)$ and $o_2 = (\mathit{rem}(a_2),\mathit{id}_2,S_2)$: We can see that $\sigma_2$ is obtained from $\sigma_1$ by inserting $(a_1,\mathit{id}_1,\mathit{true})$, and $\sigma_3$ is obtained from $\sigma_2$ by marking $a_2$ with identifiers of $S_2$ into $\mathit{false}$. Let $\sigma'_2$ be obtained from $\sigma_1$ by marking $a_2$ with identifiers of $S_2$ into $\mathit{false}$. Since $\mathit{id_1} \notin S_2$, we can see that $\sigma_1 {\xrightarrow{o_2}} \sigma'_2 {\xrightarrow{o_1}} \sigma_3$.

    \item[-] If $o_1 = (\mathit{add}(a_1),\mathit{id}_1,S_1)$ and $o_2 = (\mathit{read}() \Rightarrow l_2,\mathit{id}_2,S_2)$: Let $\sigma'_2 = \sigma_1$. Since $\mathit{id}_1 \notin S_2$, it is easy to see that $\sigma_1 {\xrightarrow{o_2}} \sigma'_2 {\xrightarrow{o_1}} \sigma_3$.

    \item[-] If $o_1 = (\mathit{rem}(a_1),\mathit{id}_1,S_1)$ and $o_2 = (\mathit{add}(a_2),\mathit{id}_2,S_2)$: We can see that $\sigma_2$ is obtained from $\sigma_1$ by marking $a_1$ with identifiers of $S_1$ into $\mathit{false}$, and $\sigma_3$ is obtained from $\sigma_2$ by inserting $(a_2,\mathit{id}_2,\mathit{true})$. Let $\sigma'_2$ be obtained from $\sigma_1$ by inserting $(a_2,\mathit{id}_2,\mathit{true})$. Since $\mathit{id}_2 \notin S_1$, we can see that $\sigma_1 {\xrightarrow{o_2}} \sigma'_2 {\xrightarrow{o_1}} \sigma_3$.

    \item[-] If $o_1 = (\mathit{rem}(a_1),\mathit{id}_1,S_1)$ and $o_2 = (\mathit{rem}(a_2),\mathit{id}_2,S_2)$: We can see that $\sigma_2$ is obtained from $\sigma_1$ by marking $a_1$ with identifiers of $S_1$ into $\mathit{false}$, and $\sigma_3$ is obtained from $\sigma_2$ by marking $a_2$ with identifiers of $S_2$ into $\mathit{false}$. Let $\sigma'_2$ be obtained from $\sigma_1$ by marking $a_2$ with identifiers of $S_2$ into $\mathit{false}$. Then, it is easy to see that $\sigma_1 {\xrightarrow{o_2}} \sigma'_2 {\xrightarrow{o_1}} \sigma_3$.

    \item[-] If $o_1 = (\mathit{rem}(a_1),\mathit{id}_1,S_1)$ and $o_2 = (\mathit{read}() \Rightarrow l_2,\mathit{id}_2,S_2)$: Let $\sigma'_2 = \sigma_1$. Since $\mathit{id}_1 \notin S_2$, it is easy to see that $\sigma_1 {\xrightarrow{o_2}} \sigma'_2 {\xrightarrow{o_1}} \sigma_3$.

    \item[-] If $o_1 = (\mathit{read}() \Rightarrow l_1,\mathit{id}_1,S_1)$ and $o_2 = (\mathit{add}(a_1),\mathit{id}_2,S_2)$: Let $\sigma'_2$ be obtained from $\sigma_1$ by inserting $(a_1,\mathit{id}_1,\mathit{true})$. Since $\mathit{id}_2 \notin S_1$, it is easy to see that $\sigma_1 {\xrightarrow{o_2}} \sigma'_2 {\xrightarrow{o_1}} \sigma_3$.

    \item[-] If $o_1 = (\mathit{read}() \Rightarrow l_1,\mathit{id}_1,S_1)$ and $o_2 = (\mathit{rem}(a_1),\mathit{id}_2,S_2)$: Let $\sigma'_2$ be obtained from $\sigma_1$ by marking $a_2$ with identifiers of $S_2$ into $\mathit{false}$. Since $\mathit{id}_2 \notin S_1$, it is easy to see that $\sigma_1 {\xrightarrow{o_2}} \sigma'_2 {\xrightarrow{o_1}} \sigma_3$.

    \item[-] If $o_1 = (\mathit{read}() \Rightarrow l_1,\mathit{id}_1,S_1)$ and $o_2 = (\mathit{read}() \Rightarrow l_2,\mathit{id}_2,S_2)$: Let $\sigma'_2 = \sigma_1$. Then, it is easy to see that $\sigma_1 {\xrightarrow{o_2}} \sigma'_2 {\xrightarrow{o_1}} \sigma_3$.
    \end{itemize}
\end{itemize}

Based on these two steps, given a linearization $\mathit{lin}$ and a sequence $\mathit{lin}' \neq \mathit{lin}$ which is consistent with visibility relation: We have $\mathit{diff}(\mathit{lin},\mathit{lin}') \neq \emptyset$. Based on the first above property, there exists $(o_1,o_2) \in \mathit{diff}(\mathit{lin},\mathit{lin}')$, such that $o_1$ and $o_2$ are concurrent, and $o_1$ and $o_2$ are adjacent in $\mathit{lin}$. Based on the second above property, $\mathit{lin}'' = \mathit{swap}(\mathit{lin},o_1,o_2)$ is also a linearization. Moreover, it is easy to see that $\mathit{diff}(\mathit{lin},\mathit{lin}') > \mathit{diff}(\mathit{lin}'',\mathit{lin}')$. Therefore, by several times of above process, we finally obtain $\mathit{lin}'$ from $\mathit{lin}$ by swapping pairs of operations, and prove that $\mathit{lin}'$ is also a linearization. This completes the proof of this lemma. $\qed$
\end {proof}



The following lemma states that $\mathit{set}_s$ is a t0-specification.

\begin{lemma}
\label{lemma:set is a t0-specification}
$\mathit{set}_s$ is a t0-specification.
\end{lemma}

\begin {proof}

We prove this lemma similarly as that of Lemma \ref{lemma:or-set is a t0-specification}. We need to prove that, given a linearization $\mathit{lin}$ and $o_1,o_2 \in \mathit{lin}$, such that $o_1$ and $o_2$ are concurrent and adjacent in $\mathit{lin}$, then, $l = \mathit{swap}(\mathit{lin},o_1,o_2)$ is also a linearization.

Let $o_1 = (\ell_1,\mathit{id}_1,S_1)$ and $o_2 = (\ell_2,\mathit{id}_2,S_2)$. Since $o_1$ and $o_2$ are concurrent, we know that $\mathit{id}_1 \notin S_2 \wedge \mathit{id}_2 \notin S_1$. Assume $\mathit{lin} = l_1 \cdot o_1 \cdot o_2 \cdot l_2$. Assume in the abstract state of $\mathit{set}_s$, we have $\sigma_0 {\xrightarrow{l_1}} \sigma_1 {\xrightarrow{o_1}} \sigma_2 {\xrightarrow{o_2}} \sigma_3 {\xrightarrow{l_2}} \sigma_4$, where $\sigma_0$ is the initial state of $\mathit{set}_s$. Then, we need to prove that, there exists $\sigma'_2$, such that $\sigma_1 {\xrightarrow{o_2}} \sigma'_2 {\xrightarrow{o_1}} \sigma_3$. We prove this by consider all the possible cases:

\begin{itemize}
\setlength{\itemsep}{0.5pt}
\item[-] If $o_1 = (\mathit{add}(a_1),\mathit{id}_1,S_1)$ and $o_2 = (\mathit{add}(a_2),\mathit{id}_2,S_2)$: We can see that, if $(a_1,\_) \in \sigma_1$, then $\sigma_2 = \sigma_1$; else, $\sigma_2$ is obtained from $\sigma_1$ by inserting $(a_1,\mathit{true})$. We can also see that, if $(a_2,\_) \in \sigma_2$, then $\sigma_3 = \sigma_2$; else, $\sigma_3$ is obtained from $\sigma_2$ by inserting $(a_2,\mathit{true})$. Let $\sigma'_2$ be: if $(a_2,\_) \in \sigma_1$, then $\sigma'_2 = \sigma_1$; else, $\sigma'_2$ is obtained from $\sigma_1$ by inserting $(a_2,\mathit{true})$. Then, it is easy to see that $\sigma_1 {\xrightarrow{o_2}} \sigma'_2 {\xrightarrow{o_1}} \sigma_3$.

\item[-] If $o_1 = (\mathit{add}(a_1),\mathit{id}_1,S_1)$ and $o_2 = (\mathit{rem}(a_2),\mathit{id}_2,S_2)$: Let $\sigma'_2$ be: if $(a_2,\mathit{false}) \in \sigma_1$, then $\sigma'_2 = \sigma_1$; else, $\sigma'_2$ is obtained from $\sigma_1$ by marking $a_2$ into $\mathit{false}$. Since $\mathit{vis}^{-1}(o_2) \cdot o_2 \in \mathit{set}_s$, we know that $(a_2,\_) \in \sigma_1$. Then, it is easy to see that $\sigma_1 {\xrightarrow{o_2}} \sigma'_2 {\xrightarrow{o_1}} \sigma_3$.

\item[-] If $o_1 = (\mathit{add}(a_1),\mathit{id}_1,S_1)$ and $o_2 = (\mathit{read}() \Rightarrow l_2,\mathit{id}_2,S_2)$: Let $\sigma'_2 = \sigma_1$. Since $\mathit{id}_1 \notin S_2$, it is easy to see that $\sigma_1 {\xrightarrow{o_2}} \sigma'_2 {\xrightarrow{o_1}} \sigma_3$.

\item[-] If $o_1 = (\mathit{rem}(a_1),\mathit{id}_1,S_1)$ and $o_2 = (\mathit{add}(a_2),\mathit{id}_2,S_2)$: Let $\sigma'_2$ be: if $(a_2,\_) \in \sigma_1$, then $\sigma'_2 = \sigma_1$; else, $\sigma'_2$ is obtained from $\sigma_1$ by inserting $(a_2,\mathit{true})$. Since $\mathit{vis}^{-1}(o_1) \cdot o_1 \in \mathit{set}_s$, we know that $(a_1,\_) \in \sigma_1$. Then, it is easy to see that $\sigma_1 {\xrightarrow{o_2}} \sigma'_2 {\xrightarrow{o_1}} \sigma_3$.

\item[-] If $o_1 = (\mathit{rem}(a_1),\mathit{id}_1,S_1)$ and $o_2 = (\mathit{rem}(a_2),\mathit{id}_2,S_2)$: Let $\sigma'_2$ be: if $(a_2,\mathit{false}) \in \sigma_1$, then $\sigma'_2 = \sigma_1$; else, $\sigma'_2$ is obtained from $\sigma_1$ by marking $a_2$ into $\mathit{false}$. Then, it is easy to see that $\sigma_1 {\xrightarrow{o_2}} \sigma'_2 {\xrightarrow{o_1}} \sigma_3$.

\item[-] If $o_1 = (\mathit{rem}(a_1),\mathit{id}_1,S_1)$ and $o_2 = (\mathit{read}() \Rightarrow l_2,\mathit{id}_2,S_2)$: Let $\sigma'_2 = \sigma_1$. Since $\mathit{id}_1 \notin S_2$, it is easy to see that $\sigma_1 {\xrightarrow{o_2}} \sigma'_2 {\xrightarrow{o_1}} \sigma_3$.

\item[-] If $o_1 = (\mathit{read}() \Rightarrow l_1,\mathit{id}_1,S_1)$ and $o_2 = (\mathit{add}(a_1),\mathit{id}_2,S_2)$: Let $\sigma'_2$ be: if $(a_2,\_) \in \sigma_1$, then $\sigma'_2 = \sigma_1$; else, $\sigma'_2$ is obtained from $\sigma_1$ by inserting $(a_2,\mathit{true})$. Since $\mathit{id}_2 \notin S_1$, it is easy to see that $\sigma_1 {\xrightarrow{o_2}} \sigma'_2 {\xrightarrow{o_1}} \sigma_3$.

\item[-] If $o_1 = (\mathit{read}() \Rightarrow l_1,\mathit{id}_1,S_1)$ and $o_2 = (\mathit{rem}(a_1),\mathit{id}_2,S_2)$: Let $\sigma'_2$ be: if $(a_2,\mathit{false}) \in \sigma_1$, then $\sigma'_2 = \sigma_1$; else, $\sigma'_2$ is obtained from $\sigma_1$ by marking $a_2$ into $\mathit{false}$. Since $\mathit{id}_2 \notin S_1$, it is easy to see that $\sigma_1 {\xrightarrow{o_2}} \sigma'_2 {\xrightarrow{o_1}} \sigma_3$.

\item[-] If $o_1 = (\mathit{read}() \Rightarrow l_1,\mathit{id}_1,S_1)$ and $o_2 = (\mathit{read}() \Rightarrow l_2,\mathit{id}_2,S_2)$: Let $\sigma'_2 = \sigma_1$. Then, it is easy to see that $\sigma_1 {\xrightarrow{o_2}} \sigma'_2 {\xrightarrow{o_1}} \sigma_3$.
\end{itemize}

This completes the proof of this lemma. $\qed$
\end {proof}




The following lemma states that $\mathit{counter}_s$ is a t0-specification.

\begin{lemma}
\label{lemma:counter is a t0-specification}
$\mathit{counter}_s$ is a t0-specification.
\end{lemma}


\begin {proof}

We prove this lemma similarly as that of Lemma \ref{lemma:or-set is a t0-specification}. We need to prove that, given a linearization $\mathit{lin}$ and $o_1,o_2 \in \mathit{lin}$, such that $o_1$ and $o_2$ are concurrent and adjacent in $\mathit{lin}$, then, $l = \mathit{swap}(\mathit{lin},o_1,o_2)$ is also a linearization.

Let $o_1 = (\ell_1,\mathit{id}_1,S_1)$ and $o_2 = (\ell_2,\mathit{id}_2,S_2)$. Since $o_1$ and $o_2$ are concurrent, we know that $\mathit{id}_1 \notin S_2 \wedge \mathit{id}_2 \notin S_1$. Assume $\mathit{lin} = l_1 \cdot o_1 \cdot o_2 \cdot l_2$. Assume in the abstract state of $\mathit{counter}_s$, we have $\sigma_0 {\xrightarrow{l_1}} \sigma_1 {\xrightarrow{o_1}} \sigma_2 {\xrightarrow{o_2}} \sigma_3 {\xrightarrow{l_2}} \sigma_4$, where $\sigma_0$ is the initial state of $\mathit{counter}_s$. Then, we need to prove that, there exists $\sigma'_2$, such that $\sigma_1 {\xrightarrow{o_2}} \sigma'_2 {\xrightarrow{o_1}} \sigma_3$. We prove this by consider all the possible cases:

\begin{itemize}
\setlength{\itemsep}{0.5pt}
\item[-] If $o_1 = (\mathit{inc},\mathit{id}_1,S_1)$ and $o_2 = (\mathit{inc},\mathit{id}_2,S_2)$: Assume that $\sigma_1 = k$, then $\sigma_2 = \mathit{k+1}$ and $\sigma_3 = \mathit{k+2}$. Let $\sigma'_2 = \mathit{k+1}$. Then, it is easy to see that $\sigma_1 {\xrightarrow{o_2}} \sigma'_2 {\xrightarrow{o_1}} \sigma_3$.

\item[-] If $o_1 = (\mathit{inc},\mathit{id}_1,S_1)$ and $o_2 = (\mathit{dec},\mathit{id}_2,S_2)$: Assume that $\sigma_1 = k$, and let $\sigma'_2 = \mathit{k-1}$. Then, it is easy to see that $\sigma_1 {\xrightarrow{o_2}} \sigma'_2 {\xrightarrow{o_1}} \sigma_3$.

\item[-] If $o_1 = (\mathit{inc},\mathit{id}_1,S_1)$ and $o_2 = (\mathit{read}() \Rightarrow k_2,\mathit{id}_2,S_2)$: Let $\sigma'_2 = \sigma_1$. Since $\mathit{id}_1 \notin S_2$, it is easy to see that $\sigma_1 {\xrightarrow{o_2}} \sigma'_2 {\xrightarrow{o_1}} \sigma_3$.

\item[-] If $o_1 = (\mathit{dec},\mathit{id}_1,S_1)$ and $o_2 = (\mathit{inc},\mathit{id}_2,S_2)$: Assume that $\sigma_1 = k$, and let $\sigma'_2 = \mathit{k+1}$. Then, it is easy to see that $\sigma_1 {\xrightarrow{o_2}} \sigma'_2 {\xrightarrow{o_1}} \sigma_3$.

\item[-] If $o_1 = (\mathit{dec},\mathit{id}_1,S_1)$ and $o_2 = (\mathit{dec},\mathit{id}_2,S_2)$: Assume that $\sigma_1 = k$, and let $\sigma'_2 = \mathit{k-1}$. Then, it is easy to see that $\sigma_1 {\xrightarrow{o_2}} \sigma'_2 {\xrightarrow{o_1}} \sigma_3$.

\item[-] If $o_1 = (\mathit{dec},\mathit{id}_1,S_1)$ and $o_2 = (\mathit{read}() \Rightarrow k_2,\mathit{id}_2,S_2)$: Let $\sigma'_2 = \sigma_1$. Since $\mathit{id}_1 \notin S_2$, it is easy to see that $\sigma_1 {\xrightarrow{o_2}} \sigma'_2 {\xrightarrow{o_1}} \sigma_3$.

\item[-] If $o_1 = (\mathit{read}() \Rightarrow k_1,\mathit{id}_1,S_1)$ and $o_2 = (\mathit{inc},\mathit{id}_2,S_2)$: Assume that $\sigma_1 = k$, and let $\sigma'_2 = \mathit{k+1}$. Since $\mathit{id}_2 \notin S_1$, it is easy to see that $\sigma_1 {\xrightarrow{o_2}} \sigma'_2 {\xrightarrow{o_1}} \sigma_3$.

\item[-] If $o_1 = (\mathit{read}() \Rightarrow k_1,\mathit{id}_1,S_1)$ and $o_2 = (\mathit{dec},\mathit{id}_2,S_2)$: Assume that $\sigma_1 = k$, and let $\sigma'_2 = \mathit{k-1}$. Since $\mathit{id}_2 \notin S_1$, it is easy to see that $\sigma_1 {\xrightarrow{o_2}} \sigma'_2 {\xrightarrow{o_1}} \sigma_3$.

\item[-] If $o_1 = (\mathit{read}() \Rightarrow k_1,\mathit{id}_1,S_1)$ and $o_2 = (\mathit{read}() \Rightarrow k_2,\mathit{id}_2,S_2)$: Let $\sigma'_2 = \sigma_1$. Then, it is easy to see that $\sigma_1 {\xrightarrow{o_2}} \sigma'_2 {\xrightarrow{o_1}} \sigma_3$.
\end{itemize}

This completes the proof of this lemma. $\qed$
\end {proof}


With Lemma \ref{lemma:or-set is a t0-specification}, Lemma \ref{lemma:set is a t0-specification} and Lemma \ref{lemma:counter is a t0-specification}, we can now prove Lemma \ref{lemma:several t0-specifications}.


\SeveralTZeroSpecifications*

\begin {proof}
This lemma holds obviously from Lemma \ref{lemma:or-set is a t0-specification}, Lemma \ref{lemma:set is a t0-specification} and Lemma \ref{lemma:counter is a t0-specification}. $\qed$
\end {proof}





\subsection{Proofs of Lemma \ref{lemma:several t1-specifications}}
\label{subsec:appendix proofs of Lemma several t1-specifications}


The following lemma states that $\mathit{list}_s^{\mathit{af}}$ is a t1-specification.

\begin{lemma}
\label{lemma:list-af is a t1-specification}
$\mathit{list}_s^{\mathit{af}}$ is a t1-specification.
\end{lemma}

\begin {proof}

Given a distributed linearizable history $h$ and a linearization $\mathit{lin}$ that is a strict time-stamp order candidate, we need to prove that, each strict time-stamp order candidate $\mathit{lin}'$ is a linearization.

We prove this by showing that each such $\mathit{lin}'$ can be obtained from $\mathit{lin}$ by several times of swapping a pair of adjacent elements. Our proof requires the following two properties:

\begin{itemize}
\setlength{\itemsep}{0.5pt}
\item[-] The first property is: Given a linarization $\mathit{lin}$ that is a strict time-stamp order candidate, and a strict time-stamp order candidate $\mathit{lin}'$. If $\mathit{diff}(\mathit{lin},\mathit{lin}') \neq \emptyset$, there exists $(o_1,o_2) \in \mathit{diff}(\mathit{lin},\mathit{lin}')$, such that $o_1$ and $o_2$ are concurrent, $o_1$ and $o_2$ are adjacent in $\mathit{lin}$, and the time-stamp of $o_1$ in $h$ equals that of $o_2$.

    We prove this by contradiction. Assume $\mathit{diff}(\mathit{lin},\mathit{lin}') \neq \emptyset$, and for each $(o_1,o_2) \in \mathit{diff}(\mathit{lin},\mathit{lin}')$, we have that either $o_1$ and $o_2$ are not concurrent, or $o_1$ and $o_2$ are not adjacent in $\mathit{lin}$, or the time-stamp of $o_1$ in $h$ is different from that of $o_2$.

    By the definition of strict time-stamp order candidate, it is easy to see that if $o_1$ and $o_2$ have different time-stamp, then their order is the same between $\mathit{lin}$ and $\mathit{lin}'$. Therefore, we know that the time-stamp of $o_1$ in $h$ equals that of $o_2$.

    Since $\mathit{diff}(\mathit{lin},\mathit{lin}') \neq \emptyset$, let $(o,o')$ be a element of $\mathit{diff}(\mathit{lin},\mathit{lin}')$, and the distance of $o_1$ and $o_2$ is minimal in $\{$ the distance between $o_1$ and $o_2 \vert (o_1,o_2) \in \mathit{diff}(\mathit{lin},\mathit{lin}') \}$. Let us prove that $o$ and $o'$ are adjacent by contradiction: If there exists $o''$ between $o$ and $o'$. Assume that in $\mathit{lin}$, $o$ is before $o''$, and $o''$ is before $o'$. By assumption, the order between $o$ and $o''$, and between $o''$ and $o'$ is the same in $\mathit{lin}$ and in $\mathit{lin}'$. This implies that $o$ is still before $o'$ in $\mathit{lin}'$, which contradicts the fact that $(o,o') \in \mathit{diff}(\mathit{lin},\mathit{lin}')$.

    Since $o$ and $o'$ are adjacent and $(o,o') \in \mathit{diff}(\mathit{lin},\mathit{lin}')$, by assumption we know that $o$ and $o'$ are not concurrent. Or we can say, $(o,o') \in \mathit{vis} \vee \mathit{o',o} \in \mathit{vis}$. This contradicts that both $\mathit{lin}$ and $\mathit{lin}'$ are consistent with visibility relation. This completes the proof of the first step.

\item[-] The second property is: Given a linearization $\mathit{lin}$ that is a strict time-stamp order candidate, and $o_1,o_2 \in \mathit{lin}$, such that $o_1$ and $o_2$ are concurrent and adjacent in $\mathit{lin}$, and $o_1$ and $o_2$ have the same time-stamp in $h$. Then, $l = \mathit{swap}(\mathit{lin},o_1,o_2)$ is also a linearization and is also a strict time-stamp order candidate. It is obvious that $l$ is still a strict time-stamp order candidate.

    Let $o_1 = (\ell_1,\mathit{id}_1,S_1)$ and $o_2 = (\ell_2,\mathit{id}_2,S_2)$. Since $o_1$ and $o_2$ are concurrent, we know that $\mathit{id}_1 \notin S_2 \wedge \mathit{id}_2 \notin S_1$. Assume $\mathit{lin} = l_1 \cdot o_1 \cdot o_2 \cdot l_2$. Assume in the abstract state of $\mathit{list}_s^{\mathit{af}}$, we have $\sigma_0 {\xrightarrow{l_1}} \sigma_1 {\xrightarrow{o_1}} \sigma_2 {\xrightarrow{o_2}} \sigma_3 {\xrightarrow{l_2}} \sigma_4$, where $\sigma_0$ is the initial state of $\mathit{OR}$-$\mathit{set}_s$. Then, we need to prove that, there exists $\sigma'_2$, such that $\sigma_1 {\xrightarrow{o_2}} \sigma'_2 {\xrightarrow{o_1}} \sigma_3$. We prove this by consider all the possible cases:

    \begin{itemize}
    \setlength{\itemsep}{0.5pt}
    \item[-] If $o_1 = (\mathit{add}(a_1,b_1),\mathit{id}_1,S_1)$ and $o_2 = (\_,\mathit{id}_2,S_2)$: This case is impossible. We can see that the time-stamp of $a$ is larger than operations in $S_1$, and thus, the time-stamp of $o_1$ is the time-stamp of $a$. Since $\mathit{id}_1 \notin S_2$, we know that the time-stamp of $o_2$ is different from that of $o_1$, contradicts the assumption that $o_1$ and $o_2$ have same time-stamp.

    \item[-] If $o_1 = (\_,\mathit{id}_1,S_1)$ and $o_2 = (\mathit{add}(a_2,b_2),\mathit{id}_2,S_2)$: Similarly, we can prove that this case is impossible.

    \item[-] If $o_1 = (\mathit{rem}(a_1),\mathit{id}_1,S_1)$ and $o_2 = (\mathit{rem}(a_2),\mathit{id}_2,S_2)$: Let $\sigma'_2$ be obtained from $\sigma_1$ by marking $a_2$ into $\mathit{false}$. Then, it is easy to see that $\sigma_1 {\xrightarrow{o_2}} \sigma'_2 {\xrightarrow{o_1}} \sigma_3$.

    \item[-] If $o_1 = (\mathit{rem}(a_1),\mathit{id}_1,S_1)$ and $o_2 = (\mathit{read}() \Rightarrow \mathit{list}_1,\mathit{id}_2,S_2)$: Let $\sigma'_2 = \sigma_1$. Since $\mathit{id}_1 \notin S_2$, it is easy to see that $\sigma_1 {\xrightarrow{o_2}} \sigma'_2 {\xrightarrow{o_1}} \sigma_3$.

    \item[-] If $o_1 = (\mathit{read}() \Rightarrow \mathit{list}_1,\mathit{id}_1,S_1)$ and $o_2 = (\mathit{read}() \Rightarrow \mathit{list}_2,\mathit{id}_2,S_2)$: Let $\sigma'_2 = \sigma_1$. Then, it is easy to see that $\sigma_1 {\xrightarrow{o_2}} \sigma'_2 {\xrightarrow{o_1}} \sigma_3$.
    \end{itemize}
\end{itemize}

Based on these two steps, given a linearization $\mathit{lin}$ that is a strict time-stamp order candidate, and a sequence $\mathit{lin}' \neq \mathit{lin}$ that is a strict time-stamp order candidate: We have $\mathit{diff}(\mathit{lin},\mathit{lin}') \neq \emptyset$. Based on the first above property, there exists $(o_1,o_2) \in \mathit{diff}(\mathit{lin},\mathit{lin}')$, such that $o_1$ and $o_2$ are concurrent, and $o_1$ and $o_2$ are adjacent in $\mathit{lin}$, and $o_1$ and $o_2$ have a same time-stamp. Based on the second above property, $\mathit{lin}'' = \mathit{swap}(\mathit{lin},o_1,o_2)$ is also a linearization, and is a strict time-stamp order candidate. Moreover, it is easy to see that $\mathit{diff}(\mathit{lin},\mathit{lin}') > \mathit{diff}(\mathit{lin}'',\mathit{lin}')$. Therefore, by several times of above process, we finally obtain $\mathit{lin}'$ from $\mathit{lin}$ by swapping pairs of operations, and prove that $\mathit{lin}'$ is also a linearization, and is a strict time-stamp order candidate. This completes the proof of this lemma. $\qed$
\end {proof}


The following lemma states that $\mathit{reg}_s$ is a t1-specification.

\begin{lemma}
\label{lemma:reg is a t1-specification}
$\mathit{reg}_s$ is a t1-specification.
\end{lemma}

\begin {proof}

We prove this lemma similarly as that of Lemma \ref{lemma:list-af is a t1-specification}. We need to prove that, given a linearization $\mathit{lin}$ that is a strict time-stamp order candidate, and $o_1,o_2 \in \mathit{lin}$, such that $o_1$ and $o_2$ are concurrent and adjacent in $\mathit{lin}$, and $o_1$ and $o_2$ have the same time-stamp in $h$. Then, $l = \mathit{swap}(\mathit{lin},o_1,o_2)$ is also a linearization and is also a strict time-stamp order candidate. It is obvious that $l$ is still a strict time-stamp order candidate.

Let $o_1 = (\ell_1,\mathit{id}_1,S_1)$ and $o_2 = (\ell_2,\mathit{id}_2,S_2)$. Since $o_1$ and $o_2$ are concurrent, we know that $\mathit{id}_1 \notin S_2 \wedge \mathit{id}_2 \notin S_1$. Assume $\mathit{lin} = l_1 \cdot o_1 \cdot o_2 \cdot l_2$. Assume in the abstract state of $\mathit{reg}_s$, we have $\sigma_0 {\xrightarrow{l_1}} \sigma_1 {\xrightarrow{o_1}} \sigma_2 {\xrightarrow{o_2}} \sigma_3 {\xrightarrow{l_2}} \sigma_4$, where $\sigma_0$ is the initial state of $\mathit{OR}$-$\mathit{set}_s$. Then, we need to prove that, there exists $\sigma'_2$, such that $\sigma_1 {\xrightarrow{o_2}} \sigma'_2 {\xrightarrow{o_1}} \sigma_3$. We prove this by consider all the possible cases:


\begin{itemize}
\setlength{\itemsep}{0.5pt}
\item[-] If $o_1 = (\mathit{write}(a_1),\mathit{id}_1,S_1)$ and $o_2 = (\_,\mathit{id}_2,S_2)$: This case is impossible. We can see that the time-stamp of $a$ is larger than operations in $S_1$, and thus, the time-stamp of $o_1$ is the time-stamp of $a$. Since $\mathit{id}_1 \notin S_2$, we know that the time-stamp of $o_2$ is different from that of $o_1$, contradicts the assumption that $o_1$ and $o_2$ have same time-stamp.

\item[-] If $o_1 = (\_,\mathit{id}_1,S_1)$ and $o_2 = (\mathit{write}(a_2),\mathit{id}_2,S_2)$: Similarly, we can prove that this case is impossible.

\item[-] If $o_1 = (\mathit{read}() \Rightarrow a_1,\mathit{id}_1,S_1)$ and $o_2 = (\mathit{read}() \Rightarrow a_2,\mathit{id}_2,S_2)$: Let $\sigma'_2 = \sigma_1$. Then, it is easy to see that $\sigma_1 {\xrightarrow{o_2}} \sigma'_2 {\xrightarrow{o_1}} \sigma_3$.
\end{itemize}
This completes the proof of this lemma. $\qed$
\end {proof}


With Lemma \ref{lemma:list-af is a t1-specification} and Lemma \ref{lemma:reg is a t1-specification}, we can now prove Lemma \ref{lemma:several t1-specifications}.

\SeveralTOneSpecifications*

\begin {proof}
This lemma holds obviously from Lemma \ref{lemma:list-af is a t1-specification} and Lemma \ref{lemma:reg is a t1-specification}. $\qed$
\end {proof}









\subsection{Proof of Lemma \ref{lemma:several t0-specifications can be composed}}
\label{subsec:appendix proofs of lemma several t0-specifications can be composed}

\composingTZero*
\begin {proof}
Assume that $h = (\mathit{Op},\mathit{ro},\mathit{vis})$. We need to prove that, if $h \uparrow_{\mathit{obj}}$ is distributed linearizable for each object $\mathit{obj}$ of $h$, then $h$ is distributed linearizable.

We construct a linearization $\mathit{lin}$ of $h$ in a process as follows:

\begin{itemize}
\setlength{\itemsep}{0.5pt}
\item[-] Initially a set $\mathit{Op}' = \mathit{Op}$ and $\mathit{lin} = \epsilon$.

\item[-] We begin a loop as follows: In each round of the loop, we choose an operation $o$ that is minimal w.r.t $\mathit{vis}$ in $\mathit{Op}'$, let $\mathit{Op}' = \mathit{Op}' \setminus \{ o \}$, and let $\mathit{lin} = \mathit{lin} \cdot o$.
\end{itemize}

If this process terminates with $\mathit{Op}' = \emptyset$: Then it is easy to see that $\mathit{lin}$ is consistent with $\mathit{vis}$, and thus, for each object $\mathit{obj}$, it is easy to see that $\mathit{lin} \uparrow_{\mathit{obj}}$ is consistent with $\mathit{vis} \uparrow_{\mathit{obj}}$. By the definition of t0-specifications, we know that, for each object $\mathit{obj}$, $\mathit{lin} \uparrow_{\mathit{obj}}$ is a linearization of $h \uparrow_{\mathit{obj}}$. Therefore, $h$ is distributed linearizable.

Let us prove that this process terminates with $\mathit{Op}' = \emptyset$ by contradiction: Assume this process terminates with $\mathit{Op}' \neq \emptyset$, then it is easy to see that $\mathit{vis}^*$ has cycle, which contradicts the assumption that $\mathit{vis}^*$ is acyclic. Therefore, this process terminates with $\mathit{Op}' = \emptyset$. $\qed$
\end {proof}





\subsection{Proof of Lemma \ref{lemma:several t0-specifications and one t1-specification can be composed}}
\label{subsec:appendix proofs of lemma several t0-specifications and one t1-specification can be composed}


\composingTZeroAndOneTOne*
\begin {proof}
Assume that $h = (\mathit{Op},\mathit{ro},\mathit{vis})$. Let $\mathit{obj}_1$ be the only object that uses t1-specification, and let $\mathit{objs}_0$ be the set of other objects. We need to prove that, if $h \uparrow_{\mathit{obj}}$ is distributed linearizable for each object $\mathit{obj}$ of $h$, then $h$ is distributed linearizable.

We construct a linearization $\mathit{lin}$ of $h$ in a process as follows:

\begin{itemize}
\setlength{\itemsep}{0.5pt}
\item[-] Initially a set $\mathit{Op}' = \mathit{Op}$ and $\mathit{lin} = \epsilon$.

\item[-] We begin a loop as follows: in each round of the loop, we choose an operation $o$ shown below, and then let $\mathit{Op}' = \mathit{Op}' \setminus \{ o \}$, and let $\mathit{lin} = \mathit{lin} \cdot o$.

    \begin{itemize}
    \setlength{\itemsep}{0.5pt}
    \item[-] either $o$ is of an operation of $\mathit{objs}_0$ and is minimal w.r.t $\mathit{vis}$ in $\mathit{Op}'$,

    \item[-] or $o$ is of an operation of $\mathit{obj}_1$, is minimal w.r.t $\mathit{vis}$ in $\mathit{Op}'$, and has the minimal time-stamp among operations of $\mathit{obj}_1$ in $\mathit{Op}'$.
    \end{itemize}
\end{itemize}

If this process terminates with $\mathit{Op}' = \emptyset$: Then it is easy to see that $\mathit{lin}$ is consistent with $\mathit{vis}$, and thus, for each object $\mathit{obj}$, it is easy to see that $\mathit{lin} \uparrow_{\mathit{obj}}$ is consistent with $\mathit{vis} \uparrow_{\mathit{obj}}$. It is also easy to see that for operation of $\mathit{obj}_1$, $\mathit{lin}$ is consistent with time-stamp. By the definition of t0-specifications, we know that, for each object $\mathit{obj} \in \mathit{objs}$, $\mathit{lin} \uparrow_{\mathit{obj}}$ is a linearization of $h \uparrow_{\mathit{obj}}$. By the definition of t1-specifications, we know that, $\mathit{lin} \uparrow_{\mathit{obj}_1}$ is a linearization of $h \uparrow_{\mathit{obj}_1}$. Therefore, $h$ is distributed linearizable.

Let us prove that this process terminates with $\mathit{Op}' = \emptyset$ by contradiction: Assume this process terminates with $\mathit{Op}' \neq \emptyset$. Let set $S_1 = \{ o' \vert o'$ is minimal w.r.t $\mathit{vis}$ in $\mathit{Op}'$ $\}$. Then, we can see that, for each operation $o \in S_1$, $o$ is of object $\mathit{obj}_1$, and $o$ does not have minimal time-stamps among operations of $\mathit{obj}_1$ in $\mathit{Op}'$. Let $o_0$ be the operation that is of object $\mathit{obj}_1$ and has minimal time-stamp among operations of $\mathit{obj}_1$ in $\mathit{Op}'$. It is obvious that $o_0 \notin S_1$. Therefore, there exists operations $o_1,\ldots,o_k$, such that $o_1 \in S_1$, $o_1$ is of object $\mathit{obj}_1$, $(o_1,o_2),\ldots,(o_k,o_0) \in \mathit{vis}$. Since the visibility is transitive, we have that $(o_1,o_0) \in \mathit{vis}$. We already know that the time-stamp of $o_0$ is less than that of $o_1$. This contradicts the assumption that time-stamp is consistent with visiblity. Therefore, this process terminates with $\mathit{Op}' = \emptyset$. $\qed$

%Let us prove that this process terminates with $\mathit{Op}' = \emptyset$ by contradiction: Assume this process terminates with $\mathit{Op}' \neq \emptyset$. Let set $S_1 = \{ o' \vert o'$ is minimal w.r.t $\mathit{vis}$ in $\mathit{Op}'$ $\}$. Then, we can see that, for each operation $o \in S_1$, $o$ is of object $\mathit{obj}_1$, and $o$ does not have minimal time-stamps among operations of $\mathit{obj}_1$ in $O'$. Thus, let $o_1 \in S_1$ be the operation that has minimal time-stamp among operations in $S_1$. We can see that there exists $o_2 \in O'$, such that $o_2$ is of object $\mathit{obj}_1$, $o_2 \notin S_1$, and the time-stamp of $o_2$ is less than that of $o_1$. Since $o_2 \notin S_1$, we can see that there exists operation $o_3 \in S_1$, such that $(o_3,o_2) \in \mathit{vis}$. We can see that the time-stamp of $o_1$ is less than that of $o_3$, and thus, the time-stamp of $o_2$ is less than that of $o_3$. Therefore, we found that the time-stamp is not consistent with visibility order for $o_2$ and $o_3$, which contradicts the assumption that time-stamp is consistent with visiblity. $\qed$
\end {proof}






\subsection{Proof of Lemma \ref{lemma:several t0-specifications and several t1-specification can be composed}}
\label{subsec:appendix proofs of lemma several t0-specifications and several t1-specification can be composed}


\composingTZeroAndTOne*
\begin {proof}
Assume that $h = (\mathit{Op},\mathit{ro},\mathit{vis})$. Let $\mathit{objs}_0$ be the set of objects that use t0-specifications in $h$, and let $\mathit{objs}_1$ be the set of objects that use t1-specifications in $h$. We need to prove that, if $h \uparrow_{\mathit{obj}}$ is distributed linearizable for each object $\mathit{obj}$ of $h$, then $h$ is distributed linearizable.

We construct a linearization $\mathit{lin}$ of $h$ in a process as follows:

\begin{itemize}
\setlength{\itemsep}{0.5pt}
\item[-] Initially a set $\mathit{Op}' = \mathit{Op}$ and $\mathit{lin} = \epsilon$.

\item[-] We begin a loop as follows: in each round of the loop, we choose an operation $o$ shown below, and then let $\mathit{Op}' = \mathit{Op}' \setminus \{ o \}$, and let $\mathit{lin} = \mathit{lin} \cdot o$.

    \begin{itemize}
    \setlength{\itemsep}{0.5pt}
    \item[-] either $o$ is of an operation of objects in $\mathit{objs}_0$ and is minimal w.r.t $\mathit{vis}$ in $\mathit{Op}'$,

    \item[-] or $o$ is of an operation of object $\mathit{obj}_1 \in \mathit{objs}_1$, is minimal w.r.t $\mathit{vis}$ in $\mathit{Op}'$, and has the minimal time-stamp among operations of $\mathit{obj}_1$ in $\mathit{Op}'$.
    \end{itemize}
\end{itemize}

If this process terminates with $\mathit{Op}' = \emptyset$: Then it is easy to see that $\mathit{lin}$ is consistent with $\mathit{vis}$, and thus, for each object $\mathit{obj}$, it is easy to see that $\mathit{lin} \uparrow_{\mathit{obj}}$ is consistent with $\mathit{vis} \uparrow_{\mathit{obj}}$. It is also easy to see that for each object $\mathit{ojb}_1 \in \mathit{objs}_1$, $\mathit{lin}$ is consistent with time-stamp of $\mathit{obj}_1$. By the definition of t0-specifications, we know that, for each object $\mathit{obj} \in \mathit{objs}_0$, $\mathit{lin} \uparrow_{\mathit{obj}}$ is a linearization of $h \uparrow_{\mathit{obj}}$. By the definition of t1-specifications, we know that, for each object $\mathit{obj}_1 \in \mathit{objs}_1$, $\mathit{lin} \uparrow_{\mathit{obj}_1}$ is a linearization of $h \uparrow_{\mathit{obj}_1}$. Therefore, $h$ is distributed linearizable.

Let us prove that this process terminates with $\mathit{Op}' = \emptyset$ by contradiction: Assume this process terminates with $\mathit{Op}' \neq \emptyset$. Let set $S_1 = \{ o' \vert o'$ is minimal w.r.t $\mathit{vis}$ in $\mathit{Op}'$ $\}$. Then, we can see that, for each operation $o \in S_1$, there exists a object $\mathit{obj}_1 \in \mathit{objs}_1$, such that $o$ is of $\mathit{obj}_1$, and $o$ does not have minimal time-stamps among operations of $\mathit{obj}_1$ in $\mathit{Op}'$.

Let $S_2 = \{ o \vert \exists \mathit{obj}_1 \in \mathit{objs}_1, o$ is of object $\mathit{obj}_1, o$ has minimal time-stamp among operations of $\mathit{obj}_1$ in $\mathit{Op}' \}$. It is easy to see that $\forall o \in S_2$, $o \notin S_1$.

Thus, it is easy to see that, for each operation $o' \in S_2$, there exists an operation $o \in S_1$ and operations $o'_1,\ldots,o'_k$, such that $(o,o'_1),(o'_1,o'_2),\ldots,(o'_k,o') \in \mathit{vis}$. Since the visibility relation is transitive, we have that $(o,o') \in \mathit{vis}$.

Let $S_3 = \{ (o,o') \vert o \in S_1, o' \in S_2, \exists o'_1,\ldots,o'_k, (o,o'_1),(o'_1,o'_2),\ldots,(o'_k,o') \in \mathit{vis} \}$. Let $S_4 = \{ (\mathit{obj},\mathit{obj}') \vert \exists (o,o') \in S_3$, $o$ is of object $\mathit{obj}$, $o'$ is of object $\mathit{obj}' \}$.

Let us prove that there is a cycle in $S_4$ by contradiction. Given $(\mathit{obj}_2,\mathit{obj}_1) \in S_4$, we know that there is a operation of object of $\mathit{obj}_2$ in $S_1$, and thus, there must exists a operation of object of $\mathit{obj}_2$ in $S_2$. By definition of $S_2$, it is easy to see that there exists $\mathit{obj}_3$, such that $(\mathit{obj}_3,\mathit{obj}_2) \in S_4$. Since $S_4$ has no cycle, we applying this process and finally terminate with $(\mathit{obj}_k,\mathit{obj}_{\mathit{k-1}}),\ldots,(\mathit{obj}_2,\mathit{obj}_1) \in S_4$ and could not found any $\mathit{obj}'$ to make $(\mathit{obj}',\mathit{obk}_k) \in S_4$. However, this implies that there is a operation of $\mathit{obj}_k$ that has minimal time-stamp among operations of $\mathit{obj}_k$ in $\mathit{Op}'$, and is in $S_1$. This contradicts our conclusion that $\forall o \in S_2$, $o \notin S_1$. Therefore, this is a cycle in $S_4$.

Let the cycle in $S_4$ be $(\mathit{obj}_1,\mathit{obj}_k),(\mathit{obj}_k,\mathit{obj}_{\mathit{k-1}}),\ldots,(\mathit{obj}_2,\mathit{obj}_1)$. Then, there exists operations $o^{0}_{\mathit{o1}}, o^{1}_{\mathit{o1}},\ldots, o^{0}_{\mathit{ok}}, o^{1}_{\mathit{ok}}$, such that

\begin{itemize}
\setlength{\itemsep}{0.5pt}
\item[-] $o^{0}_{\mathit{o1}}, o^{1}_{\mathit{o1}}$ is of object $\mathit{obj}_1$, $\ldots$, $o^{0}_{\mathit{ok}}, o^{1}_{\mathit{ok}}$ is of object $\mathit{obj}_k$.

\item[-] $(o^{1}_{\mathit{o1}},o^{0}_{\mathit{ok}}), (o^{1}_{\mathit{ok}},o^{0}_{\mathit{ok-1}})$, $\ldots$, $(o^{1}_{\mathit{o2}},o^{0}_{\mathit{o1}}) \in S_3$.
\end{itemize}

Thus, it is easy to see $(o^{1}_{\mathit{o1}},o^{0}_{\mathit{ok}}), (o^{1}_{\mathit{ok}},$ $o^{0}_{\mathit{ok-1}})$, $\ldots$, $(o^{1}_{\mathit{o2}},o^{0}_{\mathit{o1}}) \in \mathit{vis}$. By definition of $S_2$, we can see that $\mathit{ts}(o^{0}_{\mathit{o1}}) < \mathit{ts}(o^{1}_{\mathit{o1}}), \ldots, \mathit{ts}(o^{0}_{\mathit{ok}}) < \mathit{ts}(o^{1}_{\mathit{ok}})$. This contradicts the definition of causal-time-stamp. Therefore, this process terminates with $\mathit{Op}' = \emptyset$. $\qed$
\end {proof}










\section{For State-based CRDT}
\label{sec:for state-based CRDT}

\begin{example}[List with add-between interface]
\label{definition:sequential specification of list with add-after interface}
Such kind of list is similar as list with add-after interface. One difference is the $\mathit{add}$ method: $\mathit{add}(b,a,c)$ inserts item $b$ into the list at some nondeterministic position between position of $a$ and position of $c$. The other difference is that, we assume that the initial value of list is $(\circ_1,\mathit{true}) \cdot (\circ_2,\mathit{true})$ and these two nodes can not be removed. The sequential specification $\mathit{list}_s^{\mathit{ab}}$ of list is given as follows: Here $\mathit{ab}$ represents add-between. When the context is clear, in $\mathit{read}$ operation, we will omit $\circ_1$ and $\circ_2$.
\begin{itemize}
\setlength{\itemsep}{0.5pt}
\item[-] $\{ \mathit{state} = (a_1,f_1) \cdot \ldots \cdot (a_n,f_n) \wedge k < m < l \wedge b \notin \{ a_1, \ldots, a_n \} \}$ $add(b,a_k,a_l)$ $\{ \mathit{state} = (a_1,f_1) \cdot \ldots \cdot (a_m,f_m) \cdot (b,\mathit{true}) \cdot (a_{m+1},f_{m+1}) \cdot \ldots \cdot (a_n,f_n) \}$. Here the chosen of $m$ is deterministic.
\item[-] $\{ \mathit{state} = (a_1,f_1) \cdot \ldots \cdot (a_n,f_n) \wedge S = \{ a \vert (a,\mathit{true}) \in \mathit{state} \} \wedge l = a_1 \cdot \ldots \cdot a_n \uparrow_{S} \}$ $(read() \Rightarrow l)$ $\{ \mathit{state} = (a_1,f_1) \cdot \ldots \cdot (a_n,f_n) \}$.
\end{itemize}
\end{example}










Given a object $\mathit{obj}$ of a state-based CRDT with $\Sigma$ be the set of local states, we define its semantics as a set of executions generated from an LTS $\llbracket \mathit{obj} \rrbracket_s = (\mathit{Config},\mathit{config}_0,\Sigma',\rightarrow)$ as in \figurename~\ref{fig:the semantics of a state-based CRDT object}.

\begin{figure}[ht]
$\mathit{RState} = \mathbb{R} \rightarrow \Sigma$

$\mathit{TState} = \mathbb{MID} \times \mathbb{MSG} \times \mathbb{R}$.

$\mathit{Config} = \mathit{RState} \times \mathit{TState}$, $\mathit{config}_0 \in \mathit{Config}$.

$\Sigma' = \mathit{do}(\mathbb{M} \times \mathbb{D} \times \mathbb{D} \times \mathbb{R}) \cup \mathit{send}(\mathbb{MID} \times \mathbb{R}) \cup \mathit{receive}(\mathbb{MID} \times \mathbb{R})$

\[
\begin{array}{l c}
\bigfrac{ R(r) = \sigma, r.\mathit{do}(\sigma,m,a) = (\sigma',b) }
{ (R,T) {\xrightarrow{\mathit{do}(m,a,b,r)}} (R[r:\sigma'],T) }
\end{array}
\]


\[
\begin{array}{l c}
\bigfrac{ R(r) = \sigma, \mathit{unique}(\mathit{mid}) }
{ (R,T) {\xrightarrow{\mathit{send}(\mathit{mid},r)}} (R,T \cup \{ (\mathit{mid},\sigma,r) \}) }
\end{array}
\]


\[
\begin{array}{l c}
\bigfrac{ R(r) = \sigma, r.\mathit{receive}(\sigma,\sigma') = \sigma'',(\mathit{mid},\sigma',r') \in T, r \neq r'}
{ (R,T) {\xrightarrow{\mathit{receive}(\mathit{mid},r)}} (R[r:\sigma''],T) }
\end{array}
\]
\caption{The definition of semantics of $\llbracket \mathit{obj} \rrbracket_s$}
\label{fig:the semantics of a state-based CRDT object}
\end{figure}

A configuration $(R,T)$ is a snapshot of distributed system and contains two parts: $R$ gives the local state of each replica, and $T$ gives the set of messages that has been generated. Let $\mathbb{MID}$ be the set of message identifiers of message content. A message is a tuple $(\mathit{mid},\mathit{msg},r)$, where $\mathit{mid} \in \mathbb{MID}$ is the identifier, $\mathit{msg} \in \mathbb{MSG}$ is the message content, and $r$ is the original replica of message. $\mathit{config}_0$ is the initial configuration, which maps each replica into the initial local state, and has no message inside. Since $\mathit{obj}$ is a state-based CRDT, each message content is chosen from $\Sigma$.

Each element of $\Sigma'$ is called an action. $\rightarrow \in \mathit{Config} \times \Sigma' \times \mathit{Config}$ is the transition relation and describe a single step of distributed systems. The first rule in \figurename~\ref{fig:the semantics of a state-based CRDT object} describes replica $r$ performs a operation $m(a) \Rightarrow b$ and works locally. The second rule describes that a replica $r$ may nondeterministically decide to send a message with its local state as message content. Here $\mathit{unique}$ is a function that ensures $\mathit{mid}$ be a fresh message identifier. The third rule describes delivery of a message to a replica $r$ other than its origin replica $r'$.

A sequence $l$ of actions is an execution of $\llbracket \mathit{obj} \rrbracket_s = (\mathit{Config},\mathit{config}_0,\Sigma',\rightarrow)$, if there exists $(R,T) \in \mathit{Config}$, such that $\mathit{config}_0 {\xrightarrow{ l }} (R,T)$. The semantics of $\mathit{obj}$ is defined as the set of executions of $\llbracket \mathit{obj} \rrbracket_s$. Given an execution, when the context is clear, we can associate a unique operation identifier to each action. Or we can say, it is safe to assume each action is in the form of either $\mathit{do}(i,m,a,b,r)$, or $\mathit{send}(i,\mathit{mid},r)$, or $\mathit{receive}(i,\mathit{mid},r)$, where $i \in \mathbb{OID}$ is a unique operation identifier.








Given an execution $l = \alpha_1 \cdot \ldots \cdot \alpha_k$ of $\llbracket \mathit{obj} \rrbracket_s$ of state-based CRDT $\mathit{obj}$, we can obtain a corresponding history $\mathit{history}(l) = (\mathit{Op},\mathit{ro},\mathit{vis})$, such that

\begin{itemize}
\setlength{\itemsep}{0.5pt}
\item[-] Each operation in $\mathit{Op}$ is a tuple $(\ell,i,\mathit{obj})$, such that $i$ is the operation identifier of a $\mathit{do}(m,a,b,r)$ action of $l$.

\item[-] $(o_1,o_2) \in \mathit{ro}$, if they are of same replica, and the index of $o_1$ in $h$ is before that of $o_2$.

\item[-] Let us defines a delivery relation $\mathit{del} \subseteq \mathbb{OP} \times \mathbb{OP}$ as follows: $(o_1,o_2) \in \mathit{del}$, if: $o_1$ and $o_2$ are of different replicas, there exists a $\mathit{send}(\mathit{mid},r)$ action and a $\mathit{receive}(\mathit{mid},r')$ action, $o_1$ and $\mathit{send}(\mathit{mid},r)$ happen on a same replica and $o_1$ happens earlier, $\mathit{receive}(\mathit{mid},r)$ and $o_2$ happen on a same replica and $\mathit{receive}(\mathit{mid},r)$ happens earlier.

\item[-] $\mathit{vis} = (\mathit{ro}+\mathit{del})^*$.
\end{itemize}

Intuitively, each local state can be considered as the consequence of all updates it receives. Since state-based CRDT sends the modified local state as message, the visibility relation is then the transitive closure of replica order and message delivery relation. Let $\mathit{history}(\llbracket \mathit{obj} \rrbracket_s)$ be the set of histories of all executions of $\llbracket \mathit{obj} \rrbracket_s$.






\subsection{Proof Strategy of State-based CRDT}
\label{subsec:proof strategy of operation-based CRDT}

Given a state-based CRDT object $\mathit{obj}$ and a sequential specification $\mathit{spec}$, we need to construct a invariant $\mathit{inv}(\mathit{config},h,\mathit{lin},\mathit{del},\mathit{map})$, where

\begin{itemize}
\setlength{\itemsep}{0.5pt}
\item[-] $\mathit{config}$ is a configuration of $\llbracket \mathit{obj} \rrbracket_s$.

\item[-] $h$ is a history.

\item[-] $h$ is distributed linearizable w.r.t $\mathit{spec}$ and $\mathit{lin}$ is a linearization.

\item[-] $\mathit{del} \subseteq \mathbb{MID} \times \mathbb{R}$ is the message delivery relation.

\item[-] $\mathit{map} \subseteq \mathbb{MID} \times 2^{\mathbb{OID}}$ maps each message $\mathit{mid}$ to a set $S_1$ of operations. Intuitively, $S_1$ is the set of operations whose information are contained in $\mathit{mid}$.
\end{itemize}

$\mathit{inv}(\mathit{config},h,\mathit{lin},\mathit{del},\mathit{map})$ needs to satisfy the following properties:

\begin{itemize}
\setlength{\itemsep}{0.5pt}
\item[-] The visibility of $h$ is transitive.

\item[-] $\mathit{del}$ preserves causal delivery: If $(o_1,o_2) \in \mathit{vis}$ and $(o_2,r) \in \mathit{del}$, then $(o_1,r) \in \mathit{del}$.

\item[-] $\mathit{map}$ preserves causal delivery: Given $o_1,o_3 \in \mathit{map}(\mathit{mid})$, if $\exists o_2$, such that $(o_1,o_2),(o_2,o_3) \in \mathit{vis}$, then $o_2 \in \mathit{map}(\mathit{mid})$.

\item[-] $\mathit{inv}$ holds initially: $\mathit{inv}(\mathit{config}_0,\epsilon,\emptyset,\emptyset,\emptyset)$ holds, where $\mathit{config}_0$ is the initial configuration of $\llbracket \mathit{obj} \rrbracket_s$.

\item[-] $\mathit{inv}$ is a transition invariant:

    \begin{itemize}
    \setlength{\itemsep}{0.5pt}
    \item[-] If $\mathit{inv}(\mathit{config},h,\mathit{lin},\mathit{del},\mathit{map})$ holds and $\mathit{config} {\xrightarrow{\mathit{do}(m,a,b,r)}} \mathit{config}'$, then $\mathit{inv}(\mathit{config}', h \otimes i, \mathit{lin} \cdot i,\mathit{del},\mathit{map})$ holds. Note that here we always put a new operation in the last of linearization.

        Here $i$ is the identifier of the newly-generated $\mathit{do}$ action. Given $h = (\mathit{Op},\mathit{ro},\mathit{vis})$, then, $h \otimes i = (\mathit{Op}',\mathit{ro}',\mathit{vis}')$, where $\mathit{Op}' = \mathit{Op} \cup \{ (m(a) \Rightarrow b,i,\mathit{obj}) \}$, $\mathit{ro}' = \mathit{ro} \cup \{ (j,i) \vert j \in \mathit{Op}, j$ is of replica $r \}$, and $\mathit{vis}' = (\mathit{vis} \cup \{ (j,i) \vert j \in \mathit{Op},(j,r) \in \mathit{del} \} \cup \{ (j,i) \vert j \in \mathit{Op}, j$ is of replica $r \})^*$.

    \item[-] If $\mathit{inv}(\mathit{config},h,\mathit{lin},\mathit{del},\mathit{map})$ holds and $\mathit{config} {\xrightarrow{\mathit{send}(\mathit{mid},r)}} \mathit{config}'$, then $\mathit{inv}(\mathit{config}',h,\mathit{lin},\mathit{del},\mathit{map}')$ holds, where $\mathit{map}' = \mathit{map} \cup (\mathit{mid}, \mathit{vd}(h,\mathit{del},r))$.


    \item[-] If $\mathit{inv}(\mathit{config},h,\mathit{lin},\mathit{del},\mathit{map})$ holds and $\mathit{config} {\xrightarrow{\mathit{receive}(\mathit{mid},r)}} \mathit{config}'$, then $\mathit{inv}(\mathit{config}',h,\mathit{lin},\mathit{del}',\mathit{map})$ holds, where $\mathit{del}' = \mathit{del} \cup \{ (i,r) \vert i \in \mathit{map}(\mathit{mid}) \}$.
    \end{itemize}
\end{itemize}

Here $\mathit{vd}(h,\mathit{del},r) = \{ i \vert (i,j) \in h.\mathit{vis}, j$ is of replica $r \} \cup \{ i \vert (i,r) \in \mathit{del} \}$ is the set of operations that are either to some operation of replica $r$, or has been delivered into replica $r$. An invariant $\mathit{inv}$ satisfies above properties is called invariant of state-based CRDT. The following lemma states that the existence of such invariant implies distributed linearizability.

\begin{lemma}
\label{lemma:invariant of state-based CRDT implies distributed linearizability}
If there exists a invariant $\mathit{inv}$ of state-based CRDT for object $\mathit{obj}$ and sequential specification $\mathit{spec}$, then each history of $\mathit{history}(\llbracket \mathit{obj} \rrbracket_s)$ is distributed linearizable w.r.t $\mathit{spec}$.
\end{lemma}

\begin {proof}
Given an execution $l=\alpha_1 \cdot \ldots \cdot \alpha_n$, let $\mathit{config}_0 {\xrightarrow{\alpha_1}} \mathit{config}_1 \ldots {\xrightarrow{\alpha_n}} \mathit{config}_n$ be the transitions from initial configuration. We need to prove that, for each $1 \leq k \leq n$, we have $\mathit{inv}(\mathit{config}_k,h_k,\mathit{lin}_k,\mathit{del}_k,\mathit{map}_k)$ holds, where $h_k$ is the history of execution $l_k = \alpha_1 \cdot \ldots \cdot \alpha_k$, $\mathit{lin}_k$ is the linearization of $h_k$, $\mathit{del}_k$ records message delivery relation of $l_k$, and $\mathit{map}_k$ records the operations contained in each message in $l_k$.

Since $\mathit{inv}$ holds initially and is a transition invariant, it is easy to prove above requirements by induction on execution. This completes the proof of this lemma. $\qed$
\end {proof}


For many state-based CRDT implementations, $\mathit{inv}((R,T),h,\mathit{lin},\mathit{del},\mathit{map}) = C_1 \wedge C_2$, where

\begin{itemize}
\setlength{\itemsep}{0.5pt}
%\item[-] For each update operation $o$ of $h$, define $\mathit{ds}(o)$ which is a local state. %be the local state of replica $r$ at the time point immediately after $o$ is launched. Here $r$ is the replica of $o$.

\item[-] $C_1: \forall (\mathit{mid},\mathit{msg},\_) \in T$, $\mathit{msg} = \mathit{apply}(\mathit{lin},\mathit{map}(\mathit{mid}))$.

\item[-] $C_2: \forall r$, $R(r) = \mathit{apply}(\mathit{lin},\mathit{vd}(h,\mathit{del},r))$.
\end{itemize}

The function $\mathit{apply}(\mathit{lin},S)$ returns a local state by applying ``virtual messages'' of operations in $S$ according to total order $\mathit{lin}$. Here for each update operation $o$ of $h$, we need to define a local state $\mathit{ds}(o)$, which is the ``virtual messages'' of $o$. Note that state-based CRDT send message randomly, instead of each message for a update operation. This is the reason why we need to manually generate virtual message for each update operation.

To give $\mathit{inv}$, it only remains to give the virtual messages. The virtual message of state-based PN-counter and state-based multi-value register as follows. The proof of them being invariants of state-based CRDT is given in Appendix \ref{subsec:appendix proof of state-based PN-counter} and Appendix \ref{subsec:appendix proof of state-based multi-value register}, respectively.

\begin{example}[virtual messages of state-based PN-counter]
\label{example:virtual messagess of state-based PN-counter}

For each update operation $o$, $\mathit{ds}(o) = (P,N)$, where

\begin{itemize}
\setlength{\itemsep}{0.5pt}
\item[-] $\forall r'$, $P[r'] = \vert \{ o' \vert o'$ is a $\mathit{inc}$ operation of replica $r'$, $o' = o \vee (o',o) \in h.\mathit{vis} \} \vert$.

\item[-] $\forall r'$, $N[r'] = \vert \{ o' \vert o'$ is a $\mathit{dec}$ operation of replica $r'$, $o' = o \vee (o',o) \in h.\mathit{vis} \} \vert$.
\end{itemize}
\end{example}

\begin{example}[virtual messages of state-based Multi-value Register]
\label{example:virtual messages of state-based multi-value register}

For each update operation $o = (\mathit{write}(a),\_,\_)$ of replica $r$, $\mathit{ds}(o) = (a,V)$, where

\begin{itemize}
\setlength{\itemsep}{0.5pt}
\item[-] $\forall r'$, $V[r'] = \vert \{ o' \vert o'$ is a $\mathit{write}$ operation of replica $r'$, $o' = o \vee (o',o) \in h.\mathit{vis} \} \vert$.
\end{itemize}
\end{example}















\subsection{Proof of State-based PN-counter}
\label{subsec:appendix proof of state-based PN-counter}

The following lemma states that each visibility-closed set is a union of operations visible to a set of operations. Its proof is obvious and omitted here.

\begin{lemma}
\label{lemma:a transitive-closed set is a union of visibility of several sets}
Given a set $\mathit{Op}$ of operations and a transitive and acyclic visibility relation $\mathit{vis} \subseteq \mathit{Op} \times \mathit{Op}$, if given a set $S \subseteq \mathit{Op}$, if $S$ satisfies that $\forall o_1,o_2 \in S, o_2 \in S \wedge (o_1,o_2) \in \mathit{vis} \Rightarrow o_1 \in S$, then there exists a set $O \subseteq \mathit{Op}$, such that $S = \cup_{o \in O} \mathit{vis}^{-1}(o)$.
\end{lemma}

The following lemma states that given two operations $o_1,o_2$, for each replica $r$, either the set of operations of replica $r$ visible to $o_1$ is a subset of that of $o_2$, or the set of operations of replica $r$ visible to $o_2$ is a subset of that of $o_1$. Its proof is obvious and omitted here.

\begin{lemma}
\label{lemma:the view of a replica of one operation is contained in another operaiton, or vice versa}
Assume that $\mathit{inv}((R,T),h,\mathit{lin},\mathit{del},\mathit{map})$ holds. Let $S_o^r = \{ o' \vert (o',o) \in \mathit{vis}, o'$ is of replica $r \}$. Then for each operations $o_1$ and $o_2$, and for each replica $r$, $S_{\mathit{o1}}^r \subseteq S_{\mathit{o2}}^r \vee S_{\mathit{o2}}^r \subseteq S_{\mathit{o1}}^r$.
\end{lemma}


Recall that $\mathit{inv} = C_1 \wedge C_2$ with the virtual messages defined as follows: For each update operation $o$, $\mathit{ds}(o) = (P,N)$, where

\begin{itemize}
\setlength{\itemsep}{0.5pt}
\item[-] $\forall r'$, $P[r'] = \vert \{ o' \vert o'$ is a $\mathit{inc}$ operation of replica $r'$, $o' = o \vee (o',o) \in h.\mathit{vis} \} \vert$.

\item[-] $\forall r'$, $N[r'] = \vert \{ o' \vert o'$ is a $\mathit{dec}$ operation of replica $r'$, $o' = o \vee (o',o) \in h.\mathit{vis} \} \vert$.
\end{itemize}

The following lemma states that $\mathit{inv}$ is an invariant of state-based PN-counter.

\begin{lemma}
\label{lemma:inv is an invariant of state-based CRDT for state-based PN-counter}
$\mathit{inv}$ is an invariant of state-based PN-counter.
\end{lemma}

\begin {proof}

It is obvious that $\mathit{inv}(\mathit{config}_0,\epsilon,\emptyset,\emptyset,\emptyset)$ holds.

Let us prove that $\mathit{inv}$ is a transition invariant: assume $\mathit{inv}((R,T),h,\mathit{lin},\mathit{del},\mathit{map})$ holds,

\begin{itemize}
\setlength{\itemsep}{0.5pt}
\item[-] If $(R,T) {\xrightarrow{\mathit{do}(\mathit{inc},r)}} (R',T')$: Then,

    \begin{itemize}
    \setlength{\itemsep}{0.5pt}
    \item[-] It is easy to see that $R' = R[ r: ( R(r).P[r: R(r).P(r)+1 ], R(r).N ) ]$ and $T' = T$.

    \item[-] Let $h' = h \otimes i$, where $i$ is the identifier of the newly-generated $\mathit{inc}$ action.

    \item[-] Let $\mathit{lin}' = \mathit{lin} \cdot (\mathit{inc},i,\mathit{obj})$.

    \item[-] Let $\mathit{del}' = \mathit{del}$ and $\mathit{map}' = \mathit{map}$.
    \end{itemize}

    It is easy to see that $\mathit{lin}'$ is a linearization of $h'$. It is obvious that all other properties hold, except for $C_2$ for replica $r$. Therefore, let us prove that $R'(r) = \mathit{apply}(\mathit{lin}',\mathit{vd}(h',\mathit{del}',r))$.

    Since $R(r) = \mathit{apply}(\mathit{lin},\mathit{vd}(h,\mathit{del},r))$ and $\mathit{lin}' = \mathit{lin} \cdot (\mathit{inc},i,\mathit{obj})$, we know that $\mathit{apply}(\mathit{lin}',\mathit{vd}(h',\mathit{del}',r)) = \mathit{merge}(R(r),\mathit{ds}(i))$. Therefore, we need to prove that $R'(r) = \mathit{merge}(R(r),\mathit{ds}(i))$.

    Since $\mathit{vd}(h,\mathit{del},r)$ satisfies that, $\forall o_1,o_2 \in \mathit{vd}(h,\mathit{del},r), o_2 \in \mathit{vd}(h,\mathit{del},r) \wedge (o_1,o_2) \in \mathit{vis} \Rightarrow o_1 \in \mathit{vd}(h,\mathit{del},r)$, by Lemma \ref{lemma:a transitive-closed set is a union of visibility of several sets}, we know that there exists a set $O$, such that $\mathit{vd}(h,\mathit{del},r) = \cup_{o \in O} \mathit{vis}^{-1}(o)$. By Lemma \ref{lemma:the view of a replica of one operation is contained in another operaiton, or vice versa} and the construction of $\mathit{ds}$, we can see that $R(r) = (P',N')$, where for each replica $r'$, $P'[r'] = \vert \{ j \in \mathit{vd}(h,\mathit{del},r) \uparrow_{\mathit{inc}}$ and $j$ is of replica $r \} \vert$ and $N'[r'] = \vert \{ j \in \mathit{vd}(h,\mathit{del},r) \uparrow_{\mathit{dec}}$ and $j$ is of replica $r \} \vert$.

    We already know that $\mathit{ds}(i) = (P'',N'')$, where for each replica $r'$, $P''[r'] = \vert \{ j \in \mathit{vd}(h',\mathit{del}',r) \uparrow_{\mathit{inc}}$ and $j$ is of replica $r \} \vert$ and $N''[r'] = \vert \{ j \in \mathit{vd}(h',\mathit{del}',r) \uparrow_{\mathit{dec}}$ and $j$ is of replica $r \} \vert$. Then, it is obvious that $\mathit{merge}(R(r),\mathit{ds}(i)) = \mathit{ds}(i)$. It is also easy to see that $\mathit{ds}(i) = (R(r).P[r: R(r).P(r)+1], R(r).N) = R'(r)$. Therefore, $R'(r) = \mathit{merge}(R(r),\mathit{ds}(i))$.

\item[-] If $(R,T) {\xrightarrow{\mathit{do}(\mathit{dec},r)}} (R',T')$: Similarly as that of $(R,T) {\xrightarrow{\mathit{do}(\mathit{inc},r)}} (R',T')$.

\item[-] If $(R,T) {\xrightarrow{\mathit{do}(\mathit{read},k,r)}} (R',T')$: Then,

    \begin{itemize}
    \setlength{\itemsep}{0.5pt}
    \item[-] It is obvious that $R' = R$ and $T' = T$.

    \item[-] Let $h' = h \otimes i$, where $i$ is the identifier of the newly-generated $\mathit{read}$ action.

    \item[-] Let $\mathit{lin}' = \mathit{lin} \cdot ((\mathit{read}() \Rightarrow k,i,\mathit{obj}), \mathit{vd}(h,\mathit{del},r) )$.

    \item[-] Let $\mathit{del}' = \mathit{del}$ and $\mathit{map}' = \mathit{map}$.
    \end{itemize}

    It is easy to see that all other properties hold, except for $h'$ being distributed linearizable w.r.t $\mathit{spec}$ with $\mathit{lin}'$ the linearization. Let us prove that $h'$ is distributed linearizable w.r.t $\mathit{spec}$ and $\mathit{lin}'$ is a linearization. It is easy to see that only operation $i$ need to be checked.

    It is easy to see that $\mathit{vd}(h,\mathit{del},r) = \mathit{vis}^{-1}(i)$. Similarly as the case of $(R,T) {\xrightarrow{\mathit{do}(\mathit{inc},r)}} (R',T')$, we can prove that $R(r) = (P',N')$, where for each replica $r'$, $P'[r'] = \vert \{ j \in \mathit{vd}(h,\mathit{del},r) \uparrow_{\mathit{inc}}$ and $j$ is of replica $r \} \vert = \vert \{ j \in \mathit{vis}^{-1}(i) \uparrow_{\mathit{inc}}$ and $j$ is of replica $r \} \vert$ and $N'[r'] = \vert \{ j \in \mathit{vd}(h,\mathit{del},r) \uparrow_{\mathit{dec}}$ and $j$ is of replica $r \} \vert = \vert \{ j \in \mathit{vis}^{-1}(i) \uparrow_{\mathit{dec}}$ and $j$ is of replica $r \} \vert$. Since $k = \Sigma_{r'} P��[r'] - \Sigma_{r'} N'[r']$, $k$ is obtained by minus the number of all visible $\mathit{dec}$ of $i$ from the number of all visible $\mathit{inc}$ of $i$. Therefore, we can see that $((\mathit{read}() \Rightarrow k,i,\mathit{obj}), \mathit{vd}(h,\mathit{del},r) )$ of $\mathit{lin}'$ is ``correct''. Then, $h'$ is distributed linearizable w.r.t $\mathit{spec}$ and $\mathit{lin}'$ is a linearization.

\item[-] If $(R,T) {\xrightarrow{\mathit{send}(\mathit{mid},r)}} (R',T')$: Then,

    \begin{itemize}
    \setlength{\itemsep}{0.5pt}
    \item[-] It is obvious that $R' = R$. Let $T' = T \cup \{ (\mathit{mid},R(r),r) \}$.

    \item[-] Let $h' = h$.

    \item[-] Let $\mathit{lin}' = \mathit{lin}$.

    \item[-] Let $\mathit{del}' = \mathit{del}$.

    \item[-] Let $\mathit{map}' = \mathit{map} \cup \{ (\mathit{mid},\mathit{vd}(h,\mathit{del},r)) \}$.
    \end{itemize}

    It is easy to see that all other properties hold, except for checking $C_1$ for $\mathit{mid}$. This holds obviously since the message content of message $\mathit{mid}$ is $R(r)$, and we already know that $R(r) = \mathit{apply}(\mathit{lin},\mathit{vd}(h,\mathit{del},r)) = \mathit{apply}(\mathit{lin},\mathit{map}(\mathit{mid}))$.

\item[-] If $(R,T) {\xrightarrow{\mathit{receive}(\mathit{mid},r)}} (R',T')$: Then,

    \begin{itemize}
    \setlength{\itemsep}{0.5pt}
    \item[-] Let $R' = R[ r: \mathit{merge}(R(r),\mathit{msg})]$ where $(\mathit{mid},\mathit{msg},\_) \in T$. It is obvious that $T' = T$.

    \item[-] Let $h' = h$.

    \item[-] Let $\mathit{lin}' = \mathit{lin}$.

    \item[-] Let $\mathit{del}' = \mathit{del} \cup \{ (i,r) \vert i \in \mathit{map}(\mathit{mid}) \}$.

    \item[-] Let $\mathit{map}' = \mathit{map}$.
    \end{itemize}

    It is easy to see that all other properties hold, except for $C_2$ for replica $r$. Therefore, let us prove that $R'(r) = \mathit{apply}(\mathit{lin}',\mathit{vd}(h',\mathit{del}',r))$.

    We already know that $R'(r) = \mathit{merge}(R(r), \mathit{msg})$, $R(r) = \mathit{apply}(\mathit{lin},\mathit{vd}(h,\mathit{del},r))$ and $\mathit{msg} = \mathit{apply}(\mathit{lin},\mathit{map}(\mathit{mid}))$. It is easy to see that $\mathit{vd}(h',\mathit{del}',r) = \mathit{vd}(h,\mathit{del},r) \cup \mathit{map}(\mathit{mid})$. It is easy to prove that, applying messages in any order lead to the same consequence. Therefore, we have $\mathit{merge}(R(r), \mathit{msg}) = \mathit{apply}(\mathit{lin}',\mathit{vd}(h,\mathit{del},r) \cup \mathit{map}(\mathit{mid}))$. Then, we have $R'(r) = \mathit{apply}(\mathit{lin}',\mathit{vd}(h',\mathit{del}',r))$.
\end{itemize}

This completes the proof of this lemma. $\qed$
\end {proof}




\subsection{Proof of State-based Multi-value Register}
\label{subsec:appendix proof of state-based multi-value register}

Recall that $\mathit{inv} = C_1 \wedge C_2$ with the virtual messages defined as follows: For each update operation $o$, $\mathit{ds}(o) = (a,V)$, where

\begin{itemize}
\setlength{\itemsep}{0.5pt}
\item[-] $\forall r'$, $V[r'] = \vert \{ o' \vert o'$ is a $\mathit{write}$ operation of replica $r'$, $o' = o \vee (o',o) \in h.\mathit{vis} \} \vert$.
\end{itemize}

The following lemma states that $\mathit{inv}$ is an invariant of state-based multi-value register.

\begin{lemma}
\label{lemma:inv is an invariant of state-based CRDT for state-based multi-value register}
$\mathit{inv}$ is an invariant of state-based multi-value register.
\end{lemma}

\begin {proof}

It is obvious that $\mathit{inv}(\mathit{config}_0,\epsilon,\emptyset,\emptyset,\emptyset)$ holds.

Let us prove that $\mathit{inv}$ is a transition invariant: assume $\mathit{inv}((R,T),h,\mathit{lin},\mathit{del},\mathit{map})$ holds,

\begin{itemize}
\setlength{\itemsep}{0.5pt}
\item[-] If $(R,T) {\xrightarrow{\mathit{do}(\mathit{write},a,r)}} (R',T')$: Then,

    \begin{itemize}
    \setlength{\itemsep}{0.5pt}
    \item[-] $R' = R[ r: \{ (a,V') \} ], R(r).N)]$ and $T' = T$. Here $\forall r' \neq r, V'[r'] = \mathit{max} \{ V_1(r��) \vert (\_,V_1) \in R(r) \}$, and $V'[r] = \mathit{max} \{ V_1(r) \vert (\_,V_1) \in R(r) \} + 1$.

    \item[-] Let $h' = h \otimes i$, where $i$ is the identifier of the newly-generated $\mathit{inc}$ action.

    \item[-] Let $\mathit{lin}' = \mathit{lin} \cdot (\mathit{inc},i,\mathit{vis}^{-1}(i))$.

    \item[-] Let $\mathit{del}' = \mathit{del}$ and $\mathit{map}' = \mathit{map}$.
    \end{itemize}

    It is easy to see that $\mathit{lin}'$ is a linearization of $h'$. It is obvious that all other properties hold, except for $C_2$ for replica $r$. Therefore, let us prove that $R'(r) = \mathit{apply}(\mathit{lin}',\mathit{vd}(h',\mathit{del}',r))$.

    It is easy to see that $\mathit{vd}(h',\mathit{del}',r) = h'.\mathit{vis}^{-1}(i)$. And then, we need to prove that $(a,V') = \mathit{apply}(\mathit{lin}',h'.\mathit{vis}^{-1}(i))$.

    Recall that $R(r) = \mathit{apply}(\mathit{lin},\mathit{vd}(h,\mathit{del},r))$, from Lemma \ref{lemma:a transitive-closed set is a union of visibility of several sets}, we know that there exists set $O$, such that $\mathit{vd}(h,\mathit{del},r) = \cup_{o \in O} \mathit{vis}^{-1}(o)$. We can prove that, for each $o = \mathit{write}(b)$, $\mathit{apply}(\mathit{lin},\mathit{vis}^{-1}(o)) = (b,V_b)$, where $\forall r' \neq r, V_b[r'] = \vert \{ o' \vert o' \in \mathit{vis}^{-1}(o), o'$ is of replica $r' \} \vert$, and $V_b[r] = \vert \{ o' \vert o' \in \mathit{vis}^{-1}(o), o'$ is of replica $r' \} \vert + 1$.

    It is not hard to prove that the order of merging virtual message is not important, and a virtual message can be applied multiple times. By Lemma \ref{lemma:the view of a replica of one operation is contained in another operaiton, or vice versa}, we can see that $\mathit{apply}(\mathit{lin},\mathit{vd}(h,\mathit{del},r))$ is obtained by merging $\{ o \in O \vert \mathit{apply}(\mathit{lin},\mathit{vis}^{-1}(o)) \}$. Therefore, we can see that $\mathit{apply}(\mathit{lin}',h'.\mathit{vis}^{-1}(i)) = \mathit{apply}(\mathit{lin}',\mathit{vd}(h',\mathit{del}',r))$ is obtained by merging $\{ o \in O \vert \mathit{apply}(\mathit{lin},\mathit{vis}^{-1}(o)) \} \cup \{ \mathit{ds}(i) \}$. By Lemma \ref{lemma:the view of a replica of one operation is contained in another operaiton, or vice versa}, it is not hard to see that $\mathit{apply}(\mathit{lin}',h'.\mathit{vis}^{-1}(i)) = \mathit{ds}(i)$.

    Then, we need to prove that $(a,V') = \mathit{ds}(i)$. This holds since $R(r) = \mathit{apply}(\mathit{lin},\mathit{vd}(h,\mathit{del},r))$ is obtained by merging $\{ o \in O \vert \mathit{apply}(\mathit{lin},\mathit{vis}^{-1}(o)) \}$, Lemma \ref{lemma:the view of a replica of one operation is contained in another operaiton, or vice versa}, and the value of $V'$.

\item[-] If $(R,T) {\xrightarrow{\mathit{do}(\mathit{read},S,r)}} (R',T')$: Then,

    \begin{itemize}
    \setlength{\itemsep}{0.5pt}
    \item[-] It is obvious that $R' = R$ and $T' = T$.

    \item[-] Let $h' = h \otimes i$, where $i$ is the identifier of the newly-generated $\mathit{read}$ action.

    \item[-] Let $\mathit{lin}' = \mathit{lin} \cdot (\mathit{read}() \Rightarrow S,i,\mathit{vd}(h,\mathit{del},r) )$.

    \item[-] Let $\mathit{del}' = \mathit{del}$ and $\mathit{map}' = \mathit{map}$.
    \end{itemize}

    It is easy to see that all other properties hold, except for $h'$ being distributed linearizable w.r.t $\mathit{spec}$ with $\mathit{lin}'$ the linearization. Let us prove that $h'$ is distributed linearizable w.r.t $\mathit{spec}$ and $\mathit{lin}'$ is a linearization. It is easy to see that only operation $i$ need to be checked.

    It is easy to see that $\mathit{vd}(h,\mathit{del},r) = h'.\mathit{vis}^{-1}(i)$. Similarly as the case of $(R,T) {\xrightarrow{\mathit{do}(\mathit{write},a,r)}} (R',T')$, we can prove that there exists a set $O$, such that $R(r) = \mathit{apply}(\mathit{lin},\mathit{vd}(h,\mathit{del},r))$ is obtained by merging $\{ o \in O \vert \mathit{apply}(\mathit{lin},\mathit{vis}^{-1}(o)) \}$.

    By the definition of merging, it is same to assume that $O = \mathit{max}_{\mathit{vis}} \mathit{vd}(h,\mathit{del},r)$. Assume that for each operation $o = \mathit{write}(a) \in O$, $\mathit{apply}(\mathit{lin},\mathit{vis}^{-1}(o))) = (a,V_o)$. Then it is not hard to see that $R(r) = \{ (a,V_o) \vert o = \mathit{write}(a) \in O \}$. Therefore, $S = \{ a \vert o = \mathit{write}(a) \in \mathit{vis}^{-1}(i), \forall o' = \mathit{write}(\_) \in \mathit{vis}^{-1}(i), (o,o') \notin \mathit{vis} \}$. According to sequential specification $\mathit{spec}$, $(\mathit{read} \Rightarrow S,i,\mathit{obj})$ of $\mathit{lin}'$ is ``correct''. Then, $h'$ is distributed linearizable w.r.t $\mathit{spec}$ and $\mathit{lin}'$ is a linearization.

\item[-] If $(R,T) {\xrightarrow{\mathit{send}(\mathit{mid},r)}} (R',T')$: Then,

    \begin{itemize}
    \setlength{\itemsep}{0.5pt}
    \item[-] It is obvious that $R' = R$. Let $T' = T \cup \{ (\mathit{mid},R(r),r) \}$.

    \item[-] Let $h' = h$.

    \item[-] Let $\mathit{lin}' = \mathit{lin}$.

    \item[-] Let $\mathit{del}' = \mathit{del}$.

    \item[-] Let $\mathit{map}' = \mathit{map} \cup \{ (\mathit{mid},\mathit{vd}(h,\mathit{del},r)) \}$.
    \end{itemize}

    It is easy to see that all other properties hold, except for checking $C_1$ for $\mathit{mid}$. This holds obviously since the message content of message $\mathit{mid}$ is $R(r)$, and we already know that $R(r) = \mathit{apply}(\mathit{lin},\mathit{vd}(h,\mathit{del},r)) = \mathit{apply}(\mathit{lin},\mathit{map}(\mathit{mid}))$.

\item[-] If $(R,T) {\xrightarrow{\mathit{receive}(\mathit{mid},r)}} (R',T')$: Then,

    \begin{itemize}
    \setlength{\itemsep}{0.5pt}
    \item[-] Let $R' = R[ r: \mathit{merge}(R(r),\mathit{msg})]$ where $(\mathit{mid},\mathit{msg},\_) \in T$. It is obvious that $T' = T$.

    \item[-] Let $h' = h$.

    \item[-] Let $\mathit{lin}' = \mathit{lin}$.

    \item[-] Let $\mathit{del}' = \mathit{del} \cup \{ (i,r) \vert i \in \mathit{map}(\mathit{mid}) \}$.

    \item[-] Let $\mathit{map}' = \mathit{map}$.
    \end{itemize}

    It is easy to see that all other properties hold, except for $C_2$ for replica $r$. Therefore, let us prove that $R'(r) = \mathit{apply}(\mathit{lin}',\mathit{vd}(h',\mathit{del}',r))$.

    We already know that $R'(r) = \mathit{merge}(R(r), \mathit{msg})$, $R(r) = \mathit{apply}(\mathit{lin},\mathit{vd}(h,\mathit{del},r))$ and $\mathit{msg} = \mathit{apply}(\mathit{lin},\mathit{map}(\mathit{mid}))$. It is easy to see that $\mathit{vd}(h',\mathit{del}',r) = \mathit{vd}(h,\mathit{del},r) \cup \mathit{map}(\mathit{mid})$. It is easy to prove that, applying messages in any order lead to the same consequence. Therefore, we have $\mathit{merge}(R(r), \mathit{msg}) = \mathit{apply}(\mathit{lin}',\mathit{vd}(h,\mathit{del},r) \cup \mathit{map}(\mathit{mid}))$. Then, we have $R'(r) = \mathit{apply}(\mathit{lin}',\mathit{vd}(h',\mathit{del}',r))$.
\end{itemize}

This completes the proof of this lemma. $\qed$
\end {proof}
}
















\forget
{
\subsection{The WOOT Algorithm}
\label{subsec:the woot algorithm}

The WOOT algorithm of \ref{the paper of WOOT} is given in Listing~\ref{lst:woot algorithm}. Note that here $integrateIns$ is a recursive method used by $addBetween$ method.

In local of each replica, WOOT algorithm stores the list as a sequence of W-characters. A W-character $w$ is a five-tuple $<id,v,flag,id_p,id_n>$, where $id$ is the identifier of $w$; $v$ is the value of $w$; $flag \in \{ \mathit{true},\mathit{false} \}$ is the flag of $w$ and indicates whether $w$ is ``visible'' in list; $id_p$ and $id_n$ is the identifier of the previous and next W-character of $w$, respectively. The previous and the next W-characters of $w$ are the W-characters between which $w$ has been inserted on its generation state. Given $w = (id,v,flag,id_p,id_n)$, let $C_P(w) = id_p$ and $C_N(w) = id_n$ denote the previous and next W-character of $w$, respectively. A identifier $id$ of W-character is a tuple $(ctr,\arep)$, where $ctr \in \mathbb{N}$.

A W-string is an ordered sequence of W-characters $w_b \cdot w_1 \cdot \ldots \cdot w_n \cdot w_e$, where $w_b$ and $w_e$ are special W-characters that mark the beginning and the ending of the sequence. The values of $w_b$ and $w_e$ are $\circ_b$ and $\circ_e$, respectively. We define the following function for a W-string $str$:

\begin{itemize}
\setlength{\itemsep}{0.5pt}
\item[-] $\vert str \vert$ returns the length of $str$,

\item[-] $str[p]$ returns the W-character at position $p$ in $str$. Her we assume that the first element of $str$ is at position 0.

\item[-] $pos(str,w)$ returns the position of W-character $w$ in $S$.

\item[-] $insert(str,w,p)$ inserts W-character $w$ into $str$ at position $p$.

\item[-] $subseq(str,w_1,w_2)$ returns the part of $str$ between the W-characters $w_1$ and $w_2$ (excluding $w_1$ and $w_2$).

\item[-] $contains(str,a)$ returns true if there exists a W-character in $str$ with value $a$.

\item[-] $values(str)$ returns the sequence of visible (with $\mathit{true}$ flag) values of $str$.

\item[-] $getWchar(str,a)$ returns the W-character with value $a$ in $str$.

\item[-] $changeFlag(str,pos,f)$ changes the flag of $str[pos]$ into $f$.
\end{itemize}

%Note that only $values(str)$ distinguish whether a W-character is with flag $\mathit{true}$ or with flag $\mathit{false}$.

A total order $<_{id}$ is given for identifiers of W-characters for conflict resolution. Given two identifiers $(ctr_1,\arep_1)$ and $(ctr_2,\arep_2)$, we have $(ctr_1,\arep_1) <{id} (ctr_2,\arep_2)$, if $\arep_1 < \arep_2 \vee (\arep_1 = \arep_2 \wedge ctr_1 < ctr_2)$. Given a sequence $str$ and two elements $a,b$ of $str$, we write $a <_{str} b$ to indicate that $pos(str,a) < pos(str,b)$.

\begin{minipage}[t]{1.0\linewidth}
\begin{lstlisting}[frame=top,caption={Pseudo-code of WOOT algorithm},
captionpos=b,label={lst:woot algorithm}]
  payload int @|$H_s$|@, W-string @|$string_s$|@
  initial @|$H_s$|@ = 0, @|$string_s$|@ = @|$\epsilon$|@
  initial seq = @|$\epsilon$|@

  addBetween(a,b,c) :
    atSource :
      precondition :  @|$contains(string_s,b) \wedge contains(string_s,c) \wedge pos(string_s,c) - pos(string_s,b) = 1\wedge \neg contains(string_s,a)$|@
      let g = myRep()
      let @|$c_p$|@ = @|$getWchar(string_s,b)$|@
      let @|$c_n$|@ = @|$getWchar(string_s,c)$|@
      @|$H_s$|@ = @|$H_s$|@ + 1
      //@ let seq@|$'$|@ = seq@|$\,\cdot\,\alabellongind[addBetween]{a,b,c}{}{}$|@
    downStream((w,@|$c_p$|@,@|$c_n$|@)) : with @|$w = ((H_s,g),a,\mathit{true},c_p.id,c_n.id)$|@
      integrateIns(@|$w,c_p,c_n$|@)

  remove(a) :
    atSource :
      precondition : @|$contains(string_s,a)$|@
      let w = @|$getWchar(string_s,a)$|@
      //@ let seq@|$'$|@ = seq@|$\,\cdot\,\alabellongind[remove]{a}{}{}$|@
    downStream(w) :
      let p = @|$pos(string_s,w)$|@
      @|$changeFlag(string_s,p,\mathit{false})$|@

  read() :
    let s = @|$values(string_s)$|@
    //@ let seq@|$'$|@ = seq@|$\,\cdot\,\alabellongind[read]{}{s}{}$|@
    return s

  integrateIns(@|$c,c_p,c_n$|@)
    let @|$S$|@ = @|$string_s$|@
    let @|$S'$|@ = @|$subseq(S,c_p,c_n)$|@
    if @|$S' = \epsilon$|@
      then  @|$insert(S,c,pos(S,c_n))$|@
    else
      Let L = @|$c_p \cdot d_0 \cdot \ldots \cdot d_m \cdot c_n$|@, where @|$d_0, \ldots, d_m$|@ are the W-characters in @|$S'$|@
        such that for each @|$d_i$|@, @|$C_P(d_i) <_S c_p$|@ and @|$c_n <_S C_N(d_i)$|@
      Let i = 1
      while (@|$i < \vert L \vert -1 \wedge L[i] <_{id} c$|@) do
        i = i+1
      integrateIns(@|$c,L[i-1],L[i]$|@)
\end{lstlisting}
\end{minipage}

The payload of each replica is a integer value $H_s$ used to generate identifier, and a W-string $string_s$.

To do $addBetween(a,b,c)$, we first ensure that $b$ and $c$ are adjacent in $string_s$ and $a$ is not in $string_s$. Then, we generate a W-character $w$ for value $a$, and calls method $integrateIns(w,c_p,c_n)$ to put $w$ between $c_p$ and $c_n$, which are the W-characters of $b$ and $c$ in $string_s$, respectively.

$integrateIns(c,c_p,c_n)$ is a recursive method and works as follows: If there are no W-character between $c_p$ and $c_n$ (for example, in the current replica), then $w$ is put after $c_p$. Else, WOOT select a set $L$ of W-characters, such that each W-character of $L$ has a ranger ``wider than the range between $c_p$ and $c_n$''. The W-characters in $L$ are the W-characters that needs to be considered when the range is between $c_p$ and $c_n$. It can be proved that W-characters in $L$ are sorted by the $<_{id}$ order. Then, we choose the position of $c$ to be between $L[i-1]$ and $L[i]$. We can see that the range between $L[i-1]$ and $L[i]$ is strictly shorter than the range between $c_p$ and $c_n$. Since there may be W-characters in the range between $L[i-1]$ and $L[i]$ in $string_s$, we make a recursive call to $integrateIns(c,L[i-1],L[i])$ to compute the position of $c$ in the range between $L[i-1]$ and $L[i]$.

To do $remove(a)$, we just set the flag of W-character of $a$ in $string_s$ to be $\mathit{false}$. To do $read()$, we return $values(string_s)$.
}
