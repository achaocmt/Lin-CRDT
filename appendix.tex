\section{Definitions of Section \ref{sec:CRDT implementations}}
\label{sec:appendix definitions of section CRDT implementations}








\section{Proofs of Section \ref{sec:proving distributed linearizability}}
\label{sec:appendix proofs of section proving distributed linearizability}









\subsection{Proof of State-based PN-counter}
\label{subsec:appendix proof of state-based PN-counter} 

Let $\mathit{inv} = C_1 \wedge C_2$ with the virtual messages defined as follows: For each update operation $o$, $\mathit{ds}(o) = (P,N)$, where

\begin{itemize}
\setlength{\itemsep}{0.5pt}
\item[-] $\forall r'$, $P[r'] = \vert \{ o' \vert o'$ is a $\mathit{inc}$ operation of replica $r'$, $o' = o \vee (o',o) \in h.\mathit{vis} \} \vert$.

\item[-] $\forall r'$, $N[r'] = \vert \{ o' \vert o'$ is a $\mathit{dec}$ operation of replica $r'$, $o' = o \vee (o',o) \in h.\mathit{vis} \} \vert$.
\end{itemize} 

The following lemma states that $\mathit{inv}$ is an invariant of state-based PN-counter. 

\begin{lemma}
\label{lemma:inv is an invariant of state-based CRDT for state-based PN-counter} 
$\mathit{inv}$ is an invariant of state-based PN-counter. 
\end{lemma}

\begin {proof} 

It is obvious that $\mathit{inv}(\mathit{config}_0,\epsilon,\emptyset,\emptyset,\emptyset)$ holds. 

Let us prove that $\mathit{inv}$ is a transition invariant: assume $\mathit{inv}((R,T),h,\mathit{lin},\mathit{del},\mathit{map})$ holds,  

\begin{itemize}
\setlength{\itemsep}{0.5pt}
\item[-] 
\end{itemize} 

$\qed$
\end {proof}




\subsection{Proof of State-based Multi-value Register}
\label{subsec:appendix proof of state-based multi-value register}
