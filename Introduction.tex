%!TEX root = draft.tex
\section{Introduction}
\label{sec:introduction}

Convergent Replicated Data Types (CRDTs)~\cite{ShapiroPBZ11} have
recently been proposed to address the problem of availability of a
distributed application under the presence of network partitions.
%
CRDTs represent a methodological approach to the problem of retaining
some form of data-Consistency and Availability under network Partitions (CAP),
famously known to be an impossible combination of requirements by the
CAP theorem of~\citet{GilbertL02}.
%
CRDTs are data types designed to favor availability over consistency
by replicating the type instances across multiple nodes of a
network, and allowing different nodes to temporarily have different
views of the same instance.
%
However, CRDTs guarantee that the different states of the multiple
nodes will \emph{eventually} converge to a unique state common to all
nodes~\cite{ShapiroPBZ11,Burckhardt14}.
%
Importantly, this \emph{convergence property} is intrinsic to the data
type design and in general no synchronization is needed among nodes,
hence achieving availability.

\smallskip
\noindent
{\bf Availability vs. Consistency.}
To illustrate the problem we will consider the implementation of a
List-like CRDT object (usually used for text-editing applications).
%
We will base our discussion on the Replicated Growing Array (RGA) due
to~\cite{RohJKL11}.\footnote{We use a variation of code extracted
  from~\cite{AttiyaBGMYZ16}.}
%  to be consistent with the rest of the paper.}
%
RGA supports three simple operations:
\begin{inparaenum}
\item \lstinline|addAfter(a,b)| which adds the character
  \lstinline|b| -- the concrete type is inconsequential here --
  immediately after the occurrence of the character \lstinline|a|
  assumed to already be present in the list,\footnote{To simplify the
    exposition we assume elements are unique, which is easily implemented
    with timestamps.}
\item \lstinline|remove(a)| which removes the character \lstinline|a|
  assumed to be present in the list, and
\item \lstinline|read()| which returns the current contents of the
  list.
\end{inparaenum}

To make the system available under partitions RGA allows each of
multiple nodes to have a copy of the list instance.
%
We will call each of the nodes holding a copy of the list a \emph{replica}.
%
Then the question is, how can we maintain the consistency of the
different copies of the list given that the data could be at any point
in time be modified or read by any of the replicas?
%
The naive approach would require the synchronization of all the
replicas for each operation, thus maintaining coherence.
%
Alas, this naive approach would render the system unavailable if any
one replica goes off-line.
%

Instead of this naive approach, RGA takes the liberty of allowing any
of the replicas to modify the \emph{local} copy of the list immediately --
and hence return control to the client -- and lazily propagate the
updates to the other replicas at a later point in time.
%
For instance, assuming that we have an initial list containing the
sequence $\mathtt{a \cdot b \cdot e \cdot f}$~\footnote{We will use
  $s_0 \cdot s_1$ to denote the composition of sequences $s_0$ and
  $s_1$ throughout the paper.}
and two replicas, $\arep_1$ and $\arep_2$, if $\arep_1$ inserts the
letter \lstinline|c| after \lstinline|b| (by calling
\lstinline|addAfter(b,c)|), while $\arep_2$ concurrently inserts the
letter \lstinline|d| after \lstinline|b| (\lstinline|addAfter(b,d)|)
the replicas will have the states $\mathtt{a \cdot b \cdot c \cdot e
  \cdot f}$ and $\mathtt{a \cdot b \cdot d \cdot e \cdot f}$
respectively.
%
We have solved the availability problem but we have introduced
inconsistent states across the system.
%
This problem is only exacerbated if we consider that any replica could
serve any operation at any point in time.

\smallskip
\noindent
{\bf Convergence.}
To return to a consistent state across the system in the presence of
concurrent updates, CRDTs guarantee that under conflicting operations --
that is, operations that could lead to different states as shown above
--, there is a systematic way to \emph{detect conflicts}, and moreover, there
is a strategy to \emph{deterministically solve conflicts} which will be followed
by all replicas.

% {\color {red}CW: Here ``detect conflict'' may confuse the readers. When we do the downstream, we do not distinguish whether this downstream comes from a concurrent operation or from a visible operation.}
% \gpnote[nomargin, inline]{I don't think it's confusing or inaccurate.
%   The downstream + algorithm already embeds the conflict resolution
%   mechanism (timestamps, vector clocks, etc.). So the conflict
%   detection + resolution is there.}

% \ce{To simplify, the RGA uses now timestamps that are simply integers. The following text assumes that timestamps are pairs of replica ids and local clocks.}

% \ce{I would say that for RGA, two \lstinline|addAfter| operations conflict only if they try to add something after exactly the same element (like in the example given above). And this conflict is solved looking at the timestamps. The next paragraph uses a more general notion of conflict.}

In the specific case of RGA, the implementation adds metadata to each
item added to the list identifying the originating replica as well as
timestamp of the operation in that replica.\footnote{We will ignore in
  this section conflicts due to \lstinline|remove| operations.}
%
This metadata is enough to detect when conflicts have occurred.
%
Generally there are a number of assumptions that are necessary for the
metadata to detect conflicts (for instance that timestamps increase
monotonically with time, that replica identifiers are unique, etc.)
which we shall discuss in the next sections.
%
Then, for the case of the RGA data type, whose implementation will be
presented in the next section, it is enough to know whether
two \lstinline|addAfter| operations have conflicted by simply
comparing the replica identifiers and their timestamps.
%
In fact, this is a sound over-approximation of conflict since two
concurrent \lstinline|addAfter| operations have a real conflict only
if their first arguments are the same (eg. the element \lstinline|b|
in the example aforementioned).
%
In such case, the strategy to resolve the conflict will always chose
to order first in the resulting list the character added with the
highest timestamp, and in the particular case where the timestamps
should be the same, an arbitrary order among replicas will be used.
%
Going back to the example above, and assuming that the character
\lstinline|c| was added with timestamp $t_1$ and the character
\lstinline|d| was added with timestamp $t_2$, if we assume that
$t_1 < t_2$ for some order $\leq$ between timestamps, then the list will
converge to $\mathtt{a \cdot b \cdot c \cdot d \cdot e}$.
%
We obtain the same result if $t_1 = t_2$ and, assuming that we have a replica
order $<_r$, we have $\arep_1 <_r \arep_2$.
%
In any other case we obtain $\mathtt{a \cdot b \cdot d \cdot c \cdot e}$.
%
The idea of using an arbitrary order among replica identifiers is
often used in CRDT implementations to break ties among elements with
equal timestamps.
%
For this reason in the rest of the paper we will assume that metadata
provides a strict ordering relation to abstract this detail away.

If the effects of all operations are
delivered to all replicas eventually, the replicas will converge to
the same state -- assuming a quiescent period of time where no new
operations are performed.
%
We have somehow regained consistency of the data type without
giving away availability.

\smallskip
\noindent
{\bf Specifications.}
The simplicity of the list data type with the API that we have
described above allows for a somewhat simple conflict resolution
strategy.
%
Essentially, any strategy ordering conflicting concurrent insertions
to the list in a deterministic way will work.
%
However, this is not true for many other CRDT implementations.
%
It is therefore critical to provide the programmer with a clear, and
hopefully formal, specification of the allowed behaviors of the data
type under conflicts.
%
Unfortunately this is not an easy task.
%
Many times the programmer has no option but to read the implementation
of the data type to understand how the metadata is used to 
resolve conflicts, for instance by reading the algorithms
by~\citet{ShapiroPBZ11} (an exceptional case where the algorithms are
particularly well documented).
%
Recently Burckhardt et al.~\cite{BurckhardtGYZ14, Burckhardt14} have
developed a formal framework where CRDTs and other weakly
consistent systems can be specified.
%
However, we consider that reading these specifications is far from
trivial for the average programmer, let alone writing new
specifications.
%
Evidently, having a formal specification is a necessary step towards
the verification of the implementations of CRDTs.
%
These are the problems we address in this paper.

%\smallskip
\noindent
{\bf Paper Contributions.}
Inspired by linearizability~\cite{HerlihyW90} and the literature of
shared-memory concurrency we propose a \emph{new consistency criterion
  for CRDTs}, which we call \emph{\CRDTLin{}} (\CRDTLinshort{}).
%
\CRDTLinshort{} both simplifies CRDT specifications, and allows us to
give correctness proof strategies for these specifications.
%
To satisfy the \CRDTLinshort{} criterion a data type must be so that
any execution of a client interacting with an instance of the data
type
\begin{inparaenum}
\item should result in a state that can be obtained as a sequence (or
  linearization) of its updates -- where we assume that all updates
  are executed sequentially--, and
\item any operation reading the state of the data type instance should
  be justified by executing a \emph{sub-sequence} of the above
  mentioned sequence of updates.
\end{inparaenum}
For instance, for the RGA example, the state of the final
list (when all updates are delivered) should be reachable by considering a sequence where all
\lstinline|addAfter| and  operations are executed sequentially.\footnote{We
  will come back to RGA to add \lstinline|remove|
  in Section~\ref{sec:overview}.}
%
This definition shares some similarities with~\citet{PerrinMJ14}. 
%
We address the main differences in Section~\ref{sec:rel-work}.

Equipped with this criterion we show that multiple existing CRDTs are
\crdtlinearizable{}.
%
We provide both, their \CRDTLinshort{} specification, and proofs
showing that implementations respect the specification.
%
Encouragingly, this is true for most of the CRDTs
by~\citet{ShapiroPBZ11}.

Given that our criterion is inspired by
linearizability, it is natural to ask whether it
preserves the same compositionality properties, i.e.
whether the composition of a set of RA-linearizable objects is also RA-linearizable.
%
We show that this is not true in general, but we identify classes of
CRDT implementations, based on the conflict-resolution strategy that
they implement, for which compositionality can be achieved. Interestingly, the characterization of these 
classes of objects stems naturally from the proof methodology we use to show that they are RA-linearizable.

% \gpwarning[nomargin, inline]{Don't know if we should say more here.}

The paper is structured as follows:
%\begin{inparaitem}
%\item \fxwarning[nomargin, inline]{TODO}
%\end{inparaitem}
Section~\ref{sec:overview} gives an overview of our notion of RA-linearizability, Section~\ref{sec:distributed-lin} provides the formalization of the CRDT semantics and RA-linearizability, Section~\ref{sec:proofs} describes a proof methodology for establishing RA-linearizability, and Section~\ref{sec:compositionality} discusses compositionality properties. We discuss the related work in Section~\ref{sec:rel-work} and conclude in Section~\ref{sec:conclusion}.

% Here we introduce the problem of CRDTS:
% \begin{inparaenum}
% \item replication,
% \item convergence,
% \item specification.
% \end{inparaenum}
% To do so we illustrate the problem through RGA. Without much notation
% we show the code of~\cite{ShapiroPBZ11} (RGA) and explain how RGA achieves
% these three goals.
% %
% We the consider the specification of RGA. Perhaps we will need to
% explain a little of~\cite{BurckhardtGYZ14} to do so.
% %
% We argue that this process is hard a complicated, and propose
% \emph{informally} our definition of linearizability.

% We then explain the proof strategy to prove this linearizability, and
% point the reader forward to the verification sections.

% We finally shed some light into the compositionality and modularity
% aspects to be discussed at the end of the paper.

% \gpnote*{}{By the end of section 1 the reader should know what is a
%   CRDT, the intuition behind our definition of linearizability, and
%   have an idea of the structure of the paper.}

%%% Local Variables:
%%% mode: latex
%%% TeX-master: "draft"
%%% End:
