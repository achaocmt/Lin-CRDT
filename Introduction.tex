%!TEX root = draft.tex
\section{Introduction}
\label{sec:introduction}

\gpnote[inline, nomargin]{Need to complete this section}

Here we introduce the problem of CRDTS:
\begin{inparaenum}
\item replication, 
\item convergence, 
\item specification.
\end{inparaenum}
To do so we illustrate the problem through RGA. Without much notation
we show the code of~\autoref{} (RGA) and explain how RGA achieves
these three goals.
%
We the consider the specification of RGA. Perhaps we will need to
explain a little of~\cite{BurckhardtGYZ14} to do so.
%
We argue that this process is hard a complicated, and propose
\emph{informally} our definition of linearizability.

We then explain the proof strategy to prove this linearizability, and
point the reader forward to the verification sections. 

We finally shed some light into the compositionality and modularity
aspects to be discussed at the end of the paper. 

\gpnote*{}{By the end of section 1 the reader should know what is a
  CRDT, the intuition behind our definition of linearizability, and
  have an idea of the structure of the paper.}

%%% Local Variables:
%%% mode: latex
%%% TeX-master: "draft"
%%% End:
