%!TEX root = draft.tex
\section{Introduction}
\label{sec:introduction}

Convergent Replicated Data Types (CRDTs)~\cite{ShapiroPBZ11} have
recently been proposed to address the problem of availability of an
application under the presence of network partitions.
%
CRDTs represent a methodological approach to address the problem of
retaining data-Consistency and Availability under network Partitions (CAP),
famously shown to be impossible to achieve by the CAP
theorem~\cite{GilbertL02}.
%
Thus, CRDTs are data types designed to favor availability over
consistency by replicating the data type instances across multiple
nodes, and allowing different nodes to temporarily have different
views of the same instance at times.
%
However, CRDTs guarantee that the different states of the multiple
nodes will \emph{eventually} converge to a unique state common to all
nodes.
%
Importantly, this \emph{convergence property} is intrinsic to the data
type design and in general no synchronization is needed among nodes to
achieve it.

To illustrate the problem we will consider the implementation of a
data type to implement a list-like object (usually used for text
editing applications).
%
We will base our discussion on a CRDT definition called the Replicated
Growing Array (RGA) due to~\cite{RohJKL11}.\footnote{We use the code
  as shown in~\cite{ShapiroPBZ11} to be consistent with the rest of
  the paper.}
%
RGA supports three simple operations:
\begin{inparaenum}
\item \lstinline|addAfter(a, b)| which adds the character (the
  concrete type is inconsequential)
  \lstinline|b| immediately after the occurrence of the character
  \lstinline|a| assumed to already be present in the list\footnote{To
    simplify the exposition we assume elements are unique, which is easily
    implemented with timestamps.},
\item \lstinline|remove(a)| which removes the character \lstinline|a|
  assumed to be present in the list, and
\item \lstinline|read()| which returns the current contents of the
  list.
\end{inparaenum}

To make the system available under partitions RGA allows multiple
nodes to have a copy of the list instance.
%
Then the question is, how do we maintain the consistency of the
different copies of the list given that the data could be at any point
in time be modified or read by any of the replicas?
%
The naive approach would require to synchronize all the replicas for
each operation, thus rendering the system unavailable if any one
replicas becomes off-line.

\gpnote*[inline, nomargin]{Continue from here.}
{
Instead of the naive approach, RGA ensures that as operations are
propagated from one replica to another (by sending messages containing
the data of the operation), there is a single possible list that could
result for any replica that has observed all operations.
}


Here we introduce the problem of CRDTS:
\begin{inparaenum}
\item replication,
\item convergence,
\item specification.
\end{inparaenum}
To do so we illustrate the problem through RGA. Without much notation
we show the code of~\cite{ShapiroPBZ11} (RGA) and explain how RGA achieves
these three goals.
%
We the consider the specification of RGA. Perhaps we will need to
explain a little of~\cite{BurckhardtGYZ14} to do so.
%
We argue that this process is hard a complicated, and propose
\emph{informally} our definition of linearizability.

We then explain the proof strategy to prove this linearizability, and
point the reader forward to the verification sections.

We finally shed some light into the compositionality and modularity
aspects to be discussed at the end of the paper.

\gpnote*{}{By the end of section 1 the reader should know what is a
  CRDT, the intuition behind our definition of linearizability, and
  have an idea of the structure of the paper.}

%%% Local Variables:
%%% mode: latex
%%% TeX-master: "draft"
%%% End:
