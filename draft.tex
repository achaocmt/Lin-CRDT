\documentclass{llncs}
\usepackage[english]{babel}

\usepackage{hyperref}

\usepackage{epsfig}
\usepackage{amsmath}
\usepackage{color}
\usepackage{amsfonts,amssymb}
\usepackage{mathabx}
\usepackage{verbatim}
\usepackage{times}
\usepackage[linesnumbered,ruled,procnumbered,noend]{algorithm2e}
\usepackage{stmaryrd}

%added by Wang Chao%
%\usepackage{mathrsfs}
%\usepackage{extarrows}
\usepackage{thmtools}
\usepackage{thm-restate}

%add for TIkz
\usepackage[version=0.96]{pgf}
\usepackage{tikz}
\usetikzlibrary{arrows,shapes,snakes,automata,backgrounds,petri}
\usepackage{xcolor}
\usepackage{paralist}

\usepackage[margin,draft]{fixme}
\fxsetup{theme=color,mode=multiuser}
\FXRegisterAuthor{gp}{agp}{GP}
\FXRegisterAuthor{cw}{acw}{CW}

\newcommand{\gpn}[2][]{\gpnote*{#1}{#2}}
\newcommand{\gp}[1]{\gpnote{#1}}


\DeclareSymbolFont{largesymbolsA}{U}{txexa}{m}{n}
\SetSymbolFont{largesymbolsA}{bold}{U}{txexa}{bx}{n}
\DeclareFontSubstitution{U}{txexa}{m}{n}
\DeclareMathSymbol{\bigsqcupplus}{\mathop}{largesymbolsA}{"02}

\title{CRDT Linearizability}

\author{}
\institute{}

\author {Constantin Enea \and Gustavo Petri \and Chao Wang}
\institute{Institut de Recherche en Informatique Fondamentale, \\Univ. Paris Diderot (Paris 7)}

% GP: Why do we have all these macros?

\newcommand\eqdef{\stackrel{\text{def}}{=}}

%\newcommand{\nondsum}{\bigbox}
\newcommand{\nondsum}{\bigsqcupplus}
\newcommand{\probplus}[1]{\oplus_{#1}}
%\newcommand{\nondplus}{\square}
\newcommand{\bang}{!\,}
\newcommand{\nondplus}{{\textstyle\bigsqcupplus}}
\newcommand{\partmap}{\rightharpoonup}
\newcommand{\map}{\rightarrow}
%\newcommand{\exc}{\phi}
\newcommand{\exc}{\alpha}
\newcommand{\exec}{\mathit{exec}}
\newcommand{\execp}{\mathit{execp}}
\newcommand{\Act}{\mathit{Act}}
\newcommand{\Sec}{\mathit{Sec}}
\newcommand{\Obs}{\mathit{Obs}}
\newcommand{\etree}{\mathit{etree}}
\newcommand{\lstate}{\mathit{lst}}
\newcommand{\fstate}{\mathit{fst}}
%\newcommand{\STATE}{\mathcal{P}r}  $ original marked
%\newcommand{\STATE}{\mathcal{P}}   $ marked by me
%\newcommand{\st}{P}
\newcommand{\trans}{\mathcal{T}}
\newcommand{\Aut}{\mathcal{M}}
\newcommand{\init}{\mathit{init}}
\newcommand{\perr}{\mathcal{P}}

\newcommand{\calo}{\mathcal{O}}
\newcommand{\cals}{\mathcal{S}}
\newcommand{\sseq}{\vec s}
\newcommand{\oseq}{\vec o}
\newcommand{\ccsp}{CCS$_p$}

\newcommand{\bigfrac}[2]{\frac{\raisebox{1ex}{$#1$}}{\raisebox{-1.5ex}{$#2$}}}
\newcommand{\nondarr}[1]{\overset{#1}{\longrightarrow}}
\newcommand{\Nondarr}[1]{\overset{#1}{\Longrightarrow}}
\newcommand{\vectorArrow}[1]{\stackrel{\longrightarrow}{\mbox{#1}}}
\newcommand{\probarr}[1]{\overset{#1}{\dashrightarrow}}
\newcommand{\paral}{\,|\,}
\newcommand{\outp}[1]{\overline{#1}}
\renewcommand{\Pr}{{\rm Pr}}

%Commands by Chao Wang

%memory models%
\newcommand{\TSO}{\textrm{TSO}}
\newcommand{\PSO}{\textrm{PSO}}

%correctness conditions%
\newcommand{\lin}{\textrm{linearizability}}
\newcommand{\slin}{\textrm{static linearizability}}
\newcommand{\qlin}{\textrm{quasi linearizability}}
\newcommand{\TTlin}{\textrm{TSO-to-TSO linearizability}}

\newcommand{\pair}[2]{\langle #1 , #2 \rangle}% pairs
\newcommand{\setof}[2]{\{ \, #1 \mid #2 \, \}}% Sets
\newcommand{\set}[1]{\{ {#1}  \}  }
%\newcommand{\map}[3]{{#1} \colon {#2} \longmapsto {#3}} %functions
\newcommand{\den}[1]{[\![#1]\!]}% Denotation of
\newcommand{\mean}[1]{|\!|#1|\!|}
\newcommand{\forget}[1]{}

%%%%%%%%%%%%%GENERAL%%%%%%%%%%%%%%%%%%%%%%%%%%%

\newcommand{\itbox}[1]{{\it #1\/}}
\newcommand{\un}[1]{\uline{#1}}%\underline{#1}}
\newcommand{\ov}[1]{\overline{#1}}
\newcommand{\smallspace}{\vspace{10mm}}
\newcommand{\is}{\mbox{$\Longleftarrow\ $}}
\newcommand{\pright}[1]{\hfill{#1}}
\newcommand{\bnfor}{\;\;\mid\;\;}

\newcommand{\ar}[1]{\stackrel{\scriptstyle #1}{\longrightarrow}}

%%%%%%%%%italics in math mode
%%%%%%%%%%%%%%%%%%%%%%%%%%%%%%%%%%%%%%%%%

%\newcommand{\true}{{\it true}}
%\newcommand{\false}{{\it false}}
\newcommand{\calB}{{\cal B}}
\newcommand{\calF}{{\cal F}}
\newcommand{\calP}{{\cal P}}
\newcommand{\order}{{\cal O}}
\newcommand{\size}[1]{|#1|}

%other notations%
\newcommand{\LTS}{\textit{LTS}}
\newcommand{\bedt}[1]{{\color{blue}#1}}
\newcommand{\redt}[1]{{\color{red}#1}}

\newcommand{\todo}[1]{{\bf \color{red}{#1}}}

%%%%%%%%%%%%% CRDTs %%%%%%%%%%%%%%%%%%%%%%%%%%%
\newcommand{\crdtlin}{RA-linearizability}
\newcommand{\crdtlinearization}{RA-linearization}
\newcommand{\crdtlinearizable}{RA-linearizable}
\newcommand{\CRDTLin}{Replication-Aware Linearizability}
\newcommand{\CRDTLinshort}{RA\text{-}Linearizability}

\newcommand{\hwlin}{{\color{magenta} hw-linearizability}}
\newcommand{\hwlinear}{{\color{magenta} hw-linear}}
\newcommand{\hwlinearization}{{\color{magenta} hw-linearization}}
\newcommand{\hwlinearizable}{{\color{magenta} hw-linearizable}}
\newcommand{\HWLin}{{\color{magenta} HW-Linearizability}}



%%%%%%%%%%%%% CRDTs %%%%%%%%%%%%%%%%%%%%%%%%%%%

\newcommand{\aobj}{\ensuremath{\mathtt{o}}}
\newcommand{\ahist}{\ensuremath{{h}}}
\newcommand{\objs}{\ensuremath{\mathbb{O}}}
\newcommand{\acts}{\ensuremath{\mathbb{A}}}
\newcommand{\aact}{\ensuremath{a}}
\newcommand{\arep}{\ensuremath{\mathtt{r}}}
\newcommand{\reps}{\ensuremath{\mathbb{R}}}
\newcommand{\amethod}{\ensuremath{\mathsf{m}}}
\newcommand{\ainput}{\ensuremath{\mathsf{a}}}
\newcommand{\areturn}{\ensuremath{\mathsf{r}}}
\newcommand{\tsof}{\ensuremath{\mathsf{ts}}}
\newcommand{\atimestamp}{\ensuremath{\mathsf{t}}}
\newcommand{\methods}{\ensuremath{\mathbb{M}}}
\newcommand{\histories}{\ensuremath{\mathbb{H}\mathsf{ist}}}
\newcommand{\traces}{\ensuremath{\mathbb{T}\mathsf{r}}}
\newcommand{\datadomain}{\ensuremath{\mathbb{D}}}
\newcommand{\timestampdomain}{\ensuremath{\mathbb{T}}}
\newcommand{\acrdttyp}{\ensuremath{\mathsf{t}}}
\newcommand{\amethodset}{\ensuremath{\mathsf{M}}}
\newcommand{\adomain}{\ensuremath{\mathsf{D}}}
\newcommand{\atsdomain}{\ensuremath{\mathsf{T}}}
\newcommand{\powerset}[1]{\ensuremath{\mathcal{P}(#1)}}
\newcommand{\astate}{\ensuremath{\sigma}}
\newcommand{\abstate}{\ensuremath{\phi}}
\newcommand{\abstates}{\ensuremath{\Phi}}
\newcommand{\states}{\ensuremath{\Sigma}}
\newcommand{\amesg}{\ensuremath{\mathsf{m}}}
\newcommand{\messages}{\ensuremath{\mathcal{M}\mathsf{sg}}}
\newcommand{\amesgset}{\ensuremath{\mathsf{M}}}
\newcommand{\aop}{\ensuremath{\mathsf{op}}}
\newcommand{\ops}{\ensuremath{\mathbb{O}\mathsf{ps}}}
\newcommand{\aopid}{\ensuremath{\aop_{\mathsf{id}}}}
\newcommand{\opids}{\ensuremath{\mathbb{O}\textsf{p}\mathsf{ID}}}
\newcommand{\argv}{\ensuremath{a}}
\newcommand{\retv}{\ensuremath{b}}
\newcommand{\alabellong}[3][\amethod]{\ensuremath{#1(#2) \Rightarrow #3}}
\newcommand{\alabelshort}[2][\amethod]{\ensuremath{#1(#2)}}
\newcommand{\alabellongind}[4][\amethod]{\ensuremath{#1(#2) \overset{#4}{\Rightarrow}
    #3}}
\newcommand{\alabelobjind}[4][{\aobj.\amethod}]{\ensuremath{{#1}(#2) \overset{#4}{\Rightarrow}
    #3}}
\newcommand{\src}[2]{\ensuremath{\mathsf{gen}_{#1}(#2)}}
\newcommand{\dwn}[2]{\ensuremath{\mathsf{eff}_{#1}(#2)}}
\newcommand{\atsm}[1]{\mathsf{tsm}_{#1}}

\newcommand{\alabel}{\ensuremath{\ell}}
\newcommand{\refmap}{\ensuremath{\mathsf{abs}}}
\newcommand{\firstrep}{\ensuremath{\mathsf{qry}}}
\newcommand{\secondrep}{\ensuremath{\mathsf{upd}}}
\newcommand{\specOrSet}{\ensuremath{\Spec(\text{OR\text{-}Set})}}
\newcommand{\specMVReg}{\ensuremath{\Spec(\text{MV\text{-}Reg})}}
\newcommand{\specCounter}{\ensuremath{\Spec(\text{Counter})}}
\newcommand{\specRGA}{\ensuremath{\Spec(\text{RGA})}}
\newcommand{\specWooki}{\ensuremath{\Spec(\text{List-Between})}}
\newcommand{\labels}{\ensuremath{\mathbb{L}}}
\newcommand{\labelsspec}[1]{\ensuremath{\mathbb{L}\mathsf{ab}({#1})}}
\newcommand{\labelsspecrd}[1]{\ensuremath{\mathbb{L}\mathsf{ab}_{\mathsf{rd}}({#1})}}
\newcommand{\labelsspecmod}[1]{\ensuremath{\mathbb{L}\mathsf{ab}_{\mathsf{mod}}({#1})}}
\newcommand{\alabelset}{\ensuremath{\mathsf{L}}}
\newcommand{\arepord}{\ensuremath{\mathsf{ro}}}
\newcommand{\avisord}{\ensuremath{\mathsf{vis}}}
\newcommand{\atsord}[1]{\ensuremath{\prec_{#1}}}
\newcommand{\aseqord}{\ensuremath{\mathsf{seq}}}
\newcommand{\alinord}{\ensuremath{\mathsf{lin}}}
\newcommand{\absopsemplain}{\ensuremath{\Rightarrow}}
\newcommand{\absopsem}[2][\alabel]{\ensuremath{\delta_{#1}(#2)}}
\newcommand{\apre}{\ensuremath{\mathsf{pre}}}
\newcommand{\comp}{\ensuremath{\otimes}}


\newcommand{\atrace}{\ensuremath{\mathit{tr}}}
\newcommand{\hist}[1]{\ensuremath{\mathit{h}({#1)}}}
\newcommand{\Spec}{\ensuremath{\mathsf{Spec}}}
\newcommand{\updates}{\ensuremath{\mathsf{Updates}}}
\newcommand{\queries}{\ensuremath{\mathsf{Queries}}}
\newcommand{\Updates}{\ensuremath{\mathsf{Updates}}}
\newcommand{\Queries}{\ensuremath{\mathsf{Queries}}}
\newcommand{\queryupdates}{\ensuremath{\mathsf{Query\text{-}Updates}}}

\newcommand{\effector}{\ensuremath{\delta}}
\newcommand{\semop}[2][\aop]{\ensuremath{\llbracket #1
    \rrbracket}\ifthenelse{\isempty{#2}}{}{(#2)}}
\newcommand{\localstates}{\ensuremath{\mathsf{LC}}}
\newcommand{\globalstates}{\ensuremath{\mathsf{GC}}}
\newcommand{\gstates}{\ensuremath{\mathsf{G}}}
\newcommand{\dom}[1]{\ensuremath{\mathsf{dom}(#1)}}
\newcommand{\labeldom}[1]{\ensuremath{\mathsf{labels}(#1)}}
\newcommand{\downstreams}{\ensuremath{\mathsf{DS}}}

\newcommand{\specarrow}[1]{\xhookrightarrow{#1}}



\newcommand{\alabellongNoret}[2][\amethod]{\ensuremath{#1(#2)}} % m(a)

\newcommand{\alabellongNoArg}[2][\amethod]{\ensuremath{#1 \Rightarrow
    #2}} % m() \Rightarrow b

\newcommand{\ats}{\ensuremath{ts}} % a timestamp
\newcommand{\tss}{\ensuremath{\mathbb{T}\mathsf{s}}} % set of timestamps
\newcommand{\atsource}{\ensuremath{\theta}} % atsource
\newcommand{\aglobalstate}{\ensuremath{\mathtt{gc}}} % a global state
\newcommand{\aexec}{\ensuremath{\mathtt{e}}} % an execution
\newcommand{\ahis}{\ensuremath{\mathtt{h}}} % a history
\newcommand{\tzerolin}{t0-linearizability} % t0-linearizability
\newcommand{\tzerolinearization}{t0-linearization} % t0-linearization
\newcommand{\tzerolinearizable}{t0-linearizable} % t0-linearizable
\newcommand{\tonelin}{t1-linearizability} % t1-linearizability
\newcommand{\tonelinearization}{t1-linearization} % t1-linearization
\newcommand{\tonelinearizable}{t1-linearizable} % t1-linearizable

\newcommand{\tzerospec}{t0-specification} % t0-specification
\newcommand{\tonespec}{t1-specification} % t0-specification

\newcommand{\crdtimp}{crdt-implementation} % crdt-implementation
\newcommand{\gts}{global-timestamp} % a global-timestamp
\newcommand{\gtss}{\ensuremath{\mathbb{G}\mathsf{ts}}} % global-timestamps
\newcommand{\updategts}{\ensuremath{updGTs}} % update global-timestamp
\newcommand{\rupdategts}{\ensuremath{rupdGTs}} % update global-timestamp with a random value
\newcommand{\funobjtostates}{\ensuremath{\sigma}\ensuremath{s}} % a function from object to lobal states
\newcommand{\severalobj}{\ensuremath{\mathtt{os}}} % a set of objects
\newcommand{\mobj}[1]{\ensuremath{\mathsf{modify}(#1)}} % modify a object where use random update timestamp
%%%%%%%%%%%%%%%%%%%%%%%%%%%%%%%%%%%%%%%%%%%%%%
\newcommand{\gconfres}{\ensuremath{\mathsf{gcr}}} % a function for global conflict resolution. This is used for non-determnistic sequential specification
\newcommand{\igconfres}{\ensuremath{\mathsf{igcr}}} % a function for global conflict resolution in implementation. This is used for non-determnistic sequential specification

%%% Local Variables:
%%% mode: latex
%%% TeX-master: t
%%% End:


\begin{document}

\maketitle

%!TEX root = draft.tex
\begin{abstract}
  %We address the problem of specifying and verifying distributed 
  
  Geo-distributed systems often replicate data at multiple locations to achieve availability and performance, despite network partitions. 
%  
  These systems must accept updates at any replica, and propagate these updates asynchronously to the other replicas.
%  
  Convergent Replicated Data Types (CRDTs) provide a principled approach to the problem of ensuring that replicas are eventually consistent despite the asynchronous delivery of updates.

  We address the problem of specifying and verifying CRDTs, introducing a new correctness criterion called Replication-Aware Linearizability. This criterion is inspired by linearizability, the de-facto correctness criterion for (shared-memory) concurrent data structures.
%
%  In this paper we consider the problem of the specification and
%  verification of Commutative Replicated Data Types (CRDTs).
%  %
%  We provide a new specification and correctness criterion for CRDTs
%  akin to Linearizability as defined by~\citet{HerlihyW90}.
  %
  We argue that this criterion is both simple to understand, and it
  fits most known implementations of CRDTs.
  %
  We provide a proof methodology for showing that a CRDT satisfies replication-aware linearizability
  which we apply on a wide range of implementations.
%  Then we show how to prove that CRDT implementations can be
%  formally proved to be correct w.r.t. their specification.
%  %
%  In particular, we do so for many well known implementations taken
%  from~\citet{ShapiroPBZ11}.
  %
  Finally, we show that our criterion can be leveraged
  to reason modularly about the composition of CRDTs.
\end{abstract}

%%% Local Variables:
%%% mode: latex
%%% TeX-master: "draft"
%%% End:


%!TEX root = draft.tex
\section{Introduction}
\label{sec:introduction}

Convergent Replicated Data Types (CRDTs)~\cite{ShapiroPBZ11} have
recently been proposed to address the problem of availability of a
distributed application under the presence of network partitions.
%
CRDTs represent a methodological approach to the problem of retaining
some form of data-Consistency and Availability under network Partitions (CAP),
famously known to be an impossible combination of requirements by the
CAP theorem of~\citet{GilbertL02}.
%
CRDTs are data types designed to favor availability over consistency
by replicating the type instances across multiple nodes of a
network, and allowing different nodes to temporarily have different
views of the same instance.
%
However, CRDTs guarantee that the different states of the multiple
nodes will \emph{eventually} converge to a unique state common to all
nodes~\cite{ShapiroPBZ11,Burckhardt14}.
%
Importantly, this \emph{convergence property} is intrinsic to the data
type design and in general no synchronization is needed among nodes,
hence achieving availability.

\smallskip
\noindent
{\bf Availability vs. Consistency.}
To illustrate the problem we will consider the implementation of a
List-like CRDT object (usually used for text-editing applications).
%
We will base our discussion on the Replicated Growing Array (RGA) due
to~\cite{RohJKL11}.\footnote{We use a variation of code extracted
  from~\cite{AttiyaBGMYZ16}.}
%  to be consistent with the rest of the paper.}
%
RGA supports three simple operations:
\begin{inparaenum}
\item \lstinline|addAfter(a,b)| which adds the character
  \lstinline|b| -- the concrete type is inconsequential here --
  immediately after the occurrence of the character \lstinline|a|
  assumed to already be present in the list,\footnote{To simplify the
    exposition we assume elements are unique, which is easily implemented
    with timestamps.}
\item \lstinline|remove(a)| which removes the character \lstinline|a|
  assumed to be present in the list, and
\item \lstinline|read()| which returns the current contents of the
  list.
\end{inparaenum}

To make the system available under partitions RGA allows each of
multiple nodes to have a copy of the list instance.
%
We will call each of the nodes holding a copy of the list a \emph{replica}.
%
Then the question is, how can we maintain the consistency of the
different copies of the list given that the data could be at any point
in time be modified or read by any of the replicas?
%
The naive approach would require the synchronization of all the
replicas for each operation, thus maintaining coherence.
%
Alas, this naive approach would render the system unavailable if any
one replica goes off-line.
%

Instead of this naive approach, RGA takes the liberty of allowing any
of the replicas to modify the \emph{local} copy of the list immediately --
and hence return control to the client -- and lazily propagate the
updates to the other replicas at a later point in time.
%
For instance, assuming that we have an initial list containing the
sequence $\mathtt{a \cdot b \cdot e \cdot f}$~\footnote{We will use
  $s_0 \cdot s_1$ to denote the composition of sequences $s_0$ and
  $s_1$ throughout the paper.}
and two replicas, $\arep_1$ and $\arep_2$, if $\arep_1$ inserts the
letter \lstinline|c| after \lstinline|b| (by calling
\lstinline|addAfter(b,c)|), while $\arep_2$ concurrently inserts the
letter \lstinline|d| after \lstinline|b| (\lstinline|addAfter(b,d)|)
the replicas will have the states $\mathtt{a \cdot b \cdot c \cdot e
  \cdot f}$ and $\mathtt{a \cdot b \cdot d \cdot e \cdot f}$
respectively.
%
We have solved the availability problem but we have introduced
inconsistent states across the system.
%
This problem is only exacerbated if we consider that any replica could
serve any operation at any point in time.

\smallskip
\noindent
{\bf Convergence.}
To return to a consistent state across the system in the presence of
concurrent updates, CRDTs guarantee that under conflicting operations --
that is, operations that could lead to different states as shown above
--, there is a systematic way to \emph{detect conflicts}, and moreover, there
is a strategy to \emph{deterministically solve conflicts} which will be followed
by all replicas.

% {\color {red}CW: Here ``detect conflict'' may confuse the readers. When we do the downstream, we do not distinguish whether this downstream comes from a concurrent operation or from a visible operation.}
% \gpnote[nomargin, inline]{I don't think it's confusing or inaccurate.
%   The downstream + algorithm already embeds the conflict resolution
%   mechanism (timestamps, vector clocks, etc.). So the conflict
%   detection + resolution is there.}

% \ce{To simplify, the RGA uses now timestamps that are simply integers. The following text assumes that timestamps are pairs of replica ids and local clocks.}

% \ce{I would say that for RGA, two \lstinline|addAfter| operations conflict only if they try to add something after exactly the same element (like in the example given above). And this conflict is solved looking at the timestamps. The next paragraph uses a more general notion of conflict.}

In the specific case of RGA, the implementation adds metadata to each
item added to the list identifying the originating replica as well as
timestamp of the operation in that replica.\footnote{We will ignore in
  this section conflicts due to \lstinline|remove| operations.}
%
This metadata is enough to detect when conflicts have occurred.
%
Generally there are a number of assumptions that are necessary for the
metadata to detect conflicts (for instance that timestamps increase
monotonically with time, that replica identifiers are unique, etc.)
which we shall discuss in the next sections.
%
Then, for the case of the RGA data type, whose implementation will be
presented in the next section, it is enough to know whether
two \lstinline|addAfter| operations have conflicted by simply
comparing the replica identifiers and their timestamps.
%
In fact, this is a sound over-approximation of conflict since two
concurrent \lstinline|addAfter| operations have a real conflict only
if their first arguments are the same (eg. the element \lstinline|b|
in the example aforementioned).
%
In such case, the strategy to resolve the conflict will always choose
to order first in the resulting list the character added with the
highest timestamp, and in the particular case where the timestamps
should be the same, an arbitrary order among replicas will be used.
%
Going back to the example above, and assuming that the character
\lstinline|c| was added with timestamp $t_1$ and the character
\lstinline|d| was added with timestamp $t_2$, if we assume that
$t_2 < t_1$ for some order $\leq$ between timestamps, then the list will
converge to $\mathtt{a \cdot b \cdot c \cdot d \cdot e}$.
%
We obtain the same result if $t_1 = t_2$ and, assuming that we have a replica
order $<_r$, we have $\arep_2 <_r \arep_1$.
%
In any other case we obtain $\mathtt{a \cdot b \cdot d \cdot c \cdot e}$.
%
The idea of using an arbitrary order among replica identifiers is
often used in CRDT implementations to break ties among elements with
equal timestamps.
%
For this reason in the rest of the paper we will assume that metadata
provides a strict ordering relation to abstract this detail away.

If the effects of all operations are
delivered to all replicas eventually, the replicas will converge to
the same state -- assuming a quiescent period of time where no new
operations are performed.
%
We have somehow regained consistency of the data type without
giving away availability.

\smallskip
\noindent
{\bf Specifications.}
The simplicity of the list data type with the API that we have
described above allows for a somewhat simple conflict resolution
strategy.
%
Essentially, any strategy ordering conflicting concurrent insertions
to the list in a deterministic way will work.
%
However, this is not true for many other CRDT implementations.
%
It is therefore critical to provide the programmer with a clear, and
hopefully formal, specification of the allowed behaviors of the data
type under conflicts.
%
Unfortunately this is not an easy task.
%
Many times the programmer has no option but to read the implementation
of the data type to understand how the metadata is used to 
resolve conflicts, for instance by reading the algorithms
by~\citet{ShapiroPBZ11} (an exceptional case where the algorithms are
particularly well documented).
%
Recently Burckhardt et al.~\cite{BurckhardtGYZ14, Burckhardt14} have
developed a formal framework where CRDTs and other weakly
consistent systems can be specified.
%
However, we consider that reading these specifications is far from
trivial for the average programmer, let alone writing new
specifications.
%
Evidently, having a formal specification is a necessary step towards
the verification of the implementations of CRDTs.
%
These are the problems we address in this paper.

%\smallskip
\noindent
{\bf Paper Contributions.}
Inspired by linearizability~\cite{HerlihyW90} and the literature of
shared-memory concurrency we propose a \emph{new consistency criterion
  for CRDTs}, which we call \emph{\CRDTLin{}} (\CRDTLinshort{}).
%
\CRDTLinshort{} both simplifies CRDT specifications, and allows us to
give correctness proof strategies for these specifications.
%
To satisfy the \CRDTLinshort{} criterion a data type must be so that
any execution of a client interacting with an instance of the data
type
\begin{inparaenum}
\item should result in a state that can be obtained as a sequence (or
  linearization) of its updates -- where we assume that all updates
  are executed sequentially--, and
\item any operation reading the state of the data type instance should
  be justified by executing a \emph{sub-sequence} of the above
  mentioned sequence of updates.
\end{inparaenum}
For instance, for the RGA example, the state of the final
list (when all updates are delivered) should be reachable by considering a sequence where all
\lstinline|addAfter|  operations are executed sequentially.\footnote{We
  will come back to RGA to add \lstinline|remove|
  in \sectionautorefname~\ref{sec:overview}.}
%
This definition shares some similarities with~\citet{PerrinMJ14}. 
%
We address the main differences in \sectionautorefname~\ref{sec:rel-work}.

Equipped with this criterion we show that multiple existing CRDTs are
\crdtlinearizable{}.
%
We provide both, their \CRDTLinshort{} specification, and proofs
showing that implementations respect the specification.
%
Encouragingly, this is true for most of the CRDTs
by~\citet{ShapiroPBZ11}.

Given that our criterion is inspired by
linearizability, it is natural to ask whether it
preserves the same compositionality properties, i.e.
whether the composition of a set of RA-linearizable objects is also RA-linearizable.
%
We show that this is not true in general, but we identify classes of
CRDT implementations, based on the conflict-resolution strategy that
they implement, for which compositionality can be achieved. Interestingly, the characterization of these 
classes of objects stems naturally from the proof methodology we use to show that they are RA-linearizable.

% \gpwarning[nomargin, inline]{Don't know if we should say more here.}

% The paper is structured as follows:
% \sectionautorefname~\ref{sec:overview} gives an overview of our notion
% of RA-linearizability, \sectionautorefname~\ref{sec:distributed-lin}
% provides the formalization of the CRDT semantics and
% RA-linearizability, \sectionautorefname~\ref{sec:proofs} describes a
% proof methodology for establishing RA-linearizability, and
% \sectionautorefname~\ref{sec:compositionality} discusses
% compositionality properties.
% We discuss the related work in \sectionautorefname~\ref{sec:rel-work}
% and conclude in \sectionautorefname~\ref{sec:conclusion}.


%%% Local Variables:
%%% mode: latex
%%% TeX-master: "draft"
%%% End:


%!TEX root = draft.tex

\section{CRDT Implementations}
\label{sec:CRDT implementations}



%Let $\mathbb{MSG}$ be the set of message contents, such as $(a,ts_a,ts_b)$ of RGA. Then, CRDT implementations are defined as follows, where operations and receiving messages are defined as functions.
%
%\begin{definition}[CRDT implementations]
%\label{definition:operation-based CRDT implementations}
%A CRDT implementation for a type $t = (M,D)$ is a tuple $I(r) = (\Sigma, \Sigma_0, \mathit{Msg}, \mathit{do},\mathit{receive})$. Here $r \in \mathbb{R}$, $\Sigma_0 \subseteq \Sigma$, $\mathit{Msg} \subseteq \mathbb{MSG}$, $\mathit{do}:\Sigma \times M \times D \rightarrow \Sigma \times D \times (\mathit{Msg} \cup \{ \emptyset \} )$, and $\mathit{receive}: \Sigma \times \mathit{Msg} \rightarrow \Sigma$.
%\end{definition}
%
%Here $\Sigma$ is the set of local states and $\Sigma_0$ is the set of initial state. $r$ is the replica identifier of current replica, and some CRDT implementation requires current replica identifier to generate time-stamp. When the current local state is $\sigma$ and the client intends to perform a operation of method $m$ with argument $a$, a $\mathit{do}$ action is launched, which update the local states, returns a value, and generate message if $m$ is a update method. When this replica receives a message of other replica, a $\mathit{receive}$ action will be launched, which updates the current local states according to the message. If a operation has no arguments or return value, or does not generate message, then we can safely omit the corresponding tuples in $\mathit{do}$ action.
%% The formal definition of RGA can be easily obtained from its algorithms. For example, when $(a,\_,\_) \in N$, we have $\mathit{do}((N,\mathit{Tomb}),\mathit{rem},a) = ((N,\mathit{Tomb} \cup \{ a \}),a)$. Here we use $\_$ in indicate a element whose value is irrelevant.
%The formal definition of RGA, as well as more CRDT implementations, can be found in  Appendix \ref{sec:appendix definitions of section CRDT implementations}.

















\forget{
\section{CRDT Implementations}
\label{sec:CRDT implementations}

A distributed system contains multiple objects, and each objects is replicated on each replica. Each object has a type, which contains its method and data type. A client of a replica interact with the objects by calling the method and then obtaining the return value. Here we do not bound the number of replica identifiers and objects.

Let $\mathbb{OBJ}$ be the set of objects and $\mathbb{R}$ be the set of replica identifiers. We consider a finite set $\mathbb{M}$ of method names; and a possibly infinite set $\mathbb{D}$ of arguments and return values, the data domain. Each data type $t = (M,D)$ has a set $M \subseteq \mathbb{M}$ of methods and a data domain $D \subseteq \mathbb{D}$. Finally we have a infinite set $\mathbb{OID}$ of operation identifiers, corresponding to each individual operation performed on the CRDT throughout an execution.

Without loss of generality we will consider that the methods in $\mathbb{M}$ can be separated in two disjoint sets of methods: $\mathbb{Q}$ query methods that has no influence on the ``abstract state'' and normally returns an observation of the ``abstract state'' , and $\mathbb{U}$ update methods that has influence on the ``abstract state''. Note that some update operation also need to read the ``abstract state''. For example, a $add(a,b)$ operation is an operation of distributed list which intends to put item $a$ immediately after item $b$. This operation implicitly requires that item $b$ is already in list.

$\mathit{Optimistic \ replication \ algorithms}$ is a type of distributed algorithms where each client contains a copy of data structure; a client operations takes effect instantly at its replica without any synchronization, and then broadcast to other replicas and got applied. Convergent or Commutative Replicated Data Types (CRDTs) is a typical kinds of optimistic replication algorithms. In this section, we will introduce CRDT algorithms and their formation.

In practice, there are two kinds of CRDT implementations: state-based CRDT and operation-based CRDT. In state-based CRDT, a update operation take effects locally; in nondeterministic time, a replica may decide to send the (modified) local state into other replicas. The state-based PN-counter is an example of state-based CRDT algorithms and is shown below. Keyword $\mathit{payload}$ indicate the local state, and keyword $\mathit{initial}$ specifies the initial value of local state. Function $\mathit{myID}()$ returns the current replica identifier, and $\mathit{reps}()$ returns the number of replicas of the distributed system. Vector $P$ (resp., $N$) is a vector such that $P[i]$ (resp., $N[i]$) is the number of increase that is generated by replica $i$ and is observed by current replica. This algorithm assumes that the set of replica is already known and is fixed and finite.

Method $\mathit{inc}$ increase the counter by $1$, method $\mathit{dec}$ decrease the counter by $1$, and method $\mathit{read}$ returns the current counter value. Assume the replica identifier of current replica is $r$. When the current replica does $\mathit{inc}$, it modify $P[r]$ into $P[r]+1$. When the current replica does $\mathit{dec}$, it modify $N[r]$ into $N[r]+1$. When the current replica does $\mathit{read}$, it returns $\Sigma_{i}^{n} P[i] - \Sigma_{i}^{n} N[i]$. When the current replica receive a message of modified payload $Z$, it uses function $\mathit{merge}()$ to update the current local state. $\mathit{merge}$ takes the maximum of each replica in the vector.

\renewcommand{\algorithmcfname}{CRDT Implementation}
\noindent
%\begin{minipage}{.5\textwidth}
\noindent\begin{algorithm}[H]
$\mathit{payload}$ integer[$\mathit{reps}$()] P, integer[$\mathit{reps}$()] N; \\
$\mathit{initial}$ [0,\ldots,0],[0,\ldots,0]; \\

$\mathit{inc}()$ \\
%\ \ \ \ let \ g = myID();\\
\ \ \ \ P[$\mathit{myID}$()] = P[$\mathit{myID}$()] + 1; \\

$\mathit{dec}()$ \\
%\ \ \ \ let \ g = myID();\\
\ \ \ \ N[$\mathit{myID}$()] = N[$\mathit{myID}$()] + 1; \\

$\mathit{read}()$ \\
\ \ \ \ \KwRet $\Sigma_{i}^{n} P[i] - \Sigma_{i}^{n} N[i]$; \\

$\mathit{merge}(Z)$ \\
\ \ \ \ $\forall i$, $P[i] = \mathit{max}(P[i],Z.P[i])$; \\
\ \ \ \ $\forall i$, $N[i] = \mathit{max}(N[i],Z.N[i])$; \\
\caption{State-based PN-counter}
\label{Method1}
\end{algorithm}

In operation-based CRDT, an update operation not only updates its local state, but also sends a description of this operation into other replica. Here we take a more complex algorithm, replicated growable array (RGA), as an example of operation-based CRDT and it is shown below.

\renewcommand{\algorithmcfname}{CRDT Implementation}
\noindent
%\begin{minipage}{.5\textwidth}
\noindent\begin{algorithm}[H]
$\mathit{payload}$ TI-tree N, set $\mathit{Tomb}$; \\
$\mathit{initial}$ $\emptyset$,$\emptyset$; \\

$add(a,b)$ \\
\ \ $\mathit{atSource}$: \\
\ \ \ \ $\mathit{pre}$: \ $b = \circ \vee ( b \neq \circ \wedge (b,\_,\_) \in N \wedge b \notin \mathit{Tomb})$ \\

%\ \ \ \ \If {$N = \emptyset$}
%    { \ \ \ \ let \ $ts_a$ = (myID(),1); \\ }
%\ \ \ \ \Else
%    {\ \ \ \ let \ $ts_a$ = (myID(),$\mathit{max}\{ c' \vert (\_,(\_,c'),\_) \in N \} +1$); \\ }

%\ \ \ \ \If {$b = \circ$}
%    { \ \ \ \ let \ $ts_b$ = (0,0); \\ }
%\ \ \ \ \Else
%    { \ \ \ \ let \ $ts_b$ be time-stamp of $b$ in $N$; \\ }

\ \ \ \ let \ $ts_a$ = ($N = \emptyset$) ? (1,$\mathit{myID}$()) ! ($\mathit{max}\{ c' \vert (\_,(\_,c'),\_) \in N \} +1$,$\mathit{myID}$()); \\
\ \ \ \ let \ $ts_b$ = ($b = \circ$) ? (0,$r_0$) ! (the time-stamp of $b$ in $N$); \\

\ \ $\mathit{downstream}(a,ts_a,ts_b)$: \\
\ \ \ \ $\mathit{pre}$: \ $b = \circ \vee ( b \neq \circ \wedge (b,ts_b,\_) \in N)$ \\

\ \ \ \ $N = N \cup \{ (a,ts_a,ts_b) \}$.


$rem(a)$ \\
\ \ $\mathit{atSource}$: \\
\ \ \ \ $\mathit{pre}$: \ $a \neq \circ \wedge (a,\_,\_) \in N \wedge a \notin \mathit{Tomb}$ \\

\ \ $\mathit{downstream}(a)$: $\mathit{pre}$ \ $a \neq \circ \wedge (a,\_,\_) \in N)$

\ \ \ \ $\mathit{Tomb} = \mathit{Tomb} \cup \{ a \}$.

$read()$ \\
\ \ \ \ \KwRet $\mathit{trans}(N,\mathit{Tomb})$; \\

\caption{RGA}
\label{Method1}
\end{algorithm}

Each update operation of operation-based CRDT ie executed with two phases: Its first phase, marked $\mathit{atSource}$, is local to the current replica. It is enabled if its (optional) pre-condition, marked $\mathit{pre}$, is true currently in local state. It generates the information to be delivered, which is the argument of $\mathit{downstream}$. Note that this phase does not modify the local state. Its second phase, marked $\mathit{downstream}$, executed immediate after the current replica, and asynchronously at other replica when they receive the message of this operation. It is enabled if its (optional) pre-condition is true.

In RGA algorithm, a replica store the list as a timestamp insertion tree (TI-tree) $N$, and stores the deleted items in tombstone $\mathit{Tomb}$. A TI-tree $N$ is a set of tuples $(a,t,p)$, where $a$ is a item, $t$ is its unique time-stamp, and $p$ is the time-stamp of its ``parent'' node. Each time-stamp is a tuple $(c,r)$ with $c \in \mathbb{N}$ and $r \in \mathbb{R}$. A order $<_{\mathit{ts}}$ between time-stamps is defined, such that $(c_1,r_1) <_{\mathit{ts}} (c_2,r_2)$, if $c_1 < c_2 \vee (c_1 = c_2 \wedge r_1 <_r r_2)$, where $<_r$ is a total-order over $\mathbb{R}$. There is a pre-existed item $\circ$ of TI-tree with time stamp $(0,r_0)$, which are considered as the root of the tree. Each element of $N$ should have unique item and time stamp, and the elements of $N$ are required to form a tree by following the parent field. The tombstone $\mathit{Tomb}$ is a set of items and records items been removed from the list.

Method $\mathit{add}(a,b)$ intends to add item $a$ into the list immediately after a existing item $b$. Method $\mathit{rem}(a)$ removes $a$ from the list. Method $\mathit{read}$ returns the current list content. When the current replica does $\mathit{add}(a,b)$, it generate a tuple $(a,ts_a,ts_b)$ and put it into $N$. Here $ts_b$ is the time-stamp of $b$, and $ts_a$ is a new time-stamp that is larger than any time stamp in $N$. When the current replica does $\mathit{rem}(a)$, it put $a$ into tombstone. When the current replica does $\mathit{read}$, it uses function $\mathit{trans}(N,\mathit{Tomb})$ to return the list seen by the current replica, which is a sequences obtained by traversing $N$ in prefix order (children are visited in decreasing time-stamp order) and keeping only items that are not in $\mathit{Tomb}$.


Multi-value register is also a common-used data structures and its sequential specification is nondeterministic and thus different from that of the previous two examples, which are deterministic (seen in the next section). A state-based multi-value register algorithm is shown below.


\renewcommand{\algorithmcfname}{CRDT Implementation}
\noindent
%\begin{minipage}{.5\textwidth}
\noindent\begin{algorithm}[H]
$\mathit{payload}$ $S \subseteq D \times \mathbb{N}^{\mathit{reps}()}$; \\
$\mathit{initial}$ $\emptyset$; \\

$\mathit{write}(a)$ \\
\ \ \ \ let \ g = $\mathit{myID}$(); \\
\ \ \ \ let $\mathcal{V} = \{ V \vert \exists x, (x,V) \in S \}$; \\
\ \ \ \ let $V' = [ \mathit{max}_{V \in \mathcal{V}} V[j] ]_{j \neq g}$; \\
\ \ \ \ let $V'[g] = (\mathit{max}_{V \in \mathcal{V}} V[g]) + 1$; \\
\ \ \ \ $S = (a,V')$; \\

$\mathit{read}()$ \\
\ \ \ \ \KwRet $S' = \{ a \vert (a,\_) \in S \}$; \\

$\mathit{merge}(Z)$ \\
\ \ \ \ let $A' = \{ (x,V) \in S \vert \forall (x',V') \in Z.S, \exists i, V[i] \geq V'[i] \}$; \\
\ \ \ \ let $B' = \{ (x,V) \in Z.S \vert \forall (x',V') \in S, \exists i, V[i] \geq V'[i] \}$; \\
\ \ \ \ $S = A' \cup B'$; \\
\caption{state-based multi-value register}
\label{Method1}
\end{algorithm}

Each replica stores a set $S$ of items such that each item can not dominate other items. To do conflict resolution, we associate each item $a$ in $S$ with a version vector $V$. We say version vector $V$ dominates version vector $V'$, if $\forall i$, $V[i] > V'[i]$.

Method $\mathit{write}(a)$ intends to write $a$ into register. Method $\mathit{read}$ returns the current register content. When the current replica does $\mathit{write}(a)$, it generates a new version vector that dominates all previous ones in $S$. When the current replica does $\mathit{read}$, it returns the set of items in $S$. When the current replica receive a message of modified payload $Z$, we takes the union of every items in $S$ and $Z.S$ whose version vector is not dominated by that of an item in the other set. This algorithm assumes that the set of replica is already known and is fixed and finite.


To enable formally verification of CRDT algorithms, it is necessary to give formal definition of CRDT-algorithms. Let $\mathbb{MSG}$ be the set of message contents, such as $(P,N)$ of state-based PN-counter, or $(a,ts_a,ts_b)$ of RGA. Then, CRDT implementations are defined as follows, where operations and receiving messages are defined as functions.

\begin{definition}[operation-based CRDT implementations]
\label{definition:operation-based CRDT implementations}
A operation-based CRDT implementation for a type $t = (M,D)$ is a tuple $I_t(r) = (\Sigma, \Sigma_0, \mathit{Msg}, \mathit{do},\mathit{receive})$. Here $r \in \mathbb{R}$, $\Sigma_0 \subseteq \Sigma$, $\mathit{Msg} \subseteq \mathbb{MSG}$, $\mathit{do}:\Sigma \times M \times D \rightarrow \Sigma \times D \times (\mathit{Msg} \cup \{ \emptyset \} )$, and $\mathit{receive}: \Sigma \times \mathit{Msg} \rightarrow \Sigma$.
\end{definition}

Here $\Sigma$ is the set of local states and $\Sigma_0$ is the set of initial state. For example, in state-based PN-counter, since there are many possibility of total number of replicas, $\Sigma_0$ is a set of more than one elements. $r$ is the replica identifier of current replica. The reason of containing $r$ in the definition of CRDT implementations is that, some algorithms need the current replica identifier to generate time-stamp. When the current local state is $\sigma$ and the client intends to perform a operation of method $m$ with argument $a$, a $\mathit{do}$ action is launched, which update the local states, returns a value, and possibly generate messages. A $\mathit{do}$ action of update method will generate messages, while a $\mathit{do}$ action of query method will not generate message. When this replica receives a message of other replica, a $\mathit{receive}$ action will be launched, which updates the current local states according to the message. If a operation has no arguments or return value, or does not generate message, then we can safely omit the corresponding tuples in $\mathit{do}$ actions.

\begin{definition}[state-based CRDT implementations]
\label{definition:state-based CRDT implementations}
A state-based CRDT implementation for a type $t = (M,D)$ is a tuple $I_t(r) = (\Sigma, \Sigma_0, \mathit{Msg}, \mathit{do},\mathit{receive})$. Here $r \in \mathbb{R}$, $\Sigma_0 \subseteq \Sigma$, $\mathit{Msg} \subseteq \Sigma$, $\mathit{do}:\Sigma \times M \times D \rightarrow \Sigma \times D$, and $\mathit{receive}: \Sigma \times \Sigma \rightarrow \Sigma$.
\end{definition}

The state-based CRDT implementation is similarly defined. The difference is that, each operation does not send message, and the message content is fixed to be a local state. The following is an example of formal definition of state-based PN-counter. The formal definition of more CRDT implementations are given in Appendix \ref{sec:appendix definitions of section CRDT implementations}. Here we denote by $f[i:j]$ the function that has the same value as $f$ everywhere, except for $i$, where it has the value $j$. %Since each operation is executed without synchronization, it is not hard to obtain formal definition from informal algorithms.

\begin{example}[formal definition of state-based PN-counter]
\label{definition:formal definition of state-based PN-counter}
$I_t(r) = (\Sigma, \sigma_0, \mathit{Msg}, \mathit{do},\mathit{receive})$, where

\begin{itemize}
\setlength{\itemsep}{0.5pt}
\item[-] $\Sigma = \{ (P,N) \vert$, $P$ and $N$ are vector of integers with same length $\}$. $\Sigma_0 = \{ (P_0,N_0) \vert (P_0,N_0) \in \Sigma$, $P_0$ and $N_0$ maps each index into $0 \}$.

\item[-] $\mathit{Msg} \subseteq \Sigma$.

\item[-] $\mathit{do}((P,N),\mathit{inc}) = (P[r:P[r]+1],N)$,

\item[-] $\mathit{do}((P,N),\mathit{dec}) = (P,N[r:N[r]+1])$,

\item[-] $\mathit{receive}((P,N),(P',N')) = (\lambda s. \mathit{max}\{  P[s], N'[s] \}, \mathit{max}\{  N[s], N'[s] \},)$,
\end{itemize}
\end{example}
}































%%% Local Variables:
%%% mode: latex
%%% TeX-master: "draft"
%%% End:


%!TEX root = draft.tex
%\newcommand{\seqPQ}{\mathsf{SeqPQ}}


\section{Definition of Linearizability}
\label{sec:definition of linearizability} 

Let us start our formation of executions, specifications and linearizations of CRDT.  

We consider a finite set $\mathbb{M}$ of method names; and a possibly infinite set $\mathbb{D}$ of arguments and return values, the data domain. Without loss of generality we will consider that the methods in $\mathbb{M}$ can be separated in two disjoint sets of $\mathbb{Q}$ query methods, and $\mathbb{U}$ update methods. We consider replicated data types which are distributed across a set of replicas; the set of replica identifiers is denoted by $\mathbb{R}$. We assume that each replica contains a copy of the data type state. Finally we have a infinite set $\mathbb{O}$ of operation identifiers, corresponding to each individual operation performed on the CRDT throughout an execution.

Operation labels \mbox{$m(a)\Rightarrow b$} with $m \in \mathbb{M}$ and $a,b \in \mathbb{D}$, indicate that the operation is a call to method $m$ with argument $a$ and the result of the operation is the value $b$. When $m$ does not use the argument (resp., return value), we write $m()\Rightarrow b$ (resp., $m(a)$) instead. We define an operation $o$ to be a tuple $(\ell,i)$, where $\ell$ is an operation label and $i \in \mathbb{O}$ is a unique operation identifier. 

\noindent {\bf Sequential Specification:} A sequential specification is used to state the sequential intuition of operations. Let specification alphabet $\mathbb{A}$ be a set of a tuples $(o,s)$, where $o$ is an operation and $s$ is a set of operation identifiers. The reason of introducing such set of operation identifiers is that, in sequential specifications some operations should only influence a subset of previous operations. Such operation identifier set is just the operations influenced by this operation. With specification alphabet, a sequential specification of CRDT is given as a set of sequences over specification alphabets as follows. From now on, we implicitly assume that in a sequence of specification alphabets, each item has a unique operation identifier. 

\begin{definition}[Sequential Specification]
\label{definition:sequential specification} 
A sequential specification $\mathit{spec}_s \subseteq \mathbb{A}^*$ is a set of strings over specification alphabet $\mathbb{A}$. 
\end{definition} 

\noindent {\bf Distributed Specification:} Each element of distributed specification is a tuple $(s,v)$, where $s \subseteq \mathbb{A}^*$ is a sequence of specification alphabet, and $v$ is a function that maps each item $a$ of $s$ into a subset of items before $a$ in $s$. Here $s$ essentially is the linearization of an execution, while $v$ is used to ensure each specification alphabet is correct in this sequence. %Given a sequence $s \subseteq \mathbb{A}^*$ of specification alphabets and a specification alphabets $a$, let $\mathit{itm}(s)$ be the set of specification alphabets of $s$, and $\mathit{itmBef}(s,a)$ be the set of specification alphabets of $s$ that appears before $a$, or $\emptyset$ otherwise. 
Given a sequence $s$ and a set of operations $S$, let $s \uparrow_{S}$ be the projection of $s$ over $S$. Given a sequence $s$ and an item $a$ of $s$, let $\mathit{bef}(s,a)$ contains the set of items of $s$ that appear before $a$ in $s$, as well as all the subsets of this set. 

\begin{definition}[Distributed Specification]
\label{definition:distributed specification}
A distributed specification $\mathit{spec}_d$ w.r.t a sequential specification $\mathit{spec}_d$ is a set of tuples $(s,v)$, where $s \subseteq \mathbb{A}^*$, and %$v: \mathit{itm}(s) \rightarrow 2^{\mathit{items}(s)}$ is a function that maps each item $a$ into $\mathit{itmBef}(s,a)$.
%$v$ is a function that maps each specification alphabet $a$ of $s$ into a subset of specification alphabet of $s$ that appears before $a$ in $s$. 
$v$ is a function that maps each specification alphabet $a$ of $s$ into either $\{ a \} \cup x$ with $ x \in \mathit{bef}(s,a)$ or $\emptyset$. Moreover, the following condition need to be satisfied:

\begin{enumerate}[(i)]
\item For each specification alphabet $a$ of $s$, $s \uparrow_{v(a)} \in \mathit{spec}_s$. 
\end{enumerate} 
\end{definition} 

The examples of sequential specifications of typical CRDT are given below. To give sequential specification, we use the style of pre-condition and post-condition for each specification alphabet.


\begin{example}[Counter]
\label{definition:sequential specification of counter}
The sequential specification $\mathit{Counter}_s$ of counter are given as follows: Let $state$ be a natural number. 

\begin{itemize}
\setlength{\itemsep}{0.5pt}
\item[-] $\{ state = i \}$ $inc$ $\{ state = i+1 \}$.
\item[-] $\{ state = i \}$ $read() \Rightarrow i$ $\{ state = i+1 \}$.
\end{itemize} 
\end{example} 


\begin{example}[Set]
\label{definition:sequential specification of set}
The sequential specification $\mathit{Set}_s$ of set are given as follows: Here we assume that each item is put into the set only once. Let $state$ be a set and each its element $(a,flag)$ is a tuple of a data $a$ and a flag $flag \in \{ \mathit{true},\mathit{false} \}$. 

\begin{itemize}
\setlength{\itemsep}{0.5pt}
\item[-] $\{ state = S \wedge a \notin S \}$ $add(a)$ $\{ state = S \cup \{ (a,\mathit{true}) \} \}$.
\item[-] $\{ state = S \wedge S' = \{a \vert (a,\mathit{true}) \in S \} \}$ $read() \Rightarrow S'$ $\{ state = S \}$.
\item[-] $\{ state = S \wedge (a,\_) \in S \}$ $rem(a)$ $\{ state = S \setminus \{ (a,\_) \} \cup \{ (a,\mathit{false}) \} \}$.
\end{itemize} 
\end{example} 



\begin{example}[OR-Set]
\label{definition:sequential specification of or-set}
The sequential specification $\mathit{OR-Set}_s$ of OR-set are given as follows: Let $state$ be a set and each its element $(a,id,flag)$ is a tuple of a data $a$, a operation identifier $id$, and a flag $flag \in \{ \mathit{true},\mathit{false} \}$.
\begin{itemize}
\setlength{\itemsep}{0.5pt}
\item[-] $\{ state = S  \}$ $((add(a),\mathit{id}),\emptyset)$ $\{ state = S \cup \{ (a,\mathit{id},\mathit{true}) \} \}$.
\item[-] $\{ state = S \wedge S' = \{ a \vert (a,\_,\mathit{true}) \in S \} \}$ $read() \Rightarrow S'$ $\{ state = S \}$. 
\item[-] $\{ state = S  \wedge S_1 \subseteq \{a \vert (a,\_,\_) \in S\} \}$ $((rem(a),\_),S_1)$ $\{ state = S_2  \}$. Here $S_2$ is obtained from $S$ by marking each $S$ item with $\mathit{false}$ flag. 
\end{itemize}
\end{example} 


\begin{example}[Register]
\label{definition:sequential specification of register}
The sequential specification $\mathit{Reg}_s$ of register are given as follows: Let $state \in \mathbb{D}$ be a value.
\begin{itemize}
\setlength{\itemsep}{0.5pt}
\item[-] $\{ state = a  \}$ $write(b)$ $\{ state = b \}$.
\item[-] $\{ state = a \}$ $read() \Rightarrow a$ $\{ state = a \}$. 
\end{itemize}
\end{example}


\begin{example}[Multi-value Register]
\label{definition:sequential specification of multi-value register}
The sequential specification $\mathit{MVReg}_s$ of multi-value register are given as follows: Let $state$ be a set and each its element $(a,id,flag)$ is a tuple of a data $a$, a operation identifier $id$, and a flag $flag \in \{ \mathit{true},\mathit{false} \}$.
\begin{itemize}
\setlength{\itemsep}{0.5pt}
\item[-] $\{ state = S \wedge \forall x \in S_1, (b,x,\mathit{true}) \in S_1 \vee (b,x,\mathit{false}) \in S_1 \}$ $((write(b),id),S_1)$ $\{ state = S_2 \}$. Here $S_2$ is obtained from $S$ by mark each $(b,x)$ with $\mathit{false}$, and then insert $(b,id,\mathit{true})$. 
\item[-] $\{ state = S \wedge S' = \{ a \vert (a,\_,\mathit{true}) \in S \} \}$ $read() \Rightarrow S'$ $\{ state = S \}$. 
\end{itemize}
\end{example} 


\begin{example}[List with add-after interface]
\label{definition:sequential specification of list with add-after interface} 
The sequential specification $\mathit{List}_s$ of list are given as follows: Let $state$ be a sequence, where each item is a tuple $(a,flag)$ with data $a$ and flag $flag \in \{ \mathit{true},\mathit{false} \}$. 
\begin{itemize}
\setlength{\itemsep}{0.5pt}
\item[-] $\{ state = (a_1,\_) \cdot \ldots \cdot (a_n,\_) \wedge l \leq n \wedge a_k \notin \{ a_1, \ldots, a_n \} \}$ $add(a_k,a_l)$ $\{ state = (a_1,\_) \cdot \ldots \cdot (a_l,\_) \cdot (a_k,\mathit{true}) \cdot (a_{l+1},\_) \cdot \ldots \cdot (a_n,\_) \}$.
\item[-] $\{ state = (a_1,\_) \cdot \ldots \cdot (a_n,\_) \wedge S = \{ a \vert (a,\mathit{true}) \in state \} \wedge l = a_1 \cdot \ldots \cdot a_n \uparrow_{S} \}$ $read() \Rightarrow l$ $\{ state = (a_1,\_) \cdot \ldots \cdot (a_n,\_) \}$. 
\end{itemize}
\end{example} 



As customary, to capture the notion of client-observable effects of an execution over a CRDT, we will define the notion of \emph{history}. A history contains a set of operations, and the order in which they were effected in each replica. Formally, a history $h$ is a tuple of the form $h = (O,\mathit{lab},\mathit{ro})$ where $O$ is a set of operation identifiers, $\mathit{lab}$ is a function that maps each operation identifiers of $O$ into a operation label, and $\mathit{ro}$ is a union of transitive, irreflexive and total orders of $O$.

A history is distributed linearizable w.r.t a distributed specification, if we can find a visibility relation $\mathit{vis}$ of the history and a total sequence $s$ in distributed specification, such that $\mathit{vis}$ is consistent with $s$. Formally,

\begin{definition}[Distributed Linearizability] 
\label{definition:distributed linearizability} 
A history $h = (O,\mathit{ro})$ is distributed linearizable w.r.t distributed specification $\mathit{spec}_d$, if $\exists (s,v) \in \mathit{spec}_d$, $\exists \mathit{vis} \subseteq O \times O$ be a acyclic visibility relation, such that

\begin{enumerate}[(i)]
\item $\mathit{ro} \subseteq \mathit{vis}$, 
\item $\mathit{vis} \subseteq s$, 
\item For each operation identifiers $i$ of $h$, if $v(i) \neq \emptyset$, then $v(i) = \{ i \} \cup \{ a \vert (a,i) \in \mathit{vis} \}$.  
\end{enumerate}

A set $H$ of histories is is distributed linearizable w.r.t distributed specification $\mathit{spec}_d$, if each of its history is. 
\end{definition}




%!TEX root = draft.tex
%\newcommand{\seqPQ}{\mathsf{SeqPQ}}

\section{CRDT Implementation Correctness}
\label{sec:CRDT implementation semantics and correctness}

In this section, we propose the semantics of CRDT implementations. Based on our definition, we can extract histories from an execution. Then we define the correctness of an CRDT implementation. 



%\subsection{Semantics of Single Object}
%\label{subsec:semantics of single object}

CRDT implementations assume the following two guarantees of downstream:

\begin{itemize}
\setlength{\itemsep}{0.5pt}
\item[-] The downstream of each operation is applied at most once for each replica. 
\item[-] Assume the downstream of operation $\aop_1$ is applied in replica of $\aop_2$ before $\aop_2$ happens. Then, for each replica $\arep$, the downstream of $\aop_2$ can be applied only if the downstream of $\aop_1$ has already been applied. 
\end{itemize}

Given a CRDT object $\aobj$, its semantics is defined as a transition system $\llbracket \aobj \rrbracket_{\mathit{op}} = (\mathit{Config},\mathit{config}_0,\Sigma',\rightarrow)$ as in \figurename~\ref{fig:crdt-opsem}.
 


%\gpfatal[nomargin, inline]{remove if we decide to keep the other semantics}

\begin{figure}[t]
  \centering

\begin{itemize}
\item $ \Delta : \states \rightarrow \states \ni \effector$ \hspace{\fill} Effector or Downstream
%\item $\semop[]{} : \labels \times \states \rightarrow (\states \rightarrow \states)$ \hspace{\fill} Operational Semantics of a Single Operation (Label)
\item $\localstates : \powerset{\labels} \times \states \ni (\alabelset, \astate)$ \hspace{\fill} Local States
\item $\globalstates : (\reps \rightarrow \localstates) \times \powerset{\labels \times \labels} \times (\labels \rightarrow \Delta) \ni (\gstates, \avisord, \downstreams )$ \hspace{\fill} Global States
\end{itemize}


\[
  \inferrule[\text{\sc Operation}]
  {\gstates(r) = (\alabelset, \astate) \\ \mathit{do}(\sigma,\amethod,\argv,\arep) = (\retv,\effector) \\  \effector(\astate) = \astate' \\ \alabel = \alabellongind{\argv}{\retv}{i} \\ \mathit{unique}(i) }
  {(\gstates, \avisord, \downstreams) \xrightarrow{\alabel} (\gstates[\arep \leftarrow (\alabelset \cup \{\alabel\}, \astate')],
    \avisord \cup (\alabelset \times \{\alabel\}), \downstreams[\alabel \rightarrow \effector])}
\]


\[
  \inferrule[\text{\sc DownStream}]
  {\gstates(r) = (\alabelset, \astate) \\ \alabel \in \mathsf{min}_{\avisord}(\dom{\avisord} / \alabelset) \\
    \downstreams(\alabel)(\astate) = \astate'}
  {(\gstates, \avisord, \downstreams) \xrightarrow{} (\gstates[\arep \leftarrow (\alabelset \cup \{\alabel\}, \astate')], \avisord, \downstreams)}
\]

  \caption{Operational Semantics of CRDTs}
  \label{fig:crdt-opsem}
\end{figure} 

A configuration $(\gstates, \avisord, \downstreams)$ is a snapshot of distributed system. $\gstates \in \reps \rightarrow \localstates$ stores the local configuration of each replica. A local configuration $(\alabelset, \astate)$ contains a local state $\astate$ of $\aobj$ and a set $\alabelset$ of labels: $\alabel \in \alabelset$, if the downstream of $\alabel$ has already been applied in this replica. $\avisord \in \powerset{\labels \times \labels}$ stores the visibility among operations, and $\downstreams$ stores the downstream of operations. 

$\rightarrow \in \globalstates \times \labels \times \globalstates$ is the transition relation. The first rule in \figurename~\ref{fig:crdt-opsem} describes replica $\arep$ performing an operation and (possibly) generate a new downstream. $\mathit{unique}(i)$ ensures that $i$ is a unique identifier, and then $\alabel$ is a unique operation label. A new downstream $\effector$ is generated and applied to replica $\arep$. The set $\alabelset$ and $\avisord$ is updated according to $\alabel$. The second rule describes the downstream of $\alabel$ being applied to replica $\arep$. With $\alabel \in \mathsf{min}_{\avisord}(\dom{\avisord} / \alabelset)$, we always choose a operation $\alabel$ that is minimal among operations whose downstream not applied to replica $\arep$ yet. This satisfies the guarantees of downstream. 


A sequence $l$ of operation labels is an execution of $\llbracket \aobj \rrbracket_{\mathit{op}}$, if there exists configuration $(\gstates, \avisord, \downstreams)$, such that $\mathit{config}_0 {\xrightarrow{ l }} (\gstates, \avisord, \downstreams)$. The history of $l$ is a tuple $(\alabelset, \avisord')$, where $\alabelset$ is the set of labels of $l$ and $\avisord' = \avisord$ is the visibility relation recorded in the configuration along transitions.


%\subsection{Correctness of a Single Object}
%\label{subsec:correctness of a single object}

%Given an execution $l = \alpha_1 \cdot \ldots \cdot \alpha_k$ of $\llbracket \mathit{obj} \rrbracket_{\mathit{op}}$ of CRDT object $\mathit{obj}$, we can associate unique operation identifiers with each action of $l$, and then obtain a corresponding history $\mathit{history}(l) = (\mathit{Op},\mathit{ro},\mathit{vis})$, such that

%\begin{itemize}
%\setlength{\itemsep}{0.5pt}
%\item[-] Each operation in $\mathit{Op}$ is a tuple $(m(a) \Rightarrow b,i,\mathit{obj})$, such that $i$ is the operation identifier of a $\mathit{do}(m,a,b,r)$ action of $l$.

%\item[-] $(o_1,o_2) \in \mathit{ro}$, if they are of same replica, and the index of $o_1$ in $h$ is before that of $o_2$.

%\item[-] $(o_1,o_2) \in \mathit{vis}$, if the action of $o_1$ is $\mathit{send}(\mathit{mid},\_)$, and either the action of $o_2$ is $\mathit{receive}(\mathit{mid},\_)$, or there exists $o_3$, such that the action of $o_3$ is $\mathit{receive}(\mathit{mid},\_)$, and $(o_3,o_2) \in \mathit{ro}$.
%\end{itemize}

Let $\mathit{history}(\llbracket \aobj \rrbracket_{\mathit{op}})$ be the set of histories of all executions of $\llbracket \aobj \rrbracket_{\mathit{op}}$. Then, $\aobj$ is \crdtlinearizable{}, if each of its history is, as stated by the following:

\begin{definition}[Correctness of a CRDT Object]
\label{definition:correctness of a CRDT object}
A CRDT object $\aobj$ is \crdtlinearizable{} w.r.t a sequential specification $\mathit{spec}$, if each history of $\mathit{history}(\llbracket \mathit{obj} \rrbracket_{\mathit{op}})$ is \crdtlinearizable{} w.r.t $\mathit{spec}$.
\end{definition}

















\forget{
In this section, we propose the semantics of %a state-based CRDT or a operation-based CRDT.
CRDT implementations. Then we shows how to extract histories from execution, and the correctness of CRDT implementations.



\subsection{Semantics of a Single Object}
\label{subsec:semantics of a single object}

CRDT implementations assume the following two guarantees of message delivery:

\begin{itemize}
\setlength{\itemsep}{0.5pt}
\item[-] No duplicated delivery: Each message is delivered to a replica at most once.
\item[-] Causal delivery: Given two update operations $o_1$ and $o_2$, and assume $o_1$ is visible to $o_2$. Or we can say, $o_1$ is related to $o_2$ by several times of replica order and message delivery. Then, for each replica, once it receives $o_2$, it must be the case that $o_1$ has been previously received.
\end{itemize}

Given a CRDT object $\mathit{obj}$ with $\Sigma$ be the set of local states, its semantics is defined as $\llbracket \mathit{obj} \rrbracket_{\mathit{op}} = (\mathit{Config},\mathit{config}_0,\Sigma',\rightarrow)$ as in \figurename~\ref{fig:the semantics of a operation-based CRDT object}.


\gpfatal[nomargin, inline]{remove if we decide to keep the other semantics}

\begin{figure}[t]
  \centering

\begin{itemize}
\item $ \Delta : \states \rightarrow \states \ni \effector$ \hspace{\fill} Effector or Downstream
\item $\semop[]{} : \labels \times \states \rightarrow (\states \rightarrow \states)$ \hspace{\fill} Operational Semantics of a Single Operation (Label)
\item $\localstates : \powerset{\labels} \times \states \ni (\alabelset, \astate)$ \hspace{\fill} Local States
\item $\globalstates : (\reps \rightarrow \localstates) \times \powerset{\labels \times \labels} \times (\labels \rightarrow \Delta) \ni (\gstates, \avisord, \downstreams )$ \hspace{\fill} Global States
\end{itemize}


\[
  \inferrule[\text{\sc Operation}]
  {\gstates(r) = (\alabelset, \astate) \\ \semop[\alabel]{\astate} = \effector \\ \effector(\astate) = \astate'}
  {(\gstates, \avisord, \downstreams) \xrightarrow{\alabel} (\gstates[\arep \leftarrow (\alabelset \cup \{\alabel\}, \astate', \downstreams)],
    \avisord \cup (\alabelset \times \{\alabel\}), \downstreams[\alabel \rightarrow \effector])}
\]

\[
  \inferrule[\text{\sc DownStream}]
  {\gstates(r) = (\alabelset, \astate) \\ \alabel \in \mathsf{min}_{\avisord}(\dom{\avisord} / \alabelset) \\
    \downstreams(\alabel)(\astate) = \astate'}
  {(\gstates, \avisord, \downstreams) \xrightarrow{} (\gstates[\arep \leftarrow (\alabelset \cup \{\alabel\}, \astate')], \avisord, \downstreams)}
\]

  \caption{Operational Semantics of CRDTs}
  \label{fig:crdt-opsem}
\end{figure}


\begin{figure}[ht]
$\mathit{RState} = \mathbb{R} \rightarrow \Sigma$

$\mathit{TState} = \mathbb{MID} \times \mathbb{MSG} \times \mathbb{R}$

$\mathit{MsgHB} \subseteq \mathbb{MID} \times \mathbb{MID}$

$\mathit{MsgDel} \subseteq \mathbb{MID} \times \mathbb{R}$

$\mathit{Config} = \mathit{RState} \times \mathit{TState} \times \mathit{MsgHB} \times \mathit{MsgDel}$, $\mathit{config}_0 \in \mathit{Config}$.

$\Sigma' = \mathit{do}(\mathbb{M} \times \mathbb{D} \times \mathbb{D} \times \mathbb{R} \times \mathbb{MID}) \cup \mathit{do}(\mathbb{M} \times \mathbb{D} \times \mathbb{D} \times \mathbb{R}) \cup \mathit{receive}(\mathbb{MID} \times \mathbb{R})$

\begin{itemize}
\setlength{\itemsep}{0.5pt}
\item[] $\begin{array}{l c}
   \bigfrac{
   \begin{array}{c}
     R(r) = \sigma, r.\mathit{do}(\sigma,m,a) = (\sigma',b,\mathit{msg}), \mathit{msg} \neq \emptyset, \mathit{unique}(\mathit{mid}), \\
     S_1 = \{ (\mathit{mid}',\mathit{mid}) \vert (\mathit{mid'},r) \in \mathit{MsgDel} \}, S_2 = \{ (\mathit{mid}',\mathit{mid}) \vert \mathit{mid}' \in T, \mathit{mid}' \ is \ of \ replica \ r \}
   \end{array}}
     {(R,T,\mathit{msgHB},\mathit{MsgDel}) {\xrightarrow{\mathit{do}(m,a,b,r,\mathit{mid})}} (R[r:\sigma'], T \cup \{ (\mathit{mid},\mathit{msg},r) \}, (\mathit{MsgHB} \cup S_1 \cup S_2)^*,\mathit{MsgDel})}
\end{array}$

\item[] $\begin{array}{l c}
   \bigfrac{
   \begin{array}{c}
     R(r) = \sigma, r.\mathit{do}(\sigma,m,a) = (\sigma',b,\emptyset)
   \end{array}}
     {(R,T,\mathit{msgHB},\mathit{MsgDel}) {\xrightarrow{\mathit{do}(m,a,b,r)}} (R[r:\sigma'], T \cup \{ (\mathit{mid},\mathit{msg},r) \}, \mathit{MsgHB},\mathit{MsgDel})}
\end{array}$

\item[-] $\begin{array}{l c}
   \bigfrac{
   \begin{array}{c}
      R(r) = \sigma, r.\mathit{receive}(\sigma,\mathit{msg}) = \sigma', (\mathit{mid},\mathit{msg},r') \in T, r \neq r', \\
      (\mathit{mid},r) \notin \mathit{MsgDel}, \mathit{mid} \ is \ minimal \ w.r.t \ \mathit{MsgHB} \ among \ \{ \mathit{mid}' \vert (\mathit{mid}',r) \notin \mathit{MsgDel} \}
   \end{array}}
     {(R,T,\mathit{msgHB},\mathit{MsgDel}) {\xrightarrow{\mathit{receive}(\mathit{mid},r)}} (R,T,\mathit{MsgHB},\mathit{MsgDel} \cup \{ (\mathit{mid},r) \} )}
\end{array}$
\end{itemize}
\caption{The definition of semantics of $\llbracket \mathit{obj} \rrbracket_{\mathit{op}}$}
\label{fig:the semantics of a operation-based CRDT object}
\end{figure}

A configuration $(R,T,\mathit{MsgHB},\mathit{MsgDel})$ is a snapshot of distributed system. $R$ gives the local state of each replica, and $T$ gives the set of messages that has been generated. Let $\mathbb{MID}$ be the set of message identifiers of message content. A message is a tuple $(\mathit{mid},\mathit{msg},r)$, where $\mathit{mid} \in \mathbb{MID}$ is the identifier, $\mathit{msg} \in \mathbb{MSG}$ is the message content, and $r$ is the original replica of message. $\mathit{MsgHB}$ is used to record the happen-before relation between messages: two messages $(\mathit{mid}_1,\mathit{mid}_2) \in \mathit{MsgHB}$ represents that the operation of $\mathit{mid}_1$ happens before the operation of $\mathit{mid}_2$. $\mathit{MsgDel}$ is used to record the message delivery relation between messages: $(\mathit{mid},r) \in \mathit{MsgDel}$ represents that message $\mathit{mid}_1$ has already been delivered to replica $r$. $\mathit{MsgHB}$ and $\mathit{MsgDel}$ are used to ensure message delivery requirements. $\mathit{config}_0$ is the initial configuration, which maps each replica into the initial local state, has no message, and with a empty happen-before relation and a empty message delivery relation.


Each element of $\Sigma'$ is called an action. $\rightarrow \in \mathit{Config} \times \Sigma' \times \mathit{Config}$ is the transition relation and describes a single step of distributed systems. The first rule in \figurename~\ref{fig:the semantics of a operation-based CRDT object} describes replica $r$ performs a update operation $m(a) \Rightarrow b$ and generates a message with message content $\mathit{msg}$. Here $\mathit{unique}$ is a function that ensures $\mathit{mid}$ be a fresh message identifier. We insert message identifier $\mathit{mid}$ into the happen-before relation and keeps the happen-before relation be transitive. The second rule describes replica $r$ performs a query operation $m(a) \Rightarrow b$ and thus does not generate message. Since no message is generated, the $\mathit{MsgHB}$ and $\mathit{MsgDel}$ tuples remain the same. The third rule describes delivery of a message to a replica $r$ other than its origin replica $r'$. By $(\mathit{mid},r) \notin \mathit{MsgDel}$, we ensure that $\mathit{mid}$ has not been previously delivered to replica $r$, and thus, each message be delivered to each replica at most once. By $\mathit{mid}$ being minimal w.r.t $\mathit{MsgHB}$ among $\{ \mathit{mid}' \vert (\mathit{mid}',r) \notin \mathit{MsgDel} \}$, we always choose a minimal element w.r.t $\mathit{MsgHB}$ among operations not been delivered to a replica, and thus, ensures causal-delivery.



A sequence $l$ of actions is an execution of $\llbracket \mathit{obj} \rrbracket_{\mathit{op}} = (\mathit{Config},\mathit{config}_0,\Sigma',\rightarrow)$, if there exists $(R,T,\mathit{MsgHB},\mathit{MsgDel}) \in \mathit{Config}$, such that $\mathit{config}_0 {\xrightarrow{ l }} (R,T,\mathit{MsgHB},\mathit{MsgDel})$. The semantics of $\mathit{obj}$ is defined as the set of executions of $\llbracket \mathit{obj} \rrbracket_{\mathit{op}}$.




\subsection{Correctness of a Single Object}
\label{subsec:correctness of a single object}

Given an execution $l = \alpha_1 \cdot \ldots \cdot \alpha_k$ of $\llbracket \mathit{obj} \rrbracket_{\mathit{op}}$ of CRDT object $\mathit{obj}$, we can associate unique operation identifiers with each action of $l$, and then obtain a corresponding history $\mathit{history}(l) = (\mathit{Op},\mathit{ro},\mathit{vis})$, such that

\begin{itemize}
\setlength{\itemsep}{0.5pt}
\item[-] Each operation in $\mathit{Op}$ is a tuple $(m(a) \Rightarrow b,i,\mathit{obj})$, such that $i$ is the operation identifier of a $\mathit{do}(m,a,b,r)$ action of $l$.

\item[-] $(o_1,o_2) \in \mathit{ro}$, if they are of same replica, and the index of $o_1$ in $h$ is before that of $o_2$.

\item[-] $(o_1,o_2) \in \mathit{vis}$, if the action of $o_1$ is $\mathit{send}(\mathit{mid},\_)$, and either the action of $o_2$ is $\mathit{receive}(\mathit{mid},\_)$, or there exists $o_3$, such that the action of $o_3$ is $\mathit{receive}(\mathit{mid},\_)$, and $(o_3,o_2) \in \mathit{ro}$.
\end{itemize}

Let $\mathit{history}(\llbracket \mathit{obj} \rrbracket_{\mathit{op}})$ be the set of histories of all executions of $\llbracket \mathit{obj} \rrbracket_{\mathit{op}}$. Then, a object is distributed linearizable, if each of its history is, as stated by the following:

\begin{definition}[Correctness of a CRDT Object]
\label{definition:correctness of a CRDT object}
A CRDT object $\mathit{obj}$ is distributed linearizable w.r.t a sequential specification $\mathit{spec}$, if each history of $\mathit{history}(\llbracket \mathit{obj} \rrbracket_{\mathit{op}})$ is distributed linearizable w.r.t $\mathit{spec}$.
\end{definition}
}





















\forget
{
\subsection{Semantics}
\label{subsec:semantics}

Given a set $\mathit{Obj}$ of objects, we define its semantics as a set of executions generated from an LTS $\llbracket \mathit{Obj} \rrbracket = (\mathit{Config},\mathit{config}_0,\Sigma',\rightarrow)$ as in \figurename~\ref{fig:the semantics of multiple objects}.

\begin{figure}[ht]
$\mathit{RState} = \cup_{x \in \mathit{Obj}} (\mathbb{R} \rightarrow x.\Sigma)$

$\mathit{TState} = \mathbb{MID} \times \mathbb{MSG} \times \mathit{Obj} \times \mathbb{R}$

$\mathit{Config} = \mathit{RState} \times \mathit{TState}$, $\mathit{config}_0 \in \mathit{Config}$.

$\Sigma' = \mathit{do}(\mathit{Obj} \times \mathbb{M} \times \mathbb{D} \times \mathbb{D} \mathbb{MID}) \cup \mathit{receive}(\mathit{Obj} \times \mathbb{MID} \times \mathbb{R})$

\[
\begin{array}{l c}
\bigfrac{ R(x,r) = \sigma, x(r).\mathit{do}(\sigma,m,a) = (\sigma',b,\mathit{msg}), \mathit{msg} \neq \emptyset, \mathit{unique}(\mathit{mid}) }
{ (R,T) {\xrightarrow{\mathit{do}(x,m,a,b,r,\mathit{mid})}} (R[(x,r):\sigma'],T \cup \{ (\mathit{mid},\mathit{msg},x,r) \}) }
\end{array}
\]

\[
\begin{array}{l c}
\bigfrac{ R(x,r) = \sigma, x(r).\mathit{do}(\sigma,m,a) = (\sigma',b,\mathit{msg}), \mathit{msg} = \emptyset }
{ (R,T) {\xrightarrow{\mathit{do}(x,m,a,b,r)}} (R[(x,r):\sigma'],T ) }
\end{array}
\]

\[
\begin{array}{l c}
\bigfrac{ R(x,r) = \sigma, x(r).\mathit{receive}(\sigma,\mathit{msg}) = \sigma',(\mathit{mid},\mathit{msg},x,r') \in T, r \neq r'}
{ (R,T) {\xrightarrow{\mathit{receive}(x,\mathit{mid},r)}} (R[(x,r):\sigma'],T) }
\end{array}
\]
\caption{The definition of semantics of $\llbracket \mathit{Obj} \rrbracket$}
\label{fig:the semantics of multiple objects}
\end{figure}

A configuration $(R,T)$ is a snapshot of distributed system and contains two parts: $R$ gives the local state of each object at each replica, and $T$ gives the set of messages that has been generated. Let $\mathbb{MID}$ be the set of message identifiers of message content. A message is a tuple $(\mathit{mid},\mathit{msg},x,r)$, where $\mathit{msgId} \in \mathbb{MID}$ is the identifier, $\mathit{msg} \in \mathbb{MSG}$ is the message content, $x$ is the object this message pertains to, and $r$ is the original replica of message. $\mathit{config}_0$ is the initial configuration, which maps each object at each replica into its initial local state, and has no message inside.

Each element of $\Sigma'$ is called an action. $\rightarrow \in \mathit{Config} \times \Sigma' \times \mathit{Config}$ is the transition relation and describe a single step of distributed systems. The first rule in \figurename~\ref{fig:the semantics of multiple objects} describes replica $r$ performs a update operation $m(a) \Rightarrow b$ of object $x$ and generate a message with message content $\mathit{msg}$. Here $\mathit{unique}$ is a function that ensures $\mathit{mid}$ is a fresh message identifier. The first rule  describes replica $r$ performs a query operation $m(a) \Rightarrow b$ of object $x$ and thus does not generate message. The third rule describes delivery of a message of object $x$ to a replica $r$ other than its origin replica $r'$.

A sequence $l$ of actions is an execution of $\llbracket \mathit{Obj} \rrbracket = (\mathit{Config},\mathit{config}_0,\Sigma',\rightarrow)$, if there exists $(R,T) \in \mathit{Config}$, such that $\mathit{config}_0 {\xrightarrow{ l }} (R,T)$. The semantics of $\mathit{Obj}$ is defined as the set of executions of $\llbracket \mathit{Obj} \rrbracket$. Given an execution, when the context is clear, we can associate a unique operation identifier to each action. Or we can say, it is safe to assume each action is either $\mathit{do}(i,x,m,a,b,r,\mathit{mid})$ or $\mathit{receive}(i,x,\mathit{mid},r)$, where $i \mathbb{OID}$ is a unique operation identifier.





\subsection{Correctness of a CRDT Implementation}
\label{subsec:correctness of a CRDT implementation}

To state the correctness of a single CRDT implementation, we consider the semantics of a single CRDT object.



Note that the transition relation of $\llbracket \mathit{Obj} \rrbracket$ does not make any assumption about message delivery: Messages can be delivered in any order; a message can be delivery to a replica multiple times; and a message can be never delivered to a replica. However, although the state-based CRDT do not assume any guarantee about message delivery, the operation-based CRDT assume the following two guarantees about message delivery:

\begin{itemize}
\setlength{\itemsep}{0.5pt}
\item[-] Each message is delivered to a replica at most once.
\item[-] Causal delivery: Given two update operations $o_1$ and $o_2$ of a same object, and assume $o_1$ is visible to $o_2$. Or we can say, $o_1$ is related to $o_2$ by several times of replica order and message delivery (all intermediate action is of the same object of $o_1$ and $o_2$). Then, for each replica, once it receives $o_2$, it must be the case that $o_1$ has been previously received.
\end{itemize}

Given an execution $l = \alpha_1 \cdot \ldots \cdot \alpha_k$ of $\llbracket \mathit{Obj} \rrbracket$,
}

%%% Local Variables:
%%% mode: latex
%%% TeX-master: "draft"
%%% End:


%!TEX root = draft.tex
%\newcommand{\seqPQ}{\mathsf{SeqPQ}}

\section{Proving Distributed Linearizability}
\label{sec:proving distributed linearizability}

In this section, we propose the definition of t0 and t1 specification, and then propose our approach for proving distributed linearizability for t0 and t1 specification, respectively. Our proof is done by simulation.



\subsection{T0-Specification and T1-Specification}
\label{subsec:t0 specification and t1 specification}



\begin{definition}[t0-specification]
\label{definition:t0-specification}
A sequential specification $\mathit{spec}$ is a t0-specification, if for each distributed linearizable history $h=(\mathit{Op},\mathit{ro},\mathit{vis})$, a sequence $\mathit{lin}$ shown below is a linearization of $h$ w.r.t $\mathit{spec}$: 

\begin{itemize}
\setlength{\itemsep}{0.5pt}
\item[-] Each element $(\ell,i,\mathit{vis}^{-1}(i))$ of $\mathit{lin}$ is generated from an operation $(\ell,i,\_,\mathit{ts})$ of $h$.

\item[-] $\mathit{lin}$ is consistent with $\mathit{vis}$. 
\end{itemize}
%A sequential specification $\mathit{spec}$ is a t0-specification w.r.t a set $S$ of operation label pairs, if for each history $h=(\mathit{Op},\mathit{ro},\mathit{vis})$ where $h$ is distributed linearizable w.r.t $\mathit{spec}$, the following conditions hold:

%\begin{itemize}
%\setlength{\itemsep}{0.5pt}
%\item[-] Given operations $o_1,o_2 \in \mathit{Op}$, $(\ell_1,\ell_2) \in S$ implies that $(o_1,o_2) \in \mathit{vis}$. Here $\ell_1$ and $\ell_2$ is the operation labels of $o_1$ and $o_2$, respectively.

%\item[-] Each such sequence $\mathit{lin}$ is a linearization of $h$: Each element $(\ell,i,\mathit{vis}^{-1}(i))$ of $\mathit{lin}$ is generated from an operation $(\ell,i,\_)$ of $h$; $\mathit{lin}$ is consistent with $\mathit{vis}$.
%\end{itemize}
\end{definition}

For t0-specification, any sequences consistent with visibility relation is its linearization. The following lemma shows several t0-specifications. Its proof can be found in Appendix \ref{subsec:proof in subsection subsec t0 specification and t1 specification}

\begin{lemma}
\label{lemma:several t0-specifications}
$\mathit{counter}_s$, $\mathit{set}_s$ and $\mathit{OR}$-$\mathit{set}_s$ are t0-specification. 

%The following are t0-specifications:

%\begin{itemize}
%\setlength{\itemsep}{0.5pt}
%\item[-] $\mathit{counter}_s$ is a t0-specification w.r.t $\emptyset$.

%\item[-] $\mathit{set}_s^u$ is a t0-specification w.r.t $\{ (\mathit{add}(a),\mathit{rem}(a)) \vert a \in D \}$.

%\item[-] $\mathit{OR}$-$\mathit{set}_s$ is a t0-specification w.r.t $\emptyset$.
%\end{itemize}
\end{lemma}

For a history where time-stamp is used for conflict resolution, such as RGA and last-write-win register (LWW-register), we can assume that each operation also gives the information of time-stamp. Or we can say, we can implicitly assume that each operation is of the form $o = (\ell,i,\mathit{obj},\mathit{ts})$, where $\mathit{ts}$ is the ``time-stamp'' of this operation: %A set of method is selected and called special method.

\begin{itemize}
\setlength{\itemsep}{0.5pt}
\item[-] If $o$ generates a new unique time-stamp, then $\mathit{ts}$ is this new time-stamp.

\item[-] If $o$ does not generate new time-stamp, then $\mathit{ts}$ is the maximum of time-stamp among operations visible to $o$.

\item[-] Moreover, we require that given operations $o_1$ and $o_2$, if $(o_1,o_2) \in \mathit{vis}$, then the time-stamp of $o_1$ is less or equal than that of $o_2$.
\end{itemize}


\begin{definition}[t1-specification]
\label{definition:t1-specification}
A sequential specification $\mathit{spec}$ is a t1-specification, if for each distributed linearizable history $h=(\mathit{Op},\mathit{ro},\mathit{vis})$, a sequence $\mathit{lin}$ shown below is a linearization of $h$ w.r.t $\mathit{spec}$:

\begin{itemize}
\setlength{\itemsep}{0.5pt}
\item[-] Each element $(\ell,i,\mathit{vis}^{-1}(i))$ of $\mathit{lin}$ is generated from an operation $(\ell,i,\_,\mathit{ts})$ of $h$.

\item[-] $\mathit{lin}$ is consistent with $\mathit{vis}$.

\item[-] If the time-stamp of $o_1$ is less than that of $o_2$, then $o_1$ is before $o_2$ in $\mathit{lin}$. Or we could say, $\mathit{lin}$ is consistent with time-stamp. 
\end{itemize}
\end{definition}

The following lemma shows several t1-specifications. Its proof can be found in Appendix \ref{subsec:proof in subsection subsec t0 specification and t1 specification}

\begin{lemma}
\label{lemma:several t1-specifications}
$\mathit{reg}_s$ and $\mathit{list}_s^{\mathit{af}}$ are t1-specification. 
\end{lemma}






\subsection{Proof Approach for T0-Specifications}
\label{subsec:proof approach for t0-specifications}

When the context is clear, we do not distinguish an operation and its unique identifier.

For a t0-specification $\mathit{spec}$, we construct a predicate $P(\mathit{config},h,\mathit{lin},\mathit{map})$ where $\mathit{config} = (R,T,\mathit{MsgHB},\mathit{MsgDel})$ is a configuration of $\llbracket \mathit{obj} \rrbracket_{\mathit{op}}$, $h$ is a history, $\mathit{lin} \in \mathbb{A}^*$ is used for linearization, and $\mathit{map} \subseteq \mathbb{MID} \times \mathbb{OID}$ is a function used to associate messages of $\mathit{config}.T$ with operations of $h$. We require $P(\mathit{config},h,\mathit{lin},\mathit{map})$ to be a conjunction of the following statements:

\begin{itemize}
\setlength{\itemsep}{0.5pt}
\item[-] $C_1$ (linearizability): $h$ is distributed linearizable w.r.t $\mathit{spec}$ and $\mathit{lin}$ is a linearization.

\item[-] $C_2$ (correct $\mathit{map}$): $\mathit{map}$ is a bijection between messages of $\mathit{config}.T$ and operations of $h$.

    \begin{itemize}
    \setlength{\itemsep}{0.5pt}
    \item[-] For message visibility of $\mathit{config}$: $(o_1,o_2) \in h.\mathit{vis}$, if and only if $(\mathit{map}(o_1),\mathit{map}(o_2)) \in \mathit{config}.\mathit{MsgHB}$.

    \item[-] For message delivery of $\mathit{config}$: If $(\mathit{mid},r) \notin \mathit{config}.\mathit{MsgDel}$, then for each $\mathit{mid}'$ where the message of $\mathit{mid}'$ is of replica $r$, $(\mathit{map}(\mathit{mid}),\mathit{map}(\mathit{mid}')) \notin h.\mathit{vis}$.
    \end{itemize}

\item[-] $C_3$ (causal delivery): $h.\mathit{vis}$ is transitive. Moreover, if $(o_1,o_2) \in h.\mathit{vis}$ and $(\mathit{map}(o_2),r) \in \mathit{config}.\mathit{MsgDel}$, then $(\mathit{map}(o_1),r) \in \mathit{config}.\mathit{MsgDel}$.

\item[-] $C_4$ (message content): A statement about message content of each message in $\mathit{config}.T$.

\item[-] $C_5$ (sequential explanation): For each replica $r$, we have that $\mathit{config}.R(r) = \mathit{apply}(\mathit{lin},\mathit{vd}(h,\mathit{config},r))$. 
\end{itemize}

Here $\mathit{vd}(h,\mathit{config},r) = \{ o \vert (o,o') \in h.\mathit{vis}, \mathit{map}(o')$ is of replica $r \} \cup \{ o \vert (\mathit{map}(o),r) \in \mathit{config}.\mathit{MsgHB} \}$. $\mathit{vd}(h,\mathit{config},r)$ is the set of operations that are either visible to some operation of replica $r$, or has been delivered into replica $r$. The function $\mathit{apply}(\mathit{lin},S)$ returns a local state by applying messages of operations in $S$ according to total order $\mathit{lin}$.


Then, we need to prove that $P$ is a simulation relation in a way shown below. Note that since $\mathit{spec}$ is a t0-specification, we can freely choose many possible ways to locate linearization point of new operations. Among these possible ways, we choose to always insert a new operation after the tail of the old linearization. 

\begin{itemize}
\setlength{\itemsep}{0.5pt}
\item[-] $P(\mathit{config}_0,\epsilon,\epsilon,\emptyset)$ holds, where $\mathit{config}_0$ is the initial configuration. 

\item[-] If $P(\mathit{config},h,\mathit{lin},\mathit{map})$ holds and $\mathit{config} {\xrightarrow{\mathit{do}(m,a,b,r,\mathit{mid})}} \mathit{config}'$, then $P(\mathit{config}', h' = h \otimes (m(a) \Rightarrow b,i,\mathit{obj}), \mathit{lin} \cdot (m(a) \Rightarrow b,i,h'.\mathit{vis}^{-1}(i)),\mathit{map} \cup \{ (\mathit{mid}, i) \})$ holds. Here $i$ is a unique operation identifier.

\item[-] If $P(\mathit{config},h,\mathit{lin},\mathit{map})$ holds and $\mathit{config} {\xrightarrow{\mathit{do}(m,a,b,r)}} \mathit{config}'$, then $P(\mathit{config}',h' = h \otimes (m(a) \Rightarrow b,i,\mathit{obj}), \mathit{lin} \cdot (m(a) \Rightarrow b,i,h'.\mathit{vis}^{-1}(i)),\mathit{map})$ holds. Here $i$ is a unique operation identifier. 

\item[-] If $P(\mathit{config},h,\mathit{lin},\mathit{map})$ holds and $\mathit{config} {\xrightarrow{\mathit{receive}(\mathit{mid},r)}} \mathit{config}'$, then $P(\mathit{config}',h,\mathit{lin},\mathit{map})$ holds.
\end{itemize}

Given history $h = (\mathit{Op},\mathit{ro},\mathit{vis})$ and operation $o = (m(a) \Rightarrow b,i,\mathit{obj})$, $h \otimes o$ returns a history $(\mathit{Op}',\mathit{ro}',\mathit{vis}')$, where $\mathit{Op}' = \mathit{Op} \cup \{ o \}$, $\mathit{ro}' = \mathit{ro} \cup \{ (o',o) \vert \mathit{map}(o')$ is a message of replica $r \}$, and $\mathit{vis}' = (\mathit{vis} \cup \{ (o',o) \vert (\mathit{map}(o'),r) \in \mathit{config}.\mathit{MsgDel} \} \cup \{ (o',o) \vert \mathit{map}(o')$ is a message of replica $r \})^*$. A predicate $P(\mathit{config},h,\mathit{lin},\mathit{map})$ satisfies conditions $C_1$ to $C_5$, as well as be a simulation relation described above, is called an invariant.

The following lemma states that the existence of such invariant implies distributed linearizability, and each sequence consistent with visibility is a linearization. To prove this lemma, we first prove that by induction on length of executions, we obtain a linearization. Then by definition of t0-specification, we know that each sequence consistent with visibility is a linearization. 

\begin{lemma}
\label{lemma:invariant of operation-based CRDT implies distributed linearizability for t0-specification}
If there exists an invariant $P$ for a CRDT object $\mathit{obj}$ and a t0-specification $\mathit{spec}$, then each history of $\mathit{history}(\llbracket \mathit{obj} \rrbracket_{\mathit{op}})$ is distributed linearizable w.r.t $\mathit{spec}$, and each sequences that consistent with visibility is a linearization of $h$.
\end{lemma}

In definition of invariant, only $C_4$ is specific to CRDT implementation. The following is the $C_4$ for or-set implementation. The detailed of such $C_4$ defining a correct variant can be found in Appendix \ref{subsec:appendix proofs of or-set implementation}.

\begin{example}[$C_4$ for or-set implementation]
\label{example:c4 for or-set implementation}

Given history $h = (\mathit{Op},\mathit{ro},\mathit{vis})$ and a update operation $o$ of $h$, the message of $o$ is given as follows:

\begin{itemize}
\setlength{\itemsep}{0.5pt}
\item[-] If $o$ is a $\mathit{add}(a)$ operation of replica $r$:  then the message of $o$ is $(a,(c+1,r))$, where $c = \mathit{max}\{ c_1 \vert \exists o' = (\_,\_,\_,\mathit{ts}), (o',o) \in \mathit{vis}, \mathit{ts} = (c_1,\_) \}$. 

\item[-] If $o$ is a $\mathit{rem}(a)$ operation: the message of $o$ is $S_1$, where $S_1 = \{ (a,\mathit{ts}') \vert (a,\mathit{ts}') \in S(o) \}$. We also require that $S_1 \neq \emptyset$.
\end{itemize}
Here $S(o) = \{ (b,\mathit{ts}') \vert b \in D, \exists o' = (\mathit{add}(b),\_,\_,\mathit{ts}')$, $(o',o) \in \mathit{vis}$, for each $\mathit{rem}(b)$ operation $o'' \neq o, (o'',o) \in \mathit{vis} \Rightarrow (o',o'') \neq \mathit{vis} \}$.
\end{example}






\subsection{Proof Approach for T1-Specifications}
\label{subsec:proof approach for t1-specifications}

For a t1-specification $\mathit{spec}$, we use a same predicate $P(\mathit{config},h,\mathit{lin},\mathit{map})$ in subsection \ref{subsec:proof approach for t0-specifications}. We still need to prove that $P$ is a simulation relation. The difference is the construction of linearization: a operation with time-stamp $\mathit{ts}$ is put after the last operation with a time-stamp less or equal than $\mathit{ts}$.

\begin{itemize}
\setlength{\itemsep}{0.5pt}
\item[-] $P(\mathit{config}_0,\epsilon,\epsilon,\emptyset)$ holds, where $\mathit{config}_0$ is the initial configuration.

\item[-] If $P(\mathit{config},h,\mathit{lin},\mathit{map})$ holds and $\mathit{config} {\xrightarrow{\mathit{do}(m,a,b,r,\mathit{mid})}} \mathit{config}'$, then $P(\mathit{config}', h' = h \otimes (m(a) \Rightarrow b,i,\mathit{obj}),\mathit{lin}',\mathit{map} \cup \{ (\mathit{mid}, i) \})$ holds. Here $i$ is a unique operation identifier, and $\mathit{lin}'$ is obtained from $\mathit{lin}$ by inserting $(m(a) \Rightarrow b,i,h'.\mathit{vis}^{-1}(i))$ after the last operation with time-stamp less or equal than $\mathit{ts}$.

\item[-] If $P(\mathit{config},h,\mathit{lin},\mathit{map})$ holds and $\mathit{config} {\xrightarrow{\mathit{do}(m,a,b,r)}} \mathit{config}'$, then $P(\mathit{config}',h' = h \otimes (m(a) \Rightarrow b,i,\mathit{obj}),\mathit{lin}',\mathit{map})$ hold. Here $i$ is a unique operation identifier, and $\mathit{lin}'$ is obtained from $\mathit{lin}$ by inserting $(m(a) \Rightarrow b,i,h'.\mathit{vis}^{-1}(i))$ after the last operation with time-stamp less or equal than $\mathit{ts}$.

\item[-] If $P(\mathit{config},h,\mathit{lin},\mathit{map})$ holds and $\mathit{config} {\xrightarrow{\mathit{receive}(\mathit{mid},r)}} \mathit{config}'$, then $P(\mathit{config}',h,\mathit{lin},\mathit{map})$ holds.
\end{itemize}

A predicate $P(\mathit{config},h,\mathit{lin},\mathit{map})$ satisfies conditions $C_1$ to $C_5$, as well as be a simulation relation described above, is called an invariant. 

The following lemma states that the existence of such invariant implies distributed linearizability, and each sequence consistent with visibility and time-stamp is a linearization. To prove this lemma, we first prove that by induction on length of executions, we obtain a linearization. Then by definition of t1-specification, we know that each sequence consistent with visibility and time-stamp is a linearization. 

\begin{lemma}
\label{lemma:invariant of operation-based CRDT implies distributed linearizability for t1-specification}
If there exists an invariant $P$ for a CRDT object $\mathit{obj}$ and a t1-specification $\mathit{spec}$, then each history of $\mathit{history}(\llbracket \mathit{obj} \rrbracket_{\mathit{op}})$ is distributed linearizable w.r.t $\mathit{spec}$, and each sequences that consistent with visibility and time-stamp is a linearization of $h$. 
\end{lemma} 


In definition of invariant, only $C_4$ is specific to CRDT implementation. The following is the $C_4$ for rga. The detailed of such $C_4$ defining a correct variant can be found in Appendix \ref{subsec:appendix proofs of rga}. 

\begin{example}[$C_4$ for RGA]
\label{example:c4 for rga}
Given history $h = (\mathit{Op},\mathit{ro},\mathit{vis})$ and a update operation $o$ of $h$, the message of $o$ is given as follows:

\begin{itemize}
\setlength{\itemsep}{0.5pt}
\item[-] If $o$ is a $\mathit{add}(a,b)$ operation of replica $r$: the message of $o$ is $(a,\mathit{ts}_a,\mathit{ts}_b)$, where $\mathit{ts}_b$ is the time-stamp of operation $\mathit{add}(b,\_)$ in $h$, $\mathit{ts}_a = (c+1, r)$, and $c = \mathit{max}\{ c_1 \vert \exists o' = (\_,\_,\_,\mathit{ts}), (o',o) \in \mathit{vis}, \mathit{ts} = (c_1,\_) \}$. 

\item[-] If $o$ is a $\mathit{rem}(a)$ operation: the message of $o$ is $a$. 
\end{itemize} 
\end{example} 















%%% Local Variables:
%%% mode: latex
%%% TeX-master: "draft"
%%% End:


%\section{CRDT-Linearizable Implementations}
%\label{sec:crdt-lin-imp}

%\gpn{Proofs of CRDT-Linearizable implementations}

%\begin{itemize}
%\item Counters
%  \begin{itemize}
%  \item Grow-Only Counter
%  \item PN-Counter
%  \end{itemize}
%\item Registers:
%  \begin{itemize}
%  \item LWW
%  \item MVR
%  \end{itemize}
%\item Sets:
%  \begin{itemize}
%  \item LAW (OR-Set)
%  \item Grow-Only Set
%  \end{itemize}
%\item Text-Editing (Graphs/Lists)
%  \begin{itemize}
%  \item RGA (addRight)
%  \item WOOT
%  \end{itemize}
%\end{itemize}

\section{Meta-properties of CRDT-Linearizability}
\label{sec:meta-prop-lincrdt}

\subsection{Compositionality}
\label{sec:compositionality}

\gpn{Here we explain the compositionality result a la Herlihy Wing}

\begin{itemize}
\item What are the conditions for compositionality (deterministic
  specs?, conflict?)
  \begin{itemize}
  \item Can we share the same timestamp for different data structures.
  \item Coherent conflict resolution.
  \end{itemize}
\item Should we change the implementations to change the
  conflict-resolution.
\end{itemize}

\subsection{Abstraction}
\label{sec:abstraction}

\gpn{Here we explain refinement}
\begin{itemize}
\item What is the power of the client? Messages, other calls to CRDTs?
\item Observational refinement?
\item Completeness?
\end{itemize}

\bibliographystyle{plainurl}
\bibliography{draft}

\begin{thebibliography}{50}
\bibitem{Burckhardt:2014POPL}
Burckhardt, S., Gotsman, A., Yang, H., Zawirski, M.:
Replicated data types: specification, verification, optimality.
\newblock In: Jagannathan, S., Sewell, P. (eds.) POPL 2014, pp. 271-284. ACM (2014)
\end{thebibliography}



\newpage

\appendix

\section{\crdtimp{}}
\label{sec:crdt implementation}



\subsection{Multi-Value Register Implementation}
\label{subsec:multi-value register implementation}

\cite{ShapiroPBZ11} shows how to obtain a state-based \crdtimp{} from a operation-based \crdtimp{}, and we draw it in Listing~\ref{lst:operation-based emulation of state-based object}. To do an operation $f(a)$, we compute the state-based update and perform merge method in downstream. Here the precondition of downstream is empty because merge is always enabled.


\begin{minipage}[t]{1.0\linewidth}
\begin{lstlisting}[frame=top,caption={operation-based emulation of state-based object},
captionpos=b,label={lst:operation-based emulation of state-based object}]
  payload S ( the state-based payload )
  initial initial payload of S

  update method f(a)
    atSource :
      precondition : precondition of f(a)
      let s = atSource of f(a) in state-based
    downStream(s) :
      S = merge(S,s)
\end{lstlisting}
\end{minipage}

\cite{ShapiroPBZ11} gives a state-based multi-value register implementation. As discussed above, we give its operation-based version in Listing~\ref{lst:operation-based multi-value register}. Here $myRep()$ is a function that returns current replica identifier, and $reps()$ is a function that returns the number of replicas in distributed system. This implementation assumes that the number of replicas are fixed. A payload $S$ is a set of $(a,V)$ pairs, where $a$ is a value and $V$ is a vector called version vector. %Annotation1 is an annotation for the current payload, and annotation2 is an annotation for downstream of $write(a)$.
Given vector clock $V$ and $V'$, we say that $V > V'$, if for each replica $\arep$, we have $V[\arep] > V'[\arep]$. Annotation1 is an annotation for downstream of $write(a)$.


\begin{minipage}[t]{1.0\linewidth}
\begin{lstlisting}[frame=top,caption={Pseudo-code of operation-based multi-value register},
captionpos=b,label={lst:operation-based multi-value register}]
  payload Set S
  initial S = @|$\emptyset$|@
  initial seq = @|$\epsilon$|@

  write(a) :
    atSource :
      let g = myRep()
      let @|$\mathcal{V}$|@ = @|$\{ V \vert \exists x, (x,V) \in S \}$|@
      let @|$V'$|@ = @|$[ max_{V \in \mathcal{V}} V[j]]_{j \neq g}$|@
      let @|$V'[g]$|@ = @|$max_{V \in \mathcal{V}} V[g]$|@ + 1
      //@ let seq@|$'$|@ = seq@|$\,\cdot\,\alabellongind[readIds]{}{S}{}\,\cdot\,\alabelshort[write]{a,V',S}$|@
    downStream(a, V@|$'$|@) :
      let A = @|$\{ (a_1,V_1) \in S \vert \neg V' > V_1 \}$|@
      let B = @|$\{ (a,V') \}$|@, if @|$\forall (a_1,V_1) \in S, \neg V_1 > V'$|@. Otherwise, let B = @|$\emptyset$|@
      S = A @|$\cup$|@ B
      //@ Annotation1 : @|$\forall \arep, V'[\arep]$|@ = @|$\vert \{ \alabel = \alabellongind[write]{\_,\_}{\bot}{*}, \alabel$|@ happens on replica @|$\arep,  (\alabel,\alabelshort[write]{a,V'}) \in \avisord \vee \alabel = \alabelshort[write]{a,V'} \} \vert$|@

  read() :
    let @|$S_1$|@  = {a : @|$\exists$|@ V. (a,V) @|$\in$|@ S}
    //@ let seq@|$'$|@ = seq@|$\,\cdot\,\alabellongind[read]{}{S_1}{}$|@
    return @|$S_1$|@
\end{lstlisting}
\end{minipage}

%//@ Annotation1 : S =  @|$\{ (a,V) \vert \exists \alabel = \alabellongind[write]{a,V}{\bot}{*}, \alabel$|@ is maximal w.r.t @|$\avisord$|@ among write operations applied in current replica @|$\}$|@





\subsection{The WOOT Algorithm}
\label{subsec:the woot algorithm}

The WOOT algorithm of \ref{a} is given in Listing~\ref{lst:woot algorithm}. Note that here $integrateIns$ is a recursive method used by $addBetween$ method.

In local of each replica, WOOT algorithm stores the list as a sequence of W-characters. A W-character $w$ is a five-tuple $<id,v,flag,id_p,id_n>$, where $id$ is the identifier of $w$; $v$ is the value of $w$; $flag \in \{ \mathit{true},\mathit{false} \}$ is the flag of $w$ and indicates whether $w$ is ``visible'' in list; $id_p$ and $id_n$ is the identifier of the previous and next W-character of $w$, respectively. The previous and the next W-characters of $w$ are the W-characters between which $w$ has been inserted on its generation state. Given $w = (id,v,flag,id_p,id_n)$, let $C_P(w) = id_p$ and $C_N(w) = id_n$ denote the previous and next W-character of $w$, respectively. A identifier $id$ of W-character is a tuple $(ctr,\arep)$, where $ctr \in \mathbb{N}$.

A W-string is an ordered sequence of W-characters $w_b \cdot w_1 \cdot \ldots \cdot w_n \cdot w_e$, where $w_b$ and $w_e$ are special W-characters that mark the beginning and the ending of the sequence. The values of $w_b$ and $w_e$ are $\circ_b$ and $\circ_e$, respectively. We define the following function for a W-string $str$:

\begin{itemize}
\setlength{\itemsep}{0.5pt}
\item[-] $\vert str \vert$ returns the length of $str$,

\item[-] $str[p]$ returns the W-character at position $p$ in $str$. Her we assume that the first element of $str$ is at position 0.

\item[-] $pos(str,w)$ returns the position of W-character $w$ in $S$.

\item[-] $insert(str,w,p)$ inserts W-character $w$ into $str$ at position $p$.

\item[-] $subseq(str,w_1,w_2)$ returns the part of $str$ between the W-characters $w_1$ and $w_2$ (excluding $w_1$ and $w_2$).

\item[-] $contains(str,a)$ returns true if there exists a W-character in $str$ with value $a$.

\item[-] $values(str)$ returns the sequence of visible (with $\mathit{true}$ flag) values of $str$.

\item[-] $getWchar(str,a)$ returns the W-character with value $a$ in $str$.

\item[-] $changeFlag(str,pos,f)$ changes the flag of $str[pos]$ into $f$.
\end{itemize}

%Note that only $values(str)$ distinguish whether a W-character is with flag $\mathit{true}$ or with flag $\mathit{false}$.

A total order $<_{id}$ is given for identifiers of W-characters for conflict resolution. Given two identifiers $(ctr_1,\arep_1)$ and $(ctr_2,\arep_2)$, we have $(ctr_1,\arep_1) <{id} (ctr_2,\arep_2)$, if $\arep_1 < \arep_2 \vee (\arep_1 = \arep_2 \wedge ctr_1 < ctr_2)$. Given a sequence $str$ and two elements $a,b$ of $str$, we write $a <_{str} b$ to indicate that $pos(str,a) < pos(str,b)$.

\begin{minipage}[t]{1.0\linewidth}
\begin{lstlisting}[frame=top,caption={Pseudo-code of WOOT algorithm},
captionpos=b,label={lst:woot algorithm}]
  payload int @|$H_s$|@, W-string @|$string_s$|@
  initial @|$H_s$|@ = 0, @|$string_s$|@ = @|$\epsilon$|@
  initial seq = @|$\epsilon$|@

  addBetween(a,b,c) :
    atSource :
      precondition :  @|$contains(string_s,b) \wedge contains(string_s,c) \wedge pos(string_s,c) - pos(string_s,b) = 1\wedge \neg contains(string_s,a)$|@
      let g = myRep()
      let @|$c_p$|@ = @|$getWchar(string_s,b)$|@
      let @|$c_n$|@ = @|$getWchar(string_s,c)$|@
      @|$H_s$|@ = @|$H_s$|@ + 1
      //@ let seq@|$'$|@ = seq@|$\,\cdot\,\alabellongind[addBetween]{a,b,c}{}{}$|@
    downStream((w,@|$c_p$|@,@|$c_n$|@)) : with @|$w = ((H_s,g),a,\mathit{true},c_p.id,c_n.id)$|@
      integrateIns(@|$w,c_p,c_n$|@)

  remove(a) :
    atSource :
      precondition : @|$contains(string_s,a)$|@
      let w = @|$getWchar(string_s,a)$|@
      //@ let seq@|$'$|@ = seq@|$\,\cdot\,\alabellongind[remove]{a}{}{}$|@
    downStream(w) :
      let p = @|$pos(string_s,w)$|@
      @|$changeFlag(string_s,p,\mathit{false})$|@

  read() :
    let s = @|$values(string_s)$|@
    //@ let seq@|$'$|@ = seq@|$\,\cdot\,\alabellongind[read]{}{s}{}$|@
    return s

  integrateIns(@|$c,c_p,c_n$|@)
    let @|$S$|@ = @|$string_s$|@
    let @|$S'$|@ = @|$subseq(S,c_p,c_n)$|@
    if @|$S' = \epsilon$|@
      then  @|$insert(S,c,pos(S,c_n))$|@
    else
      Let L = @|$c_p \cdot d_0 \cdot \ldots \cdot d_m \cdot c_n$|@, where @|$d_0, \ldots, d_m$|@ are the W-characters in @|$S'$|@
        such that for each @|$d_i$|@, @|$C_P(d_i) <_S c_p$|@ and @|$c_n <_S C_N(d_i)$|@ 
      Let i = 1 
      while (@|$i < \vert L \vert -1 \wedge L[i] <_{id} c$|@) do
        i = i+1 
      integrateIns(@|$c,L[i-1],L[i]$|@)  
\end{lstlisting}
\end{minipage}

The payload of each replica is a integer value $H_s$ used to generate identifier, and a W-string $string_s$.

To do $addBetween(a,b,c)$, we first ensure that $b$ and $c$ are adjacent in $string_s$ and $a$ is not in $string_s$. Then, we generate a W-character $w$ for value $a$, and calls method $integrateIns(w,c_p,c_n)$ to put $w$ between $c_p$ and $c_n$, which are the W-characters of $b$ and $c$ in $string_s$, respectively. 

$integrateIns(c,c_p,c_n)$ is a recursive method and works as follows: If there are no W-character between $c_p$ and $c_n$ (for example, in the current replica), then $w$ is put after $c_p$. Else, WOOT select a set $L$ of W-characters, such that each W-character of $L$ has a ranger ``wider than the range between $c_p$ and $c_n$''. The W-characters in $L$ are the W-characters that needs to be considered when the range is between $c_p$ and $c_n$. It can be proved that W-characters in $L$ are sorted by the $<_{id}$ order. Then, we choose the position of $c$ to be between $L[i-1]$ and $L[i]$. We can see that the range between $L[i-1]$ and $L[i]$ is strictly shorter than the range between $c_p$ and $c_n$. Since there may be W-characters in the range between $L[i-1]$ and $L[i]$ in $string_s$, we make a recursive call to $integrateIns(c,L[i-1],L[i])$ to compute the position of $c$ in the range between $L[i-1]$ and $L[i]$. 

To do $remove(a)$, we just set the flag of W-character of $a$ in $string_s$ to be $\mathit{false}$. To do $read()$, we return $values(string_s)$. 







\section{Sequential Specifications}
\label{sec:sequential specifications}

%In this section we give several sequential specifications.


\subsection{The Sequential Specification of Counter}
\label{subsec:the sequential specification of counter}

The sequential specification $\mathit{counter}_s$ of counter is so that $\abstates = \mathbb{Z}$, that is the state will be an integer, and the transitions are given as follows:
\begin{itemize}
\setlength{\itemsep}{0.5pt}
\item[-] $k \xrightarrow{\alabellong[\mathsf{inc}]{}{}} k+1$
\item[-] $k \xrightarrow{\alabellong[\mathsf{dec}]{}{}} k-1$
\item[-] $k \xrightarrow{\alabellong[\mathsf{read}]{}{k}} k$
\end{itemize}

Here $inc$ increase the counter by $1$, $dec$ decrease the counter by $1$, and $read$ returns the value of the counter.



\subsection{The Sequential Specification of Multi-Value Register}
\label{subsec:the sequential specification of multi-value register}

The query-update rewriting of multi-value register is as follows: $\gamma( \alabellong[\mathsf{write}]{a}{} ) = ( \alabellong[\mathsf{readIds}]{}{S}, \alabellong[\mathsf{write}]{a,id,S}{})$.


Each abstract state $\abstate$ is a set of tuples $(a,id)$, where $a$ is a data and $id$ is a identifier. The sequential specification $\mathit{mvreg}_s$ of multi-value register is given with the transitions as follows:

\begin{itemize}
\setlength{\itemsep}{0.5pt}
\item[-] $\abstate \xrightarrow{\alabellong[\mathsf{readIds}]{}{\abstate}} \abstate$

\item[-] $\big(\abstate\ |\ (a,id) \notin \abstate \big) \xrightarrow{ \alabellong[\mathsf{write}]{a,id,S}{} }  \abstate \setminus S \cup \{ (a,id) \}$

\item[-] $\abstate \xrightarrow{\alabellong[\mathsf{read}]{}{ \{ a \vert \exists id, (a,id) \in \abstate \} }} \abstate$
\end{itemize}

Here $readIds$ returns the abstract state, $\alabellong[\mathsf{write}]{a,id,S}{}$ removes $S$ from the abstract state and puts $\{ (a,id) \}$ into the abstract state, and $read$ returns the value of multi-value register.



\subsection{The Sequential Specification of List with Add-Between Interface}
\label{subsec:the sequential specification of list with add-between interface}

Each abstract state $\abstate$ is a sequence of tuples $(a,flag)$, where $a$ is a data and $flag \in \{ \mathit{true},\mathit{false} \}$ is flag. $(a,\mathit{true})$ means that $a$ is in the list, while $(a,\mathit{false})$ means that $a$ has once been added into the list and then removed. The sequential specification $\mathit{listBet}_s$ of list with add-between interface is given with the transitions as follows:

\begin{itemize}
\setlength{\itemsep}{0.5pt}
\item[-] $\big(\abstate\ |\ \abstate = (a_1,flag_1)\cdot \ldots \cdot (a_n,flag_n) \wedge 1 \leq i \leq k < j \leq n \wedge (a,\_) \notin \abstate \big) \xrightarrow{ \alabellong[\mathsf{addBetween}]{a,a_i,a_j}{} } (a_1,flag_1)\cdot \ldots (a_k,flag_k) \cdot (a,\mathit{true}) \cdot (a_{k+1},flag_{k+1}) \cdot \ldots \cdot (a_n,flag_n)$

\item[-] $\big(\abstate\ |\ \abstate = (a_1,flag_1)\cdot \ldots \cdot (a_n,flag_n) \wedge 1 \leq k \leq n \big) \xrightarrow{ \alabellong[\mathsf{remove}]{a_k}{} } (a_1,flag_1)\cdot \ldots (a_k,\mathit{false}) \cdot \ldots \cdot (a_n,flag_n)$

\item[-] $(a_1,flag_1)\cdot \ldots \cdot (a_n,flag_n) \xrightarrow{\alabellong[\mathsf{read}]{}{ s }} (a_1,flag_1)\cdot \ldots \cdot (a_n,flag_n)$, here $s$ is the projection of $a_1 \cdot \ldots \cdot a_n$ into $\{ a \vert a$ is with flag $\mathit{true} \}$
\end{itemize}

$\alabellong[\mathsf{addBetween}]{a,a_i,a_j}{}$ puts $(a,\mathit{true})$ into a random position between $(a_i,\_)$ and $(a_j,\_)$, and we assume that each value is put into list at most once. $\alabellong[\mathsf{remove}]{a_k}{}$ removes $a_k$ from the list by setting its flag into $\mathit{false}$. $read$ returns the content of list. value of multi-value register.

Assume that the initial value of list is $(\circ_b,\mathit{true}) \cdot (\circ_e,\mathit{true})$, and $\circ_b$ and $\circ_e$ are never removed. When the context is clear, in $\mathit{read}$ operation, we will omit $\circ_b$ and $\circ_2$ in return value.










\section{Proofs of \crdtimp}
\label{sec:appendix proofs of crdt implementations}



\subsection{Proof of Operation-Based Multi-Value Register}
\label{subsec:proof of operation-based multi-value register}

Given two sequences $l_1,l_2$  such that $l_2$ is a permutation of $l_1$, let $\mathit{diff}(l_1,l_2) = \{ (a,b) \vert$ the order of $a$ and $b$ in $l_1$ is different from that of $l_2 \}$. Given a sequence $l$ and two elements $a$ and $b$ of $l$, let $\mathit{swap}(l,a,b)$ be a sequence obtained from $l$ by swapping $a$ and $b$. The following lemma states that, given two specification sequences $(\alabelset_1, \aseqord_1)$ and $(\alabelset, \aseqord_2)$ that are generated from a same history and both consistent with visibility relation, we can obtain $\aseqord_2$ from $\aseqord_1$ by several time of swapping adjacent pair of concurrent operations.

\begin{lemma}
\label{lemma:given two sequence consistent with visibility order, one can be obtained from the other}
Given a history $(\alabelset,\avisord)$ and two two specification sequences $(\alabelset_1, \aseqord_1)$ and $(\alabelset, \aseqord_2)$ that are both consistent with $\avisord$. If $\aseqord_1 \neq \aseqord_2$, then we can obtain $\aseqord_2$ from $\aseqord_1$ by several time of swapping adjacent pair of concurrent operations.
\end{lemma}

\begin {proof}

First, we need to prove that, if $\mathit{diff}(\aseqord_1,\aseqord_2) \neq \emptyset$, then, there exists $(\alabel_1,\alabel_2) \in \mathit{diff}(\aseqord_1,\aseqord_2)$, such that $l_1$ and $l_2$ are concurrent, and $l_1$ and $l_2$ are adjacent in $\aseqord_1$.

We prove this by contradiction. Assume $\mathit{diff}(\aseqord_1,\aseqord_2) \neq \emptyset$, and for each $(\alabel_1,\alabel_2) \in \mathit{diff}(\aseqord_1,\aseqord_2)$, we have that either $\alabel_1$ and $\alabel_2$ are not concurrent, or $\alabel_1$ and $\alabel_2$ are not adjacent in $\aseqord_1$.

Since $\mathit{diff}(\aseqord_1,\aseqord_2) \neq \emptyset$, let $(\alabel,\alabel')$ be a element of $\mathit{diff}(\aseqord_1,\aseqord_2)$, and the distance of $\alabel$ and $\alabel'$ is minimal in $\{$ the distance between $\alabel_1$ and $\alabel_2 \vert (\alabel_1,\alabel_2) \in \mathit{diff}(\aseqord_1,\aseqord_2) \}$. Let us prove that $\alabel$ and $\alabel'$ are adjacent by contradiction: If there exists $\alabel''$ between $\alabel$ and $\alabel'$. Assume that in $\aseqord_1$, $\alabel$ is before $\alabel''$, and $\alabel''$ is before $\alabel'$. By assumption, the order between $\alabel$ and $\alabel''$, and between $\alabel''$ and $\alabel'$ is the same in $\aseqord_1$ and in $\aseqord_2$. This implies that $\alabel$ is still before $\alabel'$ in $\aseqord_2$, which contradicts the fact that $(\alabel,\alabel') \in \mathit{diff}(\aseqord_1,\aseqord_2)$.

Since $\alabel$ and $\alabel'$ are adjacent and $(\alabel,\alabel') \in \mathit{diff}(\aseqord_1,\aseqord_2)$, by assumption we know that $\alabel$ and $\alabel'$ are not concurrent. Or we can say, $(\alabel,\alabel') \in \avisord \vee (\alabel',\alabel) \in \avisord$. This contradicts that both $\aseqord_1$ and $\aseqord_2$ are consistent with visibility relation. This completes the proof of the first part.

Since $\aseqord_1 \neq \aseqord_2$, we have $\mathit{diff}(\aseqord_1,\aseqord_2) \neq \emptyset$, and then, as discussed above, there exists $(\alabel,\alabel') \in \mathit{diff}(\aseqord_1,\aseqord_2)$, such that $\alabel$ and $\alabel'$ are concurrent, and $\alabel$ and $\alabel'$ are adjacent in $\aseqord_1$. Let $\aseqord_3 = \mathit{swap}(\aseqord_1,\alabel,\alabel')$. It is easy to see that $\mathit{diff}(\aseqord_1,\aseqord_2) > \mathit{diff}(\aseqord_3,\aseqord_2)$. Therefore, by several times of above process, we finally obtain $\aseqord_2$ from $\aseqord_1$ by swapping pairs of operations. This completes the proof of this lemma. $\qed$
\end {proof}


Then, let us prove that the operation-based multi-value register implementation is \crdtlinearizable{} w.r.t $\mathit{mvreg}_s$.

\begin{lemma}
\label{lemma:multi-value register is correct}
The operation-based multi-value register implementation is \crdtlinearizable{} w.r.t $\mathit{mvreg}_s$
\end{lemma}


\begin {proof}

Let us give two facts:

\begin{itemize}
\setlength{\itemsep}{0.5pt}
\item[-] $fact1$: Let $(a_1,V_1)$ and $(a_2,V_2)$ be the downstream for $\alabel_1$ and $\alabel_2$, respectively, and assume that $(\alabel_1,\alabel_2) \in \avisord$. Then, $V_1 < V_2$.
\item[-] $fact2$: Let $(a_1,V_1)$ and $(a_2,V_2)$ be the downstream for $\alabel_1$ and $\alabel_2$, respectively, and assume that $\alabel_1$ and $\alabel_2$ are concurrent. Then, $\neg (V_1 < V_2 \vee V_2 < V_1)$.

%\item[-] $fact3$: Let $S$ be the payload of a replica. Then, $S$ =  $\{ (a,V) \vert \exists \alabel = \alabellongind[write]{a,V}{\bot}{*}, \alabel$ is maximal w.r.t $\avisord$ among write operations applied in current replica $\}$.
\end{itemize}


\noindent Proof of $fact1$: Assume $\alabel_1$ happens on replica $\arep$. By the {\textred{causal delivery}} assumption, we know that for each replica $\arep' \neq \arep$, $\alabel_2$ see more or equal number of operations happens on replica $\arep'$ than that of $\alabel_1$, and $\alabel_2$ see more number of operations happens on replica $\arep$ than that of $\alabel_1$. By Annotation1, we know that $\forall \arep' \neq \arep$, $V_1[\arep'] \geq V_2[\arep']$, and $V_1[\arep] > V_2[\arep]$. Therefore, $V_1 < V_2$.

\noindent Proof of $fact2$: Let us prove that $\neg (V_1 < V_2 \vee V_2 < V_1)$ by contradiction. It is obvious that $\alabel_1$ and $\alabel_2$ happens on different replicas. Assume that $V_1 < V_2$, and assume that $\alabel_1$ happens on replica $\arep_1$. Since $V_1 < V_2$, we know that $V_1[\arep_1] \leq V_2[\arep_1]$. By the {\textred{causal delivery}} assumption and Annotation1, this means $(\alabel_1,\alabel_2) \in \avisord$, contradicts the assumption that $\alabel_1$ and $\alabel_2$ are concurrent. Similarly, we can see that $\neg (V_2 < V_1)$. Therefore, $\neg (V_1 < V_2 \vee V_2 < V_1)$.

Let us propose Annotation2, which is an annotation of payload and obviously holds in the initial global configuration.

\begin{itemize}
\setlength{\itemsep}{0.5pt}
\item[-] Annotation2: Let $S$ be the payload of a replica. Then, $S$ =  $\{ (a,V) \vert \exists \alabel = \alabellongind[write]{a,V}{\bot}{*}, \alabel$ is maximal w.r.t $\avisord$ among write operations applied in current replica $\}$.
\end{itemize}

Our proof of the lemma proceed as follows:

\begin{itemize}
\setlength{\itemsep}{0.5pt}
\item[-] We need to prove that $\mathsf{ReplicaStates}$ is an inductive invariant.

Since every operation is appended to the linearization when it executes {\tt atSource} it clearly follows, the linearization order is consistent with visibility order. Then, by the {\textred{causal delivery}} assumption, the order in which downstreams are applied at a given replica is also consistent with the visibility order. Let $\aseqord_1$ be the projection of linearization order into labels applied in a replica $\arep$, and $\aseqord_2$ be the order of labels applied in replica $\arep$. By Lemma \ref{lemma:given two sequence consistent with visibility order, one can be obtained from the other}, $\aseqord_2$ can be obtained from $\aseqord_1$ by several time of swapping adjacent pair of concurrent operations.

Let us prove that applying downstream of such pair of operations commute, and we only need to consider the case of two concurrent $write$ labels. Let $(a_1,V_1)$ and $(a_2,V_2)$ be the downstream of labels $\alabellongind[write]{a_1,V_1}{\bot}{*}$ and $\alabellongind[write]{a_2,V_2}{\bot}{*}$, respectively. Given a payload $S$, assume we obtained $S'$ from $S$ by applying $(a_1,V_1)$ and then applying $(a_2,V_2)$, and assume we obtained $S''$ from $S$ by applying $(a_2,V_2)$ and then applying $(a_1,V_1)$. In the process of obtaining $S'$ or $S''$ from $S$, we add $(a_1,V_1)$ and $(a_2,V_2)$ into $S$ and remove the following tuple $(a_3,V_3) \in S \cup \{(a_1,V_1),(a_2,V_2)\}$: either $V_3 < V_1$, or $V_3 < V_2$, or $(a_3,V_3) = (a_1,V_1) \wedge V_1 < V_2$, or $(a_3,V_3) = (a_2,V_2) \wedge V_2 < V_1$. Since we already know that $\alabellongind[write]{a_1,V_1}{\bot}{*}$ and $\alabellongind[write]{a_2,V_2}{\bot}{*}$ are concurrent, by  $fact_2$, we know that $\neg (V_1 < V_2 \vee V_2 < V_1)$. Therefore, $S' = S''$.



\item[-] Let we prove that the Annotation1 and Annotation2 is an inductive invariant.

We prove by induction on executions. Obvious they hold in $\aglobalstate_0$. Assume they hold along the execution $\aglobalstate_0 \xrightarrow{}^* \aglobalstate$ and there is a new transition $\aglobalstate \xrightarrow{} \aglobalstate'$. We need to prove that they still hold in $\aglobalstate'$. We only need to consider $write$ action or downstream:

    \begin{itemize}
    \setlength{\itemsep}{0.5pt}
    \item[-] For case of a $\alabellongind[write]{a,V'}{\bot}{*}$ action of replica $\arep$: Let $S$ and $S'$ be the payload of replica $\arep$ of $\aglobalstate$ and $\aglobalstate'$, respectively. Obviously $S' = \{ (a,V') \}$.

        Since $\alabellongind[write]{a,V'}{\bot}{*}$ is larger than any labels in $S$ w.r.t the visibility relation, Annotation2 still holds in $\aglobalstate'$.

        By Annotation2 we know what labels are contained in $S$, by Annotation1 we know the content of these labels. By the {\textred{causal delivery}} assumption we know that if a label $\alabel$ is visible to a label $\alabel'$ of $S$, then $\alabel$ must be already applied in replica $\arep$. let $\mathcal{V} = \{ V \vert (\_,V) \in S \}$ be the set of vector clocks of $S$. Therefore, for each replica $\arep' \neq \arep$, $max_{V \in \mathcal{V}} V[\arep']$ is the number of operations happen on replica $\arep'$ and has been applied in replica $\arep$ during $\aglobalstate_0 \xrightarrow{}^* \aglobalstate$, and $max_{V \in \mathcal{V}} V[\arep]$ is the number of operations happen on replica $\arep$ during $\aglobalstate_0 \xrightarrow{}^* \aglobalstate$. We can see that, for each replica $\arep' \neq \arep$, $V'[\arep'] = max_{V \in \mathcal{V}} V[\arep']$, and $V'[\arep] = max_{V \in \mathcal{V}} V[\arep] +1$. Therefore, Annotation1 still holds in $\aglobalstate'$.

    \item[-] For case of applying downstream $(a,V')$: We only need to consider Annotation2. Let $S$ and $S'$ be the payload of replica $\arep$ of $\aglobalstate$ and $\aglobalstate'$, respectively.

        By the {\textred{causal delivery}} assumption, if $\alabellongind[write]{a,V'}{\bot}{*}$ is visible to a operation $\alabel$, then $\alabel$ does not applied in $\aglobalstate'$ yet. By Annotation2, $fact1$ and $fact2$, we know that, $\forall (b,V) \in S$, we have $\neg(V > V')$. Therefore, we have $S' = S \setminus \{ (b,V) \vert (b,V) \in S \wedge V < V' \} \cup \{ (a,V') \}$. By $fact1$ and $fact2$, each element in $\{ (b,V) \vert (b,V) \in S \wedge V < V' \}$ is visible to $\alabellongind[write]{a,V'}{\bot}{*}$, and they are not in $S'$. Therefore, Annotation2 still holds in $\aglobalstate'$.
    \end{itemize}

\item[-] Let us prove that $\mathsf{Refinement}$ holds. We consider a refinement mapping $\refmap$ defined as the identity.

    \begin{itemize}
    \setlength{\itemsep}{0.5pt}
    \item[-] For $(a,V')$ and $write(a,V',S_1)$:

    Assume we obtain payload $S'$ from $S$ by doing downstream of $(a,V')$, in sequential specification have $\abstate \xrightarrow{write(a,V',S_1)} \abstate'$, and $\refmap(S) = \abstate$, or we can say, $S = \abstate$. We need to prove that $S' = \abstate'$.

    By the {\textred{causal delivery}} assumption, if $(\alabellongind[write]{a,V'}{\bot}{*},\alabel) \in \avisord$, then the downstream of $\alabel$ is not applied yet in the replica of $S$. By Annotation2, $fact1$ and $fact2$, we can see that, $\forall (b,V) \in S$, $\neg(V > V')$. Therefore, according to the implementaiton, we can see that $S' = S \setminus S_2 \cup \{ (a,V') \}$, where $S_2 = \{ (b,V) \vert (b,V) \in S \wedge V < V' \}$.

    According to Annotation2, we can see that, $S_1 = \{ (b,V) \vert \exists \alabel = \alabellongind[write]{b,V}{\bot}{*}, (b,V)$ is the downstream of $l, l$ is maximal among $write$ operations visible to $\alabellongind[write]{a,V'}{\bot}{*} \}$. We can see that $\abstate' = \abstate \setminus S_1 \cup \{ (a,V') \}$.

    Let us prove $S' = \abstate'$ by contradiction.

        \begin{itemize}
        \setlength{\itemsep}{0.5pt}
        \item[-] If there exists item $(c,V'')$ in $\abstate'$ but not in $S'$: we can see that $(c,V'') \in S$, $(c,V'') \notin S_1$, and $(c,V'') \in S_2$.

        Since $(c,V'') \notin S_1$, we know that there exists a $write$ operation $\alabel$, such that $(\alabellongind[write]{c,V''}{\bot}{*},\alabel),(\alabel,\alabellongind[write]{a,V'}{\bot}{*}) \in \avisord$. Since $(c,V'') \in S$, we can see that the downstream $\alabel$ is not applied yet in the replica of $S$, while in $S'$, the downstream of $\alabellongind[write]{a,V'}{\bot}{*}$ is applied. This violates the {\textred{causal delivery}} assumption.

        \item[-] If there exists item $(c,V'')$ in $S'$ but not in $S\abstate'$: we can see that $(c,V'') \in S$, $(c,V'') \notin S_2$, and $(c,V'') \in S_1$.

        Since $(c,V'') \notin S_2$, we know that $\neg(V'' < V)$. Since $(c,V'') \in S_1$, we know that $(\alabellongind[write]{c,V''}{\bot}{*},\alabellongind[write]{a,V'}{\bot}{*}) \in \avisord$. This contradicts $fact1$ and $fact2$.
        \end{itemize}

    Therefore, we know that $S' = \abstate'$, and the case of $(a,V')$ and $write(a,V',S_1)$ holds.

    \item[-] When the query-update $\alabelshort[write]{a}$ executes {\tt atSource} on a state $S$, then the query $\alabellong[readIds]{}{R}{}$ (introduced by the query-update rewriting) should be enabled in state $\refmap(S)=\abstate$, which clearly holds because the computation of $R$ in {\tt atSource} returns $S$, and the result of $\alabelshort[readIds]{}$ in the specification state $\abstate = S$ also returns $S$.

    \item[-] Applying the query $\alabelshort[read]{}$ on the payload $S$ should result in the same return value as applying the same query in the context of the specification on the same state $\abstate = \refmap(S)$, which again holds trivially.
    \end{itemize}

\item[-] Finally, we describe the proof of the fact that $\mathsf{\CRDTLinshort{}}$ is an inductive invariant. As already mentioned, appending operations to the linearization when they execute {\tt atSource} clearly implies that $\aseqord$ is consistent with the visibility. Next, the projection of $\aseqord$ on the updates is obviously admitted by the specification (the updates are always enabled from the point of view of the specification).
We also have to argue that for each query $\alabel_{\mathsf{qr}}\in\{\alabellongind[readIds]{}{R}{},\alabellongind[read]{}{A}{}\}$, the sequence $\aseqord'\cdot \alabel_{\mathsf{qr}}$ where $\aseqord'$ is the projection of $\aseqord$ on the set of updates
visible to $\alabel_{\mathsf{qr}}$ is admitted by the specification. First, by $\mathsf{ReplicaStates}$, the state $\sigma$ of the replica where $\alabel_{\mathsf{qr}}$ is applied is obtained by applying the downstreams of the operations visible to $\alabel_{\mathsf{qr}}$ in the linearization order. Then, by $\mathsf{Refinement}$, every downstream is simulated by the corresponding operation in the context of the specification. This implies that $\refmap(\sigma_0)\xRightarrow{\aseqord'}\refmap(\sigma)$, where $\sigma_0$ is the initial replica state. The query $\alabel_{\mathsf{qr}}$ is also simulated by the same operation in the context of the specification, which implies that $\refmap(\sigma)\xRightarrow{\alabel_{\mathsf{qr}}}\refmap(\sigma)$. These two facts imply that $\refmap(\sigma_0)\xRightarrow{\aseqord'\cdot \alabel_{\mathsf{qr}}}\refmap(\sigma)$ which means that $\aseqord'\cdot \alabel_{\mathsf{qr}}$ is admitted by the specification.
\end{itemize}

This completes the proof of this lemma. $\qed$
\end {proof}
















\section{\crdtlin{} with Non-Deterministic Sequential Specifications and Its Proof}
\label{sec:appendix RA-linearizability with non-deterministic sequential specifications and its proof}



\subsection{\crdtlin{} with Non-Deterministic Sequential Specifications}
\label{subsec:RA-linearizability with non-deterministic sequential specifications} 

Recall that a sequential specification \Spec{} is deterministic, if for every label, the transition from a given initial state can produce at most one final state. Otherwise, we say that \Spec{} is non-deterministic. 

Similarly as in Section \ref{subsec:definition of distributed linearizability}, let us provide the definition of \crdtlin{}. For presentation reasons, we first consider the case where all the labels in the history are either queries or updates. 

\begin{definition}
\label{definition:ralinearizability1 with non-deterministic specifications} 
A history $h = (\alabelset,\avisord)$ with $\alabelset\subseteq \queries\cup\updates$ is \crdtlinearizable{} w.r.t. a non-deterministic sequential specification \Spec{}, if there exists a specification sequence $(\alabelset, \aseqord) \in \Spec{}$, called the \emph{\crdtlinearization{}} of $h$, where we remark that the set of labels are identical, such that 
\begin{enumerate}[(i)]
\item \aseqord{} is consistent with  \avisord{}, that is: $(\avisord \cup \aseqord)^{+}$ is acyclic,

\item the projection of $\aseqord$ to \emph{updates} is admitted by $\Spec$, i.e. $\aseqord\!\downarrow_{\updates} \in \Spec$, 

\item assume $\abstate_0$ is the initial abstract value of \Spec. For each sub-sequence $s$ of $\aseqord\!\downarrow_{\updates}$ (including $s = \aseqord\!\downarrow_{\updates}$), we fix a abstract state $\abstate_s$ which is obtained by $\abstate_0 \xrightarrow{ s }^* \abstate_s$, whenever such transitions are defined.

    We require that, given two sub-sequences $s_1,s_2$ of $\aseqord\!\downarrow_{\updates}$, if $s_2$ contains more operation than $s_1$, and both $\abstate_{s_1}$ and $\abstate_{s_2}$ are defined, then, $\abstate_{s_1} \xrightarrow{ s_2 - s_1 }^* \abstate_{s_2}$, where $s_2 - s_1$ denotes the sequences obtained from $s_2$ by removing elements of $s_1$. 


\item for each query $\alabel_{\mathsf{qr}}\in \alabelset$, let $s = \avisord^{-1}(\alabel_{\mathsf{qr}})\cap \updates$. Then, we require that $\abstate_s \xrightarrow{ \alabel_{\mathsf{qr}} }^* \abstate_s$. 
\end{enumerate}
In this case we say that $(\alabelset, \aseqord)$ is an \emph{\crdtlinearization{}} of $h$ w.r.t. $\Spec{}$.
\end{definition} 

Since the sequential specification is non-deterministic, we fix the consequence $\abstate_s$ in sequential specification when applying update operation sequence $s$ from the initial state $\abstate_0$ of sequential specification. We also require that, when $s_1$ is a sub-sequence of $s_2$, the corresponding abstract of $s_1$ is a ``sub-state'' of that of $s_2$. 

The case when histories include query-updates is similarly dealt with as Definition \ref{definition:distributed linearizability}. We do it by rewriting of the original history where each query-update is decomposed into a label representing the query part and another label representing the update part. 

\begin{definition}[\CRDTLin{} with Non-Deterministic Sequential Specifications] 
\label{definition:distributed linearizability with non-deterministic sequential specifications} 
A history $h =(\alabelset,\avisord)$ is \crdtlinearizable{} w.r.t. a non-deterministic sequential specification \Spec{}, if there exists a query-update rewriting $\gamma$ such that $\gamma(h)$ is \crdtlinearizable{} w.r.t. \Spec{}.
\end{definition} 

A set $H$ of histories is called \crdtlinearizable{} w.r.t a non-deterministic sequential specification $\Spec$ when each history $h\in H$ is \crdtlinearizable{} w.r.t. $\Spec$. A data type implementation is \crdtlinearizable{} w.r.t. a non-deterministic sequential specification $\Spec$ if for any object $\aobj$ of the data type, the set $\histories(\aobj)$ is linearizable w.r.t. $\Spec$. 

Let us begin to consider convergence. Given a \crdtlinearizable{} history with two replicas $r_1,r_2$ see the same set of operations. 

According to Definition \ref{definition:ralinearizability1 with non-deterministic specifications} and Definition \ref{definition:distributed linearizability with non-deterministic sequential specifications}, we have already fix the non-deterministic choice. Therefore, it is obvious that we have convergence, as formalized in the following lemma.

\begin{lemma}
\label{lemma:distributed linarizability implies convergence for non-deterministic sequential specifications}
If a history $h$ is \crdtlinearizable{} w.r.t. a non-deterministic sequential specification \Spec, then $h$ is convergent. 
\end{lemma}






\subsection{Proof Methodology for \crdtlin{} with Non-Deterministic Sequential Specifications}
\label{subsec:proof methodology for RA-linearizability with non-deterministic sequential specifications} 

















\section{Proofs of Section \ref{sec:proving distributed linearizability}}
\label{sec:appendix proofs of section proving distributed linearizability}





\subsection{Proof of OR-set Implementation}
\label{subsec:appendix proofs of or-set implementation}

The following lemma states a property that can be generated from $P(\mathit{config},h,\mathit{lin},\mathit{map})$ for or-set.

\begin{lemma}
\label{lemma:a property that can be obtained from P for or-set}
If $P(\mathit{config},h,\mathit{lin},\mathit{map})$ holds for or-set, then each $\mathit{add}$ operation generate a new unique time-stamp. Moreover, for each replica $r'$,

\begin{itemize}
    \setlength{\itemsep}{0.5pt}
    \item[-] $R(r').S = \{ (b,\mathit{ts}') \vert b \in D, \exists o' = (\mathit{add}(b),\_,\_,\mathit{ts}'), o' \in \mathit{vd}(h,\mathit{del},r'), \forall o'' = (\mathit{rem}(b),\_,\_,\_), o'' \in \mathit{vd}(h,\mathit{config},r') \Rightarrow (o',o'') \notin h.\mathit{vis} \}$.

    \item[-] $R(r').\mathit{maxTS} = (0,r')$ if $\mathit{vd}(h,\mathit{config},r') = \emptyset$; otherwise, $R(r').\mathit{maxTS}$ is the maximal time stamp of $\mathit{add}$ operations of $\mathit{vd}(h,\mathit{config},r')$.
    \end{itemize}
\end{lemma}

\begin {proof}
By $C_4$, it is easy to see that each $\mathit{add}$ operation generate a new unique time-stamp by induction. The property of $R(r')$ can be also easily proved by induction, since the visibility relation is transitive. $\qed$
\end {proof}


The following lemma states that our $P(\mathit{config},h,\mathit{lin},\mathit{map})$ is an invariant of or-set.

\begin{lemma}
\label{lemma:P is an invariant of or-set}
$P(\mathit{config},h,\mathit{lin},\mathit{map})$ is an invariant of or-set.
\end{lemma}

\begin {proof}

Let us prove that $P$ is a simulation relation. It is obvious that $P(\mathit{config}_0,\epsilon,\emptyset,\emptyset)$ holds.

Assume $P((R,T,\mathit{MsgHB},\mathit{MsgDel}),h,\mathit{lin},\mathit{map})$ holds. Here we do not give the detailed value of $\mathit{MsgHB}'$ and $\mathit{MsgDel}'$, since it can be obtained from the definition of $\llbracket \mathit{obj} \rrbracket_{\mathit{op}}$.

\begin{itemize}
\setlength{\itemsep}{0.5pt}
\item[-] If $(R,T,\mathit{MsgHB},\mathit{MsgDel}) {\xrightarrow{\mathit{do}(\mathit{add},a,r,\mathit{mid})}} (R',T',\mathit{MsgHB}',\mathit{MsgDel}')$: Then,

    \begin{itemize}
    \setlength{\itemsep}{0.5pt}
    \item[-] $R' = R[ r: (R(r).S \cup \{ (a,\mathit{ts}) \},\mathit{ts}) ]$ and $T' = T \cup \{ (\mathit{mid},(a,\mathit{ts}),r) \}$. Here $\mathit{ts} = ( \mathit{max} \{ c \vert (\_,(c,\_)) \in R(r).S \} +1,r)$.

    \item[-] Let $h' = h \otimes i$, where $i$ is the identifier of the newly-generated $\mathit{add}$ operation.

    \item[-] Let $\mathit{lin}' = \mathit{lin} \cdot (\mathit{add}(a),i,\mathit{vd}(h,\mathit{config},r))$.

    \item[-] Let $\mathit{map}' = \mathit{map} \cup \{ (\mathit{mid},i) \}$.
    \end{itemize}

    It is easy to see that $h'$ is still distributed linearizable and $\mathit{lin}'$ is its linearization. We need to prove that $R'(r) = \mathit{apply}(\mathit{lin}',\mathit{vd}(h',\mathit{del}',r))$ and $C_4$ still holds for message $\mathit{mid}$.

    We already know that $R(r) = \mathit{apply}(\mathit{lin},\mathit{vd}(h,\mathit{del},r))$. %Based on $C_4$, it is not hard to prove that,

    %\begin{itemize}
    %\setlength{\itemsep}{0.5pt}
    %\item[-] $\mathit{Prop}_1$: Each $\mathit{add}$ operation generate a new unique time-stamp.

    %\item[-] $\mathit{Prop}_2$: for each replica $r'$, $R(r') = \{ (b,\mathit{ts}') \vert b \in D, \exists o' = (\mathit{add}(b),\_,\_,\mathit{ts}'), o' \in \mathit{vd}(h,\mathit{config},r'), \forall o'' = (\mathit{rem}(b),\_,\_,\_), o'' \in \mathit{vd}(h,\mathit{config},r') \Rightarrow (o',o'') \notin h.\mathit{vis} \}$.
    %\end{itemize}

    By Lemma \ref{lemma:a property that can be obtained from P for or-set}, it is not hard to see that $C_4$ still holds for message $\mathit{mid}$. From construction of $R'(r)$, Lemma \ref{lemma:a property that can be obtained from P for or-set} and $C_4$ holds for message $\mathit{mid}$, we can see that $R'(r) = \mathit{apply}(\mathit{lin}',\mathit{vd}(h',\mathit{del}',r))$.%, and $\mathit{Prop}_1$ and $\mathit{Prop}_2$ also hold for $(\mathit{config}',h',\mathit{lin}',\mathit{map}')$.


\item[-] If $(R,T,\mathit{MsgHB},\mathit{MsgDel}) {\xrightarrow{\mathit{do}(\mathit{rem},a,r,\mathit{mid})}} (R',T',\mathit{MsgHB}',\mathit{MsgDel}')$: Then,

    \begin{itemize}
    \setlength{\itemsep}{0.5pt}
    \item[-] $R' = R[ r: (R(r).S \setminus \{ (a,\mathit{ts}) \in R(r).S \},R(r).\mathit{maxTS}) ]$ and $T' = T \cup \{ (\mathit{mid},\{ (a,\mathit{ts}) \in R(r) \},r) \}$.

    \item[-] Let $h' = h \otimes i$, where $i$ is the identifier of the newly-generated $\mathit{rem}$ operation.

    \item[-] Let $\mathit{lin}' = \mathit{lin} \cdot (\mathit{add}(a),i,\mathit{vd}(h,\mathit{config},r))$.

    \item[-] Let $\mathit{map}' = \mathit{map} \cup \{ (\mathit{mid},i) \}$.
    \end{itemize}

    It is easy to see that $h'$ is still distributed linearizable and $\mathit{lin}'$ is its linearization. We need to prove that $R'(r) = \mathit{apply}(\mathit{lin}',\mathit{vd}(h',\mathit{del}',r))$ and $C_4$ still holds for message $\mathit{mid}$.

    By Lemma \ref{lemma:a property that can be obtained from P for or-set}, it is not hard to see that $C_4$ still holds for message $\mathit{mid}$. From construction of $R'(r)$, Lemma \ref{lemma:a property that can be obtained from P for or-set} and $C_4$ holds for message $\mathit{mid}$, we can see that $R'(r) = \mathit{apply}(\mathit{lin}',\mathit{vd}(h',\mathit{del}',r))$.%, and $\mathit{Prop}_1$ and $\mathit{Prop}_2$ also hold for $(\mathit{config}',h',\mathit{lin}',\mathit{map}')$.


\item[-] If $(R,T,\mathit{MsgHB},\mathit{MsgDel}) {\xrightarrow{\mathit{do}(\mathit{read},S_1,r)}} (R',T',\mathit{MsgHB}',\mathit{MsgDel}')$: Then,

    \begin{itemize}
    \setlength{\itemsep}{0.5pt}
    \item[-] $R' = R$ and $T' = T$.

    \item[-] Let $h' = h \otimes i$, where $i$ is the identifier of the newly-generated $\mathit{read}$ operation.

    \item[-] Let $\mathit{lin}' = \mathit{lin} \cdot (\mathit{read} \Rightarrow S_1,i,\mathit{vd}(h,\mathit{config},r))$.

    \item[-] Let $\mathit{map}'$.
    \end{itemize}

    We need to prove that $h'$ is distributed linearizable and $\mathit{lin}'$ is a linearization. Assume that in $\mathit{OR}$-$\mathit{set}_s$, $\mathit{state}_0 {\xrightarrow{\mathit{lin}}} \mathit{state}$ and $\mathit{state} {\xrightarrow{ (\mathit{read} \Rightarrow S_2, i, \mathit{vd}(h,\mathit{config},r) ) }} \mathit{state}$. Then by definition of $\mathit{OR}$-$\mathit{set}_s$, we can see that, $a \in S_2$, if there exists $(\mathit{add}(a),j,\_) \in \mathit{lin}'$, and for each $(\mathit{rem}(a),\_,S_2) \in \mathit{lin}'$, we have $j \notin S_2$. Lemma \ref{lemma:a property that can be obtained from P for or-set}, we can see that $S_1 = S_2$, and since $h$ is distributed linearizable and $\mathit{lin}$ is a linearization of $h$, we can see $h'$ is distributed linearizable and $\mathit{lin}'$ is a linearization.

\item[-] If $(R,T,\mathit{MsgHB},\mathit{MsgDel}) {\xrightarrow{\mathit{receive}(\mathit{mid},r)}} (R',T',\mathit{MsgHB}',\mathit{MsgDel}')$, where $(\mathit{mid},(a,\mathit{ts}),r') \in T$: Then,

    \begin{itemize}
    \setlength{\itemsep}{0.5pt}
    \item[-] $R' = R[ r: (R(r).S \cup \{ (a,\mathit{ts}) \},\mathit{max} \{ R(r).\mathit{maxTS},\mathit{ts} \} ) ]$ and $T' = T$.

    \item[-] Let $h' = h$.

    \item[-] Let $\mathit{lin}' = \mathit{lin}$.

    \item[-] Let $\mathit{map}' = \mathit{map}$.
    \end{itemize}

    We need to prove that $R'(r) = \mathit{apply}(\mathit{lin}',\mathit{vd}(h',\mathit{del}',r))$.

    We already know that $R(r) = \mathit{apply}(\mathit{lin},\mathit{vd}(h,\mathit{del},r))$. Since $R'(r)$ is obtained from $R(r)$ by applying message $\mathit{mid}$, and $\mathit{apply}(\mathit{lin}',\mathit{vd}(h',\mathit{del}',r))$ is obtained from $\mathit{apply}(\mathit{lin},\mathit{vd}(h,\mathit{del},r))$ by applying message $\mathit{mid}$. Therefore, $R'(r) = \mathit{apply}(\mathit{lin}',\mathit{vd}(h',\mathit{del}',r))$.

\item[-] If $(R,T,\mathit{MsgHB},\mathit{MsgDel}) {\xrightarrow{\mathit{receive}(\mathit{mid},r)}} (R',T',\mathit{MsgHB}',\mathit{MsgDel}')$, where $(\mathit{mid},S_1,r') \in T$: Then,

    \begin{itemize}
    \setlength{\itemsep}{0.5pt}
    \item[-] $R' = R[ r: (R(r).S \setminus S_1, R(r).\mathit{maxTS}) ]$ and $T' = T$.

    \item[-] Let $h' = h$.

    \item[-] Let $\mathit{lin}' = \mathit{lin}$.

    \item[-] Let $\mathit{map}' = \mathit{map}$.
    \end{itemize}

    We need to prove that $R'(r) = \mathit{apply}(\mathit{lin}',\mathit{vd}(h',\mathit{del}',r))$.

    We already know that $R(r) = \mathit{apply}(\mathit{lin},\mathit{vd}(h,\mathit{del},r))$. Since $R'(r)$ is obtained from $R(r)$ by applying message $\mathit{mid}$, and $\mathit{apply}(\mathit{lin}',\mathit{vd}(h',\mathit{del}',r))$ is obtained from $\mathit{apply}(\mathit{lin},\mathit{vd}(h,\mathit{del},r))$ by applying message $\mathit{mid}$. Therefore, $R'(r) = \mathit{apply}(\mathit{lin}',\mathit{vd}(h',\mathit{del}',r))$.
\end{itemize}

This completes the proof of this lemma. $\qed$
\end {proof}




\subsection{Proof of RGA}
\label{subsec:appendix proofs of rga}

The following lemma states a property that can be generated from $P(\mathit{config},h,\mathit{lin},\mathit{map})$ for RGA.

\begin{lemma}
\label{lemma:a property that can be obtained from P for rga}
If $P(\mathit{config},h,\mathit{lin},\mathit{map})$ holds, then each $\mathit{add}$ operation generate a new unique time-stamp. Moreover, for each replica $r'$,

\begin{itemize}
    \setlength{\itemsep}{0.5pt}
    \item[-] $R(r').N = \{ (a,\mathit{ts}_a,\mathit{ts}_b) \vert \exists o' = (\mathit{add}(\_,\_),i,\_,\_), \mathit{map}(i) = (a,\mathit{ts}_a,\mathit{ts}_b), o' \in \mathit{vd}(h,\mathit{config},r') \}$.

    \item[-] $R(r').\mathit{Tomb} = \{ a \vert \exists o = (\mathit{rem}(a),i,\_,\_), \mathit{map}(i) \in \mathit{vd}(h,\mathit{config},r') \}$.
    \end{itemize}
\end{lemma}

\begin {proof}
By $C_4$, it is easy to see that each $\mathit{add}$ operation generate a new unique time-stamp by induction. The property of $R(r')$ can be also easily proved by induction, since the visibility relation is transitive. $\qed$
\end {proof}


The following lemma states that our $P(\mathit{config},h,\mathit{lin},\mathit{map})$ is an invariant of rga.

\begin{lemma}
\label{lemma:P is an invariant of rga}
$P(\mathit{config},h,\mathit{lin},\mathit{map})$ is an invariant of rga.
\end{lemma}

\begin {proof}

Let us prove that $P$ is a simulation relation. It is obvious that $P(\mathit{config}_0,\epsilon,\emptyset,\emptyset)$ holds.

Assume $P((R,T,\mathit{MsgHB},\mathit{MsgDel}),h,\mathit{lin},\mathit{map})$ holds. Here we do not give the detailed value of $\mathit{MsgHB}'$ and $\mathit{MsgDel}'$, since it can be obtained from the definition of $\llbracket \mathit{obj} \rrbracket_{\mathit{op}}$.

\begin{itemize}
\setlength{\itemsep}{0.5pt}
\item[-] If $(R,T,\mathit{MsgHB},\mathit{MsgDel}) {\xrightarrow{\mathit{do}(\mathit{add},a,b,r,\mathit{mid})}} (R',T',\mathit{MsgHB}',\mathit{MsgDel}')$: Then,

    \begin{itemize}
    \setlength{\itemsep}{0.5pt}
    \item[-] $R' = R[ r: (R(r).N \cup \{ (a,\mathit{ts}_a,\mathit{ts}_b) \}, R(r).\mathit{Tomb}) ]$ and $T' = T \cup \{ (\mathit{mid},(a,\mathit{ts}_a,\mathit{ts}_b),r) \}$. Here $\mathit{ts}_a = ( \mathit{max} \{ c \vert (\_,(c,\_),\_) \in R(r).N \vee (\_,\_,(c,\_)) \in R(r).N \} +1,r)$, and $\mathit{ts}_b$ is the time-stamp of $b$ in $R(r).N$.

    \item[-] Let $h' = h \otimes i$, where $i$ is the identifier of the newly-generated $\mathit{add}$ action.

    \item[-] $\mathit{lin}'$ is obtained from $\mathit{lin}$ by inserting $(\mathit{add}(a,b),i,\mathit{vd}(h,\mathit{config},r))$ after the last operation with time-stamp less or equal than $\mathit{ts}_a$.

    \item[-] Let $\mathit{map}' = \mathit{map} \cup \{ (\mathit{mid},i) \}$.
    \end{itemize}

    It is easy to see that $h'$ is still distributed linearizable and $\mathit{lin}'$ is its linearization. We need to prove that $R'(r) = \mathit{apply}(\mathit{lin}',\mathit{vd}(h',\mathit{del}',r))$ and $C_4$ still holds for message $\mathit{mid}$.

    We already know that $R(r) = \mathit{apply}(\mathit{lin},\mathit{vd}(h,\mathit{del},r))$.

    By Lemma \ref{lemma:a property that can be obtained from P for rga}, it is not hard to see that $C_4$ still holds for message $\mathit{mid}$. From the fact that $\mathit{ts}_a$ is unique, the fact that there is no $\mathit{rem}(a)$ in $h$, the construction of $R'(r)$, Lemma \ref{lemma:a property that can be obtained from P for rga} and $C_4$ holds for message $\mathit{mid}$, we can see that $R'(r) = \mathit{apply}(\mathit{lin}',\mathit{vd}(h',\mathit{del}',r))$.


\item[-] If $(R,T,\mathit{MsgHB},\mathit{MsgDel}) {\xrightarrow{\mathit{do}(\mathit{rem},a,r,\mathit{mid})}} (R',T',\mathit{MsgHB}',\mathit{MsgDel}')$: Then,

    \begin{itemize}
    \setlength{\itemsep}{0.5pt}
    \item[-] $R' = R[ r: (R(r).N,R(r).\mathit{Tomb} \cup \{ a \} ) ]$ and $T' = T \cup \{ (\mathit{mid},\{ a \},r) \}$.

    \item[-] Let $h' = h \otimes i$, where $i$ is the identifier of the newly-generated $\mathit{rem}$ operation.

    \item[-] $\mathit{lin}'$ is obtained from $\mathit{lin}$ by inserting $(\mathit{rem}(a),i,\mathit{vd}(h,\mathit{config},r))$ after the last operation with time-stamp less or equal than the time-stamp of operation $i$.

    \item[-] Let $\mathit{map}' = \mathit{map} \cup \{ (\mathit{mid},i) \}$.
    \end{itemize}

    By Lemma \ref{lemma:a property that can be obtained from P for rga}, it is easy to see that $\mathit{lin} \uparrow_{\mathit{vd}(h,\mathit{config},r)}$ contains a $\mathit{add}(a,\_)$ operation $o$ and $(o,i) \in h'.\mathit{vis}$. By Lemma \ref{lemma:a property that can be obtained from P for rga}, it is easy to see that $\mathit{lin} \uparrow_{\mathit{vd}(h,\mathit{config},r)}$ does not contain $\mathit{rem}(a)$. Since $i$ does not visible to any operation in $\mathit{vd}(h,\mathit{config},r)$, we can see that $h'$ is still distributed linearizable and $\mathit{lin}'$ is its linearization.

    We need to prove that $R'(r) = \mathit{apply}(\mathit{lin}',\mathit{vd}(h',\mathit{del}',r))$ and $C_4$ still holds for message $\mathit{mid}$.

    It is obvious that $C_4$ holds for message $\mathit{mid}$. By Lemma \ref{lemma:a property that can be obtained from P for rga}, the construction of $R'(r)$, and $C_4$ holds for message $\mathit{mid}$, we can see that $R'(r) = \mathit{apply}(\mathit{lin}',\mathit{vd}(h',\mathit{del}',r))$.


\item[-] If $(R,T,\mathit{MsgHB},\mathit{MsgDel}) {\xrightarrow{\mathit{do}(\mathit{read},l,r)}} (R',T',\mathit{MsgHB}',\mathit{MsgDel}')$: Then,

    \begin{itemize}
    \setlength{\itemsep}{0.5pt}
    \item[-] $R' = R$ and $T' = T$.

    \item[-] Let $h' = h \otimes i$, where $i$ is the identifier of the newly-generated $\mathit{read}$ operation.

    \item[-] $\mathit{lin}'$ is obtained from $\mathit{lin}$ by inserting $(\mathit{rem}(a),i,\mathit{vd}(h,\mathit{config},r))$ after the last operation with time-stamp less or equal than the time-stamp of operation $i$.

    \item[-] Let $\mathit{map}' = \mathit{map}$.
    \end{itemize}

    We need to prove that $h'$ is distributed linearizable and $\mathit{lin}'$ is a linearization. Assume that in $\mathit{list}_s^{\mathit{af}}$, $\mathit{state}_0 {\xrightarrow{\mathit{lin}}} \mathit{state}$ and $\mathit{state} {\xrightarrow{ (\mathit{read} \Rightarrow l_1, i, \mathit{vd}(h,\mathit{config},r) ) }} \mathit{state}$.

    By Lemma \ref{lemma:a property that can be obtained from P for rga} and RGA implementation, we can see that $l$ and $l_1$ has the same items.

    Given items $a,b$, assume that $a$ is before $b$ in $l$, then, there are two possibilities,

    \begin{itemize}
    \setlength{\itemsep}{0.5pt}
    \item[-] $a$ is a ancestor of $b$ in $R(r).N$,

    \item[-] there exists items $c_1,c_2,c_3$, such that in $R(r).N$, $c_2$ and $c_3$ are sons of $c_1$, $c_2$ is a ancestor of $a$, $c_3$ is a ancestor of $b$, and the time-stamp of $c_2$ is larger than that of $c_3$.
    \end{itemize}

    If the first possibility holds, then there exists items $d_1,\ldots,d_k$, such that in $R(r).N$, $b$ is a son of $d_1$, $d_1$ is a son of $d_2$, $\ldots$, and $d_k$ is a son of $a$. It is easy to see that $(\mathit{add}(a,\_),\mathit{add}(d_k,a)),(\mathit{add}(d_k,a),\mathit{add}(d_{\mathit{k-1}},d_k)), \ldots, (\mathit{add}(d_1,d_2),\mathit{add}(b,d_1)) \in h.\mathit{vis}$. Since $\mathit{lin}$ is consistent with visibility relation, we know that in $\mathit{lin}$, $\mathit{add}(a,\_)$ is before $\mathit{add}(d_k,a)$, $\mathit{add}(d_k,a)$ is before $\mathit{add}(d_{\mathit{k-1}},d_k)$, $\ldots$, and $\mathit{add}(d_1,d_2)$ is before $\mathit{add}(b,d_1)$. According to $\mathit{list}_s^{\mathit{af}}$, it is easy to see that in $a$ is before $b$ in $l_1$.

    If the second possibility holds, then it is easy to see that $(\mathit{add}(c_2,c_1),\mathit{add}(a,c_2)),$ $(\mathit{add}(c_1,\_),\mathit{add}(c_2,c_1)), (\mathit{add}(c_3,c_1),\mathit{add}(b,c_3)),(\mathit{add}(c_1,\_),\mathit{add}(c_3,c_1)), \in h.\mathit{vis}$. Since $\mathit{lin}$ is consistent with visibility relation and time-stamp, we know that in $\mathit{lin}$, $\mathit{add}(c_3,c_1)$ is before $\mathit{add}(c_2,c_1)$, $\mathit{add}(c_2,c_1)$ is before $\mathit{add}(a,c_2)$, and $\mathit{add}(c_3,c_1)$ is before $\mathit{add}(b,c_3)$. According to $\mathit{list}_s^{\mathit{af}}$, it is easy to see that in $a$ is before $b$ in $l_1$.

    Therefore, $h'$ is distributed linearizable and $\mathit{lin}'$ is a linearization.

\item[-] If $(R,T,\mathit{MsgHB},\mathit{MsgDel}) {\xrightarrow{\mathit{receive}(\mathit{mid},r)}} (R',T',\mathit{MsgHB}',\mathit{MsgDel}')$, where $(\mathit{mid},(a,\mathit{ts}_a,\mathit{ts}_b),r') \in T$: Then,

    \begin{itemize}
    \setlength{\itemsep}{0.5pt}
    \item[-] $R' = R[ r: ( R(r).N \cup \{ (a,\mathit{ts}_a,\mathit{ts}_b) \}, R(r).\mathit{Tomb} ) ]$ and $T' = T$.

    \item[-] Let $h' = h$.

    \item[-] Let $\mathit{lin}' = \mathit{lin}$.

    \item[-] Let $\mathit{map}' = \mathit{map}$.
    \end{itemize}

    We need to prove that $R'(r) = \mathit{apply}(\mathit{lin}',\mathit{vd}(h',\mathit{del}',r))$.

    We already know that $R(r) = \mathit{apply}(\mathit{lin},\mathit{vd}(h,\mathit{del},r))$.

    We can see that $R'(r)$ is obtained from $R(r)$ by applying message $\mathit{mid}$, and $\mathit{apply}(\mathit{lin}',\mathit{vd}(h',\mathit{del}',r))$ is obtained from $\mathit{apply}(\mathit{lin},\mathit{vd}(h,\mathit{del},r))$ by additionally applying messages $\mathit{mid}$, but possibly in the middle of $\mathit{lin}'$. It is easy to see that $\mathit{map}(\mathit{mid})$ is a $\mathit{add}(a,\_)$ operation. By Lemma \ref{lemma:a property that can be obtained from P for rga}, we can see that there is no $\mathit{add}(a,\_)$ nor $\mathit{rem}(a)$ in $\mathit{vd}(h,\mathit{del},r)$. Thus, for each $\mathit{lin}''$ generated from $\mathit{lin}'$ by postponing message $\mathit{mid}$ to a later position, we can see that $\mathit{apply}(\mathit{lin}'',\mathit{vd}(h',\mathit{del}',r)) = \mathit{apply}(\mathit{lin}',\mathit{vd}(h',\mathit{del}',r))$.

    Therefore, $R'(r) = \mathit{apply}(\mathit{lin}',\mathit{vd}(h',\mathit{del}',r))$.

\item[-] If $(R,T,\mathit{MsgHB},\mathit{MsgDel}) {\xrightarrow{\mathit{receive}(\mathit{mid},r)}} (R',T',\mathit{MsgHB}',\mathit{MsgDel}')$, where $(\mathit{mid},a,r') \in T$: Then,

    \begin{itemize}
    \setlength{\itemsep}{0.5pt}
    \item[-] $R' = R[ r: (R(r).N,R(r).\mathit{Tomb} \cup \{ a \}) ]$ and $T' = T$.

    \item[-] Let $h' = h$.

    \item[-] Let $\mathit{lin}' = \mathit{lin}$.

    \item[-] Let $\mathit{map}' = \mathit{map}$.
    \end{itemize}

    We need to prove that $R'(r) = \mathit{apply}(\mathit{lin}',\mathit{vd}(h',\mathit{del}',r))$.

    We already know that $R(r) = \mathit{apply}(\mathit{lin},\mathit{vd}(h,\mathit{del},r))$.

    We can see that $R'(r)$ is obtained from $R(r)$ by applying message $\mathit{mid}$, and $\mathit{apply}(\mathit{lin}',\mathit{vd}(h',\mathit{del}',r))$ is obtained from $\mathit{apply}(\mathit{lin},\mathit{vd}(h,\mathit{del},r))$ by additionally applying messages $\mathit{mid}$, but possibly in the middle of $\mathit{lin}'$.

    It is easy to see that, for each $\mathit{lin}''$ generated from $\mathit{lin}'$ by postponing message $\mathit{mid}$ to a later position, we have $\mathit{apply}(\mathit{lin}'',\mathit{vd}(h',\mathit{del}',r)) = \mathit{apply}(\mathit{lin}',\mathit{vd}(h',\mathit{del}',r))$.

    Therefore, $R'(r) = \mathit{apply}(\mathit{lin}',\mathit{vd}(h',\mathit{del}',r))$.
\end{itemize}

This completes the proof of this lemma. $\qed$
\end {proof}










\section{Proofs of Section \ref{sec:compositionality of distributed linearizability}}
\label{sec:appendix proofs of section compositionality of distributed linearizability}





\subsection{Proofs of Lemma \ref{lemma:several t0-specifications}}
\label{subsec:appendix proofs of Lemma several t0-specifications}

A specification $\mathit{spec}$ is called t0-specification, if given a history $h$ that is distributed linearizable w.r.t $\mathit{spec}$, then any sequence that is consistent with visibility relation is a linearization of $h$.

Given two sequences $l_1,l_2$, let $\mathit{diff}(l_1,l_2) = \{ (o_1,o_2) \vert$ the order of $o_1$ and $o_2$ in $l_1$ is different from that of $l_2 \}$. Given a sequence $l$ and two elements $o_1$ an $o_2$ of $l$, let $\mathit{swap}(l,o_1,o_2)$ be a sequence obtained from $l$ by swapping $o_1$ and $o_2$.

The following lemma states that $\mathit{OR}$-$\mathit{set}_s$ is a t0-specification.

\begin{lemma}
\label{lemma:or-set is a t0-specification}
$\mathit{OR}$-$\mathit{set}_s$ is a t0-specification.
\end{lemma}

\begin {proof}
Given a distributed linearizable history $h$ and assume that $\mathit{lin}$ is a linearization. It is obvious that $\mathit{lin}$ is consistent with visibility relation. We need to prove that, each such sequence $\mathit{lin}'$ described below is also a linearization of $h$

\begin{itemize}
\setlength{\itemsep}{0.5pt}
\item[-] $\mathit{lin}'$ contains the same set of elements as that of $\mathit{lin}$.

\item[-] $\mathit{lin}'$ is consistent with visibility relation.
\end{itemize}

We prove this by showing that each such $\mathit{lin}'$ can be obtained from $\mathit{lin}$ by several times of swapping a pair of adjacent elements. Our proof requires the following two properties:

\begin{itemize}
\setlength{\itemsep}{0.5pt}
\item[-] The first property is: Given a linarization $\mathit{lin}$ and a sequence $\mathit{lin}'$ consistent with visibility relation of $h$, if $\mathit{diff}(\mathit{lin},\mathit{lin}') \neq \emptyset$, there exists $(o_1,o_2) \in \mathit{diff}(\mathit{lin},\mathit{lin}')$, such that $o_1$ and $o_2$ are concurrent, and $o_1$ and $o_2$ are adjacent in $\mathit{lin}$.

    We prove this by contradiction. Assume $\mathit{diff}(\mathit{lin},\mathit{lin}') \neq \emptyset$, and for each $(o_1,o_2) \in \mathit{diff}(\mathit{lin},\mathit{lin}')$, we have that either $o_1$ and $o_2$ are not concurrent, or $o_1$ and $o_2$ are not adjacent in $\mathit{lin}$.

    Since $\mathit{diff}(\mathit{lin},\mathit{lin}') \neq \emptyset$, let $(o,o')$ be a element of $\mathit{diff}(\mathit{lin},\mathit{lin}')$, and the distance of $o_1$ and $o_2$ is minimal in $\{$ the distance between $o_1$ and $o_2 \vert (o_1,o_2) \in \mathit{diff}(\mathit{lin},\mathit{lin}') \}$. Let us prove that $o$ and $o'$ are adjacent by contradiction: If there exists $o''$ between $o$ and $o'$. Assume that in $\mathit{lin}$, $o$ is before $o''$, and $o''$ is before $o'$. By assumption, the order between $o$ and $o''$, and between $o''$ and $o'$ is the same in $\mathit{lin}$ and in $\mathit{lin}'$. This implies that $o$ is still before $o'$ in $\mathit{lin}'$, which contradicts the fact that $(o,o') \in \mathit{diff}(\mathit{lin},\mathit{lin}')$.

    Since $o$ and $o'$ are adjacent and $(o,o') \in \mathit{diff}(\mathit{lin},\mathit{lin}')$, by assumption we know that $o$ and $o'$ are not concurrent. Or we can say, $(o,o') \in \mathit{vis} \vee \mathit{o',o} \in \mathit{vis}$. This contradicts that both $\mathit{lin}$ and $\mathit{lin}'$ are consistent with visibility relation. This completes the proof of the first step.

\item[-] The second property is: Given a linearization $\mathit{lin}$ and $o_1,o_2 \in \mathit{lin}$, such that $o_1$ and $o_2$ are concurrent and adjacent in $\mathit{lin}$, then, $l = \mathit{swap}(\mathit{lin},o_1,o_2)$ is also a linearization.

    Let $o_1 = (\ell_1,\mathit{id}_1,S_1)$ and $o_2 = (\ell_2,\mathit{id}_2,S_2)$. Since $o_1$ and $o_2$ are concurrent, we know that $\mathit{id}_1 \notin S_2 \wedge \mathit{id}_2 \notin S_1$. Assume $\mathit{lin} = l_1 \cdot o_1 \cdot o_2 \cdot l_2$. Assume in the abstract state of $\mathit{OR}$-$\mathit{set}_s$, we have $\sigma_0 {\xrightarrow{l_1}} \sigma_1 {\xrightarrow{o_1}} \sigma_2 {\xrightarrow{o_2}} \sigma_3 {\xrightarrow{l_2}} \sigma_4$, where $\sigma_0$ is the initial state of $\mathit{OR}$-$\mathit{set}_s$. Then, we need to prove that, there exists $\sigma'_2$, such that $\sigma_1 {\xrightarrow{o_2}} \sigma'_2 {\xrightarrow{o_1}} \sigma_3$. We prove this by consider all the possible cases:

    \begin{itemize}
    \setlength{\itemsep}{0.5pt}
    \item[-] If $o_1 = (\mathit{add}(a_1),\mathit{id}_1,S_1)$ and $o_2 = (\mathit{add}(a_2),\mathit{id}_2,S_2)$: We can see that $\sigma_2$ is obtained from $\sigma_1$ by inserting $(a_1,\mathit{id}_1,\mathit{true})$, and $\sigma_3$ is obtained from $\sigma_2$ by inserting $(a_2,\mathit{id}_2,\mathit{true})$. Let $\sigma'_2$ be obtained from $\sigma_1$ by inserting $(a_2,\mathit{id}_2,\mathit{true})$. Then, it is easy to see that $\sigma_1 {\xrightarrow{o_2}} \sigma'_2 {\xrightarrow{o_1}} \sigma_3$.

    \item[-] If $o_1 = (\mathit{add}(a_1),\mathit{id}_1,S_1)$ and $o_2 = (\mathit{rem}(a_2),\mathit{id}_2,S_2)$: We can see that $\sigma_2$ is obtained from $\sigma_1$ by inserting $(a_1,\mathit{id}_1,\mathit{true})$, and $\sigma_3$ is obtained from $\sigma_2$ by marking $a_2$ with identifiers of $S_2$ into $\mathit{false}$. Let $\sigma'_2$ be obtained from $\sigma_1$ by marking $a_2$ with identifiers of $S_2$ into $\mathit{false}$. Since $\mathit{id_1} \notin S_2$, we can see that $\sigma_1 {\xrightarrow{o_2}} \sigma'_2 {\xrightarrow{o_1}} \sigma_3$.

    \item[-] If $o_1 = (\mathit{add}(a_1),\mathit{id}_1,S_1)$ and $o_2 = (\mathit{read}() \Rightarrow l_2,\mathit{id}_2,S_2)$: Let $\sigma'_2 = \sigma_1$. Since $\mathit{id}_1 \notin S_2$, it is easy to see that $\sigma_1 {\xrightarrow{o_2}} \sigma'_2 {\xrightarrow{o_1}} \sigma_3$.

    \item[-] If $o_1 = (\mathit{rem}(a_1),\mathit{id}_1,S_1)$ and $o_2 = (\mathit{add}(a_2),\mathit{id}_2,S_2)$: We can see that $\sigma_2$ is obtained from $\sigma_1$ by marking $a_1$ with identifiers of $S_1$ into $\mathit{false}$, and $\sigma_3$ is obtained from $\sigma_2$ by inserting $(a_2,\mathit{id}_2,\mathit{true})$. Let $\sigma'_2$ be obtained from $\sigma_1$ by inserting $(a_2,\mathit{id}_2,\mathit{true})$. Since $\mathit{id}_2 \notin S_1$, we can see that $\sigma_1 {\xrightarrow{o_2}} \sigma'_2 {\xrightarrow{o_1}} \sigma_3$.

    \item[-] If $o_1 = (\mathit{rem}(a_1),\mathit{id}_1,S_1)$ and $o_2 = (\mathit{rem}(a_2),\mathit{id}_2,S_2)$: We can see that $\sigma_2$ is obtained from $\sigma_1$ by marking $a_1$ with identifiers of $S_1$ into $\mathit{false}$, and $\sigma_3$ is obtained from $\sigma_2$ by marking $a_2$ with identifiers of $S_2$ into $\mathit{false}$. Let $\sigma'_2$ be obtained from $\sigma_1$ by marking $a_2$ with identifiers of $S_2$ into $\mathit{false}$. Then, it is easy to see that $\sigma_1 {\xrightarrow{o_2}} \sigma'_2 {\xrightarrow{o_1}} \sigma_3$.

    \item[-] If $o_1 = (\mathit{rem}(a_1),\mathit{id}_1,S_1)$ and $o_2 = (\mathit{read}() \Rightarrow l_2,\mathit{id}_2,S_2)$: Let $\sigma'_2 = \sigma_1$. Since $\mathit{id}_1 \notin S_2$, it is easy to see that $\sigma_1 {\xrightarrow{o_2}} \sigma'_2 {\xrightarrow{o_1}} \sigma_3$.

    \item[-] If $o_1 = (\mathit{read}() \Rightarrow l_1,\mathit{id}_1,S_1)$ and $o_2 = (\mathit{add}(a_1),\mathit{id}_2,S_2)$: Let $\sigma'_2$ be obtained from $\sigma_1$ by inserting $(a_1,\mathit{id}_1,\mathit{true})$. Since $\mathit{id}_2 \notin S_1$, it is easy to see that $\sigma_1 {\xrightarrow{o_2}} \sigma'_2 {\xrightarrow{o_1}} \sigma_3$.

    \item[-] If $o_1 = (\mathit{read}() \Rightarrow l_1,\mathit{id}_1,S_1)$ and $o_2 = (\mathit{rem}(a_1),\mathit{id}_2,S_2)$: Let $\sigma'_2$ be obtained from $\sigma_1$ by marking $a_2$ with identifiers of $S_2$ into $\mathit{false}$. Since $\mathit{id}_2 \notin S_1$, it is easy to see that $\sigma_1 {\xrightarrow{o_2}} \sigma'_2 {\xrightarrow{o_1}} \sigma_3$.

    \item[-] If $o_1 = (\mathit{read}() \Rightarrow l_1,\mathit{id}_1,S_1)$ and $o_2 = (\mathit{read}() \Rightarrow l_2,\mathit{id}_2,S_2)$: Let $\sigma'_2 = \sigma_1$. Then, it is easy to see that $\sigma_1 {\xrightarrow{o_2}} \sigma'_2 {\xrightarrow{o_1}} \sigma_3$.
    \end{itemize}
\end{itemize}

Based on these two steps, given a linearization $\mathit{lin}$ and a sequence $\mathit{lin}' \neq \mathit{lin}$ which is consistent with visibility relation: We have $\mathit{diff}(\mathit{lin},\mathit{lin}') \neq \emptyset$. Based on the first above property, there exists $(o_1,o_2) \in \mathit{diff}(\mathit{lin},\mathit{lin}')$, such that $o_1$ and $o_2$ are concurrent, and $o_1$ and $o_2$ are adjacent in $\mathit{lin}$. Based on the second above property, $\mathit{lin}'' = \mathit{swap}(\mathit{lin},o_1,o_2)$ is also a linearization. Moreover, it is easy to see that $\mathit{diff}(\mathit{lin},\mathit{lin}') > \mathit{diff}(\mathit{lin}'',\mathit{lin}')$. Therefore, by several times of above process, we finally obtain $\mathit{lin}'$ from $\mathit{lin}$ by swapping pairs of operations, and prove that $\mathit{lin}'$ is also a linearization. This completes the proof of this lemma. $\qed$
\end {proof}



The following lemma states that $\mathit{set}_s$ is a t0-specification.

\begin{lemma}
\label{lemma:set is a t0-specification}
$\mathit{set}_s$ is a t0-specification.
\end{lemma}

\begin {proof}

We prove this lemma similarly as that of Lemma \ref{lemma:or-set is a t0-specification}. We need to prove that, given a linearization $\mathit{lin}$ and $o_1,o_2 \in \mathit{lin}$, such that $o_1$ and $o_2$ are concurrent and adjacent in $\mathit{lin}$, then, $l = \mathit{swap}(\mathit{lin},o_1,o_2)$ is also a linearization.

Let $o_1 = (\ell_1,\mathit{id}_1,S_1)$ and $o_2 = (\ell_2,\mathit{id}_2,S_2)$. Since $o_1$ and $o_2$ are concurrent, we know that $\mathit{id}_1 \notin S_2 \wedge \mathit{id}_2 \notin S_1$. Assume $\mathit{lin} = l_1 \cdot o_1 \cdot o_2 \cdot l_2$. Assume in the abstract state of $\mathit{set}_s$, we have $\sigma_0 {\xrightarrow{l_1}} \sigma_1 {\xrightarrow{o_1}} \sigma_2 {\xrightarrow{o_2}} \sigma_3 {\xrightarrow{l_2}} \sigma_4$, where $\sigma_0$ is the initial state of $\mathit{set}_s$. Then, we need to prove that, there exists $\sigma'_2$, such that $\sigma_1 {\xrightarrow{o_2}} \sigma'_2 {\xrightarrow{o_1}} \sigma_3$. We prove this by consider all the possible cases:

\begin{itemize}
\setlength{\itemsep}{0.5pt}
\item[-] If $o_1 = (\mathit{add}(a_1),\mathit{id}_1,S_1)$ and $o_2 = (\mathit{add}(a_2),\mathit{id}_2,S_2)$: We can see that, if $(a_1,\_) \in \sigma_1$, then $\sigma_2 = \sigma_1$; else, $\sigma_2$ is obtained from $\sigma_1$ by inserting $(a_1,\mathit{true})$. We can also see that, if $(a_2,\_) \in \sigma_2$, then $\sigma_3 = \sigma_2$; else, $\sigma_3$ is obtained from $\sigma_2$ by inserting $(a_2,\mathit{true})$. Let $\sigma'_2$ be: if $(a_2,\_) \in \sigma_1$, then $\sigma'_2 = \sigma_1$; else, $\sigma'_2$ is obtained from $\sigma_1$ by inserting $(a_2,\mathit{true})$. Then, it is easy to see that $\sigma_1 {\xrightarrow{o_2}} \sigma'_2 {\xrightarrow{o_1}} \sigma_3$.

\item[-] If $o_1 = (\mathit{add}(a_1),\mathit{id}_1,S_1)$ and $o_2 = (\mathit{rem}(a_2),\mathit{id}_2,S_2)$: Let $\sigma'_2$ be: if $(a_2,\mathit{false}) \in \sigma_1$, then $\sigma'_2 = \sigma_1$; else, $\sigma'_2$ is obtained from $\sigma_1$ by marking $a_2$ into $\mathit{false}$. Since $\mathit{vis}^{-1}(o_2) \cdot o_2 \in \mathit{set}_s$, we know that $(a_2,\_) \in \sigma_1$. Then, it is easy to see that $\sigma_1 {\xrightarrow{o_2}} \sigma'_2 {\xrightarrow{o_1}} \sigma_3$.

\item[-] If $o_1 = (\mathit{add}(a_1),\mathit{id}_1,S_1)$ and $o_2 = (\mathit{read}() \Rightarrow l_2,\mathit{id}_2,S_2)$: Let $\sigma'_2 = \sigma_1$. Since $\mathit{id}_1 \notin S_2$, it is easy to see that $\sigma_1 {\xrightarrow{o_2}} \sigma'_2 {\xrightarrow{o_1}} \sigma_3$.

\item[-] If $o_1 = (\mathit{rem}(a_1),\mathit{id}_1,S_1)$ and $o_2 = (\mathit{add}(a_2),\mathit{id}_2,S_2)$: Let $\sigma'_2$ be: if $(a_2,\_) \in \sigma_1$, then $\sigma'_2 = \sigma_1$; else, $\sigma'_2$ is obtained from $\sigma_1$ by inserting $(a_2,\mathit{true})$. Since $\mathit{vis}^{-1}(o_1) \cdot o_1 \in \mathit{set}_s$, we know that $(a_1,\_) \in \sigma_1$. Then, it is easy to see that $\sigma_1 {\xrightarrow{o_2}} \sigma'_2 {\xrightarrow{o_1}} \sigma_3$.

\item[-] If $o_1 = (\mathit{rem}(a_1),\mathit{id}_1,S_1)$ and $o_2 = (\mathit{rem}(a_2),\mathit{id}_2,S_2)$: Let $\sigma'_2$ be: if $(a_2,\mathit{false}) \in \sigma_1$, then $\sigma'_2 = \sigma_1$; else, $\sigma'_2$ is obtained from $\sigma_1$ by marking $a_2$ into $\mathit{false}$. Then, it is easy to see that $\sigma_1 {\xrightarrow{o_2}} \sigma'_2 {\xrightarrow{o_1}} \sigma_3$.

\item[-] If $o_1 = (\mathit{rem}(a_1),\mathit{id}_1,S_1)$ and $o_2 = (\mathit{read}() \Rightarrow l_2,\mathit{id}_2,S_2)$: Let $\sigma'_2 = \sigma_1$. Since $\mathit{id}_1 \notin S_2$, it is easy to see that $\sigma_1 {\xrightarrow{o_2}} \sigma'_2 {\xrightarrow{o_1}} \sigma_3$.

\item[-] If $o_1 = (\mathit{read}() \Rightarrow l_1,\mathit{id}_1,S_1)$ and $o_2 = (\mathit{add}(a_1),\mathit{id}_2,S_2)$: Let $\sigma'_2$ be: if $(a_2,\_) \in \sigma_1$, then $\sigma'_2 = \sigma_1$; else, $\sigma'_2$ is obtained from $\sigma_1$ by inserting $(a_2,\mathit{true})$. Since $\mathit{id}_2 \notin S_1$, it is easy to see that $\sigma_1 {\xrightarrow{o_2}} \sigma'_2 {\xrightarrow{o_1}} \sigma_3$.

\item[-] If $o_1 = (\mathit{read}() \Rightarrow l_1,\mathit{id}_1,S_1)$ and $o_2 = (\mathit{rem}(a_1),\mathit{id}_2,S_2)$: Let $\sigma'_2$ be: if $(a_2,\mathit{false}) \in \sigma_1$, then $\sigma'_2 = \sigma_1$; else, $\sigma'_2$ is obtained from $\sigma_1$ by marking $a_2$ into $\mathit{false}$. Since $\mathit{id}_2 \notin S_1$, it is easy to see that $\sigma_1 {\xrightarrow{o_2}} \sigma'_2 {\xrightarrow{o_1}} \sigma_3$.

\item[-] If $o_1 = (\mathit{read}() \Rightarrow l_1,\mathit{id}_1,S_1)$ and $o_2 = (\mathit{read}() \Rightarrow l_2,\mathit{id}_2,S_2)$: Let $\sigma'_2 = \sigma_1$. Then, it is easy to see that $\sigma_1 {\xrightarrow{o_2}} \sigma'_2 {\xrightarrow{o_1}} \sigma_3$.
\end{itemize}

This completes the proof of this lemma. $\qed$
\end {proof}




The following lemma states that $\mathit{counter}_s$ is a t0-specification.

\begin{lemma}
\label{lemma:counter is a t0-specification}
$\mathit{counter}_s$ is a t0-specification.
\end{lemma}


\begin {proof}

We prove this lemma similarly as that of Lemma \ref{lemma:or-set is a t0-specification}. We need to prove that, given a linearization $\mathit{lin}$ and $o_1,o_2 \in \mathit{lin}$, such that $o_1$ and $o_2$ are concurrent and adjacent in $\mathit{lin}$, then, $l = \mathit{swap}(\mathit{lin},o_1,o_2)$ is also a linearization.

Let $o_1 = (\ell_1,\mathit{id}_1,S_1)$ and $o_2 = (\ell_2,\mathit{id}_2,S_2)$. Since $o_1$ and $o_2$ are concurrent, we know that $\mathit{id}_1 \notin S_2 \wedge \mathit{id}_2 \notin S_1$. Assume $\mathit{lin} = l_1 \cdot o_1 \cdot o_2 \cdot l_2$. Assume in the abstract state of $\mathit{counter}_s$, we have $\sigma_0 {\xrightarrow{l_1}} \sigma_1 {\xrightarrow{o_1}} \sigma_2 {\xrightarrow{o_2}} \sigma_3 {\xrightarrow{l_2}} \sigma_4$, where $\sigma_0$ is the initial state of $\mathit{counter}_s$. Then, we need to prove that, there exists $\sigma'_2$, such that $\sigma_1 {\xrightarrow{o_2}} \sigma'_2 {\xrightarrow{o_1}} \sigma_3$. We prove this by consider all the possible cases:

\begin{itemize}
\setlength{\itemsep}{0.5pt}
\item[-] If $o_1 = (\mathit{inc},\mathit{id}_1,S_1)$ and $o_2 = (\mathit{inc},\mathit{id}_2,S_2)$: Assume that $\sigma_1 = k$, then $\sigma_2 = \mathit{k+1}$ and $\sigma_3 = \mathit{k+2}$. Let $\sigma'_2 = \mathit{k+1}$. Then, it is easy to see that $\sigma_1 {\xrightarrow{o_2}} \sigma'_2 {\xrightarrow{o_1}} \sigma_3$.

\item[-] If $o_1 = (\mathit{inc},\mathit{id}_1,S_1)$ and $o_2 = (\mathit{dec},\mathit{id}_2,S_2)$: Assume that $\sigma_1 = k$, and let $\sigma'_2 = \mathit{k-1}$. Then, it is easy to see that $\sigma_1 {\xrightarrow{o_2}} \sigma'_2 {\xrightarrow{o_1}} \sigma_3$.

\item[-] If $o_1 = (\mathit{inc},\mathit{id}_1,S_1)$ and $o_2 = (\mathit{read}() \Rightarrow k_2,\mathit{id}_2,S_2)$: Let $\sigma'_2 = \sigma_1$. Since $\mathit{id}_1 \notin S_2$, it is easy to see that $\sigma_1 {\xrightarrow{o_2}} \sigma'_2 {\xrightarrow{o_1}} \sigma_3$.

\item[-] If $o_1 = (\mathit{dec},\mathit{id}_1,S_1)$ and $o_2 = (\mathit{inc},\mathit{id}_2,S_2)$: Assume that $\sigma_1 = k$, and let $\sigma'_2 = \mathit{k+1}$. Then, it is easy to see that $\sigma_1 {\xrightarrow{o_2}} \sigma'_2 {\xrightarrow{o_1}} \sigma_3$.

\item[-] If $o_1 = (\mathit{dec},\mathit{id}_1,S_1)$ and $o_2 = (\mathit{dec},\mathit{id}_2,S_2)$: Assume that $\sigma_1 = k$, and let $\sigma'_2 = \mathit{k-1}$. Then, it is easy to see that $\sigma_1 {\xrightarrow{o_2}} \sigma'_2 {\xrightarrow{o_1}} \sigma_3$.

\item[-] If $o_1 = (\mathit{dec},\mathit{id}_1,S_1)$ and $o_2 = (\mathit{read}() \Rightarrow k_2,\mathit{id}_2,S_2)$: Let $\sigma'_2 = \sigma_1$. Since $\mathit{id}_1 \notin S_2$, it is easy to see that $\sigma_1 {\xrightarrow{o_2}} \sigma'_2 {\xrightarrow{o_1}} \sigma_3$.

\item[-] If $o_1 = (\mathit{read}() \Rightarrow k_1,\mathit{id}_1,S_1)$ and $o_2 = (\mathit{inc},\mathit{id}_2,S_2)$: Assume that $\sigma_1 = k$, and let $\sigma'_2 = \mathit{k+1}$. Since $\mathit{id}_2 \notin S_1$, it is easy to see that $\sigma_1 {\xrightarrow{o_2}} \sigma'_2 {\xrightarrow{o_1}} \sigma_3$.

\item[-] If $o_1 = (\mathit{read}() \Rightarrow k_1,\mathit{id}_1,S_1)$ and $o_2 = (\mathit{dec},\mathit{id}_2,S_2)$: Assume that $\sigma_1 = k$, and let $\sigma'_2 = \mathit{k-1}$. Since $\mathit{id}_2 \notin S_1$, it is easy to see that $\sigma_1 {\xrightarrow{o_2}} \sigma'_2 {\xrightarrow{o_1}} \sigma_3$.

\item[-] If $o_1 = (\mathit{read}() \Rightarrow k_1,\mathit{id}_1,S_1)$ and $o_2 = (\mathit{read}() \Rightarrow k_2,\mathit{id}_2,S_2)$: Let $\sigma'_2 = \sigma_1$. Then, it is easy to see that $\sigma_1 {\xrightarrow{o_2}} \sigma'_2 {\xrightarrow{o_1}} \sigma_3$.
\end{itemize}

This completes the proof of this lemma. $\qed$
\end {proof}


With Lemma \ref{lemma:or-set is a t0-specification}, Lemma \ref{lemma:set is a t0-specification} and Lemma \ref{lemma:counter is a t0-specification}, we can now prove Lemma \ref{lemma:several t0-specifications}.


\SeveralTZeroSpecifications*

\begin {proof}
This lemma holds obviously from Lemma \ref{lemma:or-set is a t0-specification}, Lemma \ref{lemma:set is a t0-specification} and Lemma \ref{lemma:counter is a t0-specification}. $\qed$
\end {proof}





\subsection{Proofs of Lemma \ref{lemma:several t1-specifications}}
\label{subsec:appendix proofs of Lemma several t1-specifications}


The following lemma states that $\mathit{list}_s^{\mathit{af}}$ is a t1-specification.

\begin{lemma}
\label{lemma:list-af is a t1-specification}
$\mathit{list}_s^{\mathit{af}}$ is a t1-specification.
\end{lemma}

\begin {proof}

Given a distributed linearizable history $h$ and a linearization $\mathit{lin}$ that is a strict time-stamp order candidate, we need to prove that, each strict time-stamp order candidate $\mathit{lin}'$ is a linearization.

We prove this by showing that each such $\mathit{lin}'$ can be obtained from $\mathit{lin}$ by several times of swapping a pair of adjacent elements. Our proof requires the following two properties:

\begin{itemize}
\setlength{\itemsep}{0.5pt}
\item[-] The first property is: Given a linarization $\mathit{lin}$ that is a strict time-stamp order candidate, and a strict time-stamp order candidate $\mathit{lin}'$. If $\mathit{diff}(\mathit{lin},\mathit{lin}') \neq \emptyset$, there exists $(o_1,o_2) \in \mathit{diff}(\mathit{lin},\mathit{lin}')$, such that $o_1$ and $o_2$ are concurrent, $o_1$ and $o_2$ are adjacent in $\mathit{lin}$, and the time-stamp of $o_1$ in $h$ equals that of $o_2$.

    We prove this by contradiction. Assume $\mathit{diff}(\mathit{lin},\mathit{lin}') \neq \emptyset$, and for each $(o_1,o_2) \in \mathit{diff}(\mathit{lin},\mathit{lin}')$, we have that either $o_1$ and $o_2$ are not concurrent, or $o_1$ and $o_2$ are not adjacent in $\mathit{lin}$, or the time-stamp of $o_1$ in $h$ is different from that of $o_2$.

    By the definition of strict time-stamp order candidate, it is easy to see that if $o_1$ and $o_2$ have different time-stamp, then their order is the same between $\mathit{lin}$ and $\mathit{lin}'$. Therefore, we know that the time-stamp of $o_1$ in $h$ equals that of $o_2$.

    Since $\mathit{diff}(\mathit{lin},\mathit{lin}') \neq \emptyset$, let $(o,o')$ be a element of $\mathit{diff}(\mathit{lin},\mathit{lin}')$, and the distance of $o_1$ and $o_2$ is minimal in $\{$ the distance between $o_1$ and $o_2 \vert (o_1,o_2) \in \mathit{diff}(\mathit{lin},\mathit{lin}') \}$. Let us prove that $o$ and $o'$ are adjacent by contradiction: If there exists $o''$ between $o$ and $o'$. Assume that in $\mathit{lin}$, $o$ is before $o''$, and $o''$ is before $o'$. By assumption, the order between $o$ and $o''$, and between $o''$ and $o'$ is the same in $\mathit{lin}$ and in $\mathit{lin}'$. This implies that $o$ is still before $o'$ in $\mathit{lin}'$, which contradicts the fact that $(o,o') \in \mathit{diff}(\mathit{lin},\mathit{lin}')$.

    Since $o$ and $o'$ are adjacent and $(o,o') \in \mathit{diff}(\mathit{lin},\mathit{lin}')$, by assumption we know that $o$ and $o'$ are not concurrent. Or we can say, $(o,o') \in \mathit{vis} \vee \mathit{o',o} \in \mathit{vis}$. This contradicts that both $\mathit{lin}$ and $\mathit{lin}'$ are consistent with visibility relation. This completes the proof of the first step.

\item[-] The second property is: Given a linearization $\mathit{lin}$ that is a strict time-stamp order candidate, and $o_1,o_2 \in \mathit{lin}$, such that $o_1$ and $o_2$ are concurrent and adjacent in $\mathit{lin}$, and $o_1$ and $o_2$ have the same time-stamp in $h$. Then, $l = \mathit{swap}(\mathit{lin},o_1,o_2)$ is also a linearization and is also a strict time-stamp order candidate. It is obvious that $l$ is still a strict time-stamp order candidate.

    Let $o_1 = (\ell_1,\mathit{id}_1,S_1)$ and $o_2 = (\ell_2,\mathit{id}_2,S_2)$. Since $o_1$ and $o_2$ are concurrent, we know that $\mathit{id}_1 \notin S_2 \wedge \mathit{id}_2 \notin S_1$. Assume $\mathit{lin} = l_1 \cdot o_1 \cdot o_2 \cdot l_2$. Assume in the abstract state of $\mathit{list}_s^{\mathit{af}}$, we have $\sigma_0 {\xrightarrow{l_1}} \sigma_1 {\xrightarrow{o_1}} \sigma_2 {\xrightarrow{o_2}} \sigma_3 {\xrightarrow{l_2}} \sigma_4$, where $\sigma_0$ is the initial state of $\mathit{OR}$-$\mathit{set}_s$. Then, we need to prove that, there exists $\sigma'_2$, such that $\sigma_1 {\xrightarrow{o_2}} \sigma'_2 {\xrightarrow{o_1}} \sigma_3$. We prove this by consider all the possible cases:

    \begin{itemize}
    \setlength{\itemsep}{0.5pt}
    \item[-] If $o_1 = (\mathit{add}(a_1,b_1),\mathit{id}_1,S_1)$ and $o_2 = (\_,\mathit{id}_2,S_2)$: This case is impossible. We can see that the time-stamp of $a$ is larger than operations in $S_1$, and thus, the time-stamp of $o_1$ is the time-stamp of $a$. Since $\mathit{id}_1 \notin S_2$, we know that the time-stamp of $o_2$ is different from that of $o_1$, contradicts the assumption that $o_1$ and $o_2$ have same time-stamp.

    \item[-] If $o_1 = (\_,\mathit{id}_1,S_1)$ and $o_2 = (\mathit{add}(a_2,b_2),\mathit{id}_2,S_2)$: Similarly, we can prove that this case is impossible.

    \item[-] If $o_1 = (\mathit{rem}(a_1),\mathit{id}_1,S_1)$ and $o_2 = (\mathit{rem}(a_2),\mathit{id}_2,S_2)$: Let $\sigma'_2$ be obtained from $\sigma_1$ by marking $a_2$ into $\mathit{false}$. Then, it is easy to see that $\sigma_1 {\xrightarrow{o_2}} \sigma'_2 {\xrightarrow{o_1}} \sigma_3$.

    \item[-] If $o_1 = (\mathit{rem}(a_1),\mathit{id}_1,S_1)$ and $o_2 = (\mathit{read}() \Rightarrow \mathit{list}_1,\mathit{id}_2,S_2)$: Let $\sigma'_2 = \sigma_1$. Since $\mathit{id}_1 \notin S_2$, it is easy to see that $\sigma_1 {\xrightarrow{o_2}} \sigma'_2 {\xrightarrow{o_1}} \sigma_3$.

    \item[-] If $o_1 = (\mathit{read}() \Rightarrow \mathit{list}_1,\mathit{id}_1,S_1)$ and $o_2 = (\mathit{read}() \Rightarrow \mathit{list}_2,\mathit{id}_2,S_2)$: Let $\sigma'_2 = \sigma_1$. Then, it is easy to see that $\sigma_1 {\xrightarrow{o_2}} \sigma'_2 {\xrightarrow{o_1}} \sigma_3$.
    \end{itemize}
\end{itemize}

Based on these two steps, given a linearization $\mathit{lin}$ that is a strict time-stamp order candidate, and a sequence $\mathit{lin}' \neq \mathit{lin}$ that is a strict time-stamp order candidate: We have $\mathit{diff}(\mathit{lin},\mathit{lin}') \neq \emptyset$. Based on the first above property, there exists $(o_1,o_2) \in \mathit{diff}(\mathit{lin},\mathit{lin}')$, such that $o_1$ and $o_2$ are concurrent, and $o_1$ and $o_2$ are adjacent in $\mathit{lin}$, and $o_1$ and $o_2$ have a same time-stamp. Based on the second above property, $\mathit{lin}'' = \mathit{swap}(\mathit{lin},o_1,o_2)$ is also a linearization, and is a strict time-stamp order candidate. Moreover, it is easy to see that $\mathit{diff}(\mathit{lin},\mathit{lin}') > \mathit{diff}(\mathit{lin}'',\mathit{lin}')$. Therefore, by several times of above process, we finally obtain $\mathit{lin}'$ from $\mathit{lin}$ by swapping pairs of operations, and prove that $\mathit{lin}'$ is also a linearization, and is a strict time-stamp order candidate. This completes the proof of this lemma. $\qed$
\end {proof}


The following lemma states that $\mathit{reg}_s$ is a t1-specification.

\begin{lemma}
\label{lemma:reg is a t1-specification}
$\mathit{reg}_s$ is a t1-specification.
\end{lemma}

\begin {proof}

We prove this lemma similarly as that of Lemma \ref{lemma:list-af is a t1-specification}. We need to prove that, given a linearization $\mathit{lin}$ that is a strict time-stamp order candidate, and $o_1,o_2 \in \mathit{lin}$, such that $o_1$ and $o_2$ are concurrent and adjacent in $\mathit{lin}$, and $o_1$ and $o_2$ have the same time-stamp in $h$. Then, $l = \mathit{swap}(\mathit{lin},o_1,o_2)$ is also a linearization and is also a strict time-stamp order candidate. It is obvious that $l$ is still a strict time-stamp order candidate.

Let $o_1 = (\ell_1,\mathit{id}_1,S_1)$ and $o_2 = (\ell_2,\mathit{id}_2,S_2)$. Since $o_1$ and $o_2$ are concurrent, we know that $\mathit{id}_1 \notin S_2 \wedge \mathit{id}_2 \notin S_1$. Assume $\mathit{lin} = l_1 \cdot o_1 \cdot o_2 \cdot l_2$. Assume in the abstract state of $\mathit{reg}_s$, we have $\sigma_0 {\xrightarrow{l_1}} \sigma_1 {\xrightarrow{o_1}} \sigma_2 {\xrightarrow{o_2}} \sigma_3 {\xrightarrow{l_2}} \sigma_4$, where $\sigma_0$ is the initial state of $\mathit{OR}$-$\mathit{set}_s$. Then, we need to prove that, there exists $\sigma'_2$, such that $\sigma_1 {\xrightarrow{o_2}} \sigma'_2 {\xrightarrow{o_1}} \sigma_3$. We prove this by consider all the possible cases:


\begin{itemize}
\setlength{\itemsep}{0.5pt}
\item[-] If $o_1 = (\mathit{write}(a_1),\mathit{id}_1,S_1)$ and $o_2 = (\_,\mathit{id}_2,S_2)$: This case is impossible. We can see that the time-stamp of $a$ is larger than operations in $S_1$, and thus, the time-stamp of $o_1$ is the time-stamp of $a$. Since $\mathit{id}_1 \notin S_2$, we know that the time-stamp of $o_2$ is different from that of $o_1$, contradicts the assumption that $o_1$ and $o_2$ have same time-stamp.

\item[-] If $o_1 = (\_,\mathit{id}_1,S_1)$ and $o_2 = (\mathit{write}(a_2),\mathit{id}_2,S_2)$: Similarly, we can prove that this case is impossible.

\item[-] If $o_1 = (\mathit{read}() \Rightarrow a_1,\mathit{id}_1,S_1)$ and $o_2 = (\mathit{read}() \Rightarrow a_2,\mathit{id}_2,S_2)$: Let $\sigma'_2 = \sigma_1$. Then, it is easy to see that $\sigma_1 {\xrightarrow{o_2}} \sigma'_2 {\xrightarrow{o_1}} \sigma_3$.
\end{itemize}
This completes the proof of this lemma. $\qed$
\end {proof}


With Lemma \ref{lemma:list-af is a t1-specification} and Lemma \ref{lemma:reg is a t1-specification}, we can now prove Lemma \ref{lemma:several t1-specifications}.

\SeveralTOneSpecifications*

\begin {proof}
This lemma holds obviously from Lemma \ref{lemma:list-af is a t1-specification} and Lemma \ref{lemma:reg is a t1-specification}. $\qed$
\end {proof}









\subsection{Proof of Lemma \ref{lemma:several t0-specifications can be composed}}
\label{subsec:appendix proofs of lemma several t0-specifications can be composed}

\composingTZero*
\begin {proof}
Assume that $h = (\mathit{Op},\mathit{ro},\mathit{vis})$. We need to prove that, if $h \uparrow_{\mathit{obj}}$ is distributed linearizable for each object $\mathit{obj}$ of $h$, then $h$ is distributed linearizable.

We construct a linearization $\mathit{lin}$ of $h$ in a process as follows:

\begin{itemize}
\setlength{\itemsep}{0.5pt}
\item[-] Initially a set $\mathit{Op}' = \mathit{Op}$ and $\mathit{lin} = \epsilon$.

\item[-] We begin a loop as follows: In each round of the loop, we choose an operation $o$ that is minimal w.r.t $\mathit{vis}$ in $\mathit{Op}'$, let $\mathit{Op}' = \mathit{Op}' \setminus \{ o \}$, and let $\mathit{lin} = \mathit{lin} \cdot o$.
\end{itemize}

If this process terminates with $\mathit{Op}' = \emptyset$: Then it is easy to see that $\mathit{lin}$ is consistent with $\mathit{vis}$, and thus, for each object $\mathit{obj}$, it is easy to see that $\mathit{lin} \uparrow_{\mathit{obj}}$ is consistent with $\mathit{vis} \uparrow_{\mathit{obj}}$. By the definition of t0-specifications, we know that, for each object $\mathit{obj}$, $\mathit{lin} \uparrow_{\mathit{obj}}$ is a linearization of $h \uparrow_{\mathit{obj}}$. Therefore, $h$ is distributed linearizable.

Let us prove that this process terminates with $\mathit{Op}' = \emptyset$ by contradiction: Assume this process terminates with $\mathit{Op}' \neq \emptyset$, then it is easy to see that $\mathit{vis}^*$ has cycle, which contradicts the assumption that $\mathit{vis}^*$ is acyclic. Therefore, this process terminates with $\mathit{Op}' = \emptyset$. $\qed$
\end {proof}





\subsection{Proof of Lemma \ref{lemma:several t0-specifications and one t1-specification can be composed}}
\label{subsec:appendix proofs of lemma several t0-specifications and one t1-specification can be composed}


\composingTZeroAndOneTOne*
\begin {proof}
Assume that $h = (\mathit{Op},\mathit{ro},\mathit{vis})$. Let $\mathit{obj}_1$ be the only object that uses t1-specification, and let $\mathit{objs}_0$ be the set of other objects. We need to prove that, if $h \uparrow_{\mathit{obj}}$ is distributed linearizable for each object $\mathit{obj}$ of $h$, then $h$ is distributed linearizable.

We construct a linearization $\mathit{lin}$ of $h$ in a process as follows:

\begin{itemize}
\setlength{\itemsep}{0.5pt}
\item[-] Initially a set $\mathit{Op}' = \mathit{Op}$ and $\mathit{lin} = \epsilon$.

\item[-] We begin a loop as follows: in each round of the loop, we choose an operation $o$ shown below, and then let $\mathit{Op}' = \mathit{Op}' \setminus \{ o \}$, and let $\mathit{lin} = \mathit{lin} \cdot o$.

    \begin{itemize}
    \setlength{\itemsep}{0.5pt}
    \item[-] either $o$ is of an operation of $\mathit{objs}_0$ and is minimal w.r.t $\mathit{vis}$ in $\mathit{Op}'$,

    \item[-] or $o$ is of an operation of $\mathit{obj}_1$, is minimal w.r.t $\mathit{vis}$ in $\mathit{Op}'$, and has the minimal time-stamp among operations of $\mathit{obj}_1$ in $\mathit{Op}'$.
    \end{itemize}
\end{itemize}

If this process terminates with $\mathit{Op}' = \emptyset$: Then it is easy to see that $\mathit{lin}$ is consistent with $\mathit{vis}$, and thus, for each object $\mathit{obj}$, it is easy to see that $\mathit{lin} \uparrow_{\mathit{obj}}$ is consistent with $\mathit{vis} \uparrow_{\mathit{obj}}$. It is also easy to see that for operation of $\mathit{obj}_1$, $\mathit{lin}$ is consistent with time-stamp. By the definition of t0-specifications, we know that, for each object $\mathit{obj} \in \mathit{objs}$, $\mathit{lin} \uparrow_{\mathit{obj}}$ is a linearization of $h \uparrow_{\mathit{obj}}$. By the definition of t1-specifications, we know that, $\mathit{lin} \uparrow_{\mathit{obj}_1}$ is a linearization of $h \uparrow_{\mathit{obj}_1}$. Therefore, $h$ is distributed linearizable.

Let us prove that this process terminates with $\mathit{Op}' = \emptyset$ by contradiction: Assume this process terminates with $\mathit{Op}' \neq \emptyset$. Let set $S_1 = \{ o' \vert o'$ is minimal w.r.t $\mathit{vis}$ in $\mathit{Op}'$ $\}$. Then, we can see that, for each operation $o \in S_1$, $o$ is of object $\mathit{obj}_1$, and $o$ does not have minimal time-stamps among operations of $\mathit{obj}_1$ in $\mathit{Op}'$. Let $o_0$ be the operation that is of object $\mathit{obj}_1$ and has minimal time-stamp among operations of $\mathit{obj}_1$ in $\mathit{Op}'$. It is obvious that $o_0 \notin S_1$. Therefore, there exists operations $o_1,\ldots,o_k$, such that $o_1 \in S_1$, $o_1$ is of object $\mathit{obj}_1$, $(o_1,o_2),\ldots,(o_k,o_0) \in \mathit{vis}$. Since the visibility is transitive, we have that $(o_1,o_0) \in \mathit{vis}$. We already know that the time-stamp of $o_0$ is less than that of $o_1$. This contradicts the assumption that time-stamp is consistent with visiblity. Therefore, this process terminates with $\mathit{Op}' = \emptyset$. $\qed$

%Let us prove that this process terminates with $\mathit{Op}' = \emptyset$ by contradiction: Assume this process terminates with $\mathit{Op}' \neq \emptyset$. Let set $S_1 = \{ o' \vert o'$ is minimal w.r.t $\mathit{vis}$ in $\mathit{Op}'$ $\}$. Then, we can see that, for each operation $o \in S_1$, $o$ is of object $\mathit{obj}_1$, and $o$ does not have minimal time-stamps among operations of $\mathit{obj}_1$ in $O'$. Thus, let $o_1 \in S_1$ be the operation that has minimal time-stamp among operations in $S_1$. We can see that there exists $o_2 \in O'$, such that $o_2$ is of object $\mathit{obj}_1$, $o_2 \notin S_1$, and the time-stamp of $o_2$ is less than that of $o_1$. Since $o_2 \notin S_1$, we can see that there exists operation $o_3 \in S_1$, such that $(o_3,o_2) \in \mathit{vis}$. We can see that the time-stamp of $o_1$ is less than that of $o_3$, and thus, the time-stamp of $o_2$ is less than that of $o_3$. Therefore, we found that the time-stamp is not consistent with visibility order for $o_2$ and $o_3$, which contradicts the assumption that time-stamp is consistent with visiblity. $\qed$
\end {proof}






\subsection{Proof of Lemma \ref{lemma:several t0-specifications and several t1-specification can be composed}}
\label{subsec:appendix proofs of lemma several t0-specifications and several t1-specification can be composed}


\composingTZeroAndTOne*
\begin {proof}
Assume that $h = (\mathit{Op},\mathit{ro},\mathit{vis})$. Let $\mathit{objs}_0$ be the set of objects that use t0-specifications in $h$, and let $\mathit{objs}_1$ be the set of objects that use t1-specifications in $h$. We need to prove that, if $h \uparrow_{\mathit{obj}}$ is distributed linearizable for each object $\mathit{obj}$ of $h$, then $h$ is distributed linearizable.

We construct a linearization $\mathit{lin}$ of $h$ in a process as follows:

\begin{itemize}
\setlength{\itemsep}{0.5pt}
\item[-] Initially a set $\mathit{Op}' = \mathit{Op}$ and $\mathit{lin} = \epsilon$.

\item[-] We begin a loop as follows: in each round of the loop, we choose an operation $o$ shown below, and then let $\mathit{Op}' = \mathit{Op}' \setminus \{ o \}$, and let $\mathit{lin} = \mathit{lin} \cdot o$.

    \begin{itemize}
    \setlength{\itemsep}{0.5pt}
    \item[-] either $o$ is of an operation of objects in $\mathit{objs}_0$ and is minimal w.r.t $\mathit{vis}$ in $\mathit{Op}'$,

    \item[-] or $o$ is of an operation of object $\mathit{obj}_1 \in \mathit{objs}_1$, is minimal w.r.t $\mathit{vis}$ in $\mathit{Op}'$, and has the minimal time-stamp among operations of $\mathit{obj}_1$ in $\mathit{Op}'$.
    \end{itemize}
\end{itemize}

If this process terminates with $\mathit{Op}' = \emptyset$: Then it is easy to see that $\mathit{lin}$ is consistent with $\mathit{vis}$, and thus, for each object $\mathit{obj}$, it is easy to see that $\mathit{lin} \uparrow_{\mathit{obj}}$ is consistent with $\mathit{vis} \uparrow_{\mathit{obj}}$. It is also easy to see that for each object $\mathit{ojb}_1 \in \mathit{objs}_1$, $\mathit{lin}$ is consistent with time-stamp of $\mathit{obj}_1$. By the definition of t0-specifications, we know that, for each object $\mathit{obj} \in \mathit{objs}_0$, $\mathit{lin} \uparrow_{\mathit{obj}}$ is a linearization of $h \uparrow_{\mathit{obj}}$. By the definition of t1-specifications, we know that, for each object $\mathit{obj}_1 \in \mathit{objs}_1$, $\mathit{lin} \uparrow_{\mathit{obj}_1}$ is a linearization of $h \uparrow_{\mathit{obj}_1}$. Therefore, $h$ is distributed linearizable.

Let us prove that this process terminates with $\mathit{Op}' = \emptyset$ by contradiction: Assume this process terminates with $\mathit{Op}' \neq \emptyset$. Let set $S_1 = \{ o' \vert o'$ is minimal w.r.t $\mathit{vis}$ in $\mathit{Op}'$ $\}$. Then, we can see that, for each operation $o \in S_1$, there exists a object $\mathit{obj}_1 \in \mathit{objs}_1$, such that $o$ is of $\mathit{obj}_1$, and $o$ does not have minimal time-stamps among operations of $\mathit{obj}_1$ in $\mathit{Op}'$.

Let $S_2 = \{ o \vert \exists \mathit{obj}_1 \in \mathit{objs}_1, o$ is of object $\mathit{obj}_1, o$ has minimal time-stamp among operations of $\mathit{obj}_1$ in $\mathit{Op}' \}$. It is easy to see that $\forall o \in S_2$, $o \notin S_1$.

Thus, it is easy to see that, for each operation $o' \in S_2$, there exists an operation $o \in S_1$ and operations $o'_1,\ldots,o'_k$, such that $(o,o'_1),(o'_1,o'_2),\ldots,(o'_k,o') \in \mathit{vis}$. Since the visibility relation is transitive, we have that $(o,o') \in \mathit{vis}$.

Let $S_3 = \{ (o,o') \vert o \in S_1, o' \in S_2, \exists o'_1,\ldots,o'_k, (o,o'_1),(o'_1,o'_2),\ldots,(o'_k,o') \in \mathit{vis} \}$. Let $S_4 = \{ (\mathit{obj},\mathit{obj}') \vert \exists (o,o') \in S_3$, $o$ is of object $\mathit{obj}$, $o'$ is of object $\mathit{obj}' \}$.

Let us prove that there is a cycle in $S_4$ by contradiction. Given $(\mathit{obj}_2,\mathit{obj}_1) \in S_4$, we know that there is a operation of object of $\mathit{obj}_2$ in $S_1$, and thus, there must exists a operation of object of $\mathit{obj}_2$ in $S_2$. By definition of $S_2$, it is easy to see that there exists $\mathit{obj}_3$, such that $(\mathit{obj}_3,\mathit{obj}_2) \in S_4$. Since $S_4$ has no cycle, we applying this process and finally terminate with $(\mathit{obj}_k,\mathit{obj}_{\mathit{k-1}}),\ldots,(\mathit{obj}_2,\mathit{obj}_1) \in S_4$ and could not found any $\mathit{obj}'$ to make $(\mathit{obj}',\mathit{obk}_k) \in S_4$. However, this implies that there is a operation of $\mathit{obj}_k$ that has minimal time-stamp among operations of $\mathit{obj}_k$ in $\mathit{Op}'$, and is in $S_1$. This contradicts our conclusion that $\forall o \in S_2$, $o \notin S_1$. Therefore, this is a cycle in $S_4$.

Let the cycle in $S_4$ be $(\mathit{obj}_1,\mathit{obj}_k),(\mathit{obj}_k,\mathit{obj}_{\mathit{k-1}}),\ldots,(\mathit{obj}_2,\mathit{obj}_1)$. Then, there exists operations $o^{0}_{\mathit{o1}}, o^{1}_{\mathit{o1}},\ldots, o^{0}_{\mathit{ok}}, o^{1}_{\mathit{ok}}$, such that

\begin{itemize}
\setlength{\itemsep}{0.5pt}
\item[-] $o^{0}_{\mathit{o1}}, o^{1}_{\mathit{o1}}$ is of object $\mathit{obj}_1$, $\ldots$, $o^{0}_{\mathit{ok}}, o^{1}_{\mathit{ok}}$ is of object $\mathit{obj}_k$.

\item[-] $(o^{1}_{\mathit{o1}},o^{0}_{\mathit{ok}}), (o^{1}_{\mathit{ok}},o^{0}_{\mathit{ok-1}})$, $\ldots$, $(o^{1}_{\mathit{o2}},o^{0}_{\mathit{o1}}) \in S_3$.
\end{itemize}

Thus, it is easy to see $(o^{1}_{\mathit{o1}},o^{0}_{\mathit{ok}}), (o^{1}_{\mathit{ok}},$ $o^{0}_{\mathit{ok-1}})$, $\ldots$, $(o^{1}_{\mathit{o2}},o^{0}_{\mathit{o1}}) \in \mathit{vis}$. By definition of $S_2$, we can see that $\mathit{ts}(o^{0}_{\mathit{o1}}) < \mathit{ts}(o^{1}_{\mathit{o1}}), \ldots, \mathit{ts}(o^{0}_{\mathit{ok}}) < \mathit{ts}(o^{1}_{\mathit{ok}})$. This contradicts the definition of causal-time-stamp. Therefore, this process terminates with $\mathit{Op}' = \emptyset$. $\qed$
\end {proof}










\section{For State-based CRDT}
\label{sec:for state-based CRDT}

\begin{example}[List with add-between interface]
\label{definition:sequential specification of list with add-after interface}
Such kind of list is similar as list with add-after interface. One difference is the $\mathit{add}$ method: $\mathit{add}(b,a,c)$ inserts item $b$ into the list at some nondeterministic position between position of $a$ and position of $c$. The other difference is that, we assume that the initial value of list is $(\circ_1,\mathit{true}) \cdot (\circ_2,\mathit{true})$ and these two nodes can not be removed. The sequential specification $\mathit{list}_s^{\mathit{ab}}$ of list is given as follows: Here $\mathit{ab}$ represents add-between. When the context is clear, in $\mathit{read}$ operation, we will omit $\circ_1$ and $\circ_2$.
\begin{itemize}
\setlength{\itemsep}{0.5pt}
\item[-] $\{ \mathit{state} = (a_1,f_1) \cdot \ldots \cdot (a_n,f_n) \wedge k < m < l \wedge b \notin \{ a_1, \ldots, a_n \} \}$ $add(b,a_k,a_l)$ $\{ \mathit{state} = (a_1,f_1) \cdot \ldots \cdot (a_m,f_m) \cdot (b,\mathit{true}) \cdot (a_{m+1},f_{m+1}) \cdot \ldots \cdot (a_n,f_n) \}$. Here the chosen of $m$ is deterministic.
\item[-] $\{ \mathit{state} = (a_1,f_1) \cdot \ldots \cdot (a_n,f_n) \wedge S = \{ a \vert (a,\mathit{true}) \in \mathit{state} \} \wedge l = a_1 \cdot \ldots \cdot a_n \uparrow_{S} \}$ $(read() \Rightarrow l)$ $\{ \mathit{state} = (a_1,f_1) \cdot \ldots \cdot (a_n,f_n) \}$.
\end{itemize}
\end{example}










Given a object $\mathit{obj}$ of a state-based CRDT with $\Sigma$ be the set of local states, we define its semantics as a set of executions generated from an LTS $\llbracket \mathit{obj} \rrbracket_s = (\mathit{Config},\mathit{config}_0,\Sigma',\rightarrow)$ as in \figurename~\ref{fig:the semantics of a state-based CRDT object}.

\begin{figure}[ht]
$\mathit{RState} = \mathbb{R} \rightarrow \Sigma$

$\mathit{TState} = \mathbb{MID} \times \mathbb{MSG} \times \mathbb{R}$.

$\mathit{Config} = \mathit{RState} \times \mathit{TState}$, $\mathit{config}_0 \in \mathit{Config}$.

$\Sigma' = \mathit{do}(\mathbb{M} \times \mathbb{D} \times \mathbb{D} \times \mathbb{R}) \cup \mathit{send}(\mathbb{MID} \times \mathbb{R}) \cup \mathit{receive}(\mathbb{MID} \times \mathbb{R})$

\[
\begin{array}{l c}
\bigfrac{ R(r) = \sigma, r.\mathit{do}(\sigma,m,a) = (\sigma',b) }
{ (R,T) {\xrightarrow{\mathit{do}(m,a,b,r)}} (R[r:\sigma'],T) }
\end{array}
\]


\[
\begin{array}{l c}
\bigfrac{ R(r) = \sigma, \mathit{unique}(\mathit{mid}) }
{ (R,T) {\xrightarrow{\mathit{send}(\mathit{mid},r)}} (R,T \cup \{ (\mathit{mid},\sigma,r) \}) }
\end{array}
\]


\[
\begin{array}{l c}
\bigfrac{ R(r) = \sigma, r.\mathit{receive}(\sigma,\sigma') = \sigma'',(\mathit{mid},\sigma',r') \in T, r \neq r'}
{ (R,T) {\xrightarrow{\mathit{receive}(\mathit{mid},r)}} (R[r:\sigma''],T) }
\end{array}
\]
\caption{The definition of semantics of $\llbracket \mathit{obj} \rrbracket_s$}
\label{fig:the semantics of a state-based CRDT object}
\end{figure}

A configuration $(R,T)$ is a snapshot of distributed system and contains two parts: $R$ gives the local state of each replica, and $T$ gives the set of messages that has been generated. Let $\mathbb{MID}$ be the set of message identifiers of message content. A message is a tuple $(\mathit{mid},\mathit{msg},r)$, where $\mathit{mid} \in \mathbb{MID}$ is the identifier, $\mathit{msg} \in \mathbb{MSG}$ is the message content, and $r$ is the original replica of message. $\mathit{config}_0$ is the initial configuration, which maps each replica into the initial local state, and has no message inside. Since $\mathit{obj}$ is a state-based CRDT, each message content is chosen from $\Sigma$.

Each element of $\Sigma'$ is called an action. $\rightarrow \in \mathit{Config} \times \Sigma' \times \mathit{Config}$ is the transition relation and describe a single step of distributed systems. The first rule in \figurename~\ref{fig:the semantics of a state-based CRDT object} describes replica $r$ performs a operation $m(a) \Rightarrow b$ and works locally. The second rule describes that a replica $r$ may nondeterministically decide to send a message with its local state as message content. Here $\mathit{unique}$ is a function that ensures $\mathit{mid}$ be a fresh message identifier. The third rule describes delivery of a message to a replica $r$ other than its origin replica $r'$.

A sequence $l$ of actions is an execution of $\llbracket \mathit{obj} \rrbracket_s = (\mathit{Config},\mathit{config}_0,\Sigma',\rightarrow)$, if there exists $(R,T) \in \mathit{Config}$, such that $\mathit{config}_0 {\xrightarrow{ l }} (R,T)$. The semantics of $\mathit{obj}$ is defined as the set of executions of $\llbracket \mathit{obj} \rrbracket_s$. Given an execution, when the context is clear, we can associate a unique operation identifier to each action. Or we can say, it is safe to assume each action is in the form of either $\mathit{do}(i,m,a,b,r)$, or $\mathit{send}(i,\mathit{mid},r)$, or $\mathit{receive}(i,\mathit{mid},r)$, where $i \in \mathbb{OID}$ is a unique operation identifier.








Given an execution $l = \alpha_1 \cdot \ldots \cdot \alpha_k$ of $\llbracket \mathit{obj} \rrbracket_s$ of state-based CRDT $\mathit{obj}$, we can obtain a corresponding history $\mathit{history}(l) = (\mathit{Op},\mathit{ro},\mathit{vis})$, such that

\begin{itemize}
\setlength{\itemsep}{0.5pt}
\item[-] Each operation in $\mathit{Op}$ is a tuple $(\ell,i,\mathit{obj})$, such that $i$ is the operation identifier of a $\mathit{do}(m,a,b,r)$ action of $l$.

\item[-] $(o_1,o_2) \in \mathit{ro}$, if they are of same replica, and the index of $o_1$ in $h$ is before that of $o_2$.

\item[-] Let us defines a delivery relation $\mathit{del} \subseteq \mathbb{OP} \times \mathbb{OP}$ as follows: $(o_1,o_2) \in \mathit{del}$, if: $o_1$ and $o_2$ are of different replicas, there exists a $\mathit{send}(\mathit{mid},r)$ action and a $\mathit{receive}(\mathit{mid},r')$ action, $o_1$ and $\mathit{send}(\mathit{mid},r)$ happen on a same replica and $o_1$ happens earlier, $\mathit{receive}(\mathit{mid},r)$ and $o_2$ happen on a same replica and $\mathit{receive}(\mathit{mid},r)$ happens earlier.

\item[-] $\mathit{vis} = (\mathit{ro}+\mathit{del})^*$.
\end{itemize}

Intuitively, each local state can be considered as the consequence of all updates it receives. Since state-based CRDT sends the modified local state as message, the visibility relation is then the transitive closure of replica order and message delivery relation. Let $\mathit{history}(\llbracket \mathit{obj} \rrbracket_s)$ be the set of histories of all executions of $\llbracket \mathit{obj} \rrbracket_s$.






\subsection{Proof Strategy of State-based CRDT}
\label{subsec:proof strategy of operation-based CRDT}

Given a state-based CRDT object $\mathit{obj}$ and a sequential specification $\mathit{spec}$, we need to construct a invariant $\mathit{inv}(\mathit{config},h,\mathit{lin},\mathit{del},\mathit{map})$, where

\begin{itemize}
\setlength{\itemsep}{0.5pt}
\item[-] $\mathit{config}$ is a configuration of $\llbracket \mathit{obj} \rrbracket_s$.

\item[-] $h$ is a history.

\item[-] $h$ is distributed linearizable w.r.t $\mathit{spec}$ and $\mathit{lin}$ is a linearization.

\item[-] $\mathit{del} \subseteq \mathbb{MID} \times \mathbb{R}$ is the message delivery relation.

\item[-] $\mathit{map} \subseteq \mathbb{MID} \times 2^{\mathbb{OID}}$ maps each message $\mathit{mid}$ to a set $S_1$ of operations. Intuitively, $S_1$ is the set of operations whose information are contained in $\mathit{mid}$.
\end{itemize}

$\mathit{inv}(\mathit{config},h,\mathit{lin},\mathit{del},\mathit{map})$ needs to satisfy the following properties:

\begin{itemize}
\setlength{\itemsep}{0.5pt}
\item[-] The visibility of $h$ is transitive.

\item[-] $\mathit{del}$ preserves causal delivery: If $(o_1,o_2) \in \mathit{vis}$ and $(o_2,r) \in \mathit{del}$, then $(o_1,r) \in \mathit{del}$.

\item[-] $\mathit{map}$ preserves causal delivery: Given $o_1,o_3 \in \mathit{map}(\mathit{mid})$, if $\exists o_2$, such that $(o_1,o_2),(o_2,o_3) \in \mathit{vis}$, then $o_2 \in \mathit{map}(\mathit{mid})$.

\item[-] $\mathit{inv}$ holds initially: $\mathit{inv}(\mathit{config}_0,\epsilon,\emptyset,\emptyset,\emptyset)$ holds, where $\mathit{config}_0$ is the initial configuration of $\llbracket \mathit{obj} \rrbracket_s$.

\item[-] $\mathit{inv}$ is a transition invariant:

    \begin{itemize}
    \setlength{\itemsep}{0.5pt}
    \item[-] If $\mathit{inv}(\mathit{config},h,\mathit{lin},\mathit{del},\mathit{map})$ holds and $\mathit{config} {\xrightarrow{\mathit{do}(m,a,b,r)}} \mathit{config}'$, then $\mathit{inv}(\mathit{config}', h \otimes i, \mathit{lin} \cdot i,\mathit{del},\mathit{map})$ holds. Note that here we always put a new operation in the last of linearization.

        Here $i$ is the identifier of the newly-generated $\mathit{do}$ action. Given $h = (\mathit{Op},\mathit{ro},\mathit{vis})$, then, $h \otimes i = (\mathit{Op}',\mathit{ro}',\mathit{vis}')$, where $\mathit{Op}' = \mathit{Op} \cup \{ (m(a) \Rightarrow b,i,\mathit{obj}) \}$, $\mathit{ro}' = \mathit{ro} \cup \{ (j,i) \vert j \in \mathit{Op}, j$ is of replica $r \}$, and $\mathit{vis}' = (\mathit{vis} \cup \{ (j,i) \vert j \in \mathit{Op},(j,r) \in \mathit{del} \} \cup \{ (j,i) \vert j \in \mathit{Op}, j$ is of replica $r \})^*$.

    \item[-] If $\mathit{inv}(\mathit{config},h,\mathit{lin},\mathit{del},\mathit{map})$ holds and $\mathit{config} {\xrightarrow{\mathit{send}(\mathit{mid},r)}} \mathit{config}'$, then $\mathit{inv}(\mathit{config}',h,\mathit{lin},\mathit{del},\mathit{map}')$ holds, where $\mathit{map}' = \mathit{map} \cup (\mathit{mid}, \mathit{vd}(h,\mathit{del},r))$.


    \item[-] If $\mathit{inv}(\mathit{config},h,\mathit{lin},\mathit{del},\mathit{map})$ holds and $\mathit{config} {\xrightarrow{\mathit{receive}(\mathit{mid},r)}} \mathit{config}'$, then $\mathit{inv}(\mathit{config}',h,\mathit{lin},\mathit{del}',\mathit{map})$ holds, where $\mathit{del}' = \mathit{del} \cup \{ (i,r) \vert i \in \mathit{map}(\mathit{mid}) \}$.
    \end{itemize}
\end{itemize}

Here $\mathit{vd}(h,\mathit{del},r) = \{ i \vert (i,j) \in h.\mathit{vis}, j$ is of replica $r \} \cup \{ i \vert (i,r) \in \mathit{del} \}$ is the set of operations that are either to some operation of replica $r$, or has been delivered into replica $r$. An invariant $\mathit{inv}$ satisfies above properties is called invariant of state-based CRDT. The following lemma states that the existence of such invariant implies distributed linearizability.

\begin{lemma}
\label{lemma:invariant of state-based CRDT implies distributed linearizability}
If there exists a invariant $\mathit{inv}$ of state-based CRDT for object $\mathit{obj}$ and sequential specification $\mathit{spec}$, then each history of $\mathit{history}(\llbracket \mathit{obj} \rrbracket_s)$ is distributed linearizable w.r.t $\mathit{spec}$.
\end{lemma}

\begin {proof}
Given an execution $l=\alpha_1 \cdot \ldots \cdot \alpha_n$, let $\mathit{config}_0 {\xrightarrow{\alpha_1}} \mathit{config}_1 \ldots {\xrightarrow{\alpha_n}} \mathit{config}_n$ be the transitions from initial configuration. We need to prove that, for each $1 \leq k \leq n$, we have $\mathit{inv}(\mathit{config}_k,h_k,\mathit{lin}_k,\mathit{del}_k,\mathit{map}_k)$ holds, where $h_k$ is the history of execution $l_k = \alpha_1 \cdot \ldots \cdot \alpha_k$, $\mathit{lin}_k$ is the linearization of $h_k$, $\mathit{del}_k$ records message delivery relation of $l_k$, and $\mathit{map}_k$ records the operations contained in each message in $l_k$.

Since $\mathit{inv}$ holds initially and is a transition invariant, it is easy to prove above requirements by induction on execution. This completes the proof of this lemma. $\qed$
\end {proof}


For many state-based CRDT implementations, $\mathit{inv}((R,T),h,\mathit{lin},\mathit{del},\mathit{map}) = C_1 \wedge C_2$, where

\begin{itemize}
\setlength{\itemsep}{0.5pt}
%\item[-] For each update operation $o$ of $h$, define $\mathit{ds}(o)$ which is a local state. %be the local state of replica $r$ at the time point immediately after $o$ is launched. Here $r$ is the replica of $o$.

\item[-] $C_1: \forall (\mathit{mid},\mathit{msg},\_) \in T$, $\mathit{msg} = \mathit{apply}(\mathit{lin},\mathit{map}(\mathit{mid}))$.

\item[-] $C_2: \forall r$, $R(r) = \mathit{apply}(\mathit{lin},\mathit{vd}(h,\mathit{del},r))$.
\end{itemize}

The function $\mathit{apply}(\mathit{lin},S)$ returns a local state by applying ``virtual messages'' of operations in $S$ according to total order $\mathit{lin}$. Here for each update operation $o$ of $h$, we need to define a local state $\mathit{ds}(o)$, which is the ``virtual messages'' of $o$. Note that state-based CRDT send message randomly, instead of each message for a update operation. This is the reason why we need to manually generate virtual message for each update operation.

To give $\mathit{inv}$, it only remains to give the virtual messages. The virtual message of state-based PN-counter and state-based multi-value register as follows. The proof of them being invariants of state-based CRDT is given in Appendix \ref{subsec:appendix proof of state-based PN-counter} and Appendix \ref{subsec:appendix proof of state-based multi-value register}, respectively.

\begin{example}[virtual messages of state-based PN-counter]
\label{example:virtual messagess of state-based PN-counter}

For each update operation $o$, $\mathit{ds}(o) = (P,N)$, where

\begin{itemize}
\setlength{\itemsep}{0.5pt}
\item[-] $\forall r'$, $P[r'] = \vert \{ o' \vert o'$ is a $\mathit{inc}$ operation of replica $r'$, $o' = o \vee (o',o) \in h.\mathit{vis} \} \vert$.

\item[-] $\forall r'$, $N[r'] = \vert \{ o' \vert o'$ is a $\mathit{dec}$ operation of replica $r'$, $o' = o \vee (o',o) \in h.\mathit{vis} \} \vert$.
\end{itemize}
\end{example}

\begin{example}[virtual messages of state-based Multi-value Register]
\label{example:virtual messages of state-based multi-value register}

For each update operation $o = (\mathit{write}(a),\_,\_)$ of replica $r$, $\mathit{ds}(o) = (a,V)$, where

\begin{itemize}
\setlength{\itemsep}{0.5pt}
\item[-] $\forall r'$, $V[r'] = \vert \{ o' \vert o'$ is a $\mathit{write}$ operation of replica $r'$, $o' = o \vee (o',o) \in h.\mathit{vis} \} \vert$.
\end{itemize}
\end{example}















\subsection{Proof of State-based PN-counter}
\label{subsec:appendix proof of state-based PN-counter}

The following lemma states that each visibility-closed set is a union of operations visible to a set of operations. Its proof is obvious and omitted here.

\begin{lemma}
\label{lemma:a transitive-closed set is a union of visibility of several sets}
Given a set $\mathit{Op}$ of operations and a transitive and acyclic visibility relation $\mathit{vis} \subseteq \mathit{Op} \times \mathit{Op}$, if given a set $S \subseteq \mathit{Op}$, if $S$ satisfies that $\forall o_1,o_2 \in S, o_2 \in S \wedge (o_1,o_2) \in \mathit{vis} \Rightarrow o_1 \in S$, then there exists a set $O \subseteq \mathit{Op}$, such that $S = \cup_{o \in O} \mathit{vis}^{-1}(o)$.
\end{lemma}

The following lemma states that given two operations $o_1,o_2$, for each replica $r$, either the set of operations of replica $r$ visible to $o_1$ is a subset of that of $o_2$, or the set of operations of replica $r$ visible to $o_2$ is a subset of that of $o_1$. Its proof is obvious and omitted here.

\begin{lemma}
\label{lemma:the view of a replica of one operation is contained in another operaiton, or vice versa}
Assume that $\mathit{inv}((R,T),h,\mathit{lin},\mathit{del},\mathit{map})$ holds. Let $S_o^r = \{ o' \vert (o',o) \in \mathit{vis}, o'$ is of replica $r \}$. Then for each operations $o_1$ and $o_2$, and for each replica $r$, $S_{\mathit{o1}}^r \subseteq S_{\mathit{o2}}^r \vee S_{\mathit{o2}}^r \subseteq S_{\mathit{o1}}^r$.
\end{lemma}


Recall that $\mathit{inv} = C_1 \wedge C_2$ with the virtual messages defined as follows: For each update operation $o$, $\mathit{ds}(o) = (P,N)$, where

\begin{itemize}
\setlength{\itemsep}{0.5pt}
\item[-] $\forall r'$, $P[r'] = \vert \{ o' \vert o'$ is a $\mathit{inc}$ operation of replica $r'$, $o' = o \vee (o',o) \in h.\mathit{vis} \} \vert$.

\item[-] $\forall r'$, $N[r'] = \vert \{ o' \vert o'$ is a $\mathit{dec}$ operation of replica $r'$, $o' = o \vee (o',o) \in h.\mathit{vis} \} \vert$.
\end{itemize}

The following lemma states that $\mathit{inv}$ is an invariant of state-based PN-counter.

\begin{lemma}
\label{lemma:inv is an invariant of state-based CRDT for state-based PN-counter}
$\mathit{inv}$ is an invariant of state-based PN-counter.
\end{lemma}

\begin {proof}

It is obvious that $\mathit{inv}(\mathit{config}_0,\epsilon,\emptyset,\emptyset,\emptyset)$ holds.

Let us prove that $\mathit{inv}$ is a transition invariant: assume $\mathit{inv}((R,T),h,\mathit{lin},\mathit{del},\mathit{map})$ holds,

\begin{itemize}
\setlength{\itemsep}{0.5pt}
\item[-] If $(R,T) {\xrightarrow{\mathit{do}(\mathit{inc},r)}} (R',T')$: Then,

    \begin{itemize}
    \setlength{\itemsep}{0.5pt}
    \item[-] It is easy to see that $R' = R[ r: ( R(r).P[r: R(r).P(r)+1 ], R(r).N ) ]$ and $T' = T$.

    \item[-] Let $h' = h \otimes i$, where $i$ is the identifier of the newly-generated $\mathit{inc}$ action.

    \item[-] Let $\mathit{lin}' = \mathit{lin} \cdot (\mathit{inc},i,\mathit{obj})$.

    \item[-] Let $\mathit{del}' = \mathit{del}$ and $\mathit{map}' = \mathit{map}$.
    \end{itemize}

    It is easy to see that $\mathit{lin}'$ is a linearization of $h'$. It is obvious that all other properties hold, except for $C_2$ for replica $r$. Therefore, let us prove that $R'(r) = \mathit{apply}(\mathit{lin}',\mathit{vd}(h',\mathit{del}',r))$.

    Since $R(r) = \mathit{apply}(\mathit{lin},\mathit{vd}(h,\mathit{del},r))$ and $\mathit{lin}' = \mathit{lin} \cdot (\mathit{inc},i,\mathit{obj})$, we know that $\mathit{apply}(\mathit{lin}',\mathit{vd}(h',\mathit{del}',r)) = \mathit{merge}(R(r),\mathit{ds}(i))$. Therefore, we need to prove that $R'(r) = \mathit{merge}(R(r),\mathit{ds}(i))$.

    Since $\mathit{vd}(h,\mathit{del},r)$ satisfies that, $\forall o_1,o_2 \in \mathit{vd}(h,\mathit{del},r), o_2 \in \mathit{vd}(h,\mathit{del},r) \wedge (o_1,o_2) \in \mathit{vis} \Rightarrow o_1 \in \mathit{vd}(h,\mathit{del},r)$, by Lemma \ref{lemma:a transitive-closed set is a union of visibility of several sets}, we know that there exists a set $O$, such that $\mathit{vd}(h,\mathit{del},r) = \cup_{o \in O} \mathit{vis}^{-1}(o)$. By Lemma \ref{lemma:the view of a replica of one operation is contained in another operaiton, or vice versa} and the construction of $\mathit{ds}$, we can see that $R(r) = (P',N')$, where for each replica $r'$, $P'[r'] = \vert \{ j \in \mathit{vd}(h,\mathit{del},r) \uparrow_{\mathit{inc}}$ and $j$ is of replica $r \} \vert$ and $N'[r'] = \vert \{ j \in \mathit{vd}(h,\mathit{del},r) \uparrow_{\mathit{dec}}$ and $j$ is of replica $r \} \vert$.

    We already know that $\mathit{ds}(i) = (P'',N'')$, where for each replica $r'$, $P''[r'] = \vert \{ j \in \mathit{vd}(h',\mathit{del}',r) \uparrow_{\mathit{inc}}$ and $j$ is of replica $r \} \vert$ and $N''[r'] = \vert \{ j \in \mathit{vd}(h',\mathit{del}',r) \uparrow_{\mathit{dec}}$ and $j$ is of replica $r \} \vert$. Then, it is obvious that $\mathit{merge}(R(r),\mathit{ds}(i)) = \mathit{ds}(i)$. It is also easy to see that $\mathit{ds}(i) = (R(r).P[r: R(r).P(r)+1], R(r).N) = R'(r)$. Therefore, $R'(r) = \mathit{merge}(R(r),\mathit{ds}(i))$.

\item[-] If $(R,T) {\xrightarrow{\mathit{do}(\mathit{dec},r)}} (R',T')$: Similarly as that of $(R,T) {\xrightarrow{\mathit{do}(\mathit{inc},r)}} (R',T')$.

\item[-] If $(R,T) {\xrightarrow{\mathit{do}(\mathit{read},k,r)}} (R',T')$: Then,

    \begin{itemize}
    \setlength{\itemsep}{0.5pt}
    \item[-] It is obvious that $R' = R$ and $T' = T$.

    \item[-] Let $h' = h \otimes i$, where $i$ is the identifier of the newly-generated $\mathit{read}$ action.

    \item[-] Let $\mathit{lin}' = \mathit{lin} \cdot ((\mathit{read}() \Rightarrow k,i,\mathit{obj}), \mathit{vd}(h,\mathit{del},r) )$.

    \item[-] Let $\mathit{del}' = \mathit{del}$ and $\mathit{map}' = \mathit{map}$.
    \end{itemize}

    It is easy to see that all other properties hold, except for $h'$ being distributed linearizable w.r.t $\mathit{spec}$ with $\mathit{lin}'$ the linearization. Let us prove that $h'$ is distributed linearizable w.r.t $\mathit{spec}$ and $\mathit{lin}'$ is a linearization. It is easy to see that only operation $i$ need to be checked.

    It is easy to see that $\mathit{vd}(h,\mathit{del},r) = \mathit{vis}^{-1}(i)$. Similarly as the case of $(R,T) {\xrightarrow{\mathit{do}(\mathit{inc},r)}} (R',T')$, we can prove that $R(r) = (P',N')$, where for each replica $r'$, $P'[r'] = \vert \{ j \in \mathit{vd}(h,\mathit{del},r) \uparrow_{\mathit{inc}}$ and $j$ is of replica $r \} \vert = \vert \{ j \in \mathit{vis}^{-1}(i) \uparrow_{\mathit{inc}}$ and $j$ is of replica $r \} \vert$ and $N'[r'] = \vert \{ j \in \mathit{vd}(h,\mathit{del},r) \uparrow_{\mathit{dec}}$ and $j$ is of replica $r \} \vert = \vert \{ j \in \mathit{vis}^{-1}(i) \uparrow_{\mathit{dec}}$ and $j$ is of replica $r \} \vert$. Since $k = \Sigma_{r'} P��[r'] - \Sigma_{r'} N'[r']$, $k$ is obtained by minus the number of all visible $\mathit{dec}$ of $i$ from the number of all visible $\mathit{inc}$ of $i$. Therefore, we can see that $((\mathit{read}() \Rightarrow k,i,\mathit{obj}), \mathit{vd}(h,\mathit{del},r) )$ of $\mathit{lin}'$ is ``correct''. Then, $h'$ is distributed linearizable w.r.t $\mathit{spec}$ and $\mathit{lin}'$ is a linearization.

\item[-] If $(R,T) {\xrightarrow{\mathit{send}(\mathit{mid},r)}} (R',T')$: Then,

    \begin{itemize}
    \setlength{\itemsep}{0.5pt}
    \item[-] It is obvious that $R' = R$. Let $T' = T \cup \{ (\mathit{mid},R(r),r) \}$.

    \item[-] Let $h' = h$.

    \item[-] Let $\mathit{lin}' = \mathit{lin}$.

    \item[-] Let $\mathit{del}' = \mathit{del}$.

    \item[-] Let $\mathit{map}' = \mathit{map} \cup \{ (\mathit{mid},\mathit{vd}(h,\mathit{del},r)) \}$.
    \end{itemize}

    It is easy to see that all other properties hold, except for checking $C_1$ for $\mathit{mid}$. This holds obviously since the message content of message $\mathit{mid}$ is $R(r)$, and we already know that $R(r) = \mathit{apply}(\mathit{lin},\mathit{vd}(h,\mathit{del},r)) = \mathit{apply}(\mathit{lin},\mathit{map}(\mathit{mid}))$.

\item[-] If $(R,T) {\xrightarrow{\mathit{receive}(\mathit{mid},r)}} (R',T')$: Then,

    \begin{itemize}
    \setlength{\itemsep}{0.5pt}
    \item[-] Let $R' = R[ r: \mathit{merge}(R(r),\mathit{msg})]$ where $(\mathit{mid},\mathit{msg},\_) \in T$. It is obvious that $T' = T$.

    \item[-] Let $h' = h$.

    \item[-] Let $\mathit{lin}' = \mathit{lin}$.

    \item[-] Let $\mathit{del}' = \mathit{del} \cup \{ (i,r) \vert i \in \mathit{map}(\mathit{mid}) \}$.

    \item[-] Let $\mathit{map}' = \mathit{map}$.
    \end{itemize}

    It is easy to see that all other properties hold, except for $C_2$ for replica $r$. Therefore, let us prove that $R'(r) = \mathit{apply}(\mathit{lin}',\mathit{vd}(h',\mathit{del}',r))$.

    We already know that $R'(r) = \mathit{merge}(R(r), \mathit{msg})$, $R(r) = \mathit{apply}(\mathit{lin},\mathit{vd}(h,\mathit{del},r))$ and $\mathit{msg} = \mathit{apply}(\mathit{lin},\mathit{map}(\mathit{mid}))$. It is easy to see that $\mathit{vd}(h',\mathit{del}',r) = \mathit{vd}(h,\mathit{del},r) \cup \mathit{map}(\mathit{mid})$. It is easy to prove that, applying messages in any order lead to the same consequence. Therefore, we have $\mathit{merge}(R(r), \mathit{msg}) = \mathit{apply}(\mathit{lin}',\mathit{vd}(h,\mathit{del},r) \cup \mathit{map}(\mathit{mid}))$. Then, we have $R'(r) = \mathit{apply}(\mathit{lin}',\mathit{vd}(h',\mathit{del}',r))$.
\end{itemize}

This completes the proof of this lemma. $\qed$
\end {proof}




\subsection{Proof of State-based Multi-value Register}
\label{subsec:appendix proof of state-based multi-value register}

Recall that $\mathit{inv} = C_1 \wedge C_2$ with the virtual messages defined as follows: For each update operation $o$, $\mathit{ds}(o) = (a,V)$, where

\begin{itemize}
\setlength{\itemsep}{0.5pt}
\item[-] $\forall r'$, $V[r'] = \vert \{ o' \vert o'$ is a $\mathit{write}$ operation of replica $r'$, $o' = o \vee (o',o) \in h.\mathit{vis} \} \vert$.
\end{itemize}

The following lemma states that $\mathit{inv}$ is an invariant of state-based multi-value register.

\begin{lemma}
\label{lemma:inv is an invariant of state-based CRDT for state-based multi-value register}
$\mathit{inv}$ is an invariant of state-based multi-value register.
\end{lemma}

\begin {proof}

It is obvious that $\mathit{inv}(\mathit{config}_0,\epsilon,\emptyset,\emptyset,\emptyset)$ holds.

Let us prove that $\mathit{inv}$ is a transition invariant: assume $\mathit{inv}((R,T),h,\mathit{lin},\mathit{del},\mathit{map})$ holds,

\begin{itemize}
\setlength{\itemsep}{0.5pt}
\item[-] If $(R,T) {\xrightarrow{\mathit{do}(\mathit{write},a,r)}} (R',T')$: Then,

    \begin{itemize}
    \setlength{\itemsep}{0.5pt}
    \item[-] $R' = R[ r: \{ (a,V') \} ], R(r).N)]$ and $T' = T$. Here $\forall r' \neq r, V'[r'] = \mathit{max} \{ V_1(r��) \vert (\_,V_1) \in R(r) \}$, and $V'[r] = \mathit{max} \{ V_1(r) \vert (\_,V_1) \in R(r) \} + 1$.

    \item[-] Let $h' = h \otimes i$, where $i$ is the identifier of the newly-generated $\mathit{inc}$ action.

    \item[-] Let $\mathit{lin}' = \mathit{lin} \cdot (\mathit{inc},i,\mathit{vis}^{-1}(i))$.

    \item[-] Let $\mathit{del}' = \mathit{del}$ and $\mathit{map}' = \mathit{map}$.
    \end{itemize}

    It is easy to see that $\mathit{lin}'$ is a linearization of $h'$. It is obvious that all other properties hold, except for $C_2$ for replica $r$. Therefore, let us prove that $R'(r) = \mathit{apply}(\mathit{lin}',\mathit{vd}(h',\mathit{del}',r))$.

    It is easy to see that $\mathit{vd}(h',\mathit{del}',r) = h'.\mathit{vis}^{-1}(i)$. And then, we need to prove that $(a,V') = \mathit{apply}(\mathit{lin}',h'.\mathit{vis}^{-1}(i))$.

    Recall that $R(r) = \mathit{apply}(\mathit{lin},\mathit{vd}(h,\mathit{del},r))$, from Lemma \ref{lemma:a transitive-closed set is a union of visibility of several sets}, we know that there exists set $O$, such that $\mathit{vd}(h,\mathit{del},r) = \cup_{o \in O} \mathit{vis}^{-1}(o)$. We can prove that, for each $o = \mathit{write}(b)$, $\mathit{apply}(\mathit{lin},\mathit{vis}^{-1}(o)) = (b,V_b)$, where $\forall r' \neq r, V_b[r'] = \vert \{ o' \vert o' \in \mathit{vis}^{-1}(o), o'$ is of replica $r' \} \vert$, and $V_b[r] = \vert \{ o' \vert o' \in \mathit{vis}^{-1}(o), o'$ is of replica $r' \} \vert + 1$.

    It is not hard to prove that the order of merging virtual message is not important, and a virtual message can be applied multiple times. By Lemma \ref{lemma:the view of a replica of one operation is contained in another operaiton, or vice versa}, we can see that $\mathit{apply}(\mathit{lin},\mathit{vd}(h,\mathit{del},r))$ is obtained by merging $\{ o \in O \vert \mathit{apply}(\mathit{lin},\mathit{vis}^{-1}(o)) \}$. Therefore, we can see that $\mathit{apply}(\mathit{lin}',h'.\mathit{vis}^{-1}(i)) = \mathit{apply}(\mathit{lin}',\mathit{vd}(h',\mathit{del}',r))$ is obtained by merging $\{ o \in O \vert \mathit{apply}(\mathit{lin},\mathit{vis}^{-1}(o)) \} \cup \{ \mathit{ds}(i) \}$. By Lemma \ref{lemma:the view of a replica of one operation is contained in another operaiton, or vice versa}, it is not hard to see that $\mathit{apply}(\mathit{lin}',h'.\mathit{vis}^{-1}(i)) = \mathit{ds}(i)$.

    Then, we need to prove that $(a,V') = \mathit{ds}(i)$. This holds since $R(r) = \mathit{apply}(\mathit{lin},\mathit{vd}(h,\mathit{del},r))$ is obtained by merging $\{ o \in O \vert \mathit{apply}(\mathit{lin},\mathit{vis}^{-1}(o)) \}$, Lemma \ref{lemma:the view of a replica of one operation is contained in another operaiton, or vice versa}, and the value of $V'$.

\item[-] If $(R,T) {\xrightarrow{\mathit{do}(\mathit{read},S,r)}} (R',T')$: Then,

    \begin{itemize}
    \setlength{\itemsep}{0.5pt}
    \item[-] It is obvious that $R' = R$ and $T' = T$.

    \item[-] Let $h' = h \otimes i$, where $i$ is the identifier of the newly-generated $\mathit{read}$ action.

    \item[-] Let $\mathit{lin}' = \mathit{lin} \cdot (\mathit{read}() \Rightarrow S,i,\mathit{vd}(h,\mathit{del},r) )$.

    \item[-] Let $\mathit{del}' = \mathit{del}$ and $\mathit{map}' = \mathit{map}$.
    \end{itemize}

    It is easy to see that all other properties hold, except for $h'$ being distributed linearizable w.r.t $\mathit{spec}$ with $\mathit{lin}'$ the linearization. Let us prove that $h'$ is distributed linearizable w.r.t $\mathit{spec}$ and $\mathit{lin}'$ is a linearization. It is easy to see that only operation $i$ need to be checked.

    It is easy to see that $\mathit{vd}(h,\mathit{del},r) = h'.\mathit{vis}^{-1}(i)$. Similarly as the case of $(R,T) {\xrightarrow{\mathit{do}(\mathit{write},a,r)}} (R',T')$, we can prove that there exists a set $O$, such that $R(r) = \mathit{apply}(\mathit{lin},\mathit{vd}(h,\mathit{del},r))$ is obtained by merging $\{ o \in O \vert \mathit{apply}(\mathit{lin},\mathit{vis}^{-1}(o)) \}$.

    By the definition of merging, it is same to assume that $O = \mathit{max}_{\mathit{vis}} \mathit{vd}(h,\mathit{del},r)$. Assume that for each operation $o = \mathit{write}(a) \in O$, $\mathit{apply}(\mathit{lin},\mathit{vis}^{-1}(o))) = (a,V_o)$. Then it is not hard to see that $R(r) = \{ (a,V_o) \vert o = \mathit{write}(a) \in O \}$. Therefore, $S = \{ a \vert o = \mathit{write}(a) \in \mathit{vis}^{-1}(i), \forall o' = \mathit{write}(\_) \in \mathit{vis}^{-1}(i), (o,o') \notin \mathit{vis} \}$. According to sequential specification $\mathit{spec}$, $(\mathit{read} \Rightarrow S,i,\mathit{obj})$ of $\mathit{lin}'$ is ``correct''. Then, $h'$ is distributed linearizable w.r.t $\mathit{spec}$ and $\mathit{lin}'$ is a linearization.

\item[-] If $(R,T) {\xrightarrow{\mathit{send}(\mathit{mid},r)}} (R',T')$: Then,

    \begin{itemize}
    \setlength{\itemsep}{0.5pt}
    \item[-] It is obvious that $R' = R$. Let $T' = T \cup \{ (\mathit{mid},R(r),r) \}$.

    \item[-] Let $h' = h$.

    \item[-] Let $\mathit{lin}' = \mathit{lin}$.

    \item[-] Let $\mathit{del}' = \mathit{del}$.

    \item[-] Let $\mathit{map}' = \mathit{map} \cup \{ (\mathit{mid},\mathit{vd}(h,\mathit{del},r)) \}$.
    \end{itemize}

    It is easy to see that all other properties hold, except for checking $C_1$ for $\mathit{mid}$. This holds obviously since the message content of message $\mathit{mid}$ is $R(r)$, and we already know that $R(r) = \mathit{apply}(\mathit{lin},\mathit{vd}(h,\mathit{del},r)) = \mathit{apply}(\mathit{lin},\mathit{map}(\mathit{mid}))$.

\item[-] If $(R,T) {\xrightarrow{\mathit{receive}(\mathit{mid},r)}} (R',T')$: Then,

    \begin{itemize}
    \setlength{\itemsep}{0.5pt}
    \item[-] Let $R' = R[ r: \mathit{merge}(R(r),\mathit{msg})]$ where $(\mathit{mid},\mathit{msg},\_) \in T$. It is obvious that $T' = T$.

    \item[-] Let $h' = h$.

    \item[-] Let $\mathit{lin}' = \mathit{lin}$.

    \item[-] Let $\mathit{del}' = \mathit{del} \cup \{ (i,r) \vert i \in \mathit{map}(\mathit{mid}) \}$.

    \item[-] Let $\mathit{map}' = \mathit{map}$.
    \end{itemize}

    It is easy to see that all other properties hold, except for $C_2$ for replica $r$. Therefore, let us prove that $R'(r) = \mathit{apply}(\mathit{lin}',\mathit{vd}(h',\mathit{del}',r))$.

    We already know that $R'(r) = \mathit{merge}(R(r), \mathit{msg})$, $R(r) = \mathit{apply}(\mathit{lin},\mathit{vd}(h,\mathit{del},r))$ and $\mathit{msg} = \mathit{apply}(\mathit{lin},\mathit{map}(\mathit{mid}))$. It is easy to see that $\mathit{vd}(h',\mathit{del}',r) = \mathit{vd}(h,\mathit{del},r) \cup \mathit{map}(\mathit{mid})$. It is easy to prove that, applying messages in any order lead to the same consequence. Therefore, we have $\mathit{merge}(R(r), \mathit{msg}) = \mathit{apply}(\mathit{lin}',\mathit{vd}(h,\mathit{del},r) \cup \mathit{map}(\mathit{mid}))$. Then, we have $R'(r) = \mathit{apply}(\mathit{lin}',\mathit{vd}(h',\mathit{del}',r))$.
\end{itemize}

This completes the proof of this lemma. $\qed$
\end {proof}
















































\forget{
\section{\crdtimp{}}
\label{sec:crdt implementation}



\subsection{Multi-Value Register Implementation}
\label{subsec:multi-value register implementation}

\cite{ShapiroPBZ11} shows how to obtain a state-based \crdtimp{} from a operation-based \crdtimp{}, and we draw it in Listing~\ref{lst:operation-based emulation of state-based object}. To do an operation $f(a)$, we compute the state-based update and perform merge method in downstream. Here the precondition of downstream is empty because merge is always enabled.


\begin{minipage}[t]{1.0\linewidth}
\begin{lstlisting}[frame=top,caption={operation-based emulation of state-based object},
captionpos=b,label={lst:operation-based emulation of state-based object}]
  payload S ( the state-based payload )
  initial initial payload of S

  update method f(a)
    atSource :
      precondition : precondition of f
      let s = atSource of f(a) in state-based
    downStream(s) :
      S = merge(S,s)
\end{lstlisting}
\end{minipage}

\cite{ShapiroPBZ11} gives a state-based multi-value register implementation. As discussed above, we give its operation-based version in Listing~\ref{lst:operation-based multi-value register}.


The following is a multi-value register implementation.

\begin{minipage}[t]{1.0\linewidth}
\begin{lstlisting}[frame=top,caption={Pseudo-code of the or-set CRDT},
captionpos=b,label={lst:operation-based multi-value register}]
  payload Set S
  initial S = @|$\emptyset$|@
  initial seq = @|$\epsilon$|@

  add(a) :
    atSource :
      let k = getUniqueIdentifier()
      //@ let seq@|$'$|@ = seq@|$\,\cdot\,\alabelshort[add]{a,k}$|@
    downStream(a, k) :
      S = S @|$\cup$|@ {(a, k)}
      //@ S@|$'$|@ = S @|$\cup$|@ {(a, k)}

  remove(a) :
    atSource :
      let R = @|$\{$|@ (a,k): (a,k) @|$\in$|@ S @|$\}$|@
      //@ let seq@|$'$|@ = seq@|$\,\cdot\,\alabellongind[readIds]{a}{R}{}\,\cdot\,\alabelshort[remove]{a,R}$|@
    downStream(R) :
      S = S @|$\setminus$|@ R
      //@ R = @|$\{ (a,k): \exists\ \alabel = \alabellongind[add]{a,k}{\bot}{*}.\ (\alabel, \alabelshort[remove]{a,R}) \in \avisord$|@
                       @|$\land\,\forall\ \alabel' = \alabellongind[remove]{a,*}{\bot}{*}.\ \{(\alabel,\alabel'),(\alabel',\alabelshort[remove]{a,R})\}\not\subseteq \avisord\}$|@
      //@ S@|$'$|@ = S @|$\setminus$|@ R

  read() :
    let A = {a : @|$\exists$|@ k. (a,k) @|$\in$|@ S}
    //@ let seq@|$'$|@ = seq@|$\,\cdot\,\alabellongind[read]{}{A}{}$|@
    return A
\end{lstlisting}
\end{minipage}









\subsection{OR-set Implementation and Formation}
\label{subsec:or-set implementation and formation}

The or-set implementation is shown below. Here function $\mathit{myRep}()$ returns the current replica identifier.

\renewcommand{\algorithmcfname}{CRDT Implementation}
\noindent
%\begin{minipage}{.5\textwidth}
\noindent\begin{algorithm}[H]
$\mathit{payload}$ set $S$; $\mathit{maxTS}$\\
$\mathit{initial}$ $\emptyset$; $(0,\mathit{myRep}())$\\

$add(a)$ \\
\ \ $\mathit{atSource}$: \\
\ \ \ \ assume $\mathit{maxTS} = (c,r')$; \\
\ \ \ \ let $\mathit{ts}' =(c+1,\mathit{myRep}())$; \\

\ \ $\mathit{downstream}((a,\mathit{ts}'))$: \\
\ \ \ \ $S = S \cup \{ (a,\mathit{ts}') \}$; \\
\ \ \ \ $\mathit{maxTS} = \mathit{max} \{ \mathit{maxTS},\mathit{ts}' \}$;


$rem(a)$ \\
\ \ $\mathit{atSource}$: \\
\ \ \ \ $\mathit{pre}$: \ $\exists \mathit{ts}', (a,\mathit{ts}') \in S$ \\
\ \ \ \ let $S_1 = \{ (a,\mathit{ts}') \vert (a,\mathit{ts}') \in S \}$; \\

\ \ $\mathit{downstream}(S_1)$: \\
\ \ \ \ $S = S \setminus S_1$.

$read()$ \\
\ \ \ \ \KwRet $\{ a \vert \exists \mathit{ts}, (a,\mathit{ts}) \in S \}$; \\

\caption{OR-set}
\label{Method-or-set}
\end{algorithm}


The formation of or-set is as follows: $I(r) = (\Sigma, \sigma_0, \mathit{Msg}, \mathit{do},\mathit{receive})$, where

\begin{itemize}
\setlength{\itemsep}{0.5pt}
\item[-] $\Sigma = \{ (S,\mathit{ts}) \vert$, $S$ is a set, each item of $S$ is of the form $(a',\mathit{ts}')$ with $a' \in D$ and $\mathit{ts}' \in \mathbb{N} \times \mathbb{R}.$ $\mathit{ts} \in \mathbb{N} \times \mathbb{R} \}$. $\Sigma_0 = (\emptyset,(0,\mathit{myRep}()))$.

\item[-] Each message content in $\mathit{Msg}$ is either in $D \times \mathbb{N} \times \mathbb{R}$, or a subset of $D \times \mathbb{N} \times \mathbb{R}$.

\item[-] $\mathit{do}((S,(c,r')),\mathit{add},a) = ((S \cup \{ (a, (c+1,r)) \}, (c+1,r)),(a,(c+1,r)))$.

\item[-] If $\exists \mathit{ts}', (a,\mathit{ts}') \in S$, then $\mathit{do}((S,\mathit{ts}),\mathit{rem},a) = ((S \setminus S_1,\mathit{ts}), S_1)$, where $S_1 = \{ (a,\mathit{ts}'') \in S \}$.

\item[-] $\mathit{do}((S,\mathit{ts}),\mathit{read}) = ((S,\mathit{ts}),\{ a \vert \exists \mathit{ts}', (a,\mathit{ts})' \in S \})$.

\item[-] $\mathit{receive}((S,\mathit{ts}),(a,\mathit{ts}')) = (S \cup \{ (a,\mathit{ts}') \}, \mathit{max}( \mathit{ts},\mathit{ts}' ))$,

\item[-] $\mathit{receive}((S,\mathit{ts}),S_1) = (S \setminus S_1,\mathit{ts})$,
\end{itemize}






\section{\Spec{}}
\label{sec:specification}


















\section{Proofs of Section \ref{sec:proving distributed linearizability}}
\label{sec:appendix proofs of section proving distributed linearizability}





\subsection{Proof of OR-set Implementation}
\label{subsec:appendix proofs of or-set implementation}

The following lemma states a property that can be generated from $P(\mathit{config},h,\mathit{lin},\mathit{map})$ for or-set.

\begin{lemma}
\label{lemma:a property that can be obtained from P for or-set}
If $P(\mathit{config},h,\mathit{lin},\mathit{map})$ holds for or-set, then each $\mathit{add}$ operation generate a new unique time-stamp. Moreover, for each replica $r'$,

\begin{itemize}
    \setlength{\itemsep}{0.5pt}
    \item[-] $R(r').S = \{ (b,\mathit{ts}') \vert b \in D, \exists o' = (\mathit{add}(b),\_,\_,\mathit{ts}'), o' \in \mathit{vd}(h,\mathit{del},r'), \forall o'' = (\mathit{rem}(b),\_,\_,\_), o'' \in \mathit{vd}(h,\mathit{config},r') \Rightarrow (o',o'') \notin h.\mathit{vis} \}$.

    \item[-] $R(r').\mathit{maxTS} = (0,r')$ if $\mathit{vd}(h,\mathit{config},r') = \emptyset$; otherwise, $R(r').\mathit{maxTS}$ is the maximal time stamp of $\mathit{add}$ operations of $\mathit{vd}(h,\mathit{config},r')$.
    \end{itemize}
\end{lemma}

\begin {proof}
By $C_4$, it is easy to see that each $\mathit{add}$ operation generate a new unique time-stamp by induction. The property of $R(r')$ can be also easily proved by induction, since the visibility relation is transitive. $\qed$
\end {proof}


The following lemma states that our $P(\mathit{config},h,\mathit{lin},\mathit{map})$ is an invariant of or-set.

\begin{lemma}
\label{lemma:P is an invariant of or-set}
$P(\mathit{config},h,\mathit{lin},\mathit{map})$ is an invariant of or-set.
\end{lemma}

\begin {proof}

Let us prove that $P$ is a simulation relation. It is obvious that $P(\mathit{config}_0,\epsilon,\emptyset,\emptyset)$ holds.

Assume $P((R,T,\mathit{MsgHB},\mathit{MsgDel}),h,\mathit{lin},\mathit{map})$ holds. Here we do not give the detailed value of $\mathit{MsgHB}'$ and $\mathit{MsgDel}'$, since it can be obtained from the definition of $\llbracket \mathit{obj} \rrbracket_{\mathit{op}}$.

\begin{itemize}
\setlength{\itemsep}{0.5pt}
\item[-] If $(R,T,\mathit{MsgHB},\mathit{MsgDel}) {\xrightarrow{\mathit{do}(\mathit{add},a,r,\mathit{mid})}} (R',T',\mathit{MsgHB}',\mathit{MsgDel}')$: Then,

    \begin{itemize}
    \setlength{\itemsep}{0.5pt}
    \item[-] $R' = R[ r: (R(r).S \cup \{ (a,\mathit{ts}) \},\mathit{ts}) ]$ and $T' = T \cup \{ (\mathit{mid},(a,\mathit{ts}),r) \}$. Here $\mathit{ts} = ( \mathit{max} \{ c \vert (\_,(c,\_)) \in R(r).S \} +1,r)$.

    \item[-] Let $h' = h \otimes i$, where $i$ is the identifier of the newly-generated $\mathit{add}$ operation.

    \item[-] Let $\mathit{lin}' = \mathit{lin} \cdot (\mathit{add}(a),i,\mathit{vd}(h,\mathit{config},r))$.

    \item[-] Let $\mathit{map}' = \mathit{map} \cup \{ (\mathit{mid},i) \}$.
    \end{itemize}

    It is easy to see that $h'$ is still distributed linearizable and $\mathit{lin}'$ is its linearization. We need to prove that $R'(r) = \mathit{apply}(\mathit{lin}',\mathit{vd}(h',\mathit{del}',r))$ and $C_4$ still holds for message $\mathit{mid}$.

    We already know that $R(r) = \mathit{apply}(\mathit{lin},\mathit{vd}(h,\mathit{del},r))$. %Based on $C_4$, it is not hard to prove that,

    %\begin{itemize}
    %\setlength{\itemsep}{0.5pt}
    %\item[-] $\mathit{Prop}_1$: Each $\mathit{add}$ operation generate a new unique time-stamp.

    %\item[-] $\mathit{Prop}_2$: for each replica $r'$, $R(r') = \{ (b,\mathit{ts}') \vert b \in D, \exists o' = (\mathit{add}(b),\_,\_,\mathit{ts}'), o' \in \mathit{vd}(h,\mathit{config},r'), \forall o'' = (\mathit{rem}(b),\_,\_,\_), o'' \in \mathit{vd}(h,\mathit{config},r') \Rightarrow (o',o'') \notin h.\mathit{vis} \}$.
    %\end{itemize}

    By Lemma \ref{lemma:a property that can be obtained from P for or-set}, it is not hard to see that $C_4$ still holds for message $\mathit{mid}$. From construction of $R'(r)$, Lemma \ref{lemma:a property that can be obtained from P for or-set} and $C_4$ holds for message $\mathit{mid}$, we can see that $R'(r) = \mathit{apply}(\mathit{lin}',\mathit{vd}(h',\mathit{del}',r))$.%, and $\mathit{Prop}_1$ and $\mathit{Prop}_2$ also hold for $(\mathit{config}',h',\mathit{lin}',\mathit{map}')$.


\item[-] If $(R,T,\mathit{MsgHB},\mathit{MsgDel}) {\xrightarrow{\mathit{do}(\mathit{rem},a,r,\mathit{mid})}} (R',T',\mathit{MsgHB}',\mathit{MsgDel}')$: Then,

    \begin{itemize}
    \setlength{\itemsep}{0.5pt}
    \item[-] $R' = R[ r: (R(r).S \setminus \{ (a,\mathit{ts}) \in R(r).S \},R(r).\mathit{maxTS}) ]$ and $T' = T \cup \{ (\mathit{mid},\{ (a,\mathit{ts}) \in R(r) \},r) \}$.

    \item[-] Let $h' = h \otimes i$, where $i$ is the identifier of the newly-generated $\mathit{rem}$ operation.

    \item[-] Let $\mathit{lin}' = \mathit{lin} \cdot (\mathit{add}(a),i,\mathit{vd}(h,\mathit{config},r))$.

    \item[-] Let $\mathit{map}' = \mathit{map} \cup \{ (\mathit{mid},i) \}$.
    \end{itemize}

    It is easy to see that $h'$ is still distributed linearizable and $\mathit{lin}'$ is its linearization. We need to prove that $R'(r) = \mathit{apply}(\mathit{lin}',\mathit{vd}(h',\mathit{del}',r))$ and $C_4$ still holds for message $\mathit{mid}$.

    By Lemma \ref{lemma:a property that can be obtained from P for or-set}, it is not hard to see that $C_4$ still holds for message $\mathit{mid}$. From construction of $R'(r)$, Lemma \ref{lemma:a property that can be obtained from P for or-set} and $C_4$ holds for message $\mathit{mid}$, we can see that $R'(r) = \mathit{apply}(\mathit{lin}',\mathit{vd}(h',\mathit{del}',r))$.%, and $\mathit{Prop}_1$ and $\mathit{Prop}_2$ also hold for $(\mathit{config}',h',\mathit{lin}',\mathit{map}')$.


\item[-] If $(R,T,\mathit{MsgHB},\mathit{MsgDel}) {\xrightarrow{\mathit{do}(\mathit{read},S_1,r)}} (R',T',\mathit{MsgHB}',\mathit{MsgDel}')$: Then,

    \begin{itemize}
    \setlength{\itemsep}{0.5pt}
    \item[-] $R' = R$ and $T' = T$.

    \item[-] Let $h' = h \otimes i$, where $i$ is the identifier of the newly-generated $\mathit{read}$ operation.

    \item[-] Let $\mathit{lin}' = \mathit{lin} \cdot (\mathit{read} \Rightarrow S_1,i,\mathit{vd}(h,\mathit{config},r))$.

    \item[-] Let $\mathit{map}'$.
    \end{itemize}

    We need to prove that $h'$ is distributed linearizable and $\mathit{lin}'$ is a linearization. Assume that in $\mathit{OR}$-$\mathit{set}_s$, $\mathit{state}_0 {\xrightarrow{\mathit{lin}}} \mathit{state}$ and $\mathit{state} {\xrightarrow{ (\mathit{read} \Rightarrow S_2, i, \mathit{vd}(h,\mathit{config},r) ) }} \mathit{state}$. Then by definition of $\mathit{OR}$-$\mathit{set}_s$, we can see that, $a \in S_2$, if there exists $(\mathit{add}(a),j,\_) \in \mathit{lin}'$, and for each $(\mathit{rem}(a),\_,S_2) \in \mathit{lin}'$, we have $j \notin S_2$. Lemma \ref{lemma:a property that can be obtained from P for or-set}, we can see that $S_1 = S_2$, and since $h$ is distributed linearizable and $\mathit{lin}$ is a linearization of $h$, we can see $h'$ is distributed linearizable and $\mathit{lin}'$ is a linearization.

\item[-] If $(R,T,\mathit{MsgHB},\mathit{MsgDel}) {\xrightarrow{\mathit{receive}(\mathit{mid},r)}} (R',T',\mathit{MsgHB}',\mathit{MsgDel}')$, where $(\mathit{mid},(a,\mathit{ts}),r') \in T$: Then,

    \begin{itemize}
    \setlength{\itemsep}{0.5pt}
    \item[-] $R' = R[ r: (R(r).S \cup \{ (a,\mathit{ts}) \},\mathit{max} \{ R(r).\mathit{maxTS},\mathit{ts} \} ) ]$ and $T' = T$.

    \item[-] Let $h' = h$.

    \item[-] Let $\mathit{lin}' = \mathit{lin}$.

    \item[-] Let $\mathit{map}' = \mathit{map}$.
    \end{itemize}

    We need to prove that $R'(r) = \mathit{apply}(\mathit{lin}',\mathit{vd}(h',\mathit{del}',r))$.

    We already know that $R(r) = \mathit{apply}(\mathit{lin},\mathit{vd}(h,\mathit{del},r))$. Since $R'(r)$ is obtained from $R(r)$ by applying message $\mathit{mid}$, and $\mathit{apply}(\mathit{lin}',\mathit{vd}(h',\mathit{del}',r))$ is obtained from $\mathit{apply}(\mathit{lin},\mathit{vd}(h,\mathit{del},r))$ by applying message $\mathit{mid}$. Therefore, $R'(r) = \mathit{apply}(\mathit{lin}',\mathit{vd}(h',\mathit{del}',r))$.

\item[-] If $(R,T,\mathit{MsgHB},\mathit{MsgDel}) {\xrightarrow{\mathit{receive}(\mathit{mid},r)}} (R',T',\mathit{MsgHB}',\mathit{MsgDel}')$, where $(\mathit{mid},S_1,r') \in T$: Then,

    \begin{itemize}
    \setlength{\itemsep}{0.5pt}
    \item[-] $R' = R[ r: (R(r).S \setminus S_1, R(r).\mathit{maxTS}) ]$ and $T' = T$.

    \item[-] Let $h' = h$.

    \item[-] Let $\mathit{lin}' = \mathit{lin}$.

    \item[-] Let $\mathit{map}' = \mathit{map}$.
    \end{itemize}

    We need to prove that $R'(r) = \mathit{apply}(\mathit{lin}',\mathit{vd}(h',\mathit{del}',r))$.

    We already know that $R(r) = \mathit{apply}(\mathit{lin},\mathit{vd}(h,\mathit{del},r))$. Since $R'(r)$ is obtained from $R(r)$ by applying message $\mathit{mid}$, and $\mathit{apply}(\mathit{lin}',\mathit{vd}(h',\mathit{del}',r))$ is obtained from $\mathit{apply}(\mathit{lin},\mathit{vd}(h,\mathit{del},r))$ by applying message $\mathit{mid}$. Therefore, $R'(r) = \mathit{apply}(\mathit{lin}',\mathit{vd}(h',\mathit{del}',r))$.
\end{itemize}

This completes the proof of this lemma. $\qed$
\end {proof}




\subsection{Proof of RGA}
\label{subsec:appendix proofs of rga}

The following lemma states a property that can be generated from $P(\mathit{config},h,\mathit{lin},\mathit{map})$ for RGA.

\begin{lemma}
\label{lemma:a property that can be obtained from P for rga}
If $P(\mathit{config},h,\mathit{lin},\mathit{map})$ holds, then each $\mathit{add}$ operation generate a new unique time-stamp. Moreover, for each replica $r'$,

\begin{itemize}
    \setlength{\itemsep}{0.5pt}
    \item[-] $R(r').N = \{ (a,\mathit{ts}_a,\mathit{ts}_b) \vert \exists o' = (\mathit{add}(\_,\_),i,\_,\_), \mathit{map}(i) = (a,\mathit{ts}_a,\mathit{ts}_b), o' \in \mathit{vd}(h,\mathit{config},r') \}$.

    \item[-] $R(r').\mathit{Tomb} = \{ a \vert \exists o = (\mathit{rem}(a),i,\_,\_), \mathit{map}(i) \in \mathit{vd}(h,\mathit{config},r') \}$.
    \end{itemize}
\end{lemma}

\begin {proof}
By $C_4$, it is easy to see that each $\mathit{add}$ operation generate a new unique time-stamp by induction. The property of $R(r')$ can be also easily proved by induction, since the visibility relation is transitive. $\qed$
\end {proof}


The following lemma states that our $P(\mathit{config},h,\mathit{lin},\mathit{map})$ is an invariant of rga.

\begin{lemma}
\label{lemma:P is an invariant of rga}
$P(\mathit{config},h,\mathit{lin},\mathit{map})$ is an invariant of rga.
\end{lemma}

\begin {proof}

Let us prove that $P$ is a simulation relation. It is obvious that $P(\mathit{config}_0,\epsilon,\emptyset,\emptyset)$ holds.

Assume $P((R,T,\mathit{MsgHB},\mathit{MsgDel}),h,\mathit{lin},\mathit{map})$ holds. Here we do not give the detailed value of $\mathit{MsgHB}'$ and $\mathit{MsgDel}'$, since it can be obtained from the definition of $\llbracket \mathit{obj} \rrbracket_{\mathit{op}}$.

\begin{itemize}
\setlength{\itemsep}{0.5pt}
\item[-] If $(R,T,\mathit{MsgHB},\mathit{MsgDel}) {\xrightarrow{\mathit{do}(\mathit{add},a,b,r,\mathit{mid})}} (R',T',\mathit{MsgHB}',\mathit{MsgDel}')$: Then,

    \begin{itemize}
    \setlength{\itemsep}{0.5pt}
    \item[-] $R' = R[ r: (R(r).N \cup \{ (a,\mathit{ts}_a,\mathit{ts}_b) \}, R(r).\mathit{Tomb}) ]$ and $T' = T \cup \{ (\mathit{mid},(a,\mathit{ts}_a,\mathit{ts}_b),r) \}$. Here $\mathit{ts}_a = ( \mathit{max} \{ c \vert (\_,(c,\_),\_) \in R(r).N \vee (\_,\_,(c,\_)) \in R(r).N \} +1,r)$, and $\mathit{ts}_b$ is the time-stamp of $b$ in $R(r).N$.

    \item[-] Let $h' = h \otimes i$, where $i$ is the identifier of the newly-generated $\mathit{add}$ action.

    \item[-] $\mathit{lin}'$ is obtained from $\mathit{lin}$ by inserting $(\mathit{add}(a,b),i,\mathit{vd}(h,\mathit{config},r))$ after the last operation with time-stamp less or equal than $\mathit{ts}_a$.

    \item[-] Let $\mathit{map}' = \mathit{map} \cup \{ (\mathit{mid},i) \}$.
    \end{itemize}

    It is easy to see that $h'$ is still distributed linearizable and $\mathit{lin}'$ is its linearization. We need to prove that $R'(r) = \mathit{apply}(\mathit{lin}',\mathit{vd}(h',\mathit{del}',r))$ and $C_4$ still holds for message $\mathit{mid}$.

    We already know that $R(r) = \mathit{apply}(\mathit{lin},\mathit{vd}(h,\mathit{del},r))$.

    By Lemma \ref{lemma:a property that can be obtained from P for rga}, it is not hard to see that $C_4$ still holds for message $\mathit{mid}$. From the fact that $\mathit{ts}_a$ is unique, the fact that there is no $\mathit{rem}(a)$ in $h$, the construction of $R'(r)$, Lemma \ref{lemma:a property that can be obtained from P for rga} and $C_4$ holds for message $\mathit{mid}$, we can see that $R'(r) = \mathit{apply}(\mathit{lin}',\mathit{vd}(h',\mathit{del}',r))$.


\item[-] If $(R,T,\mathit{MsgHB},\mathit{MsgDel}) {\xrightarrow{\mathit{do}(\mathit{rem},a,r,\mathit{mid})}} (R',T',\mathit{MsgHB}',\mathit{MsgDel}')$: Then,

    \begin{itemize}
    \setlength{\itemsep}{0.5pt}
    \item[-] $R' = R[ r: (R(r).N,R(r).\mathit{Tomb} \cup \{ a \} ) ]$ and $T' = T \cup \{ (\mathit{mid},\{ a \},r) \}$.

    \item[-] Let $h' = h \otimes i$, where $i$ is the identifier of the newly-generated $\mathit{rem}$ operation.

    \item[-] $\mathit{lin}'$ is obtained from $\mathit{lin}$ by inserting $(\mathit{rem}(a),i,\mathit{vd}(h,\mathit{config},r))$ after the last operation with time-stamp less or equal than the time-stamp of operation $i$.

    \item[-] Let $\mathit{map}' = \mathit{map} \cup \{ (\mathit{mid},i) \}$.
    \end{itemize}

    By Lemma \ref{lemma:a property that can be obtained from P for rga}, it is easy to see that $\mathit{lin} \uparrow_{\mathit{vd}(h,\mathit{config},r)}$ contains a $\mathit{add}(a,\_)$ operation $o$ and $(o,i) \in h'.\mathit{vis}$. By Lemma \ref{lemma:a property that can be obtained from P for rga}, it is easy to see that $\mathit{lin} \uparrow_{\mathit{vd}(h,\mathit{config},r)}$ does not contain $\mathit{rem}(a)$. Since $i$ does not visible to any operation in $\mathit{vd}(h,\mathit{config},r)$, we can see that $h'$ is still distributed linearizable and $\mathit{lin}'$ is its linearization.

    We need to prove that $R'(r) = \mathit{apply}(\mathit{lin}',\mathit{vd}(h',\mathit{del}',r))$ and $C_4$ still holds for message $\mathit{mid}$.

    It is obvious that $C_4$ holds for message $\mathit{mid}$. By Lemma \ref{lemma:a property that can be obtained from P for rga}, the construction of $R'(r)$, and $C_4$ holds for message $\mathit{mid}$, we can see that $R'(r) = \mathit{apply}(\mathit{lin}',\mathit{vd}(h',\mathit{del}',r))$.


\item[-] If $(R,T,\mathit{MsgHB},\mathit{MsgDel}) {\xrightarrow{\mathit{do}(\mathit{read},l,r)}} (R',T',\mathit{MsgHB}',\mathit{MsgDel}')$: Then,

    \begin{itemize}
    \setlength{\itemsep}{0.5pt}
    \item[-] $R' = R$ and $T' = T$.

    \item[-] Let $h' = h \otimes i$, where $i$ is the identifier of the newly-generated $\mathit{read}$ operation.

    \item[-] $\mathit{lin}'$ is obtained from $\mathit{lin}$ by inserting $(\mathit{rem}(a),i,\mathit{vd}(h,\mathit{config},r))$ after the last operation with time-stamp less or equal than the time-stamp of operation $i$.

    \item[-] Let $\mathit{map}' = \mathit{map}$.
    \end{itemize}

    We need to prove that $h'$ is distributed linearizable and $\mathit{lin}'$ is a linearization. Assume that in $\mathit{list}_s^{\mathit{af}}$, $\mathit{state}_0 {\xrightarrow{\mathit{lin}}} \mathit{state}$ and $\mathit{state} {\xrightarrow{ (\mathit{read} \Rightarrow l_1, i, \mathit{vd}(h,\mathit{config},r) ) }} \mathit{state}$.

    By Lemma \ref{lemma:a property that can be obtained from P for rga} and RGA implementation, we can see that $l$ and $l_1$ has the same items.

    Given items $a,b$, assume that $a$ is before $b$ in $l$, then, there are two possibilities,

    \begin{itemize}
    \setlength{\itemsep}{0.5pt}
    \item[-] $a$ is a ancestor of $b$ in $R(r).N$,

    \item[-] there exists items $c_1,c_2,c_3$, such that in $R(r).N$, $c_2$ and $c_3$ are sons of $c_1$, $c_2$ is a ancestor of $a$, $c_3$ is a ancestor of $b$, and the time-stamp of $c_2$ is larger than that of $c_3$.
    \end{itemize}

    If the first possibility holds, then there exists items $d_1,\ldots,d_k$, such that in $R(r).N$, $b$ is a son of $d_1$, $d_1$ is a son of $d_2$, $\ldots$, and $d_k$ is a son of $a$. It is easy to see that $(\mathit{add}(a,\_),\mathit{add}(d_k,a)),(\mathit{add}(d_k,a),\mathit{add}(d_{\mathit{k-1}},d_k)), \ldots, (\mathit{add}(d_1,d_2),\mathit{add}(b,d_1)) \in h.\mathit{vis}$. Since $\mathit{lin}$ is consistent with visibility relation, we know that in $\mathit{lin}$, $\mathit{add}(a,\_)$ is before $\mathit{add}(d_k,a)$, $\mathit{add}(d_k,a)$ is before $\mathit{add}(d_{\mathit{k-1}},d_k)$, $\ldots$, and $\mathit{add}(d_1,d_2)$ is before $\mathit{add}(b,d_1)$. According to $\mathit{list}_s^{\mathit{af}}$, it is easy to see that in $a$ is before $b$ in $l_1$.

    If the second possibility holds, then it is easy to see that $(\mathit{add}(c_2,c_1),\mathit{add}(a,c_2)),$ $(\mathit{add}(c_1,\_),\mathit{add}(c_2,c_1)), (\mathit{add}(c_3,c_1),\mathit{add}(b,c_3)),(\mathit{add}(c_1,\_),\mathit{add}(c_3,c_1)), \in h.\mathit{vis}$. Since $\mathit{lin}$ is consistent with visibility relation and time-stamp, we know that in $\mathit{lin}$, $\mathit{add}(c_3,c_1)$ is before $\mathit{add}(c_2,c_1)$, $\mathit{add}(c_2,c_1)$ is before $\mathit{add}(a,c_2)$, and $\mathit{add}(c_3,c_1)$ is before $\mathit{add}(b,c_3)$. According to $\mathit{list}_s^{\mathit{af}}$, it is easy to see that in $a$ is before $b$ in $l_1$.

    Therefore, $h'$ is distributed linearizable and $\mathit{lin}'$ is a linearization.

\item[-] If $(R,T,\mathit{MsgHB},\mathit{MsgDel}) {\xrightarrow{\mathit{receive}(\mathit{mid},r)}} (R',T',\mathit{MsgHB}',\mathit{MsgDel}')$, where $(\mathit{mid},(a,\mathit{ts}_a,\mathit{ts}_b),r') \in T$: Then,

    \begin{itemize}
    \setlength{\itemsep}{0.5pt}
    \item[-] $R' = R[ r: ( R(r).N \cup \{ (a,\mathit{ts}_a,\mathit{ts}_b) \}, R(r).\mathit{Tomb} ) ]$ and $T' = T$.

    \item[-] Let $h' = h$.

    \item[-] Let $\mathit{lin}' = \mathit{lin}$.

    \item[-] Let $\mathit{map}' = \mathit{map}$.
    \end{itemize}

    We need to prove that $R'(r) = \mathit{apply}(\mathit{lin}',\mathit{vd}(h',\mathit{del}',r))$.

    We already know that $R(r) = \mathit{apply}(\mathit{lin},\mathit{vd}(h,\mathit{del},r))$.

    We can see that $R'(r)$ is obtained from $R(r)$ by applying message $\mathit{mid}$, and $\mathit{apply}(\mathit{lin}',\mathit{vd}(h',\mathit{del}',r))$ is obtained from $\mathit{apply}(\mathit{lin},\mathit{vd}(h,\mathit{del},r))$ by additionally applying messages $\mathit{mid}$, but possibly in the middle of $\mathit{lin}'$. It is easy to see that $\mathit{map}(\mathit{mid})$ is a $\mathit{add}(a,\_)$ operation. By Lemma \ref{lemma:a property that can be obtained from P for rga}, we can see that there is no $\mathit{add}(a,\_)$ nor $\mathit{rem}(a)$ in $\mathit{vd}(h,\mathit{del},r)$. Thus, for each $\mathit{lin}''$ generated from $\mathit{lin}'$ by postponing message $\mathit{mid}$ to a later position, we can see that $\mathit{apply}(\mathit{lin}'',\mathit{vd}(h',\mathit{del}',r)) = \mathit{apply}(\mathit{lin}',\mathit{vd}(h',\mathit{del}',r))$.

    Therefore, $R'(r) = \mathit{apply}(\mathit{lin}',\mathit{vd}(h',\mathit{del}',r))$.

\item[-] If $(R,T,\mathit{MsgHB},\mathit{MsgDel}) {\xrightarrow{\mathit{receive}(\mathit{mid},r)}} (R',T',\mathit{MsgHB}',\mathit{MsgDel}')$, where $(\mathit{mid},a,r') \in T$: Then,

    \begin{itemize}
    \setlength{\itemsep}{0.5pt}
    \item[-] $R' = R[ r: (R(r).N,R(r).\mathit{Tomb} \cup \{ a \}) ]$ and $T' = T$.

    \item[-] Let $h' = h$.

    \item[-] Let $\mathit{lin}' = \mathit{lin}$.

    \item[-] Let $\mathit{map}' = \mathit{map}$.
    \end{itemize}

    We need to prove that $R'(r) = \mathit{apply}(\mathit{lin}',\mathit{vd}(h',\mathit{del}',r))$.

    We already know that $R(r) = \mathit{apply}(\mathit{lin},\mathit{vd}(h,\mathit{del},r))$.

    We can see that $R'(r)$ is obtained from $R(r)$ by applying message $\mathit{mid}$, and $\mathit{apply}(\mathit{lin}',\mathit{vd}(h',\mathit{del}',r))$ is obtained from $\mathit{apply}(\mathit{lin},\mathit{vd}(h,\mathit{del},r))$ by additionally applying messages $\mathit{mid}$, but possibly in the middle of $\mathit{lin}'$.

    It is easy to see that, for each $\mathit{lin}''$ generated from $\mathit{lin}'$ by postponing message $\mathit{mid}$ to a later position, we have $\mathit{apply}(\mathit{lin}'',\mathit{vd}(h',\mathit{del}',r)) = \mathit{apply}(\mathit{lin}',\mathit{vd}(h',\mathit{del}',r))$.

    Therefore, $R'(r) = \mathit{apply}(\mathit{lin}',\mathit{vd}(h',\mathit{del}',r))$.
\end{itemize}

This completes the proof of this lemma. $\qed$
\end {proof}










\section{Proofs of Section \ref{sec:compositionality of distributed linearizability}}
\label{sec:appendix proofs of section compositionality of distributed linearizability}





\subsection{Proofs of Lemma \ref{lemma:several t0-specifications}}
\label{subsec:appendix proofs of Lemma several t0-specifications}

A specification $\mathit{spec}$ is called t0-specification, if given a history $h$ that is distributed linearizable w.r.t $\mathit{spec}$, then any sequence that is consistent with visibility relation is a linearization of $h$.

Given two sequences $l_1,l_2$, let $\mathit{diff}(l_1,l_2) = \{ (o_1,o_2) \vert$ the order of $o_1$ and $o_2$ in $l_1$ is different from that of $l_2 \}$. Given a sequence $l$ and two elements $o_1$ an $o_2$ of $l$, let $\mathit{swap}(l,o_1,o_2)$ be a sequence obtained from $l$ by swapping $o_1$ and $o_2$.

The following lemma states that $\mathit{OR}$-$\mathit{set}_s$ is a t0-specification.

\begin{lemma}
\label{lemma:or-set is a t0-specification}
$\mathit{OR}$-$\mathit{set}_s$ is a t0-specification.
\end{lemma}

\begin {proof}
Given a distributed linearizable history $h$ and assume that $\mathit{lin}$ is a linearization. It is obvious that $\mathit{lin}$ is consistent with visibility relation. We need to prove that, each such sequence $\mathit{lin}'$ described below is also a linearization of $h$

\begin{itemize}
\setlength{\itemsep}{0.5pt}
\item[-] $\mathit{lin}'$ contains the same set of elements as that of $\mathit{lin}$.

\item[-] $\mathit{lin}'$ is consistent with visibility relation.
\end{itemize}

We prove this by showing that each such $\mathit{lin}'$ can be obtained from $\mathit{lin}$ by several times of swapping a pair of adjacent elements. Our proof requires the following two properties:

\begin{itemize}
\setlength{\itemsep}{0.5pt}
\item[-] The first property is: Given a linarization $\mathit{lin}$ and a sequence $\mathit{lin}'$ consistent with visibility relation of $h$, if $\mathit{diff}(\mathit{lin},\mathit{lin}') \neq \emptyset$, there exists $(o_1,o_2) \in \mathit{diff}(\mathit{lin},\mathit{lin}')$, such that $o_1$ and $o_2$ are concurrent, and $o_1$ and $o_2$ are adjacent in $\mathit{lin}$.

    We prove this by contradiction. Assume $\mathit{diff}(\mathit{lin},\mathit{lin}') \neq \emptyset$, and for each $(o_1,o_2) \in \mathit{diff}(\mathit{lin},\mathit{lin}')$, we have that either $o_1$ and $o_2$ are not concurrent, or $o_1$ and $o_2$ are not adjacent in $\mathit{lin}$.

    Since $\mathit{diff}(\mathit{lin},\mathit{lin}') \neq \emptyset$, let $(o,o')$ be a element of $\mathit{diff}(\mathit{lin},\mathit{lin}')$, and the distance of $o_1$ and $o_2$ is minimal in $\{$ the distance between $o_1$ and $o_2 \vert (o_1,o_2) \in \mathit{diff}(\mathit{lin},\mathit{lin}') \}$. Let us prove that $o$ and $o'$ are adjacent by contradiction: If there exists $o''$ between $o$ and $o'$. Assume that in $\mathit{lin}$, $o$ is before $o''$, and $o''$ is before $o'$. By assumption, the order between $o$ and $o''$, and between $o''$ and $o'$ is the same in $\mathit{lin}$ and in $\mathit{lin}'$. This implies that $o$ is still before $o'$ in $\mathit{lin}'$, which contradicts the fact that $(o,o') \in \mathit{diff}(\mathit{lin},\mathit{lin}')$.

    Since $o$ and $o'$ are adjacent and $(o,o') \in \mathit{diff}(\mathit{lin},\mathit{lin}')$, by assumption we know that $o$ and $o'$ are not concurrent. Or we can say, $(o,o') \in \mathit{vis} \vee \mathit{o',o} \in \mathit{vis}$. This contradicts that both $\mathit{lin}$ and $\mathit{lin}'$ are consistent with visibility relation. This completes the proof of the first step.

\item[-] The second property is: Given a linearization $\mathit{lin}$ and $o_1,o_2 \in \mathit{lin}$, such that $o_1$ and $o_2$ are concurrent and adjacent in $\mathit{lin}$, then, $l = \mathit{swap}(\mathit{lin},o_1,o_2)$ is also a linearization.

    Let $o_1 = (\ell_1,\mathit{id}_1,S_1)$ and $o_2 = (\ell_2,\mathit{id}_2,S_2)$. Since $o_1$ and $o_2$ are concurrent, we know that $\mathit{id}_1 \notin S_2 \wedge \mathit{id}_2 \notin S_1$. Assume $\mathit{lin} = l_1 \cdot o_1 \cdot o_2 \cdot l_2$. Assume in the abstract state of $\mathit{OR}$-$\mathit{set}_s$, we have $\sigma_0 {\xrightarrow{l_1}} \sigma_1 {\xrightarrow{o_1}} \sigma_2 {\xrightarrow{o_2}} \sigma_3 {\xrightarrow{l_2}} \sigma_4$, where $\sigma_0$ is the initial state of $\mathit{OR}$-$\mathit{set}_s$. Then, we need to prove that, there exists $\sigma'_2$, such that $\sigma_1 {\xrightarrow{o_2}} \sigma'_2 {\xrightarrow{o_1}} \sigma_3$. We prove this by consider all the possible cases:

    \begin{itemize}
    \setlength{\itemsep}{0.5pt}
    \item[-] If $o_1 = (\mathit{add}(a_1),\mathit{id}_1,S_1)$ and $o_2 = (\mathit{add}(a_2),\mathit{id}_2,S_2)$: We can see that $\sigma_2$ is obtained from $\sigma_1$ by inserting $(a_1,\mathit{id}_1,\mathit{true})$, and $\sigma_3$ is obtained from $\sigma_2$ by inserting $(a_2,\mathit{id}_2,\mathit{true})$. Let $\sigma'_2$ be obtained from $\sigma_1$ by inserting $(a_2,\mathit{id}_2,\mathit{true})$. Then, it is easy to see that $\sigma_1 {\xrightarrow{o_2}} \sigma'_2 {\xrightarrow{o_1}} \sigma_3$.

    \item[-] If $o_1 = (\mathit{add}(a_1),\mathit{id}_1,S_1)$ and $o_2 = (\mathit{rem}(a_2),\mathit{id}_2,S_2)$: We can see that $\sigma_2$ is obtained from $\sigma_1$ by inserting $(a_1,\mathit{id}_1,\mathit{true})$, and $\sigma_3$ is obtained from $\sigma_2$ by marking $a_2$ with identifiers of $S_2$ into $\mathit{false}$. Let $\sigma'_2$ be obtained from $\sigma_1$ by marking $a_2$ with identifiers of $S_2$ into $\mathit{false}$. Since $\mathit{id_1} \notin S_2$, we can see that $\sigma_1 {\xrightarrow{o_2}} \sigma'_2 {\xrightarrow{o_1}} \sigma_3$.

    \item[-] If $o_1 = (\mathit{add}(a_1),\mathit{id}_1,S_1)$ and $o_2 = (\mathit{read}() \Rightarrow l_2,\mathit{id}_2,S_2)$: Let $\sigma'_2 = \sigma_1$. Since $\mathit{id}_1 \notin S_2$, it is easy to see that $\sigma_1 {\xrightarrow{o_2}} \sigma'_2 {\xrightarrow{o_1}} \sigma_3$.

    \item[-] If $o_1 = (\mathit{rem}(a_1),\mathit{id}_1,S_1)$ and $o_2 = (\mathit{add}(a_2),\mathit{id}_2,S_2)$: We can see that $\sigma_2$ is obtained from $\sigma_1$ by marking $a_1$ with identifiers of $S_1$ into $\mathit{false}$, and $\sigma_3$ is obtained from $\sigma_2$ by inserting $(a_2,\mathit{id}_2,\mathit{true})$. Let $\sigma'_2$ be obtained from $\sigma_1$ by inserting $(a_2,\mathit{id}_2,\mathit{true})$. Since $\mathit{id}_2 \notin S_1$, we can see that $\sigma_1 {\xrightarrow{o_2}} \sigma'_2 {\xrightarrow{o_1}} \sigma_3$.

    \item[-] If $o_1 = (\mathit{rem}(a_1),\mathit{id}_1,S_1)$ and $o_2 = (\mathit{rem}(a_2),\mathit{id}_2,S_2)$: We can see that $\sigma_2$ is obtained from $\sigma_1$ by marking $a_1$ with identifiers of $S_1$ into $\mathit{false}$, and $\sigma_3$ is obtained from $\sigma_2$ by marking $a_2$ with identifiers of $S_2$ into $\mathit{false}$. Let $\sigma'_2$ be obtained from $\sigma_1$ by marking $a_2$ with identifiers of $S_2$ into $\mathit{false}$. Then, it is easy to see that $\sigma_1 {\xrightarrow{o_2}} \sigma'_2 {\xrightarrow{o_1}} \sigma_3$.

    \item[-] If $o_1 = (\mathit{rem}(a_1),\mathit{id}_1,S_1)$ and $o_2 = (\mathit{read}() \Rightarrow l_2,\mathit{id}_2,S_2)$: Let $\sigma'_2 = \sigma_1$. Since $\mathit{id}_1 \notin S_2$, it is easy to see that $\sigma_1 {\xrightarrow{o_2}} \sigma'_2 {\xrightarrow{o_1}} \sigma_3$.

    \item[-] If $o_1 = (\mathit{read}() \Rightarrow l_1,\mathit{id}_1,S_1)$ and $o_2 = (\mathit{add}(a_1),\mathit{id}_2,S_2)$: Let $\sigma'_2$ be obtained from $\sigma_1$ by inserting $(a_1,\mathit{id}_1,\mathit{true})$. Since $\mathit{id}_2 \notin S_1$, it is easy to see that $\sigma_1 {\xrightarrow{o_2}} \sigma'_2 {\xrightarrow{o_1}} \sigma_3$.

    \item[-] If $o_1 = (\mathit{read}() \Rightarrow l_1,\mathit{id}_1,S_1)$ and $o_2 = (\mathit{rem}(a_1),\mathit{id}_2,S_2)$: Let $\sigma'_2$ be obtained from $\sigma_1$ by marking $a_2$ with identifiers of $S_2$ into $\mathit{false}$. Since $\mathit{id}_2 \notin S_1$, it is easy to see that $\sigma_1 {\xrightarrow{o_2}} \sigma'_2 {\xrightarrow{o_1}} \sigma_3$.

    \item[-] If $o_1 = (\mathit{read}() \Rightarrow l_1,\mathit{id}_1,S_1)$ and $o_2 = (\mathit{read}() \Rightarrow l_2,\mathit{id}_2,S_2)$: Let $\sigma'_2 = \sigma_1$. Then, it is easy to see that $\sigma_1 {\xrightarrow{o_2}} \sigma'_2 {\xrightarrow{o_1}} \sigma_3$.
    \end{itemize}
\end{itemize}

Based on these two steps, given a linearization $\mathit{lin}$ and a sequence $\mathit{lin}' \neq \mathit{lin}$ which is consistent with visibility relation: We have $\mathit{diff}(\mathit{lin},\mathit{lin}') \neq \emptyset$. Based on the first above property, there exists $(o_1,o_2) \in \mathit{diff}(\mathit{lin},\mathit{lin}')$, such that $o_1$ and $o_2$ are concurrent, and $o_1$ and $o_2$ are adjacent in $\mathit{lin}$. Based on the second above property, $\mathit{lin}'' = \mathit{swap}(\mathit{lin},o_1,o_2)$ is also a linearization. Moreover, it is easy to see that $\mathit{diff}(\mathit{lin},\mathit{lin}') > \mathit{diff}(\mathit{lin}'',\mathit{lin}')$. Therefore, by several times of above process, we finally obtain $\mathit{lin}'$ from $\mathit{lin}$ by swapping pairs of operations, and prove that $\mathit{lin}'$ is also a linearization. This completes the proof of this lemma. $\qed$
\end {proof}



The following lemma states that $\mathit{set}_s$ is a t0-specification.

\begin{lemma}
\label{lemma:set is a t0-specification}
$\mathit{set}_s$ is a t0-specification.
\end{lemma}

\begin {proof}

We prove this lemma similarly as that of Lemma \ref{lemma:or-set is a t0-specification}. We need to prove that, given a linearization $\mathit{lin}$ and $o_1,o_2 \in \mathit{lin}$, such that $o_1$ and $o_2$ are concurrent and adjacent in $\mathit{lin}$, then, $l = \mathit{swap}(\mathit{lin},o_1,o_2)$ is also a linearization.

Let $o_1 = (\ell_1,\mathit{id}_1,S_1)$ and $o_2 = (\ell_2,\mathit{id}_2,S_2)$. Since $o_1$ and $o_2$ are concurrent, we know that $\mathit{id}_1 \notin S_2 \wedge \mathit{id}_2 \notin S_1$. Assume $\mathit{lin} = l_1 \cdot o_1 \cdot o_2 \cdot l_2$. Assume in the abstract state of $\mathit{set}_s$, we have $\sigma_0 {\xrightarrow{l_1}} \sigma_1 {\xrightarrow{o_1}} \sigma_2 {\xrightarrow{o_2}} \sigma_3 {\xrightarrow{l_2}} \sigma_4$, where $\sigma_0$ is the initial state of $\mathit{set}_s$. Then, we need to prove that, there exists $\sigma'_2$, such that $\sigma_1 {\xrightarrow{o_2}} \sigma'_2 {\xrightarrow{o_1}} \sigma_3$. We prove this by consider all the possible cases:

\begin{itemize}
\setlength{\itemsep}{0.5pt}
\item[-] If $o_1 = (\mathit{add}(a_1),\mathit{id}_1,S_1)$ and $o_2 = (\mathit{add}(a_2),\mathit{id}_2,S_2)$: We can see that, if $(a_1,\_) \in \sigma_1$, then $\sigma_2 = \sigma_1$; else, $\sigma_2$ is obtained from $\sigma_1$ by inserting $(a_1,\mathit{true})$. We can also see that, if $(a_2,\_) \in \sigma_2$, then $\sigma_3 = \sigma_2$; else, $\sigma_3$ is obtained from $\sigma_2$ by inserting $(a_2,\mathit{true})$. Let $\sigma'_2$ be: if $(a_2,\_) \in \sigma_1$, then $\sigma'_2 = \sigma_1$; else, $\sigma'_2$ is obtained from $\sigma_1$ by inserting $(a_2,\mathit{true})$. Then, it is easy to see that $\sigma_1 {\xrightarrow{o_2}} \sigma'_2 {\xrightarrow{o_1}} \sigma_3$.

\item[-] If $o_1 = (\mathit{add}(a_1),\mathit{id}_1,S_1)$ and $o_2 = (\mathit{rem}(a_2),\mathit{id}_2,S_2)$: Let $\sigma'_2$ be: if $(a_2,\mathit{false}) \in \sigma_1$, then $\sigma'_2 = \sigma_1$; else, $\sigma'_2$ is obtained from $\sigma_1$ by marking $a_2$ into $\mathit{false}$. Since $\mathit{vis}^{-1}(o_2) \cdot o_2 \in \mathit{set}_s$, we know that $(a_2,\_) \in \sigma_1$. Then, it is easy to see that $\sigma_1 {\xrightarrow{o_2}} \sigma'_2 {\xrightarrow{o_1}} \sigma_3$.

\item[-] If $o_1 = (\mathit{add}(a_1),\mathit{id}_1,S_1)$ and $o_2 = (\mathit{read}() \Rightarrow l_2,\mathit{id}_2,S_2)$: Let $\sigma'_2 = \sigma_1$. Since $\mathit{id}_1 \notin S_2$, it is easy to see that $\sigma_1 {\xrightarrow{o_2}} \sigma'_2 {\xrightarrow{o_1}} \sigma_3$.

\item[-] If $o_1 = (\mathit{rem}(a_1),\mathit{id}_1,S_1)$ and $o_2 = (\mathit{add}(a_2),\mathit{id}_2,S_2)$: Let $\sigma'_2$ be: if $(a_2,\_) \in \sigma_1$, then $\sigma'_2 = \sigma_1$; else, $\sigma'_2$ is obtained from $\sigma_1$ by inserting $(a_2,\mathit{true})$. Since $\mathit{vis}^{-1}(o_1) \cdot o_1 \in \mathit{set}_s$, we know that $(a_1,\_) \in \sigma_1$. Then, it is easy to see that $\sigma_1 {\xrightarrow{o_2}} \sigma'_2 {\xrightarrow{o_1}} \sigma_3$.

\item[-] If $o_1 = (\mathit{rem}(a_1),\mathit{id}_1,S_1)$ and $o_2 = (\mathit{rem}(a_2),\mathit{id}_2,S_2)$: Let $\sigma'_2$ be: if $(a_2,\mathit{false}) \in \sigma_1$, then $\sigma'_2 = \sigma_1$; else, $\sigma'_2$ is obtained from $\sigma_1$ by marking $a_2$ into $\mathit{false}$. Then, it is easy to see that $\sigma_1 {\xrightarrow{o_2}} \sigma'_2 {\xrightarrow{o_1}} \sigma_3$.

\item[-] If $o_1 = (\mathit{rem}(a_1),\mathit{id}_1,S_1)$ and $o_2 = (\mathit{read}() \Rightarrow l_2,\mathit{id}_2,S_2)$: Let $\sigma'_2 = \sigma_1$. Since $\mathit{id}_1 \notin S_2$, it is easy to see that $\sigma_1 {\xrightarrow{o_2}} \sigma'_2 {\xrightarrow{o_1}} \sigma_3$.

\item[-] If $o_1 = (\mathit{read}() \Rightarrow l_1,\mathit{id}_1,S_1)$ and $o_2 = (\mathit{add}(a_1),\mathit{id}_2,S_2)$: Let $\sigma'_2$ be: if $(a_2,\_) \in \sigma_1$, then $\sigma'_2 = \sigma_1$; else, $\sigma'_2$ is obtained from $\sigma_1$ by inserting $(a_2,\mathit{true})$. Since $\mathit{id}_2 \notin S_1$, it is easy to see that $\sigma_1 {\xrightarrow{o_2}} \sigma'_2 {\xrightarrow{o_1}} \sigma_3$.

\item[-] If $o_1 = (\mathit{read}() \Rightarrow l_1,\mathit{id}_1,S_1)$ and $o_2 = (\mathit{rem}(a_1),\mathit{id}_2,S_2)$: Let $\sigma'_2$ be: if $(a_2,\mathit{false}) \in \sigma_1$, then $\sigma'_2 = \sigma_1$; else, $\sigma'_2$ is obtained from $\sigma_1$ by marking $a_2$ into $\mathit{false}$. Since $\mathit{id}_2 \notin S_1$, it is easy to see that $\sigma_1 {\xrightarrow{o_2}} \sigma'_2 {\xrightarrow{o_1}} \sigma_3$.

\item[-] If $o_1 = (\mathit{read}() \Rightarrow l_1,\mathit{id}_1,S_1)$ and $o_2 = (\mathit{read}() \Rightarrow l_2,\mathit{id}_2,S_2)$: Let $\sigma'_2 = \sigma_1$. Then, it is easy to see that $\sigma_1 {\xrightarrow{o_2}} \sigma'_2 {\xrightarrow{o_1}} \sigma_3$.
\end{itemize}

This completes the proof of this lemma. $\qed$
\end {proof}




The following lemma states that $\mathit{counter}_s$ is a t0-specification.

\begin{lemma}
\label{lemma:counter is a t0-specification}
$\mathit{counter}_s$ is a t0-specification.
\end{lemma}


\begin {proof}

We prove this lemma similarly as that of Lemma \ref{lemma:or-set is a t0-specification}. We need to prove that, given a linearization $\mathit{lin}$ and $o_1,o_2 \in \mathit{lin}$, such that $o_1$ and $o_2$ are concurrent and adjacent in $\mathit{lin}$, then, $l = \mathit{swap}(\mathit{lin},o_1,o_2)$ is also a linearization.

Let $o_1 = (\ell_1,\mathit{id}_1,S_1)$ and $o_2 = (\ell_2,\mathit{id}_2,S_2)$. Since $o_1$ and $o_2$ are concurrent, we know that $\mathit{id}_1 \notin S_2 \wedge \mathit{id}_2 \notin S_1$. Assume $\mathit{lin} = l_1 \cdot o_1 \cdot o_2 \cdot l_2$. Assume in the abstract state of $\mathit{counter}_s$, we have $\sigma_0 {\xrightarrow{l_1}} \sigma_1 {\xrightarrow{o_1}} \sigma_2 {\xrightarrow{o_2}} \sigma_3 {\xrightarrow{l_2}} \sigma_4$, where $\sigma_0$ is the initial state of $\mathit{counter}_s$. Then, we need to prove that, there exists $\sigma'_2$, such that $\sigma_1 {\xrightarrow{o_2}} \sigma'_2 {\xrightarrow{o_1}} \sigma_3$. We prove this by consider all the possible cases:

\begin{itemize}
\setlength{\itemsep}{0.5pt}
\item[-] If $o_1 = (\mathit{inc},\mathit{id}_1,S_1)$ and $o_2 = (\mathit{inc},\mathit{id}_2,S_2)$: Assume that $\sigma_1 = k$, then $\sigma_2 = \mathit{k+1}$ and $\sigma_3 = \mathit{k+2}$. Let $\sigma'_2 = \mathit{k+1}$. Then, it is easy to see that $\sigma_1 {\xrightarrow{o_2}} \sigma'_2 {\xrightarrow{o_1}} \sigma_3$.

\item[-] If $o_1 = (\mathit{inc},\mathit{id}_1,S_1)$ and $o_2 = (\mathit{dec},\mathit{id}_2,S_2)$: Assume that $\sigma_1 = k$, and let $\sigma'_2 = \mathit{k-1}$. Then, it is easy to see that $\sigma_1 {\xrightarrow{o_2}} \sigma'_2 {\xrightarrow{o_1}} \sigma_3$.

\item[-] If $o_1 = (\mathit{inc},\mathit{id}_1,S_1)$ and $o_2 = (\mathit{read}() \Rightarrow k_2,\mathit{id}_2,S_2)$: Let $\sigma'_2 = \sigma_1$. Since $\mathit{id}_1 \notin S_2$, it is easy to see that $\sigma_1 {\xrightarrow{o_2}} \sigma'_2 {\xrightarrow{o_1}} \sigma_3$.

\item[-] If $o_1 = (\mathit{dec},\mathit{id}_1,S_1)$ and $o_2 = (\mathit{inc},\mathit{id}_2,S_2)$: Assume that $\sigma_1 = k$, and let $\sigma'_2 = \mathit{k+1}$. Then, it is easy to see that $\sigma_1 {\xrightarrow{o_2}} \sigma'_2 {\xrightarrow{o_1}} \sigma_3$.

\item[-] If $o_1 = (\mathit{dec},\mathit{id}_1,S_1)$ and $o_2 = (\mathit{dec},\mathit{id}_2,S_2)$: Assume that $\sigma_1 = k$, and let $\sigma'_2 = \mathit{k-1}$. Then, it is easy to see that $\sigma_1 {\xrightarrow{o_2}} \sigma'_2 {\xrightarrow{o_1}} \sigma_3$.

\item[-] If $o_1 = (\mathit{dec},\mathit{id}_1,S_1)$ and $o_2 = (\mathit{read}() \Rightarrow k_2,\mathit{id}_2,S_2)$: Let $\sigma'_2 = \sigma_1$. Since $\mathit{id}_1 \notin S_2$, it is easy to see that $\sigma_1 {\xrightarrow{o_2}} \sigma'_2 {\xrightarrow{o_1}} \sigma_3$.

\item[-] If $o_1 = (\mathit{read}() \Rightarrow k_1,\mathit{id}_1,S_1)$ and $o_2 = (\mathit{inc},\mathit{id}_2,S_2)$: Assume that $\sigma_1 = k$, and let $\sigma'_2 = \mathit{k+1}$. Since $\mathit{id}_2 \notin S_1$, it is easy to see that $\sigma_1 {\xrightarrow{o_2}} \sigma'_2 {\xrightarrow{o_1}} \sigma_3$.

\item[-] If $o_1 = (\mathit{read}() \Rightarrow k_1,\mathit{id}_1,S_1)$ and $o_2 = (\mathit{dec},\mathit{id}_2,S_2)$: Assume that $\sigma_1 = k$, and let $\sigma'_2 = \mathit{k-1}$. Since $\mathit{id}_2 \notin S_1$, it is easy to see that $\sigma_1 {\xrightarrow{o_2}} \sigma'_2 {\xrightarrow{o_1}} \sigma_3$.

\item[-] If $o_1 = (\mathit{read}() \Rightarrow k_1,\mathit{id}_1,S_1)$ and $o_2 = (\mathit{read}() \Rightarrow k_2,\mathit{id}_2,S_2)$: Let $\sigma'_2 = \sigma_1$. Then, it is easy to see that $\sigma_1 {\xrightarrow{o_2}} \sigma'_2 {\xrightarrow{o_1}} \sigma_3$.
\end{itemize}

This completes the proof of this lemma. $\qed$
\end {proof}


With Lemma \ref{lemma:or-set is a t0-specification}, Lemma \ref{lemma:set is a t0-specification} and Lemma \ref{lemma:counter is a t0-specification}, we can now prove Lemma \ref{lemma:several t0-specifications}.


\SeveralTZeroSpecifications*

\begin {proof}
This lemma holds obviously from Lemma \ref{lemma:or-set is a t0-specification}, Lemma \ref{lemma:set is a t0-specification} and Lemma \ref{lemma:counter is a t0-specification}. $\qed$
\end {proof}





\subsection{Proofs of Lemma \ref{lemma:several t1-specifications}}
\label{subsec:appendix proofs of Lemma several t1-specifications}


The following lemma states that $\mathit{list}_s^{\mathit{af}}$ is a t1-specification.

\begin{lemma}
\label{lemma:list-af is a t1-specification}
$\mathit{list}_s^{\mathit{af}}$ is a t1-specification.
\end{lemma}

\begin {proof}

Given a distributed linearizable history $h$ and a linearization $\mathit{lin}$ that is a strict time-stamp order candidate, we need to prove that, each strict time-stamp order candidate $\mathit{lin}'$ is a linearization.

We prove this by showing that each such $\mathit{lin}'$ can be obtained from $\mathit{lin}$ by several times of swapping a pair of adjacent elements. Our proof requires the following two properties:

\begin{itemize}
\setlength{\itemsep}{0.5pt}
\item[-] The first property is: Given a linarization $\mathit{lin}$ that is a strict time-stamp order candidate, and a strict time-stamp order candidate $\mathit{lin}'$. If $\mathit{diff}(\mathit{lin},\mathit{lin}') \neq \emptyset$, there exists $(o_1,o_2) \in \mathit{diff}(\mathit{lin},\mathit{lin}')$, such that $o_1$ and $o_2$ are concurrent, $o_1$ and $o_2$ are adjacent in $\mathit{lin}$, and the time-stamp of $o_1$ in $h$ equals that of $o_2$.

    We prove this by contradiction. Assume $\mathit{diff}(\mathit{lin},\mathit{lin}') \neq \emptyset$, and for each $(o_1,o_2) \in \mathit{diff}(\mathit{lin},\mathit{lin}')$, we have that either $o_1$ and $o_2$ are not concurrent, or $o_1$ and $o_2$ are not adjacent in $\mathit{lin}$, or the time-stamp of $o_1$ in $h$ is different from that of $o_2$.

    By the definition of strict time-stamp order candidate, it is easy to see that if $o_1$ and $o_2$ have different time-stamp, then their order is the same between $\mathit{lin}$ and $\mathit{lin}'$. Therefore, we know that the time-stamp of $o_1$ in $h$ equals that of $o_2$.

    Since $\mathit{diff}(\mathit{lin},\mathit{lin}') \neq \emptyset$, let $(o,o')$ be a element of $\mathit{diff}(\mathit{lin},\mathit{lin}')$, and the distance of $o_1$ and $o_2$ is minimal in $\{$ the distance between $o_1$ and $o_2 \vert (o_1,o_2) \in \mathit{diff}(\mathit{lin},\mathit{lin}') \}$. Let us prove that $o$ and $o'$ are adjacent by contradiction: If there exists $o''$ between $o$ and $o'$. Assume that in $\mathit{lin}$, $o$ is before $o''$, and $o''$ is before $o'$. By assumption, the order between $o$ and $o''$, and between $o''$ and $o'$ is the same in $\mathit{lin}$ and in $\mathit{lin}'$. This implies that $o$ is still before $o'$ in $\mathit{lin}'$, which contradicts the fact that $(o,o') \in \mathit{diff}(\mathit{lin},\mathit{lin}')$.

    Since $o$ and $o'$ are adjacent and $(o,o') \in \mathit{diff}(\mathit{lin},\mathit{lin}')$, by assumption we know that $o$ and $o'$ are not concurrent. Or we can say, $(o,o') \in \mathit{vis} \vee \mathit{o',o} \in \mathit{vis}$. This contradicts that both $\mathit{lin}$ and $\mathit{lin}'$ are consistent with visibility relation. This completes the proof of the first step.

\item[-] The second property is: Given a linearization $\mathit{lin}$ that is a strict time-stamp order candidate, and $o_1,o_2 \in \mathit{lin}$, such that $o_1$ and $o_2$ are concurrent and adjacent in $\mathit{lin}$, and $o_1$ and $o_2$ have the same time-stamp in $h$. Then, $l = \mathit{swap}(\mathit{lin},o_1,o_2)$ is also a linearization and is also a strict time-stamp order candidate. It is obvious that $l$ is still a strict time-stamp order candidate.

    Let $o_1 = (\ell_1,\mathit{id}_1,S_1)$ and $o_2 = (\ell_2,\mathit{id}_2,S_2)$. Since $o_1$ and $o_2$ are concurrent, we know that $\mathit{id}_1 \notin S_2 \wedge \mathit{id}_2 \notin S_1$. Assume $\mathit{lin} = l_1 \cdot o_1 \cdot o_2 \cdot l_2$. Assume in the abstract state of $\mathit{list}_s^{\mathit{af}}$, we have $\sigma_0 {\xrightarrow{l_1}} \sigma_1 {\xrightarrow{o_1}} \sigma_2 {\xrightarrow{o_2}} \sigma_3 {\xrightarrow{l_2}} \sigma_4$, where $\sigma_0$ is the initial state of $\mathit{OR}$-$\mathit{set}_s$. Then, we need to prove that, there exists $\sigma'_2$, such that $\sigma_1 {\xrightarrow{o_2}} \sigma'_2 {\xrightarrow{o_1}} \sigma_3$. We prove this by consider all the possible cases:

    \begin{itemize}
    \setlength{\itemsep}{0.5pt}
    \item[-] If $o_1 = (\mathit{add}(a_1,b_1),\mathit{id}_1,S_1)$ and $o_2 = (\_,\mathit{id}_2,S_2)$: This case is impossible. We can see that the time-stamp of $a$ is larger than operations in $S_1$, and thus, the time-stamp of $o_1$ is the time-stamp of $a$. Since $\mathit{id}_1 \notin S_2$, we know that the time-stamp of $o_2$ is different from that of $o_1$, contradicts the assumption that $o_1$ and $o_2$ have same time-stamp.

    \item[-] If $o_1 = (\_,\mathit{id}_1,S_1)$ and $o_2 = (\mathit{add}(a_2,b_2),\mathit{id}_2,S_2)$: Similarly, we can prove that this case is impossible.

    \item[-] If $o_1 = (\mathit{rem}(a_1),\mathit{id}_1,S_1)$ and $o_2 = (\mathit{rem}(a_2),\mathit{id}_2,S_2)$: Let $\sigma'_2$ be obtained from $\sigma_1$ by marking $a_2$ into $\mathit{false}$. Then, it is easy to see that $\sigma_1 {\xrightarrow{o_2}} \sigma'_2 {\xrightarrow{o_1}} \sigma_3$.

    \item[-] If $o_1 = (\mathit{rem}(a_1),\mathit{id}_1,S_1)$ and $o_2 = (\mathit{read}() \Rightarrow \mathit{list}_1,\mathit{id}_2,S_2)$: Let $\sigma'_2 = \sigma_1$. Since $\mathit{id}_1 \notin S_2$, it is easy to see that $\sigma_1 {\xrightarrow{o_2}} \sigma'_2 {\xrightarrow{o_1}} \sigma_3$.

    \item[-] If $o_1 = (\mathit{read}() \Rightarrow \mathit{list}_1,\mathit{id}_1,S_1)$ and $o_2 = (\mathit{read}() \Rightarrow \mathit{list}_2,\mathit{id}_2,S_2)$: Let $\sigma'_2 = \sigma_1$. Then, it is easy to see that $\sigma_1 {\xrightarrow{o_2}} \sigma'_2 {\xrightarrow{o_1}} \sigma_3$.
    \end{itemize}
\end{itemize}

Based on these two steps, given a linearization $\mathit{lin}$ that is a strict time-stamp order candidate, and a sequence $\mathit{lin}' \neq \mathit{lin}$ that is a strict time-stamp order candidate: We have $\mathit{diff}(\mathit{lin},\mathit{lin}') \neq \emptyset$. Based on the first above property, there exists $(o_1,o_2) \in \mathit{diff}(\mathit{lin},\mathit{lin}')$, such that $o_1$ and $o_2$ are concurrent, and $o_1$ and $o_2$ are adjacent in $\mathit{lin}$, and $o_1$ and $o_2$ have a same time-stamp. Based on the second above property, $\mathit{lin}'' = \mathit{swap}(\mathit{lin},o_1,o_2)$ is also a linearization, and is a strict time-stamp order candidate. Moreover, it is easy to see that $\mathit{diff}(\mathit{lin},\mathit{lin}') > \mathit{diff}(\mathit{lin}'',\mathit{lin}')$. Therefore, by several times of above process, we finally obtain $\mathit{lin}'$ from $\mathit{lin}$ by swapping pairs of operations, and prove that $\mathit{lin}'$ is also a linearization, and is a strict time-stamp order candidate. This completes the proof of this lemma. $\qed$
\end {proof}


The following lemma states that $\mathit{reg}_s$ is a t1-specification.

\begin{lemma}
\label{lemma:reg is a t1-specification}
$\mathit{reg}_s$ is a t1-specification.
\end{lemma}

\begin {proof}

We prove this lemma similarly as that of Lemma \ref{lemma:list-af is a t1-specification}. We need to prove that, given a linearization $\mathit{lin}$ that is a strict time-stamp order candidate, and $o_1,o_2 \in \mathit{lin}$, such that $o_1$ and $o_2$ are concurrent and adjacent in $\mathit{lin}$, and $o_1$ and $o_2$ have the same time-stamp in $h$. Then, $l = \mathit{swap}(\mathit{lin},o_1,o_2)$ is also a linearization and is also a strict time-stamp order candidate. It is obvious that $l$ is still a strict time-stamp order candidate.

Let $o_1 = (\ell_1,\mathit{id}_1,S_1)$ and $o_2 = (\ell_2,\mathit{id}_2,S_2)$. Since $o_1$ and $o_2$ are concurrent, we know that $\mathit{id}_1 \notin S_2 \wedge \mathit{id}_2 \notin S_1$. Assume $\mathit{lin} = l_1 \cdot o_1 \cdot o_2 \cdot l_2$. Assume in the abstract state of $\mathit{reg}_s$, we have $\sigma_0 {\xrightarrow{l_1}} \sigma_1 {\xrightarrow{o_1}} \sigma_2 {\xrightarrow{o_2}} \sigma_3 {\xrightarrow{l_2}} \sigma_4$, where $\sigma_0$ is the initial state of $\mathit{OR}$-$\mathit{set}_s$. Then, we need to prove that, there exists $\sigma'_2$, such that $\sigma_1 {\xrightarrow{o_2}} \sigma'_2 {\xrightarrow{o_1}} \sigma_3$. We prove this by consider all the possible cases:


\begin{itemize}
\setlength{\itemsep}{0.5pt}
\item[-] If $o_1 = (\mathit{write}(a_1),\mathit{id}_1,S_1)$ and $o_2 = (\_,\mathit{id}_2,S_2)$: This case is impossible. We can see that the time-stamp of $a$ is larger than operations in $S_1$, and thus, the time-stamp of $o_1$ is the time-stamp of $a$. Since $\mathit{id}_1 \notin S_2$, we know that the time-stamp of $o_2$ is different from that of $o_1$, contradicts the assumption that $o_1$ and $o_2$ have same time-stamp.

\item[-] If $o_1 = (\_,\mathit{id}_1,S_1)$ and $o_2 = (\mathit{write}(a_2),\mathit{id}_2,S_2)$: Similarly, we can prove that this case is impossible.

\item[-] If $o_1 = (\mathit{read}() \Rightarrow a_1,\mathit{id}_1,S_1)$ and $o_2 = (\mathit{read}() \Rightarrow a_2,\mathit{id}_2,S_2)$: Let $\sigma'_2 = \sigma_1$. Then, it is easy to see that $\sigma_1 {\xrightarrow{o_2}} \sigma'_2 {\xrightarrow{o_1}} \sigma_3$.
\end{itemize}
This completes the proof of this lemma. $\qed$
\end {proof}


With Lemma \ref{lemma:list-af is a t1-specification} and Lemma \ref{lemma:reg is a t1-specification}, we can now prove Lemma \ref{lemma:several t1-specifications}.

\SeveralTOneSpecifications*

\begin {proof}
This lemma holds obviously from Lemma \ref{lemma:list-af is a t1-specification} and Lemma \ref{lemma:reg is a t1-specification}. $\qed$
\end {proof}









\subsection{Proof of Lemma \ref{lemma:several t0-specifications can be composed}}
\label{subsec:appendix proofs of lemma several t0-specifications can be composed}

\composingTZero*
\begin {proof}
Assume that $h = (\mathit{Op},\mathit{ro},\mathit{vis})$. We need to prove that, if $h \uparrow_{\mathit{obj}}$ is distributed linearizable for each object $\mathit{obj}$ of $h$, then $h$ is distributed linearizable.

We construct a linearization $\mathit{lin}$ of $h$ in a process as follows:

\begin{itemize}
\setlength{\itemsep}{0.5pt}
\item[-] Initially a set $\mathit{Op}' = \mathit{Op}$ and $\mathit{lin} = \epsilon$.

\item[-] We begin a loop as follows: In each round of the loop, we choose an operation $o$ that is minimal w.r.t $\mathit{vis}$ in $\mathit{Op}'$, let $\mathit{Op}' = \mathit{Op}' \setminus \{ o \}$, and let $\mathit{lin} = \mathit{lin} \cdot o$.
\end{itemize}

If this process terminates with $\mathit{Op}' = \emptyset$: Then it is easy to see that $\mathit{lin}$ is consistent with $\mathit{vis}$, and thus, for each object $\mathit{obj}$, it is easy to see that $\mathit{lin} \uparrow_{\mathit{obj}}$ is consistent with $\mathit{vis} \uparrow_{\mathit{obj}}$. By the definition of t0-specifications, we know that, for each object $\mathit{obj}$, $\mathit{lin} \uparrow_{\mathit{obj}}$ is a linearization of $h \uparrow_{\mathit{obj}}$. Therefore, $h$ is distributed linearizable.

Let us prove that this process terminates with $\mathit{Op}' = \emptyset$ by contradiction: Assume this process terminates with $\mathit{Op}' \neq \emptyset$, then it is easy to see that $\mathit{vis}^*$ has cycle, which contradicts the assumption that $\mathit{vis}^*$ is acyclic. Therefore, this process terminates with $\mathit{Op}' = \emptyset$. $\qed$
\end {proof}





\subsection{Proof of Lemma \ref{lemma:several t0-specifications and one t1-specification can be composed}}
\label{subsec:appendix proofs of lemma several t0-specifications and one t1-specification can be composed}


\composingTZeroAndOneTOne*
\begin {proof}
Assume that $h = (\mathit{Op},\mathit{ro},\mathit{vis})$. Let $\mathit{obj}_1$ be the only object that uses t1-specification, and let $\mathit{objs}_0$ be the set of other objects. We need to prove that, if $h \uparrow_{\mathit{obj}}$ is distributed linearizable for each object $\mathit{obj}$ of $h$, then $h$ is distributed linearizable.

We construct a linearization $\mathit{lin}$ of $h$ in a process as follows:

\begin{itemize}
\setlength{\itemsep}{0.5pt}
\item[-] Initially a set $\mathit{Op}' = \mathit{Op}$ and $\mathit{lin} = \epsilon$.

\item[-] We begin a loop as follows: in each round of the loop, we choose an operation $o$ shown below, and then let $\mathit{Op}' = \mathit{Op}' \setminus \{ o \}$, and let $\mathit{lin} = \mathit{lin} \cdot o$.

    \begin{itemize}
    \setlength{\itemsep}{0.5pt}
    \item[-] either $o$ is of an operation of $\mathit{objs}_0$ and is minimal w.r.t $\mathit{vis}$ in $\mathit{Op}'$,

    \item[-] or $o$ is of an operation of $\mathit{obj}_1$, is minimal w.r.t $\mathit{vis}$ in $\mathit{Op}'$, and has the minimal time-stamp among operations of $\mathit{obj}_1$ in $\mathit{Op}'$.
    \end{itemize}
\end{itemize}

If this process terminates with $\mathit{Op}' = \emptyset$: Then it is easy to see that $\mathit{lin}$ is consistent with $\mathit{vis}$, and thus, for each object $\mathit{obj}$, it is easy to see that $\mathit{lin} \uparrow_{\mathit{obj}}$ is consistent with $\mathit{vis} \uparrow_{\mathit{obj}}$. It is also easy to see that for operation of $\mathit{obj}_1$, $\mathit{lin}$ is consistent with time-stamp. By the definition of t0-specifications, we know that, for each object $\mathit{obj} \in \mathit{objs}$, $\mathit{lin} \uparrow_{\mathit{obj}}$ is a linearization of $h \uparrow_{\mathit{obj}}$. By the definition of t1-specifications, we know that, $\mathit{lin} \uparrow_{\mathit{obj}_1}$ is a linearization of $h \uparrow_{\mathit{obj}_1}$. Therefore, $h$ is distributed linearizable.

Let us prove that this process terminates with $\mathit{Op}' = \emptyset$ by contradiction: Assume this process terminates with $\mathit{Op}' \neq \emptyset$. Let set $S_1 = \{ o' \vert o'$ is minimal w.r.t $\mathit{vis}$ in $\mathit{Op}'$ $\}$. Then, we can see that, for each operation $o \in S_1$, $o$ is of object $\mathit{obj}_1$, and $o$ does not have minimal time-stamps among operations of $\mathit{obj}_1$ in $\mathit{Op}'$. Let $o_0$ be the operation that is of object $\mathit{obj}_1$ and has minimal time-stamp among operations of $\mathit{obj}_1$ in $\mathit{Op}'$. It is obvious that $o_0 \notin S_1$. Therefore, there exists operations $o_1,\ldots,o_k$, such that $o_1 \in S_1$, $o_1$ is of object $\mathit{obj}_1$, $(o_1,o_2),\ldots,(o_k,o_0) \in \mathit{vis}$. Since the visibility is transitive, we have that $(o_1,o_0) \in \mathit{vis}$. We already know that the time-stamp of $o_0$ is less than that of $o_1$. This contradicts the assumption that time-stamp is consistent with visiblity. Therefore, this process terminates with $\mathit{Op}' = \emptyset$. $\qed$

%Let us prove that this process terminates with $\mathit{Op}' = \emptyset$ by contradiction: Assume this process terminates with $\mathit{Op}' \neq \emptyset$. Let set $S_1 = \{ o' \vert o'$ is minimal w.r.t $\mathit{vis}$ in $\mathit{Op}'$ $\}$. Then, we can see that, for each operation $o \in S_1$, $o$ is of object $\mathit{obj}_1$, and $o$ does not have minimal time-stamps among operations of $\mathit{obj}_1$ in $O'$. Thus, let $o_1 \in S_1$ be the operation that has minimal time-stamp among operations in $S_1$. We can see that there exists $o_2 \in O'$, such that $o_2$ is of object $\mathit{obj}_1$, $o_2 \notin S_1$, and the time-stamp of $o_2$ is less than that of $o_1$. Since $o_2 \notin S_1$, we can see that there exists operation $o_3 \in S_1$, such that $(o_3,o_2) \in \mathit{vis}$. We can see that the time-stamp of $o_1$ is less than that of $o_3$, and thus, the time-stamp of $o_2$ is less than that of $o_3$. Therefore, we found that the time-stamp is not consistent with visibility order for $o_2$ and $o_3$, which contradicts the assumption that time-stamp is consistent with visiblity. $\qed$
\end {proof}






\subsection{Proof of Lemma \ref{lemma:several t0-specifications and several t1-specification can be composed}}
\label{subsec:appendix proofs of lemma several t0-specifications and several t1-specification can be composed}


\composingTZeroAndTOne*
\begin {proof}
Assume that $h = (\mathit{Op},\mathit{ro},\mathit{vis})$. Let $\mathit{objs}_0$ be the set of objects that use t0-specifications in $h$, and let $\mathit{objs}_1$ be the set of objects that use t1-specifications in $h$. We need to prove that, if $h \uparrow_{\mathit{obj}}$ is distributed linearizable for each object $\mathit{obj}$ of $h$, then $h$ is distributed linearizable.

We construct a linearization $\mathit{lin}$ of $h$ in a process as follows:

\begin{itemize}
\setlength{\itemsep}{0.5pt}
\item[-] Initially a set $\mathit{Op}' = \mathit{Op}$ and $\mathit{lin} = \epsilon$.

\item[-] We begin a loop as follows: in each round of the loop, we choose an operation $o$ shown below, and then let $\mathit{Op}' = \mathit{Op}' \setminus \{ o \}$, and let $\mathit{lin} = \mathit{lin} \cdot o$.

    \begin{itemize}
    \setlength{\itemsep}{0.5pt}
    \item[-] either $o$ is of an operation of objects in $\mathit{objs}_0$ and is minimal w.r.t $\mathit{vis}$ in $\mathit{Op}'$,

    \item[-] or $o$ is of an operation of object $\mathit{obj}_1 \in \mathit{objs}_1$, is minimal w.r.t $\mathit{vis}$ in $\mathit{Op}'$, and has the minimal time-stamp among operations of $\mathit{obj}_1$ in $\mathit{Op}'$.
    \end{itemize}
\end{itemize}

If this process terminates with $\mathit{Op}' = \emptyset$: Then it is easy to see that $\mathit{lin}$ is consistent with $\mathit{vis}$, and thus, for each object $\mathit{obj}$, it is easy to see that $\mathit{lin} \uparrow_{\mathit{obj}}$ is consistent with $\mathit{vis} \uparrow_{\mathit{obj}}$. It is also easy to see that for each object $\mathit{ojb}_1 \in \mathit{objs}_1$, $\mathit{lin}$ is consistent with time-stamp of $\mathit{obj}_1$. By the definition of t0-specifications, we know that, for each object $\mathit{obj} \in \mathit{objs}_0$, $\mathit{lin} \uparrow_{\mathit{obj}}$ is a linearization of $h \uparrow_{\mathit{obj}}$. By the definition of t1-specifications, we know that, for each object $\mathit{obj}_1 \in \mathit{objs}_1$, $\mathit{lin} \uparrow_{\mathit{obj}_1}$ is a linearization of $h \uparrow_{\mathit{obj}_1}$. Therefore, $h$ is distributed linearizable.

Let us prove that this process terminates with $\mathit{Op}' = \emptyset$ by contradiction: Assume this process terminates with $\mathit{Op}' \neq \emptyset$. Let set $S_1 = \{ o' \vert o'$ is minimal w.r.t $\mathit{vis}$ in $\mathit{Op}'$ $\}$. Then, we can see that, for each operation $o \in S_1$, there exists a object $\mathit{obj}_1 \in \mathit{objs}_1$, such that $o$ is of $\mathit{obj}_1$, and $o$ does not have minimal time-stamps among operations of $\mathit{obj}_1$ in $\mathit{Op}'$.

Let $S_2 = \{ o \vert \exists \mathit{obj}_1 \in \mathit{objs}_1, o$ is of object $\mathit{obj}_1, o$ has minimal time-stamp among operations of $\mathit{obj}_1$ in $\mathit{Op}' \}$. It is easy to see that $\forall o \in S_2$, $o \notin S_1$.

Thus, it is easy to see that, for each operation $o' \in S_2$, there exists an operation $o \in S_1$ and operations $o'_1,\ldots,o'_k$, such that $(o,o'_1),(o'_1,o'_2),\ldots,(o'_k,o') \in \mathit{vis}$. Since the visibility relation is transitive, we have that $(o,o') \in \mathit{vis}$.

Let $S_3 = \{ (o,o') \vert o \in S_1, o' \in S_2, \exists o'_1,\ldots,o'_k, (o,o'_1),(o'_1,o'_2),\ldots,(o'_k,o') \in \mathit{vis} \}$. Let $S_4 = \{ (\mathit{obj},\mathit{obj}') \vert \exists (o,o') \in S_3$, $o$ is of object $\mathit{obj}$, $o'$ is of object $\mathit{obj}' \}$.

Let us prove that there is a cycle in $S_4$ by contradiction. Given $(\mathit{obj}_2,\mathit{obj}_1) \in S_4$, we know that there is a operation of object of $\mathit{obj}_2$ in $S_1$, and thus, there must exists a operation of object of $\mathit{obj}_2$ in $S_2$. By definition of $S_2$, it is easy to see that there exists $\mathit{obj}_3$, such that $(\mathit{obj}_3,\mathit{obj}_2) \in S_4$. Since $S_4$ has no cycle, we applying this process and finally terminate with $(\mathit{obj}_k,\mathit{obj}_{\mathit{k-1}}),\ldots,(\mathit{obj}_2,\mathit{obj}_1) \in S_4$ and could not found any $\mathit{obj}'$ to make $(\mathit{obj}',\mathit{obk}_k) \in S_4$. However, this implies that there is a operation of $\mathit{obj}_k$ that has minimal time-stamp among operations of $\mathit{obj}_k$ in $\mathit{Op}'$, and is in $S_1$. This contradicts our conclusion that $\forall o \in S_2$, $o \notin S_1$. Therefore, this is a cycle in $S_4$.

Let the cycle in $S_4$ be $(\mathit{obj}_1,\mathit{obj}_k),(\mathit{obj}_k,\mathit{obj}_{\mathit{k-1}}),\ldots,(\mathit{obj}_2,\mathit{obj}_1)$. Then, there exists operations $o^{0}_{\mathit{o1}}, o^{1}_{\mathit{o1}},\ldots, o^{0}_{\mathit{ok}}, o^{1}_{\mathit{ok}}$, such that

\begin{itemize}
\setlength{\itemsep}{0.5pt}
\item[-] $o^{0}_{\mathit{o1}}, o^{1}_{\mathit{o1}}$ is of object $\mathit{obj}_1$, $\ldots$, $o^{0}_{\mathit{ok}}, o^{1}_{\mathit{ok}}$ is of object $\mathit{obj}_k$.

\item[-] $(o^{1}_{\mathit{o1}},o^{0}_{\mathit{ok}}), (o^{1}_{\mathit{ok}},o^{0}_{\mathit{ok-1}})$, $\ldots$, $(o^{1}_{\mathit{o2}},o^{0}_{\mathit{o1}}) \in S_3$.
\end{itemize}

Thus, it is easy to see $(o^{1}_{\mathit{o1}},o^{0}_{\mathit{ok}}), (o^{1}_{\mathit{ok}},$ $o^{0}_{\mathit{ok-1}})$, $\ldots$, $(o^{1}_{\mathit{o2}},o^{0}_{\mathit{o1}}) \in \mathit{vis}$. By definition of $S_2$, we can see that $\mathit{ts}(o^{0}_{\mathit{o1}}) < \mathit{ts}(o^{1}_{\mathit{o1}}), \ldots, \mathit{ts}(o^{0}_{\mathit{ok}}) < \mathit{ts}(o^{1}_{\mathit{ok}})$. This contradicts the definition of causal-time-stamp. Therefore, this process terminates with $\mathit{Op}' = \emptyset$. $\qed$
\end {proof}










\section{For State-based CRDT}
\label{sec:for state-based CRDT}

\begin{example}[List with add-between interface]
\label{definition:sequential specification of list with add-after interface}
Such kind of list is similar as list with add-after interface. One difference is the $\mathit{add}$ method: $\mathit{add}(b,a,c)$ inserts item $b$ into the list at some nondeterministic position between position of $a$ and position of $c$. The other difference is that, we assume that the initial value of list is $(\circ_1,\mathit{true}) \cdot (\circ_2,\mathit{true})$ and these two nodes can not be removed. The sequential specification $\mathit{list}_s^{\mathit{ab}}$ of list is given as follows: Here $\mathit{ab}$ represents add-between. When the context is clear, in $\mathit{read}$ operation, we will omit $\circ_1$ and $\circ_2$.
\begin{itemize}
\setlength{\itemsep}{0.5pt}
\item[-] $\{ \mathit{state} = (a_1,f_1) \cdot \ldots \cdot (a_n,f_n) \wedge k < m < l \wedge b \notin \{ a_1, \ldots, a_n \} \}$ $add(b,a_k,a_l)$ $\{ \mathit{state} = (a_1,f_1) \cdot \ldots \cdot (a_m,f_m) \cdot (b,\mathit{true}) \cdot (a_{m+1},f_{m+1}) \cdot \ldots \cdot (a_n,f_n) \}$. Here the chosen of $m$ is deterministic.
\item[-] $\{ \mathit{state} = (a_1,f_1) \cdot \ldots \cdot (a_n,f_n) \wedge S = \{ a \vert (a,\mathit{true}) \in \mathit{state} \} \wedge l = a_1 \cdot \ldots \cdot a_n \uparrow_{S} \}$ $(read() \Rightarrow l)$ $\{ \mathit{state} = (a_1,f_1) \cdot \ldots \cdot (a_n,f_n) \}$.
\end{itemize}
\end{example}










Given a object $\mathit{obj}$ of a state-based CRDT with $\Sigma$ be the set of local states, we define its semantics as a set of executions generated from an LTS $\llbracket \mathit{obj} \rrbracket_s = (\mathit{Config},\mathit{config}_0,\Sigma',\rightarrow)$ as in \figurename~\ref{fig:the semantics of a state-based CRDT object}.

\begin{figure}[ht]
$\mathit{RState} = \mathbb{R} \rightarrow \Sigma$

$\mathit{TState} = \mathbb{MID} \times \mathbb{MSG} \times \mathbb{R}$.

$\mathit{Config} = \mathit{RState} \times \mathit{TState}$, $\mathit{config}_0 \in \mathit{Config}$.

$\Sigma' = \mathit{do}(\mathbb{M} \times \mathbb{D} \times \mathbb{D} \times \mathbb{R}) \cup \mathit{send}(\mathbb{MID} \times \mathbb{R}) \cup \mathit{receive}(\mathbb{MID} \times \mathbb{R})$

\[
\begin{array}{l c}
\bigfrac{ R(r) = \sigma, r.\mathit{do}(\sigma,m,a) = (\sigma',b) }
{ (R,T) {\xrightarrow{\mathit{do}(m,a,b,r)}} (R[r:\sigma'],T) }
\end{array}
\]


\[
\begin{array}{l c}
\bigfrac{ R(r) = \sigma, \mathit{unique}(\mathit{mid}) }
{ (R,T) {\xrightarrow{\mathit{send}(\mathit{mid},r)}} (R,T \cup \{ (\mathit{mid},\sigma,r) \}) }
\end{array}
\]


\[
\begin{array}{l c}
\bigfrac{ R(r) = \sigma, r.\mathit{receive}(\sigma,\sigma') = \sigma'',(\mathit{mid},\sigma',r') \in T, r \neq r'}
{ (R,T) {\xrightarrow{\mathit{receive}(\mathit{mid},r)}} (R[r:\sigma''],T) }
\end{array}
\]
\caption{The definition of semantics of $\llbracket \mathit{obj} \rrbracket_s$}
\label{fig:the semantics of a state-based CRDT object}
\end{figure}

A configuration $(R,T)$ is a snapshot of distributed system and contains two parts: $R$ gives the local state of each replica, and $T$ gives the set of messages that has been generated. Let $\mathbb{MID}$ be the set of message identifiers of message content. A message is a tuple $(\mathit{mid},\mathit{msg},r)$, where $\mathit{mid} \in \mathbb{MID}$ is the identifier, $\mathit{msg} \in \mathbb{MSG}$ is the message content, and $r$ is the original replica of message. $\mathit{config}_0$ is the initial configuration, which maps each replica into the initial local state, and has no message inside. Since $\mathit{obj}$ is a state-based CRDT, each message content is chosen from $\Sigma$.

Each element of $\Sigma'$ is called an action. $\rightarrow \in \mathit{Config} \times \Sigma' \times \mathit{Config}$ is the transition relation and describe a single step of distributed systems. The first rule in \figurename~\ref{fig:the semantics of a state-based CRDT object} describes replica $r$ performs a operation $m(a) \Rightarrow b$ and works locally. The second rule describes that a replica $r$ may nondeterministically decide to send a message with its local state as message content. Here $\mathit{unique}$ is a function that ensures $\mathit{mid}$ be a fresh message identifier. The third rule describes delivery of a message to a replica $r$ other than its origin replica $r'$.

A sequence $l$ of actions is an execution of $\llbracket \mathit{obj} \rrbracket_s = (\mathit{Config},\mathit{config}_0,\Sigma',\rightarrow)$, if there exists $(R,T) \in \mathit{Config}$, such that $\mathit{config}_0 {\xrightarrow{ l }} (R,T)$. The semantics of $\mathit{obj}$ is defined as the set of executions of $\llbracket \mathit{obj} \rrbracket_s$. Given an execution, when the context is clear, we can associate a unique operation identifier to each action. Or we can say, it is safe to assume each action is in the form of either $\mathit{do}(i,m,a,b,r)$, or $\mathit{send}(i,\mathit{mid},r)$, or $\mathit{receive}(i,\mathit{mid},r)$, where $i \in \mathbb{OID}$ is a unique operation identifier.








Given an execution $l = \alpha_1 \cdot \ldots \cdot \alpha_k$ of $\llbracket \mathit{obj} \rrbracket_s$ of state-based CRDT $\mathit{obj}$, we can obtain a corresponding history $\mathit{history}(l) = (\mathit{Op},\mathit{ro},\mathit{vis})$, such that

\begin{itemize}
\setlength{\itemsep}{0.5pt}
\item[-] Each operation in $\mathit{Op}$ is a tuple $(\ell,i,\mathit{obj})$, such that $i$ is the operation identifier of a $\mathit{do}(m,a,b,r)$ action of $l$.

\item[-] $(o_1,o_2) \in \mathit{ro}$, if they are of same replica, and the index of $o_1$ in $h$ is before that of $o_2$.

\item[-] Let us defines a delivery relation $\mathit{del} \subseteq \mathbb{OP} \times \mathbb{OP}$ as follows: $(o_1,o_2) \in \mathit{del}$, if: $o_1$ and $o_2$ are of different replicas, there exists a $\mathit{send}(\mathit{mid},r)$ action and a $\mathit{receive}(\mathit{mid},r')$ action, $o_1$ and $\mathit{send}(\mathit{mid},r)$ happen on a same replica and $o_1$ happens earlier, $\mathit{receive}(\mathit{mid},r)$ and $o_2$ happen on a same replica and $\mathit{receive}(\mathit{mid},r)$ happens earlier.

\item[-] $\mathit{vis} = (\mathit{ro}+\mathit{del})^*$.
\end{itemize}

Intuitively, each local state can be considered as the consequence of all updates it receives. Since state-based CRDT sends the modified local state as message, the visibility relation is then the transitive closure of replica order and message delivery relation. Let $\mathit{history}(\llbracket \mathit{obj} \rrbracket_s)$ be the set of histories of all executions of $\llbracket \mathit{obj} \rrbracket_s$.






\subsection{Proof Strategy of State-based CRDT}
\label{subsec:proof strategy of operation-based CRDT}

Given a state-based CRDT object $\mathit{obj}$ and a sequential specification $\mathit{spec}$, we need to construct a invariant $\mathit{inv}(\mathit{config},h,\mathit{lin},\mathit{del},\mathit{map})$, where

\begin{itemize}
\setlength{\itemsep}{0.5pt}
\item[-] $\mathit{config}$ is a configuration of $\llbracket \mathit{obj} \rrbracket_s$.

\item[-] $h$ is a history.

\item[-] $h$ is distributed linearizable w.r.t $\mathit{spec}$ and $\mathit{lin}$ is a linearization.

\item[-] $\mathit{del} \subseteq \mathbb{MID} \times \mathbb{R}$ is the message delivery relation.

\item[-] $\mathit{map} \subseteq \mathbb{MID} \times 2^{\mathbb{OID}}$ maps each message $\mathit{mid}$ to a set $S_1$ of operations. Intuitively, $S_1$ is the set of operations whose information are contained in $\mathit{mid}$.
\end{itemize}

$\mathit{inv}(\mathit{config},h,\mathit{lin},\mathit{del},\mathit{map})$ needs to satisfy the following properties:

\begin{itemize}
\setlength{\itemsep}{0.5pt}
\item[-] The visibility of $h$ is transitive.

\item[-] $\mathit{del}$ preserves causal delivery: If $(o_1,o_2) \in \mathit{vis}$ and $(o_2,r) \in \mathit{del}$, then $(o_1,r) \in \mathit{del}$.

\item[-] $\mathit{map}$ preserves causal delivery: Given $o_1,o_3 \in \mathit{map}(\mathit{mid})$, if $\exists o_2$, such that $(o_1,o_2),(o_2,o_3) \in \mathit{vis}$, then $o_2 \in \mathit{map}(\mathit{mid})$.

\item[-] $\mathit{inv}$ holds initially: $\mathit{inv}(\mathit{config}_0,\epsilon,\emptyset,\emptyset,\emptyset)$ holds, where $\mathit{config}_0$ is the initial configuration of $\llbracket \mathit{obj} \rrbracket_s$.

\item[-] $\mathit{inv}$ is a transition invariant:

    \begin{itemize}
    \setlength{\itemsep}{0.5pt}
    \item[-] If $\mathit{inv}(\mathit{config},h,\mathit{lin},\mathit{del},\mathit{map})$ holds and $\mathit{config} {\xrightarrow{\mathit{do}(m,a,b,r)}} \mathit{config}'$, then $\mathit{inv}(\mathit{config}', h \otimes i, \mathit{lin} \cdot i,\mathit{del},\mathit{map})$ holds. Note that here we always put a new operation in the last of linearization.

        Here $i$ is the identifier of the newly-generated $\mathit{do}$ action. Given $h = (\mathit{Op},\mathit{ro},\mathit{vis})$, then, $h \otimes i = (\mathit{Op}',\mathit{ro}',\mathit{vis}')$, where $\mathit{Op}' = \mathit{Op} \cup \{ (m(a) \Rightarrow b,i,\mathit{obj}) \}$, $\mathit{ro}' = \mathit{ro} \cup \{ (j,i) \vert j \in \mathit{Op}, j$ is of replica $r \}$, and $\mathit{vis}' = (\mathit{vis} \cup \{ (j,i) \vert j \in \mathit{Op},(j,r) \in \mathit{del} \} \cup \{ (j,i) \vert j \in \mathit{Op}, j$ is of replica $r \})^*$.

    \item[-] If $\mathit{inv}(\mathit{config},h,\mathit{lin},\mathit{del},\mathit{map})$ holds and $\mathit{config} {\xrightarrow{\mathit{send}(\mathit{mid},r)}} \mathit{config}'$, then $\mathit{inv}(\mathit{config}',h,\mathit{lin},\mathit{del},\mathit{map}')$ holds, where $\mathit{map}' = \mathit{map} \cup (\mathit{mid}, \mathit{vd}(h,\mathit{del},r))$.


    \item[-] If $\mathit{inv}(\mathit{config},h,\mathit{lin},\mathit{del},\mathit{map})$ holds and $\mathit{config} {\xrightarrow{\mathit{receive}(\mathit{mid},r)}} \mathit{config}'$, then $\mathit{inv}(\mathit{config}',h,\mathit{lin},\mathit{del}',\mathit{map})$ holds, where $\mathit{del}' = \mathit{del} \cup \{ (i,r) \vert i \in \mathit{map}(\mathit{mid}) \}$.
    \end{itemize}
\end{itemize}

Here $\mathit{vd}(h,\mathit{del},r) = \{ i \vert (i,j) \in h.\mathit{vis}, j$ is of replica $r \} \cup \{ i \vert (i,r) \in \mathit{del} \}$ is the set of operations that are either to some operation of replica $r$, or has been delivered into replica $r$. An invariant $\mathit{inv}$ satisfies above properties is called invariant of state-based CRDT. The following lemma states that the existence of such invariant implies distributed linearizability.

\begin{lemma}
\label{lemma:invariant of state-based CRDT implies distributed linearizability}
If there exists a invariant $\mathit{inv}$ of state-based CRDT for object $\mathit{obj}$ and sequential specification $\mathit{spec}$, then each history of $\mathit{history}(\llbracket \mathit{obj} \rrbracket_s)$ is distributed linearizable w.r.t $\mathit{spec}$.
\end{lemma}

\begin {proof}
Given an execution $l=\alpha_1 \cdot \ldots \cdot \alpha_n$, let $\mathit{config}_0 {\xrightarrow{\alpha_1}} \mathit{config}_1 \ldots {\xrightarrow{\alpha_n}} \mathit{config}_n$ be the transitions from initial configuration. We need to prove that, for each $1 \leq k \leq n$, we have $\mathit{inv}(\mathit{config}_k,h_k,\mathit{lin}_k,\mathit{del}_k,\mathit{map}_k)$ holds, where $h_k$ is the history of execution $l_k = \alpha_1 \cdot \ldots \cdot \alpha_k$, $\mathit{lin}_k$ is the linearization of $h_k$, $\mathit{del}_k$ records message delivery relation of $l_k$, and $\mathit{map}_k$ records the operations contained in each message in $l_k$.

Since $\mathit{inv}$ holds initially and is a transition invariant, it is easy to prove above requirements by induction on execution. This completes the proof of this lemma. $\qed$
\end {proof}


For many state-based CRDT implementations, $\mathit{inv}((R,T),h,\mathit{lin},\mathit{del},\mathit{map}) = C_1 \wedge C_2$, where

\begin{itemize}
\setlength{\itemsep}{0.5pt}
%\item[-] For each update operation $o$ of $h$, define $\mathit{ds}(o)$ which is a local state. %be the local state of replica $r$ at the time point immediately after $o$ is launched. Here $r$ is the replica of $o$.

\item[-] $C_1: \forall (\mathit{mid},\mathit{msg},\_) \in T$, $\mathit{msg} = \mathit{apply}(\mathit{lin},\mathit{map}(\mathit{mid}))$.

\item[-] $C_2: \forall r$, $R(r) = \mathit{apply}(\mathit{lin},\mathit{vd}(h,\mathit{del},r))$.
\end{itemize}

The function $\mathit{apply}(\mathit{lin},S)$ returns a local state by applying ``virtual messages'' of operations in $S$ according to total order $\mathit{lin}$. Here for each update operation $o$ of $h$, we need to define a local state $\mathit{ds}(o)$, which is the ``virtual messages'' of $o$. Note that state-based CRDT send message randomly, instead of each message for a update operation. This is the reason why we need to manually generate virtual message for each update operation.

To give $\mathit{inv}$, it only remains to give the virtual messages. The virtual message of state-based PN-counter and state-based multi-value register as follows. The proof of them being invariants of state-based CRDT is given in Appendix \ref{subsec:appendix proof of state-based PN-counter} and Appendix \ref{subsec:appendix proof of state-based multi-value register}, respectively.

\begin{example}[virtual messages of state-based PN-counter]
\label{example:virtual messagess of state-based PN-counter}

For each update operation $o$, $\mathit{ds}(o) = (P,N)$, where

\begin{itemize}
\setlength{\itemsep}{0.5pt}
\item[-] $\forall r'$, $P[r'] = \vert \{ o' \vert o'$ is a $\mathit{inc}$ operation of replica $r'$, $o' = o \vee (o',o) \in h.\mathit{vis} \} \vert$.

\item[-] $\forall r'$, $N[r'] = \vert \{ o' \vert o'$ is a $\mathit{dec}$ operation of replica $r'$, $o' = o \vee (o',o) \in h.\mathit{vis} \} \vert$.
\end{itemize}
\end{example}

\begin{example}[virtual messages of state-based Multi-value Register]
\label{example:virtual messages of state-based multi-value register}

For each update operation $o = (\mathit{write}(a),\_,\_)$ of replica $r$, $\mathit{ds}(o) = (a,V)$, where

\begin{itemize}
\setlength{\itemsep}{0.5pt}
\item[-] $\forall r'$, $V[r'] = \vert \{ o' \vert o'$ is a $\mathit{write}$ operation of replica $r'$, $o' = o \vee (o',o) \in h.\mathit{vis} \} \vert$.
\end{itemize}
\end{example}















\subsection{Proof of State-based PN-counter}
\label{subsec:appendix proof of state-based PN-counter}

The following lemma states that each visibility-closed set is a union of operations visible to a set of operations. Its proof is obvious and omitted here.

\begin{lemma}
\label{lemma:a transitive-closed set is a union of visibility of several sets}
Given a set $\mathit{Op}$ of operations and a transitive and acyclic visibility relation $\mathit{vis} \subseteq \mathit{Op} \times \mathit{Op}$, if given a set $S \subseteq \mathit{Op}$, if $S$ satisfies that $\forall o_1,o_2 \in S, o_2 \in S \wedge (o_1,o_2) \in \mathit{vis} \Rightarrow o_1 \in S$, then there exists a set $O \subseteq \mathit{Op}$, such that $S = \cup_{o \in O} \mathit{vis}^{-1}(o)$.
\end{lemma}

The following lemma states that given two operations $o_1,o_2$, for each replica $r$, either the set of operations of replica $r$ visible to $o_1$ is a subset of that of $o_2$, or the set of operations of replica $r$ visible to $o_2$ is a subset of that of $o_1$. Its proof is obvious and omitted here.

\begin{lemma}
\label{lemma:the view of a replica of one operation is contained in another operaiton, or vice versa}
Assume that $\mathit{inv}((R,T),h,\mathit{lin},\mathit{del},\mathit{map})$ holds. Let $S_o^r = \{ o' \vert (o',o) \in \mathit{vis}, o'$ is of replica $r \}$. Then for each operations $o_1$ and $o_2$, and for each replica $r$, $S_{\mathit{o1}}^r \subseteq S_{\mathit{o2}}^r \vee S_{\mathit{o2}}^r \subseteq S_{\mathit{o1}}^r$.
\end{lemma}


Recall that $\mathit{inv} = C_1 \wedge C_2$ with the virtual messages defined as follows: For each update operation $o$, $\mathit{ds}(o) = (P,N)$, where

\begin{itemize}
\setlength{\itemsep}{0.5pt}
\item[-] $\forall r'$, $P[r'] = \vert \{ o' \vert o'$ is a $\mathit{inc}$ operation of replica $r'$, $o' = o \vee (o',o) \in h.\mathit{vis} \} \vert$.

\item[-] $\forall r'$, $N[r'] = \vert \{ o' \vert o'$ is a $\mathit{dec}$ operation of replica $r'$, $o' = o \vee (o',o) \in h.\mathit{vis} \} \vert$.
\end{itemize}

The following lemma states that $\mathit{inv}$ is an invariant of state-based PN-counter.

\begin{lemma}
\label{lemma:inv is an invariant of state-based CRDT for state-based PN-counter}
$\mathit{inv}$ is an invariant of state-based PN-counter.
\end{lemma}

\begin {proof}

It is obvious that $\mathit{inv}(\mathit{config}_0,\epsilon,\emptyset,\emptyset,\emptyset)$ holds.

Let us prove that $\mathit{inv}$ is a transition invariant: assume $\mathit{inv}((R,T),h,\mathit{lin},\mathit{del},\mathit{map})$ holds,

\begin{itemize}
\setlength{\itemsep}{0.5pt}
\item[-] If $(R,T) {\xrightarrow{\mathit{do}(\mathit{inc},r)}} (R',T')$: Then,

    \begin{itemize}
    \setlength{\itemsep}{0.5pt}
    \item[-] It is easy to see that $R' = R[ r: ( R(r).P[r: R(r).P(r)+1 ], R(r).N ) ]$ and $T' = T$.

    \item[-] Let $h' = h \otimes i$, where $i$ is the identifier of the newly-generated $\mathit{inc}$ action.

    \item[-] Let $\mathit{lin}' = \mathit{lin} \cdot (\mathit{inc},i,\mathit{obj})$.

    \item[-] Let $\mathit{del}' = \mathit{del}$ and $\mathit{map}' = \mathit{map}$.
    \end{itemize}

    It is easy to see that $\mathit{lin}'$ is a linearization of $h'$. It is obvious that all other properties hold, except for $C_2$ for replica $r$. Therefore, let us prove that $R'(r) = \mathit{apply}(\mathit{lin}',\mathit{vd}(h',\mathit{del}',r))$.

    Since $R(r) = \mathit{apply}(\mathit{lin},\mathit{vd}(h,\mathit{del},r))$ and $\mathit{lin}' = \mathit{lin} \cdot (\mathit{inc},i,\mathit{obj})$, we know that $\mathit{apply}(\mathit{lin}',\mathit{vd}(h',\mathit{del}',r)) = \mathit{merge}(R(r),\mathit{ds}(i))$. Therefore, we need to prove that $R'(r) = \mathit{merge}(R(r),\mathit{ds}(i))$.

    Since $\mathit{vd}(h,\mathit{del},r)$ satisfies that, $\forall o_1,o_2 \in \mathit{vd}(h,\mathit{del},r), o_2 \in \mathit{vd}(h,\mathit{del},r) \wedge (o_1,o_2) \in \mathit{vis} \Rightarrow o_1 \in \mathit{vd}(h,\mathit{del},r)$, by Lemma \ref{lemma:a transitive-closed set is a union of visibility of several sets}, we know that there exists a set $O$, such that $\mathit{vd}(h,\mathit{del},r) = \cup_{o \in O} \mathit{vis}^{-1}(o)$. By Lemma \ref{lemma:the view of a replica of one operation is contained in another operaiton, or vice versa} and the construction of $\mathit{ds}$, we can see that $R(r) = (P',N')$, where for each replica $r'$, $P'[r'] = \vert \{ j \in \mathit{vd}(h,\mathit{del},r) \uparrow_{\mathit{inc}}$ and $j$ is of replica $r \} \vert$ and $N'[r'] = \vert \{ j \in \mathit{vd}(h,\mathit{del},r) \uparrow_{\mathit{dec}}$ and $j$ is of replica $r \} \vert$.

    We already know that $\mathit{ds}(i) = (P'',N'')$, where for each replica $r'$, $P''[r'] = \vert \{ j \in \mathit{vd}(h',\mathit{del}',r) \uparrow_{\mathit{inc}}$ and $j$ is of replica $r \} \vert$ and $N''[r'] = \vert \{ j \in \mathit{vd}(h',\mathit{del}',r) \uparrow_{\mathit{dec}}$ and $j$ is of replica $r \} \vert$. Then, it is obvious that $\mathit{merge}(R(r),\mathit{ds}(i)) = \mathit{ds}(i)$. It is also easy to see that $\mathit{ds}(i) = (R(r).P[r: R(r).P(r)+1], R(r).N) = R'(r)$. Therefore, $R'(r) = \mathit{merge}(R(r),\mathit{ds}(i))$.

\item[-] If $(R,T) {\xrightarrow{\mathit{do}(\mathit{dec},r)}} (R',T')$: Similarly as that of $(R,T) {\xrightarrow{\mathit{do}(\mathit{inc},r)}} (R',T')$.

\item[-] If $(R,T) {\xrightarrow{\mathit{do}(\mathit{read},k,r)}} (R',T')$: Then,

    \begin{itemize}
    \setlength{\itemsep}{0.5pt}
    \item[-] It is obvious that $R' = R$ and $T' = T$.

    \item[-] Let $h' = h \otimes i$, where $i$ is the identifier of the newly-generated $\mathit{read}$ action.

    \item[-] Let $\mathit{lin}' = \mathit{lin} \cdot ((\mathit{read}() \Rightarrow k,i,\mathit{obj}), \mathit{vd}(h,\mathit{del},r) )$.

    \item[-] Let $\mathit{del}' = \mathit{del}$ and $\mathit{map}' = \mathit{map}$.
    \end{itemize}

    It is easy to see that all other properties hold, except for $h'$ being distributed linearizable w.r.t $\mathit{spec}$ with $\mathit{lin}'$ the linearization. Let us prove that $h'$ is distributed linearizable w.r.t $\mathit{spec}$ and $\mathit{lin}'$ is a linearization. It is easy to see that only operation $i$ need to be checked.

    It is easy to see that $\mathit{vd}(h,\mathit{del},r) = \mathit{vis}^{-1}(i)$. Similarly as the case of $(R,T) {\xrightarrow{\mathit{do}(\mathit{inc},r)}} (R',T')$, we can prove that $R(r) = (P',N')$, where for each replica $r'$, $P'[r'] = \vert \{ j \in \mathit{vd}(h,\mathit{del},r) \uparrow_{\mathit{inc}}$ and $j$ is of replica $r \} \vert = \vert \{ j \in \mathit{vis}^{-1}(i) \uparrow_{\mathit{inc}}$ and $j$ is of replica $r \} \vert$ and $N'[r'] = \vert \{ j \in \mathit{vd}(h,\mathit{del},r) \uparrow_{\mathit{dec}}$ and $j$ is of replica $r \} \vert = \vert \{ j \in \mathit{vis}^{-1}(i) \uparrow_{\mathit{dec}}$ and $j$ is of replica $r \} \vert$. Since $k = \Sigma_{r'} P��[r'] - \Sigma_{r'} N'[r']$, $k$ is obtained by minus the number of all visible $\mathit{dec}$ of $i$ from the number of all visible $\mathit{inc}$ of $i$. Therefore, we can see that $((\mathit{read}() \Rightarrow k,i,\mathit{obj}), \mathit{vd}(h,\mathit{del},r) )$ of $\mathit{lin}'$ is ``correct''. Then, $h'$ is distributed linearizable w.r.t $\mathit{spec}$ and $\mathit{lin}'$ is a linearization.

\item[-] If $(R,T) {\xrightarrow{\mathit{send}(\mathit{mid},r)}} (R',T')$: Then,

    \begin{itemize}
    \setlength{\itemsep}{0.5pt}
    \item[-] It is obvious that $R' = R$. Let $T' = T \cup \{ (\mathit{mid},R(r),r) \}$.

    \item[-] Let $h' = h$.

    \item[-] Let $\mathit{lin}' = \mathit{lin}$.

    \item[-] Let $\mathit{del}' = \mathit{del}$.

    \item[-] Let $\mathit{map}' = \mathit{map} \cup \{ (\mathit{mid},\mathit{vd}(h,\mathit{del},r)) \}$.
    \end{itemize}

    It is easy to see that all other properties hold, except for checking $C_1$ for $\mathit{mid}$. This holds obviously since the message content of message $\mathit{mid}$ is $R(r)$, and we already know that $R(r) = \mathit{apply}(\mathit{lin},\mathit{vd}(h,\mathit{del},r)) = \mathit{apply}(\mathit{lin},\mathit{map}(\mathit{mid}))$.

\item[-] If $(R,T) {\xrightarrow{\mathit{receive}(\mathit{mid},r)}} (R',T')$: Then,

    \begin{itemize}
    \setlength{\itemsep}{0.5pt}
    \item[-] Let $R' = R[ r: \mathit{merge}(R(r),\mathit{msg})]$ where $(\mathit{mid},\mathit{msg},\_) \in T$. It is obvious that $T' = T$.

    \item[-] Let $h' = h$.

    \item[-] Let $\mathit{lin}' = \mathit{lin}$.

    \item[-] Let $\mathit{del}' = \mathit{del} \cup \{ (i,r) \vert i \in \mathit{map}(\mathit{mid}) \}$.

    \item[-] Let $\mathit{map}' = \mathit{map}$.
    \end{itemize}

    It is easy to see that all other properties hold, except for $C_2$ for replica $r$. Therefore, let us prove that $R'(r) = \mathit{apply}(\mathit{lin}',\mathit{vd}(h',\mathit{del}',r))$.

    We already know that $R'(r) = \mathit{merge}(R(r), \mathit{msg})$, $R(r) = \mathit{apply}(\mathit{lin},\mathit{vd}(h,\mathit{del},r))$ and $\mathit{msg} = \mathit{apply}(\mathit{lin},\mathit{map}(\mathit{mid}))$. It is easy to see that $\mathit{vd}(h',\mathit{del}',r) = \mathit{vd}(h,\mathit{del},r) \cup \mathit{map}(\mathit{mid})$. It is easy to prove that, applying messages in any order lead to the same consequence. Therefore, we have $\mathit{merge}(R(r), \mathit{msg}) = \mathit{apply}(\mathit{lin}',\mathit{vd}(h,\mathit{del},r) \cup \mathit{map}(\mathit{mid}))$. Then, we have $R'(r) = \mathit{apply}(\mathit{lin}',\mathit{vd}(h',\mathit{del}',r))$.
\end{itemize}

This completes the proof of this lemma. $\qed$
\end {proof}




\subsection{Proof of State-based Multi-value Register}
\label{subsec:appendix proof of state-based multi-value register}

Recall that $\mathit{inv} = C_1 \wedge C_2$ with the virtual messages defined as follows: For each update operation $o$, $\mathit{ds}(o) = (a,V)$, where

\begin{itemize}
\setlength{\itemsep}{0.5pt}
\item[-] $\forall r'$, $V[r'] = \vert \{ o' \vert o'$ is a $\mathit{write}$ operation of replica $r'$, $o' = o \vee (o',o) \in h.\mathit{vis} \} \vert$.
\end{itemize}

The following lemma states that $\mathit{inv}$ is an invariant of state-based multi-value register.

\begin{lemma}
\label{lemma:inv is an invariant of state-based CRDT for state-based multi-value register}
$\mathit{inv}$ is an invariant of state-based multi-value register.
\end{lemma}

\begin {proof}

It is obvious that $\mathit{inv}(\mathit{config}_0,\epsilon,\emptyset,\emptyset,\emptyset)$ holds.

Let us prove that $\mathit{inv}$ is a transition invariant: assume $\mathit{inv}((R,T),h,\mathit{lin},\mathit{del},\mathit{map})$ holds,

\begin{itemize}
\setlength{\itemsep}{0.5pt}
\item[-] If $(R,T) {\xrightarrow{\mathit{do}(\mathit{write},a,r)}} (R',T')$: Then,

    \begin{itemize}
    \setlength{\itemsep}{0.5pt}
    \item[-] $R' = R[ r: \{ (a,V') \} ], R(r).N)]$ and $T' = T$. Here $\forall r' \neq r, V'[r'] = \mathit{max} \{ V_1(r��) \vert (\_,V_1) \in R(r) \}$, and $V'[r] = \mathit{max} \{ V_1(r) \vert (\_,V_1) \in R(r) \} + 1$.

    \item[-] Let $h' = h \otimes i$, where $i$ is the identifier of the newly-generated $\mathit{inc}$ action.

    \item[-] Let $\mathit{lin}' = \mathit{lin} \cdot (\mathit{inc},i,\mathit{vis}^{-1}(i))$.

    \item[-] Let $\mathit{del}' = \mathit{del}$ and $\mathit{map}' = \mathit{map}$.
    \end{itemize}

    It is easy to see that $\mathit{lin}'$ is a linearization of $h'$. It is obvious that all other properties hold, except for $C_2$ for replica $r$. Therefore, let us prove that $R'(r) = \mathit{apply}(\mathit{lin}',\mathit{vd}(h',\mathit{del}',r))$.

    It is easy to see that $\mathit{vd}(h',\mathit{del}',r) = h'.\mathit{vis}^{-1}(i)$. And then, we need to prove that $(a,V') = \mathit{apply}(\mathit{lin}',h'.\mathit{vis}^{-1}(i))$.

    Recall that $R(r) = \mathit{apply}(\mathit{lin},\mathit{vd}(h,\mathit{del},r))$, from Lemma \ref{lemma:a transitive-closed set is a union of visibility of several sets}, we know that there exists set $O$, such that $\mathit{vd}(h,\mathit{del},r) = \cup_{o \in O} \mathit{vis}^{-1}(o)$. We can prove that, for each $o = \mathit{write}(b)$, $\mathit{apply}(\mathit{lin},\mathit{vis}^{-1}(o)) = (b,V_b)$, where $\forall r' \neq r, V_b[r'] = \vert \{ o' \vert o' \in \mathit{vis}^{-1}(o), o'$ is of replica $r' \} \vert$, and $V_b[r] = \vert \{ o' \vert o' \in \mathit{vis}^{-1}(o), o'$ is of replica $r' \} \vert + 1$.

    It is not hard to prove that the order of merging virtual message is not important, and a virtual message can be applied multiple times. By Lemma \ref{lemma:the view of a replica of one operation is contained in another operaiton, or vice versa}, we can see that $\mathit{apply}(\mathit{lin},\mathit{vd}(h,\mathit{del},r))$ is obtained by merging $\{ o \in O \vert \mathit{apply}(\mathit{lin},\mathit{vis}^{-1}(o)) \}$. Therefore, we can see that $\mathit{apply}(\mathit{lin}',h'.\mathit{vis}^{-1}(i)) = \mathit{apply}(\mathit{lin}',\mathit{vd}(h',\mathit{del}',r))$ is obtained by merging $\{ o \in O \vert \mathit{apply}(\mathit{lin},\mathit{vis}^{-1}(o)) \} \cup \{ \mathit{ds}(i) \}$. By Lemma \ref{lemma:the view of a replica of one operation is contained in another operaiton, or vice versa}, it is not hard to see that $\mathit{apply}(\mathit{lin}',h'.\mathit{vis}^{-1}(i)) = \mathit{ds}(i)$.

    Then, we need to prove that $(a,V') = \mathit{ds}(i)$. This holds since $R(r) = \mathit{apply}(\mathit{lin},\mathit{vd}(h,\mathit{del},r))$ is obtained by merging $\{ o \in O \vert \mathit{apply}(\mathit{lin},\mathit{vis}^{-1}(o)) \}$, Lemma \ref{lemma:the view of a replica of one operation is contained in another operaiton, or vice versa}, and the value of $V'$.

\item[-] If $(R,T) {\xrightarrow{\mathit{do}(\mathit{read},S,r)}} (R',T')$: Then,

    \begin{itemize}
    \setlength{\itemsep}{0.5pt}
    \item[-] It is obvious that $R' = R$ and $T' = T$.

    \item[-] Let $h' = h \otimes i$, where $i$ is the identifier of the newly-generated $\mathit{read}$ action.

    \item[-] Let $\mathit{lin}' = \mathit{lin} \cdot (\mathit{read}() \Rightarrow S,i,\mathit{vd}(h,\mathit{del},r) )$.

    \item[-] Let $\mathit{del}' = \mathit{del}$ and $\mathit{map}' = \mathit{map}$.
    \end{itemize}

    It is easy to see that all other properties hold, except for $h'$ being distributed linearizable w.r.t $\mathit{spec}$ with $\mathit{lin}'$ the linearization. Let us prove that $h'$ is distributed linearizable w.r.t $\mathit{spec}$ and $\mathit{lin}'$ is a linearization. It is easy to see that only operation $i$ need to be checked.

    It is easy to see that $\mathit{vd}(h,\mathit{del},r) = h'.\mathit{vis}^{-1}(i)$. Similarly as the case of $(R,T) {\xrightarrow{\mathit{do}(\mathit{write},a,r)}} (R',T')$, we can prove that there exists a set $O$, such that $R(r) = \mathit{apply}(\mathit{lin},\mathit{vd}(h,\mathit{del},r))$ is obtained by merging $\{ o \in O \vert \mathit{apply}(\mathit{lin},\mathit{vis}^{-1}(o)) \}$.

    By the definition of merging, it is same to assume that $O = \mathit{max}_{\mathit{vis}} \mathit{vd}(h,\mathit{del},r)$. Assume that for each operation $o = \mathit{write}(a) \in O$, $\mathit{apply}(\mathit{lin},\mathit{vis}^{-1}(o))) = (a,V_o)$. Then it is not hard to see that $R(r) = \{ (a,V_o) \vert o = \mathit{write}(a) \in O \}$. Therefore, $S = \{ a \vert o = \mathit{write}(a) \in \mathit{vis}^{-1}(i), \forall o' = \mathit{write}(\_) \in \mathit{vis}^{-1}(i), (o,o') \notin \mathit{vis} \}$. According to sequential specification $\mathit{spec}$, $(\mathit{read} \Rightarrow S,i,\mathit{obj})$ of $\mathit{lin}'$ is ``correct''. Then, $h'$ is distributed linearizable w.r.t $\mathit{spec}$ and $\mathit{lin}'$ is a linearization.

\item[-] If $(R,T) {\xrightarrow{\mathit{send}(\mathit{mid},r)}} (R',T')$: Then,

    \begin{itemize}
    \setlength{\itemsep}{0.5pt}
    \item[-] It is obvious that $R' = R$. Let $T' = T \cup \{ (\mathit{mid},R(r),r) \}$.

    \item[-] Let $h' = h$.

    \item[-] Let $\mathit{lin}' = \mathit{lin}$.

    \item[-] Let $\mathit{del}' = \mathit{del}$.

    \item[-] Let $\mathit{map}' = \mathit{map} \cup \{ (\mathit{mid},\mathit{vd}(h,\mathit{del},r)) \}$.
    \end{itemize}

    It is easy to see that all other properties hold, except for checking $C_1$ for $\mathit{mid}$. This holds obviously since the message content of message $\mathit{mid}$ is $R(r)$, and we already know that $R(r) = \mathit{apply}(\mathit{lin},\mathit{vd}(h,\mathit{del},r)) = \mathit{apply}(\mathit{lin},\mathit{map}(\mathit{mid}))$.

\item[-] If $(R,T) {\xrightarrow{\mathit{receive}(\mathit{mid},r)}} (R',T')$: Then,

    \begin{itemize}
    \setlength{\itemsep}{0.5pt}
    \item[-] Let $R' = R[ r: \mathit{merge}(R(r),\mathit{msg})]$ where $(\mathit{mid},\mathit{msg},\_) \in T$. It is obvious that $T' = T$.

    \item[-] Let $h' = h$.

    \item[-] Let $\mathit{lin}' = \mathit{lin}$.

    \item[-] Let $\mathit{del}' = \mathit{del} \cup \{ (i,r) \vert i \in \mathit{map}(\mathit{mid}) \}$.

    \item[-] Let $\mathit{map}' = \mathit{map}$.
    \end{itemize}

    It is easy to see that all other properties hold, except for $C_2$ for replica $r$. Therefore, let us prove that $R'(r) = \mathit{apply}(\mathit{lin}',\mathit{vd}(h',\mathit{del}',r))$.

    We already know that $R'(r) = \mathit{merge}(R(r), \mathit{msg})$, $R(r) = \mathit{apply}(\mathit{lin},\mathit{vd}(h,\mathit{del},r))$ and $\mathit{msg} = \mathit{apply}(\mathit{lin},\mathit{map}(\mathit{mid}))$. It is easy to see that $\mathit{vd}(h',\mathit{del}',r) = \mathit{vd}(h,\mathit{del},r) \cup \mathit{map}(\mathit{mid})$. It is easy to prove that, applying messages in any order lead to the same consequence. Therefore, we have $\mathit{merge}(R(r), \mathit{msg}) = \mathit{apply}(\mathit{lin}',\mathit{vd}(h,\mathit{del},r) \cup \mathit{map}(\mathit{mid}))$. Then, we have $R'(r) = \mathit{apply}(\mathit{lin}',\mathit{vd}(h',\mathit{del}',r))$.
\end{itemize}

This completes the proof of this lemma. $\qed$
\end {proof}
}


\end{document}

%%% Local Variables:
%%% mode: latex
%%% TeX-master: t
%%% End:
