\documentclass{llncs}
\usepackage[english]{babel}

\usepackage{hyperref}

\usepackage{epsfig}
\usepackage{amsmath}
\usepackage{color}
\usepackage{amsfonts,amssymb}
\usepackage{mathabx}
\usepackage{verbatim}
\usepackage{times}
\usepackage[linesnumbered,ruled,procnumbered,noend]{algorithm2e}
\usepackage{stmaryrd}

%added by Wang Chao%
%\usepackage{mathrsfs}
%\usepackage{extarrows}
\usepackage{thmtools}
\usepackage{thm-restate}

%add for TIkz
\usepackage[version=0.96]{pgf}
\usepackage{tikz}
\usetikzlibrary{arrows,shapes,snakes,automata,backgrounds,petri}
\usepackage{xcolor}
\usepackage{paralist}

\usepackage[margin,draft]{fixme}
\fxsetup{theme=color,mode=multiuser}
\FXRegisterAuthor{gp}{agp}{GP}
\FXRegisterAuthor{cw}{acw}{CW}

\newcommand{\gpn}[2][]{\gpnote*{#1}{#2}}
\newcommand{\gp}[1]{\gpnote{#1}}


\DeclareSymbolFont{largesymbolsA}{U}{txexa}{m}{n}
\SetSymbolFont{largesymbolsA}{bold}{U}{txexa}{bx}{n}
\DeclareFontSubstitution{U}{txexa}{m}{n}
\DeclareMathSymbol{\bigsqcupplus}{\mathop}{largesymbolsA}{"02}

\title{CRDT Linearizability}

\author{}
\institute{}

\author {Constantin Enea \and Gustavo Petri \and Chao Wang}
\institute{Institut de Recherche en Informatique Fondamentale, \\Univ. Paris Diderot (Paris 7)}

% GP: Why do we have all these macros?

\newcommand\eqdef{\stackrel{\text{def}}{=}}

%\newcommand{\nondsum}{\bigbox}
\newcommand{\nondsum}{\bigsqcupplus}
\newcommand{\probplus}[1]{\oplus_{#1}}
%\newcommand{\nondplus}{\square}
\newcommand{\bang}{!\,}
\newcommand{\nondplus}{{\textstyle\bigsqcupplus}}
\newcommand{\partmap}{\rightharpoonup}
\newcommand{\map}{\rightarrow}
%\newcommand{\exc}{\phi}
\newcommand{\exc}{\alpha}
\newcommand{\exec}{\mathit{exec}}
\newcommand{\execp}{\mathit{execp}}
\newcommand{\Act}{\mathit{Act}}
\newcommand{\Sec}{\mathit{Sec}}
\newcommand{\Obs}{\mathit{Obs}}
\newcommand{\etree}{\mathit{etree}}
\newcommand{\lstate}{\mathit{lst}}
\newcommand{\fstate}{\mathit{fst}}
%\newcommand{\STATE}{\mathcal{P}r}  $ original marked
%\newcommand{\STATE}{\mathcal{P}}   $ marked by me
%\newcommand{\st}{P}
\newcommand{\trans}{\mathcal{T}}
\newcommand{\Aut}{\mathcal{M}}
\newcommand{\init}{\mathit{init}}
\newcommand{\perr}{\mathcal{P}}

\newcommand{\calo}{\mathcal{O}}
\newcommand{\cals}{\mathcal{S}}
\newcommand{\sseq}{\vec s}
\newcommand{\oseq}{\vec o}
\newcommand{\ccsp}{CCS$_p$}

\newcommand{\bigfrac}[2]{\frac{\raisebox{1ex}{$#1$}}{\raisebox{-1.5ex}{$#2$}}}
\newcommand{\nondarr}[1]{\overset{#1}{\longrightarrow}}
\newcommand{\Nondarr}[1]{\overset{#1}{\Longrightarrow}}
\newcommand{\vectorArrow}[1]{\stackrel{\longrightarrow}{\mbox{#1}}}
\newcommand{\probarr}[1]{\overset{#1}{\dashrightarrow}}
\newcommand{\paral}{\,|\,}
\newcommand{\outp}[1]{\overline{#1}}
\renewcommand{\Pr}{{\rm Pr}}

%Commands by Chao Wang

%memory models%
\newcommand{\TSO}{\textrm{TSO}}
\newcommand{\PSO}{\textrm{PSO}}

%correctness conditions%
\newcommand{\lin}{\textrm{linearizability}}
\newcommand{\slin}{\textrm{static linearizability}}
\newcommand{\qlin}{\textrm{quasi linearizability}}
\newcommand{\TTlin}{\textrm{TSO-to-TSO linearizability}}

\newcommand{\pair}[2]{\langle #1 , #2 \rangle}% pairs
\newcommand{\setof}[2]{\{ \, #1 \mid #2 \, \}}% Sets
\newcommand{\set}[1]{\{ {#1}  \}  }
%\newcommand{\map}[3]{{#1} \colon {#2} \longmapsto {#3}} %functions
\newcommand{\den}[1]{[\![#1]\!]}% Denotation of
\newcommand{\mean}[1]{|\!|#1|\!|}
\newcommand{\forget}[1]{}

%%%%%%%%%%%%%GENERAL%%%%%%%%%%%%%%%%%%%%%%%%%%%

\newcommand{\itbox}[1]{{\it #1\/}}
\newcommand{\un}[1]{\uline{#1}}%\underline{#1}}
\newcommand{\ov}[1]{\overline{#1}}
\newcommand{\smallspace}{\vspace{10mm}}
\newcommand{\is}{\mbox{$\Longleftarrow\ $}}
\newcommand{\pright}[1]{\hfill{#1}}
\newcommand{\bnfor}{\;\;\mid\;\;}

\newcommand{\ar}[1]{\stackrel{\scriptstyle #1}{\longrightarrow}}

%%%%%%%%%italics in math mode
%%%%%%%%%%%%%%%%%%%%%%%%%%%%%%%%%%%%%%%%%

%\newcommand{\true}{{\it true}}
%\newcommand{\false}{{\it false}}
\newcommand{\calB}{{\cal B}}
\newcommand{\calF}{{\cal F}}
\newcommand{\calP}{{\cal P}}
\newcommand{\order}{{\cal O}}
\newcommand{\size}[1]{|#1|}

%other notations%
\newcommand{\LTS}{\textit{LTS}}
\newcommand{\bedt}[1]{{\color{blue}#1}}
\newcommand{\redt}[1]{{\color{red}#1}}

\newcommand{\todo}[1]{{\bf \color{red}{#1}}}

%%%%%%%%%%%%% CRDTs %%%%%%%%%%%%%%%%%%%%%%%%%%%
\newcommand{\crdtlin}{RA-linearizability}
\newcommand{\crdtlinearization}{RA-linearization}
\newcommand{\crdtlinearizable}{RA-linearizable}
\newcommand{\CRDTLin}{Replication-Aware Linearizability}
\newcommand{\CRDTLinshort}{RA-linearizability}

\newcommand{\hwlin}{{\color{magenta} hw-linearizability}}
\newcommand{\hwlinear}{{\color{magenta} hw-linear}}
\newcommand{\hwlinearization}{{\color{magenta} hw-linearization}}
\newcommand{\hwlinearizable}{{\color{magenta} hw-linearizable}}
\newcommand{\HWLin}{{\color{magenta} HW-Linearizability}}



%%%%%%%%%%%%% CRDTs %%%%%%%%%%%%%%%%%%%%%%%%%%%

\newcommand{\aobj}{\ensuremath{\mathtt{o}}}
\newcommand{\objs}{\ensuremath{\mathbb{O}\mathsf{bs}}}
\newcommand{\arep}{\ensuremath{\mathtt{r}}}
\newcommand{\reps}{\ensuremath{\mathbb{R}\mathsf{eps}}}
\newcommand{\amethod}{\ensuremath{\mathsf{m}}}
\newcommand{\ainput}{\ensuremath{\mathsf{a}}}
\newcommand{\areturn}{\ensuremath{\mathsf{r}}}
\newcommand{\atimestamp}{\ensuremath{\mathsf{t}}}
\newcommand{\methods}{\ensuremath{\mathbb{M}\mathsf{th}}}
\newcommand{\histories}{\ensuremath{\mathbb{H}\mathsf{ist}}}
\newcommand{\datadomain}{\ensuremath{\mathbb{D}\mathsf{om}}}
\newcommand{\timestampdomain}{\ensuremath{\mathbb{T}\mathsf{om}}}
\newcommand{\acrdttyp}{\ensuremath{\mathsf{t}}}
\newcommand{\amethodset}{\ensuremath{\mathsf{M}}}
\newcommand{\adomain}{\ensuremath{\mathsf{D}}}
\newcommand{\atsdomain}{\ensuremath{\mathsf{T}}}
\newcommand{\powerset}[1]{\ensuremath{\mathcal{P}(#1)}}
\newcommand{\astate}{\ensuremath{\sigma}}
\newcommand{\abstate}{\ensuremath{\phi}}
\newcommand{\abstates}{\ensuremath{\Phi}}
\newcommand{\states}{\ensuremath{\Sigma}}
\newcommand{\amesg}{\ensuremath{\mathsf{m}}}
\newcommand{\messages}{\ensuremath{\mathcal{M}\mathsf{sg}}}
\newcommand{\amesgset}{\ensuremath{\mathsf{M}}}
\newcommand{\aop}{\ensuremath{\mathsf{op}}}
\newcommand{\ops}{\ensuremath{\mathbb{O}\mathsf{ps}}}
\newcommand{\aopid}{\ensuremath{\aop_{\mathsf{id}}}}
\newcommand{\opids}{\ensuremath{\mathbb{O}\textsf{p}\mathsf{ID}}}
\newcommand{\argv}{\ensuremath{a}}
\newcommand{\retv}{\ensuremath{b}}
\newcommand{\alabellong}[3][\amethod]{\ensuremath{#1(#2) \Rightarrow
    #3}}
\newcommand{\alabellongind}[4][\amethod]{\ensuremath{#1(#2) \overset{#4}{\Rightarrow}
    #3}}
\newcommand{\alabelobjind}[4][{\aobj.\amethod}]{\ensuremath{{#1}(#2) \overset{#4}{\Rightarrow}
    #3}}
\newcommand{\alabel}{\ensuremath{\ell}}
\newcommand{\labels}{\ensuremath{\mathbb{L}\mathsf{ab}}}
\newcommand{\alabelset}{\ensuremath{\mathsf{L}}}
\newcommand{\arepord}{\ensuremath{\mathsf{ro}}}
\newcommand{\avisord}{\ensuremath{\mathsf{vis}}}
\newcommand{\aseqord}{\ensuremath{\mathsf{seq}}}
\newcommand{\absopsemplain}{\ensuremath{\Rightarrow}}
\newcommand{\absopsem}[2][\alabel]{\ensuremath{\delta_{#1}(#2)}}
\newcommand{\apre}{\ensuremath{\mathsf{pre}}}
\newcommand{\ts}{\ensuremath{\mathit{ts}}}


\newcommand{\Spec}{\ensuremath{\mathsf{Spec}}}
\newcommand{\updates}{\ensuremath{\mathsf{Updates}}}
\newcommand{\queries}{\ensuremath{\mathsf{Queries}}}
\newcommand{\Updates}{\ensuremath{\mathsf{Updates}}}
\newcommand{\Queries}{\ensuremath{\mathsf{Queries}}}
\newcommand{\queryupdates}{\ensuremath{\mathsf{Query\text{-}Updates}}}

\newcommand{\effector}{\ensuremath{\delta}}
\newcommand{\semop}[2][\aop]{\ensuremath{\llbracket #1
    \rrbracket}\ifthenelse{\isempty{#2}}{}{(#2)}}
\newcommand{\localstates}{\ensuremath{\mathsf{LC}}}
\newcommand{\globalstates}{\ensuremath{\mathsf{GC}}}
\newcommand{\gstates}{\ensuremath{\mathsf{G}}}
\newcommand{\dom}[1]{\ensuremath{\mathsf{dom}(#1)}}
\newcommand{\labeldom}[1]{\ensuremath{\mathsf{labels}(#1)}}
\newcommand{\downstreams}{\ensuremath{\mathsf{DS}}}




\newcommand{\alabellongNoret}[2][\amethod]{\ensuremath{#1(#2)}} % m(a)

\newcommand{\alabellongNoArg}[2][\amethod]{\ensuremath{#1 \Rightarrow
    #2}} % m() \Rightarrow b

\newcommand{\ats}{\ensuremath{\mathtt{ts}}} % a timestamp
\newcommand{\tss}{\ensuremath{\mathbb{T}\mathsf{s}}} % set of timestamps 
\newcommand{\atsource}{\ensuremath{\theta}} % atsource
\newcommand{\aglobalstate}{\ensuremath{\mathtt{gc}}} % a global state
\newcommand{\aexec}{\ensuremath{\mathtt{e}}} % an execution

%%%%%%%%%%%%%%%%%%%%%%%%%%%%%%%%%%%%%%%%%%%%%%


%%% Local Variables:
%%% mode: latex
%%% TeX-master: t
%%% End:


\begin{document}

\maketitle

\begin{abstract}
  In this paper we consider the problem of the specification and
  verification of Commutative Replicated Data Types (CRDTs).
  %
  We provide a new specification and correctness criterion for CRDTs
  akin to Linearizability as defined by~\cite{HerlihyW90}.
  %
  We argue that this criterion is both simple to understand, and it
  fits most well known implementations of known CRDTs.
  %
  Then we show how to prove that CRDT implementations can be
  formally proved to be correct w.r.t. their specification.
  %
  In particular, we do so for many well known implementations taken
  from~\cite{ShapiroPBZ11}.
  %
  We conclude the paper by showing how our criterion can be leveraged
  to reason modularly about the composition of CRDTs.
\end{abstract}

%%% Local Variables:
%%% mode: latex
%%% TeX-master: "draft"
%%% End:


%!TEX root = draft.tex
\section{Introduction}
\label{sec:introduction}

Introduction.

\gpn{Use Grow-Only Counter (GOC) to motivate the problem.}
%

%%% Local Variables:
%%% mode: latex
%%% TeX-master: "draft"
%%% End:


%!TEX root = draft.tex

\section{CRDT Implementations}
\label{sec:CRDT implementations}

\textblue{
Here we define:
\begin{itemize}
\item the notations required for labels: $\aobj.\alabellongind{\argv}{\retv}{i,\ts}$ (partition queries/updates)
\item the notion of history: $(\alabelset, \avisord)$
\item the notion of sequential specification: a set of sequences $(\alabelset, \aseqord)$. Talk about per-object specification (and a representation of these specs based on pre/post conditions) and a specification for a set of objects (defined by interleavings)
\item the notion of \CRDTLin{}: label rewriting + linearization of the visibilities (make it in one shot for both - the intuition should be clear from the overview). Also, in one shot for multiple objects, given that we already defined specifications for multiple objects.
\item should we talk here about non-determinism and convergence ?
  (maybe left for later in a discussion section)
  \gpnote*{}{Maybe talk about it when we introduce WOOT}
\end{itemize}}


\begin{figure}
  \centering

  \(
  \begin{array}[t]{rcll}
    \aobj & \in  & \objs & \text{Objects} \\
    \arep & \in & \reps & \text{Replicas} \\
    \amethod & \in & \methods & \text{Methods}\\
    \adomain & \subseteq & \datadomain & \text{Data Domain} \\
    \acrdttyp & \in & \powerset{\methods} \times \powerset{\adomain} & \text{CRDT definition} \\
    \astate & \in & \states & \text{States} \\
    \amesg & \in & \messages & \text{Messages} \\
    \amesgset & \subseteq & \messages & \text{Message Set} \\
    \aop & \in & \ops \equiv \opids & \text{Operations}\ (ID) \\
    \alabellong[\amethod]{\argv}{\retv} \equiv \alabel  & \in & \methods \times \adomain \times \adomain \equiv \labels & \text{Operation labels} \\
    \alabelset & \subseteq & \labels & \text{Label Set}\\
    \arepord & \subseteq & \ops \times \ops & \text{Replica Order} \\
    \avisord & \subseteq & \ops \times \ops & \text{Visibility Order}
  \end{array}
  \)
  \caption{Summary of notations}
  \label{fig:notations}
\end{figure}


To formalize our CRDT correctness criterion we will introduce the
following semantic domains.
%
We let $\aobj \in \objs$ be an object in the set of objects $\objs$. Similarly,
$\arep \in \mathbb{R}$ is a replica in the set of replicas
$\reps$.
We assume that both objects and replicas are uniquely identified, and
therefore we will equate an object with its identifier, with the same
convention applying to replicas.

We will assume that the specification of a CRDT is given by a set of
method names $\amethod \in \methods$, and that each method has a
number of arguments sampled from a data domain $\datadomain$, and a
return value also from the data domain.
We ignore here the issues of typing which should be ensured by the
underlying programming language.
Hence, a data type $\acrdttyp = (\amethodset, \adomain)$ is given by a
set of method names $\amethodset \subseteq \methods$ and a data domain
$\adomain \subseteq \datadomain$.

%A distributed system contains multiple objects, and each objects is replicated on each replica. Each object has a type, which contains its method and data type. A client of a replica interact with the objects by calling the method and then obtaining the return value. Here we do not bound the number of replica identifiers and objects.

% Let $\objs$ be the set of objects and $\mathbb{R}$ be the set of replica identifiers. We consider a finite set $\mathbb{M}$ of method names; and a possibly infinite set $\mathbb{D}$ of arguments and return values, the data domain. Each data type $t = (M,D)$ has a set $M \subseteq \mathbb{M}$ of methods and a data domain $D \subseteq \mathbb{D}$.

Without loss of generality we will consider that the methods in $\mathbb{M}$ can be separated in two disjoint sets of methods: $\mathbb{Q}$ query methods that has no influence on the ``abstract state'' and normally returns an observation of the ``abstract state'' , and $\mathbb{U}$ update methods that has influence on the ``abstract state''. %Note that some update operation also need to read the ``abstract state''. For example, a $add(a,b)$ operation is an operation of distributed list which intends to put item $a$ immediately after item $b$. This operation implicitly requires that item $b$ is already in list.

$\mathit{Optimistic \ replication \ algorithms}$ is a type of distributed algorithms where each client contains a copy of data structure; a client operations takes effect instantly at its replica without any synchronization, and then broadcast to other replicas and got applied. Convergent or Commutative Replicated Data Types (CRDTs) is a typical kinds of optimistic replication algorithms. In this section, we will introduce CRDT algorithms and their formation \footnote{To be exact, there are two kinds of CRDT algorithms: state-based and operation based. The implementations we discussed in this section is operation-based. The state-based CRDT can be similarly formalized and verified, and we discuss state-based CRDT in the discussion section of this paper.}.

In CRDT, each query operation works locally, while each update operation will inform other replica about its update. We takes a algorithm, replicated growable array (RGA), as an example of operation-based CRDT and it is shown below.


\begin{lstlisting}[caption={Pseudo-code of the Replicated Growable Array (RGA) CRDT}, captionpos=b,label={lst:rga}]
  payload Ti-Tree N, Set Tomb
  initial N = @|$\emptyset$|@, Tomb = @|$\emptyset$|@

  addAfter(a,b) :
    atSource :
      precondition : b = @|$\circ$|@ or (b != @|$\circ$|@ and (b,_,_) @|$\in$|@ N and b @|$\not\in$|@ Tomb)
      let ts@|$_{\mathtt{a}}$|@ = (N == @|$\emptyset$|@)?(1,myID()):(max{c | (_,(_,c),_) @|$\in$|@ N},myID())
      let ts@|$_{\mathtt{b}}$|@ = (b == @|$\circ$|@)?(0,r@|$_{0}$|@):(timestamp of b in N)
    downStream(a, ts@|$_{\mathtt{a}}$|@, ts@|$_{\mathtt{b}}$|@) :
      precondition: b = @|$\circ$|@ or (b != @|$\circ$|@ and (b, ts@|$_{\mathtt{b}}$|@,_) @|$\in$|@ N)
      N = N ts@|$\cup$|@ {(a, ts@|$_{\mathtt{a}}$|@, ts@|$_{\mathtt{b}}$|@)}

  remove(a) :
    atSource :
      precondition : a != @|$\emptyset$|@ and (a,_,_) @|$\in$|@ N and a @|$\notin$|@ Tomb
    downStream(a) :
      precondition : a != @|$\emptyset$|@ and (a,_,_) @|$\in$|@ N
      Tomb = Tomb @|$\cup$|@ {a}

  read() :
    return traverse(N, Tomb)
\end{lstlisting}


\renewcommand{\algorithmcfname}{CRDT Implementation}
\noindent
%\begin{minipage}{.5\textwidth}
\noindent% \begin{algorithm}[H]
% $\mathit{payload}$ TI-tree N, set $\mathit{Tomb}$; \\
% $\mathit{initial}$ $\emptyset$,$\emptyset$; \\

% $add(a,b)$ \\
% \ \ $\mathit{atSource}$: \\
% \ \ \ \ $\mathit{pre}$: \ $b = \circ \vee$ %( b \neq \circ \wedge (b,\_,\_) \in N \wedge b \notin \mathit{Tomb})$ \\

% % \ \ \ \ \If {$N = \emptyset$}
% % { \ \ \ \ let \ $ts_a$ = (myID(),1); \\ }
% % \ \ \ \ \Else
% % {\ \ \ \ let \ $ts_a$ = (myID(),$\mathit{max}\{ c' \vert (\_,(\_,c'),\_) \in N \} +1$); \\ }

% % \ \ \ \ \%If {$b = \circ$}
% %    { \ \ \ \ let \ $ts_b$ = (0,0); \\ }
% %\ \ \ \ \Else
% %    { \ \ \ \ let \ $ts_b$ be time-stamp of $b$ in $N$; \\ }

% \ \ \ \ let \ $ts_a$ = ($N = \emptyset$) ? (1,$\mathit{myID}$()) ! ($\mathit{max}\{ c' \vert (\_,(\_,c'),\_) \in N \} +1$,$\mathit{myID}$()); \\
% \ \ \ \ let \ $ts_b$ = ($b = \circ$) ? (0,$r_0$) ! (the time-stamp of $b$ in $N$); \\

% \ \ $\mathit{downstream}(a,ts_a,ts_b)$: \\
% \ \ \ \ $\mathit{pre}$: \ $b = \circ \vee ( b \neq \circ \wedge (b,ts_b,\_) \in N)$ \\

% \ \ \ \ $N = N \cup \{ (a,ts_a,ts_b) \}$.


%Let $\mathbb{MSG}$ be the set of message contents, such as $(a,ts_a,ts_b)$ of RGA. Then, CRDT implementations are defined as follows, where operations and receiving messages are defined as functions.
%
%\begin{definition}[CRDT implementations]
%\label{definition:operation-based CRDT implementations}
%A CRDT implementation for a type $t = (M,D)$ is a tuple $I(r) = (\Sigma, \Sigma_0, \mathit{Msg}, \mathit{do},\mathit{receive})$. Here $r \in \mathbb{R}$, $\Sigma_0 \subseteq \Sigma$, $\mathit{Msg} \subseteq \mathbb{MSG}$, $\mathit{do}:\Sigma \times M \times D \rightarrow \Sigma \times D \times (\mathit{Msg} \cup \{ \emptyset \} )$, and $\mathit{receive}: \Sigma \times \mathit{Msg} \rightarrow \Sigma$.
%\end{definition}
%
%Here $\Sigma$ is the set of local states and $\Sigma_0$ is the set of initial state. $r$ is the replica identifier of current replica, and some CRDT implementation requires current replica identifier to generate time-stamp. When the current local state is $\sigma$ and the client intends to perform a operation of method $m$ with argument $a$, a $\mathit{do}$ action is launched, which update the local states, returns a value, and generate message if $m$ is a update method. When this replica receives a message of other replica, a $\mathit{receive}$ action will be launched, which updates the current local states according to the message. If a operation has no arguments or return value, or does not generate message, then we can safely omit the corresponding tuples in $\mathit{do}$ action.
%% The formal definition of RGA can be easily obtained from its algorithms. For example, when $(a,\_,\_) \in N$, we have $\mathit{do}((N,\mathit{Tomb}),\mathit{rem},a) = ((N,\mathit{Tomb} \cup \{ a \}),a)$. Here we use $\_$ in indicate a element whose value is irrelevant.
%The formal definition of RGA, as well as more CRDT implementations, can be found in  Appendix \ref{sec:appendix definitions of section CRDT implementations}.

















\forget{
\section{CRDT Implementations}
\label{sec:CRDT implementations}

A distributed system contains multiple objects, and each objects is replicated on each replica. Each object has a type, which contains its method and data type. A client of a replica interact with the objects by calling the method and then obtaining the return value. Here we do not bound the number of replica identifiers and objects.

Let $\mathbb{OBJ}$ be the set of objects and $\mathbb{R}$ be the set of replica identifiers. We consider a finite set $\mathbb{M}$ of method names; and a possibly infinite set $\mathbb{D}$ of arguments and return values, the data domain. Each data type $t = (M,D)$ has a set $M \subseteq \mathbb{M}$ of methods and a data domain $D \subseteq \mathbb{D}$. Finally we have a infinite set $\mathbb{OID}$ of operation identifiers, corresponding to each individual operation performed on the CRDT throughout an execution.

Without loss of generality we will consider that the methods in $\mathbb{M}$ can be separated in two disjoint sets of methods: $\mathbb{Q}$ query methods that has no influence on the ``abstract state'' and normally returns an observation of the ``abstract state'' , and $\mathbb{U}$ update methods that has influence on the ``abstract state''. Note that some update operation also need to read the ``abstract state''. For example, a $add(a,b)$ operation is an operation of distributed list which intends to put item $a$ immediately after item $b$. This operation implicitly requires that item $b$ is already in list.

$\mathit{Optimistic \ replication \ algorithms}$ is a type of distributed algorithms where each client contains a copy of data structure; a client operations takes effect instantly at its replica without any synchronization, and then broadcast to other replicas and got applied. Convergent or Commutative Replicated Data Types (CRDTs) is a typical kinds of optimistic replication algorithms. In this section, we will introduce CRDT algorithms and their formation.

In practice, there are two kinds of CRDT implementations: state-based CRDT and operation-based CRDT. In state-based CRDT, a update operation take effects locally; in nondeterministic time, a replica may decide to send the (modified) local state into other replicas. The state-based PN-counter is an example of state-based CRDT algorithms and is shown below. Keyword $\mathit{payload}$ indicate the local state, and keyword $\mathit{initial}$ specifies the initial value of local state. Function $\mathit{myID}()$ returns the current replica identifier, and $\mathit{reps}()$ returns the number of replicas of the distributed system. Vector $P$ (resp., $N$) is a vector such that $P[i]$ (resp., $N[i]$) is the number of increase that is generated by replica $i$ and is observed by current replica. This algorithm assumes that the set of replica is already known and is fixed and finite.

Method $\mathit{inc}$ increase the counter by $1$, method $\mathit{dec}$ decrease the counter by $1$, and method $\mathit{read}$ returns the current counter value. Assume the replica identifier of current replica is $r$. When the current replica does $\mathit{inc}$, it modify $P[r]$ into $P[r]+1$. When the current replica does $\mathit{dec}$, it modify $N[r]$ into $N[r]+1$. When the current replica does $\mathit{read}$, it returns $\Sigma_{i}^{n} P[i] - \Sigma_{i}^{n} N[i]$. When the current replica receive a message of modified payload $Z$, it uses function $\mathit{merge}()$ to update the current local state. $\mathit{merge}$ takes the maximum of each replica in the vector.

\renewcommand{\algorithmcfname}{CRDT Implementation}
\noindent
%\begin{minipage}{.5\textwidth}
\noindent\begin{algorithm}[H]
$\mathit{payload}$ integer[$\mathit{reps}$()] P, integer[$\mathit{reps}$()] N; \\
$\mathit{initial}$ [0,\ldots,0],[0,\ldots,0]; \\

$\mathit{inc}()$ \\
%\ \ \ \ let \ g = myID();\\
\ \ \ \ P[$\mathit{myID}$()] = P[$\mathit{myID}$()] + 1; \\

$\mathit{dec}()$ \\
%\ \ \ \ let \ g = myID();\\
\ \ \ \ N[$\mathit{myID}$()] = N[$\mathit{myID}$()] + 1; \\

$\mathit{read}()$ \\
\ \ \ \ \KwRet $\Sigma_{i}^{n} P[i] - \Sigma_{i}^{n} N[i]$; \\

$\mathit{merge}(Z)$ \\
\ \ \ \ $\forall i$, $P[i] = \mathit{max}(P[i],Z.P[i])$; \\
\ \ \ \ $\forall i$, $N[i] = \mathit{max}(N[i],Z.N[i])$; \\
\caption{State-based PN-counter}
\label{Method1}
\end{algorithm}

In operation-based CRDT, an update operation not only updates its local state, but also sends a description of this operation into other replica. Here we take a more complex algorithm, replicated growable array (RGA), as an example of operation-based CRDT and it is shown below.

\renewcommand{\algorithmcfname}{CRDT Implementation}
\noindent
%\begin{minipage}{.5\textwidth}
\noindent\begin{algorithm}[H]
$\mathit{payload}$ TI-tree N, set $\mathit{Tomb}$; \\
$\mathit{initial}$ $\emptyset$,$\emptyset$; \\

$add(a,b)$ \\
\ \ $\mathit{atSource}$: \\
\ \ \ \ $\mathit{pre}$: \ $b = \circ \vee ( b \neq \circ \wedge (b,\_,\_) \in N \wedge b \notin \mathit{Tomb})$ \\

%\ \ \ \ \If {$N = \emptyset$}
%    { \ \ \ \ let \ $ts_a$ = (myID(),1); \\ }
%\ \ \ \ \Else
%    {\ \ \ \ let \ $ts_a$ = (myID(),$\mathit{max}\{ c' \vert (\_,(\_,c'),\_) \in N \} +1$); \\ }

%\ \ \ \ \If {$b = \circ$}
%    { \ \ \ \ let \ $ts_b$ = (0,0); \\ }
%\ \ \ \ \Else
%    { \ \ \ \ let \ $ts_b$ be time-stamp of $b$ in $N$; \\ }

\ \ \ \ let \ $ts_a$ = ($N = \emptyset$) ? (1,$\mathit{myID}$()) ! ($\mathit{max}\{ c' \vert (\_,(\_,c'),\_) \in N \} +1$,$\mathit{myID}$()); \\
\ \ \ \ let \ $ts_b$ = ($b = \circ$) ? (0,$r_0$) ! (the time-stamp of $b$ in $N$); \\

\ \ $\mathit{downstream}(a,ts_a,ts_b)$: \\
\ \ \ \ $\mathit{pre}$: \ $b = \circ \vee ( b \neq \circ \wedge (b,ts_b,\_) \in N)$ \\

\ \ \ \ $N = N \cup \{ (a,ts_a,ts_b) \}$.


$rem(a)$ \\
\ \ $\mathit{atSource}$: \\
\ \ \ \ $\mathit{pre}$: \ $a \neq \circ \wedge (a,\_,\_) \in N \wedge a \notin \mathit{Tomb}$ \\

\ \ $\mathit{downstream}(a)$: $\mathit{pre}$ \ $a \neq \circ \wedge (a,\_,\_) \in N)$

\ \ \ \ $\mathit{Tomb} = \mathit{Tomb} \cup \{ a \}$.

$read()$ \\
\ \ \ \ \KwRet $\mathit{trans}(N,\mathit{Tomb})$; \\

\caption{RGA}
\label{Method1}
\end{algorithm}

Each update operation of operation-based CRDT ie executed with two phases: Its first phase, marked $\mathit{atSource}$, is local to the current replica. It is enabled if its (optional) pre-condition, marked $\mathit{pre}$, is true currently in local state. It generates the information to be delivered, which is the argument of $\mathit{downstream}$. Note that this phase does not modify the local state. Its second phase, marked $\mathit{downstream}$, executed immediate after the current replica, and asynchronously at other replica when they receive the message of this operation. It is enabled if its (optional) pre-condition is true.

In RGA algorithm, a replica store the list as a timestamp insertion tree (TI-tree) $N$, and stores the deleted items in tombstone $\mathit{Tomb}$. A TI-tree $N$ is a set of tuples $(a,t,p)$, where $a$ is a item, $t$ is its unique time-stamp, and $p$ is the time-stamp of its ``parent'' node. Each time-stamp is a tuple $(c,r)$ with $c \in \mathbb{N}$ and $r \in \mathbb{R}$. A order $<_{\mathit{ts}}$ between time-stamps is defined, such that $(c_1,r_1) <_{\mathit{ts}} (c_2,r_2)$, if $c_1 < c_2 \vee (c_1 = c_2 \wedge r_1 <_r r_2)$, where $<_r$ is a total-order over $\mathbb{R}$. There is a pre-existed item $\circ$ of TI-tree with time stamp $(0,r_0)$, which are considered as the root of the tree. Each element of $N$ should have unique item and time stamp, and the elements of $N$ are required to form a tree by following the parent field. The tombstone $\mathit{Tomb}$ is a set of items and records items been removed from the list.

Method $\mathit{add}(a,b)$ intends to add item $a$ into the list immediately after a existing item $b$. Method $\mathit{rem}(a)$ removes $a$ from the list. Method $\mathit{read}$ returns the current list content. When the current replica does $\mathit{add}(a,b)$, it generate a tuple $(a,ts_a,ts_b)$ and put it into $N$. Here $ts_b$ is the time-stamp of $b$, and $ts_a$ is a new time-stamp that is larger than any time stamp in $N$. When the current replica does $\mathit{rem}(a)$, it put $a$ into tombstone. When the current replica does $\mathit{read}$, it uses function $\mathit{trans}(N,\mathit{Tomb})$ to return the list seen by the current replica, which is a sequences obtained by traversing $N$ in prefix order (children are visited in decreasing time-stamp order) and keeping only items that are not in $\mathit{Tomb}$.


Multi-value register is also a common-used data structures and its sequential specification is nondeterministic and thus different from that of the previous two examples, which are deterministic (seen in the next section). A state-based multi-value register algorithm is shown below.


\renewcommand{\algorithmcfname}{CRDT Implementation}
\noindent
%\begin{minipage}{.5\textwidth}
\noindent\begin{algorithm}[H]
$\mathit{payload}$ $S \subseteq D \times \mathbb{N}^{\mathit{reps}()}$; \\
$\mathit{initial}$ $\emptyset$; \\

$\mathit{write}(a)$ \\
\ \ \ \ let \ g = $\mathit{myID}$(); \\
\ \ \ \ let $\mathcal{V} = \{ V \vert \exists x, (x,V) \in S \}$; \\
\ \ \ \ let $V' = [ \mathit{max}_{V \in \mathcal{V}} V[j] ]_{j \neq g}$; \\
\ \ \ \ let $V'[g] = (\mathit{max}_{V \in \mathcal{V}} V[g]) + 1$; \\
\ \ \ \ $S = (a,V')$; \\

$\mathit{read}()$ \\
\ \ \ \ \KwRet $S' = \{ a \vert (a,\_) \in S \}$; \\

$\mathit{merge}(Z)$ \\
\ \ \ \ let $A' = \{ (x,V) \in S \vert \forall (x',V') \in Z.S, \exists i, V[i] \geq V'[i] \}$; \\
\ \ \ \ let $B' = \{ (x,V) \in Z.S \vert \forall (x',V') \in S, \exists i, V[i] \geq V'[i] \}$; \\
\ \ \ \ $S = A' \cup B'$; \\
\caption{state-based multi-value register}
\label{Method1}
\end{algorithm}

Each replica stores a set $S$ of items such that each item can not dominate other items. To do conflict resolution, we associate each item $a$ in $S$ with a version vector $V$. We say version vector $V$ dominates version vector $V'$, if $\forall i$, $V[i] > V'[i]$.

Method $\mathit{write}(a)$ intends to write $a$ into register. Method $\mathit{read}$ returns the current register content. When the current replica does $\mathit{write}(a)$, it generates a new version vector that dominates all previous ones in $S$. When the current replica does $\mathit{read}$, it returns the set of items in $S$. When the current replica receive a message of modified payload $Z$, we takes the union of every items in $S$ and $Z.S$ whose version vector is not dominated by that of an item in the other set. This algorithm assumes that the set of replica is already known and is fixed and finite.


To enable formally verification of CRDT algorithms, it is necessary to give formal definition of CRDT-algorithms. Let $\mathbb{MSG}$ be the set of message contents, such as $(P,N)$ of state-based PN-counter, or $(a,ts_a,ts_b)$ of RGA. Then, CRDT implementations are defined as follows, where operations and receiving messages are defined as functions.

\begin{definition}[operation-based CRDT implementations]
\label{definition:operation-based CRDT implementations}
A operation-based CRDT implementation for a type $t = (M,D)$ is a tuple $I_t(r) = (\Sigma, \Sigma_0, \mathit{Msg}, \mathit{do},\mathit{receive})$. Here $r \in \mathbb{R}$, $\Sigma_0 \subseteq \Sigma$, $\mathit{Msg} \subseteq \mathbb{MSG}$, $\mathit{do}:\Sigma \times M \times D \rightarrow \Sigma \times D \times (\mathit{Msg} \cup \{ \emptyset \} )$, and $\mathit{receive}: \Sigma \times \mathit{Msg} \rightarrow \Sigma$.
\end{definition}

Here $\Sigma$ is the set of local states and $\Sigma_0$ is the set of initial state. For example, in state-based PN-counter, since there are many possibility of total number of replicas, $\Sigma_0$ is a set of more than one elements. $r$ is the replica identifier of current replica. The reason of containing $r$ in the definition of CRDT implementations is that, some algorithms need the current replica identifier to generate time-stamp. When the current local state is $\sigma$ and the client intends to perform a operation of method $m$ with argument $a$, a $\mathit{do}$ action is launched, which update the local states, returns a value, and possibly generate messages. A $\mathit{do}$ action of update method will generate messages, while a $\mathit{do}$ action of query method will not generate message. When this replica receives a message of other replica, a $\mathit{receive}$ action will be launched, which updates the current local states according to the message. If a operation has no arguments or return value, or does not generate message, then we can safely omit the corresponding tuples in $\mathit{do}$ actions.

\begin{definition}[state-based CRDT implementations]
\label{definition:state-based CRDT implementations}
A state-based CRDT implementation for a type $t = (M,D)$ is a tuple $I_t(r) = (\Sigma, \Sigma_0, \mathit{Msg}, \mathit{do},\mathit{receive})$. Here $r \in \mathbb{R}$, $\Sigma_0 \subseteq \Sigma$, $\mathit{Msg} \subseteq \Sigma$, $\mathit{do}:\Sigma \times M \times D \rightarrow \Sigma \times D$, and $\mathit{receive}: \Sigma \times \Sigma \rightarrow \Sigma$.
\end{definition}

The state-based CRDT implementation is similarly defined. The difference is that, each operation does not send message, and the message content is fixed to be a local state. The following is an example of formal definition of state-based PN-counter. The formal definition of more CRDT implementations are given in Appendix \ref{sec:appendix definitions of section CRDT implementations}. Here we denote by $f[i:j]$ the function that has the same value as $f$ everywhere, except for $i$, where it has the value $j$. %Since each operation is executed without synchronization, it is not hard to obtain formal definition from informal algorithms.

\begin{example}[formal definition of state-based PN-counter]
\label{definition:formal definition of state-based PN-counter}
$I_t(r) = (\Sigma, \sigma_0, \mathit{Msg}, \mathit{do},\mathit{receive})$, where

\begin{itemize}
\setlength{\itemsep}{0.5pt}
\item[-] $\Sigma = \{ (P,N) \vert$, $P$ and $N$ are vector of integers with same length $\}$. $\Sigma_0 = \{ (P_0,N_0) \vert (P_0,N_0) \in \Sigma$, $P_0$ and $N_0$ maps each index into $0 \}$.

\item[-] $\mathit{Msg} \subseteq \Sigma$.

\item[-] $\mathit{do}((P,N),\mathit{inc}) = (P[r:P[r]+1],N)$,

\item[-] $\mathit{do}((P,N),\mathit{dec}) = (P,N[r:N[r]+1])$,

\item[-] $\mathit{receive}((P,N),(P',N')) = (\lambda s. \mathit{max}\{  P[s], N'[s] \}, \mathit{max}\{  N[s], N'[s] \},)$,
\end{itemize}
\end{example}
}































%%% Local Variables:
%%% mode: latex
%%% TeX-master: "draft"
%%% End:


%!TEX root = draft.tex
%\newcommand{\seqPQ}{\mathsf{SeqPQ}}


\section{Definition of Linearizability}
\label{sec:definition of linearizability} 

Let us start our formation of executions, specifications and linearizations of CRDT.  

We consider a finite set $\mathbb{M}$ of method names; and a possibly infinite set $\mathbb{D}$ of arguments and return values, the data domain. Without loss of generality we will consider that the methods in $\mathbb{M}$ can be separated in two disjoint sets of $\mathbb{Q}$ query methods, and $\mathbb{U}$ update methods. We consider replicated data types which are distributed across a set of replicas; the set of replica identifiers is denoted by $\mathbb{R}$. We assume that each replica contains a copy of the data type state. Finally we have a infinite set $\mathbb{O}$ of operation identifiers, corresponding to each individual operation performed on the CRDT throughout an execution.

Operation labels \mbox{$m(a)\Rightarrow b$} with $m \in \mathbb{M}$ and $a,b \in \mathbb{D}$, indicate that the operation is a call to method $m$ with argument $a$ and the result of the operation is the value $b$. When $m$ does not use the argument (resp., return value), we write $m()\Rightarrow b$ (resp., $m(a)$) instead. We define an operation $o$ to be a tuple $(\ell,i)$, where $\ell$ is an operation label and $i \in \mathbb{O}$ is a unique operation identifier. 

\noindent {\bf Sequential Specification:} A sequential specification is used to state the sequential intuition of operations. Let specification alphabet $\mathbb{A}$ be a set of a tuples $(o,s)$, where $o$ is an operation and $s$ is a set of operation identifiers. The reason of introducing such set of operation identifiers is that, in sequential specifications some operations should only influence a subset of previous operations. Such operation identifier set is just the operations influenced by this operation. With specification alphabet, a sequential specification of CRDT is given as a set of sequences over specification alphabets as follows. From now on, we implicitly assume that in a sequence of specification alphabets, each item has a unique operation identifier. 

\begin{definition}[Sequential Specification]
\label{definition:sequential specification} 
A sequential specification $\mathit{spec}_s \subseteq \mathbb{A}^*$ is a set of strings over specification alphabet $\mathbb{A}$. 
\end{definition} 

\noindent {\bf Distributed Specification:} Each element of distributed specification is a tuple $(s,v)$, where $s \subseteq \mathbb{A}^*$ is a sequence of specification alphabet, and $v$ is a function that maps each item $a$ of $s$ into a subset of items before $a$ in $s$. Here $s$ essentially is the linearization of an execution, while $v$ is used to ensure each specification alphabet is correct in this sequence. %Given a sequence $s \subseteq \mathbb{A}^*$ of specification alphabets and a specification alphabets $a$, let $\mathit{itm}(s)$ be the set of specification alphabets of $s$, and $\mathit{itmBef}(s,a)$ be the set of specification alphabets of $s$ that appears before $a$, or $\emptyset$ otherwise. 
Given a sequence $s$ and a set of operations $S$, let $s \uparrow_{S}$ be the projection of $s$ over $S$. Given a sequence $s$ and an item $a$ of $s$, let $\mathit{bef}(s,a)$ contains the set of items of $s$ that appear before $a$ in $s$, as well as all the subsets of this set. 

\begin{definition}[Distributed Specification]
\label{definition:distributed specification}
A distributed specification $\mathit{spec}_d$ w.r.t a sequential specification $\mathit{spec}_d$ is a set of tuples $(s,v)$, where $s \subseteq \mathbb{A}^*$, and %$v: \mathit{itm}(s) \rightarrow 2^{\mathit{items}(s)}$ is a function that maps each item $a$ into $\mathit{itmBef}(s,a)$.
%$v$ is a function that maps each specification alphabet $a$ of $s$ into a subset of specification alphabet of $s$ that appears before $a$ in $s$. 
$v$ is a function that maps each specification alphabet $a$ of $s$ into either $\{ a \} \cup x$ with $ x \in \mathit{bef}(s,a)$ or $\emptyset$. Moreover, the following condition need to be satisfied:

\begin{enumerate}[(i)]
\item For each specification alphabet $a$ of $s$, $s \uparrow_{v(a)} \in \mathit{spec}_s$. 
\end{enumerate} 
\end{definition} 

The examples of sequential specifications of typical CRDT are given below. To give sequential specification, we use the style of pre-condition and post-condition for each specification alphabet.


\begin{example}[Counter]
\label{definition:sequential specification of counter}
The sequential specification $\mathit{Counter}_s$ of counter are given as follows: Let $state$ be a natural number. 

\begin{itemize}
\setlength{\itemsep}{0.5pt}
\item[-] $\{ state = i \}$ $inc$ $\{ state = i+1 \}$.
\item[-] $\{ state = i \}$ $read() \Rightarrow i$ $\{ state = i+1 \}$.
\end{itemize} 
\end{example} 


\begin{example}[Set]
\label{definition:sequential specification of set}
The sequential specification $\mathit{Set}_s$ of set are given as follows: Here we assume that each item is put into the set only once. Let $state$ be a set and each its element $(a,flag)$ is a tuple of a data $a$ and a flag $flag \in \{ \mathit{true},\mathit{false} \}$. 

\begin{itemize}
\setlength{\itemsep}{0.5pt}
\item[-] $\{ state = S \wedge a \notin S \}$ $add(a)$ $\{ state = S \cup \{ (a,\mathit{true}) \} \}$.
\item[-] $\{ state = S \wedge S' = \{a \vert (a,\mathit{true}) \in S \} \}$ $read() \Rightarrow S'$ $\{ state = S \}$.
\item[-] $\{ state = S \wedge (a,\_) \in S \}$ $rem(a)$ $\{ state = S \setminus \{ (a,\_) \} \cup \{ (a,\mathit{false}) \} \}$.
\end{itemize} 
\end{example} 



\begin{example}[OR-Set]
\label{definition:sequential specification of or-set}
The sequential specification $\mathit{OR-Set}_s$ of OR-set are given as follows: Let $state$ be a set and each its element $(a,id,flag)$ is a tuple of a data $a$, a operation identifier $id$, and a flag $flag \in \{ \mathit{true},\mathit{false} \}$.
\begin{itemize}
\setlength{\itemsep}{0.5pt}
\item[-] $\{ state = S  \}$ $((add(a),\mathit{id}),\emptyset)$ $\{ state = S \cup \{ (a,\mathit{id},\mathit{true}) \} \}$.
\item[-] $\{ state = S \wedge S' = \{ a \vert (a,\_,\mathit{true}) \in S \} \}$ $read() \Rightarrow S'$ $\{ state = S \}$. 
\item[-] $\{ state = S  \wedge S_1 \subseteq \{a \vert (a,\_,\_) \in S\} \}$ $((rem(a),\_),S_1)$ $\{ state = S_2  \}$. Here $S_2$ is obtained from $S$ by marking each $S$ item with $\mathit{false}$ flag. 
\end{itemize}
\end{example} 


\begin{example}[Register]
\label{definition:sequential specification of register}
The sequential specification $\mathit{Reg}_s$ of register are given as follows: Let $state \in \mathbb{D}$ be a value.
\begin{itemize}
\setlength{\itemsep}{0.5pt}
\item[-] $\{ state = a  \}$ $write(b)$ $\{ state = b \}$.
\item[-] $\{ state = a \}$ $read() \Rightarrow a$ $\{ state = a \}$. 
\end{itemize}
\end{example}


\begin{example}[Multi-value Register]
\label{definition:sequential specification of multi-value register}
The sequential specification $\mathit{MVReg}_s$ of multi-value register are given as follows: Let $state$ be a set and each its element $(a,id,flag)$ is a tuple of a data $a$, a operation identifier $id$, and a flag $flag \in \{ \mathit{true},\mathit{false} \}$.
\begin{itemize}
\setlength{\itemsep}{0.5pt}
\item[-] $\{ state = S \wedge \forall x \in S_1, (b,x,\mathit{true}) \in S_1 \vee (b,x,\mathit{false}) \in S_1 \}$ $((write(b),id),S_1)$ $\{ state = S_2 \}$. Here $S_2$ is obtained from $S$ by mark each $(b,x)$ with $\mathit{false}$, and then insert $(b,id,\mathit{true})$. 
\item[-] $\{ state = S \wedge S' = \{ a \vert (a,\_,\mathit{true}) \in S \} \}$ $read() \Rightarrow S'$ $\{ state = S \}$. 
\end{itemize}
\end{example} 


\begin{example}[List with add-after interface]
\label{definition:sequential specification of list with add-after interface} 
The sequential specification $\mathit{List}_s$ of list are given as follows: Let $state$ be a sequence, where each item is a tuple $(a,flag)$ with data $a$ and flag $flag \in \{ \mathit{true},\mathit{false} \}$. 
\begin{itemize}
\setlength{\itemsep}{0.5pt}
\item[-] $\{ state = (a_1,\_) \cdot \ldots \cdot (a_n,\_) \wedge l \leq n \wedge a_k \notin \{ a_1, \ldots, a_n \} \}$ $add(a_k,a_l)$ $\{ state = (a_1,\_) \cdot \ldots \cdot (a_l,\_) \cdot (a_k,\mathit{true}) \cdot (a_{l+1},\_) \cdot \ldots \cdot (a_n,\_) \}$.
\item[-] $\{ state = (a_1,\_) \cdot \ldots \cdot (a_n,\_) \wedge S = \{ a \vert (a,\mathit{true}) \in state \} \wedge l = a_1 \cdot \ldots \cdot a_n \uparrow_{S} \}$ $read() \Rightarrow l$ $\{ state = (a_1,\_) \cdot \ldots \cdot (a_n,\_) \}$. 
\end{itemize}
\end{example} 



As customary, to capture the notion of client-observable effects of an execution over a CRDT, we will define the notion of \emph{history}. A history contains a set of operations, and the order in which they were effected in each replica. Formally, a history $h$ is a tuple of the form $h = (O,\mathit{lab},\mathit{ro})$ where $O$ is a set of operation identifiers, $\mathit{lab}$ is a function that maps each operation identifiers of $O$ into a operation label, and $\mathit{ro}$ is a union of transitive, irreflexive and total orders of $O$.

A history is distributed linearizable w.r.t a distributed specification, if we can find a visibility relation $\mathit{vis}$ of the history and a total sequence $s$ in distributed specification, such that $\mathit{vis}$ is consistent with $s$. Formally,

\begin{definition}[Distributed Linearizability] 
\label{definition:distributed linearizability} 
A history $h = (O,\mathit{ro})$ is distributed linearizable w.r.t distributed specification $\mathit{spec}_d$, if $\exists (s,v) \in \mathit{spec}_d$, $\exists \mathit{vis} \subseteq O \times O$ be a acyclic visibility relation, such that

\begin{enumerate}[(i)]
\item $\mathit{ro} \subseteq \mathit{vis}$, 
\item $\mathit{vis} \subseteq s$, 
\item For each operation identifiers $i$ of $h$, if $v(i) \neq \emptyset$, then $v(i) = \{ i \} \cup \{ a \vert (a,i) \in \mathit{vis} \}$.  
\end{enumerate}

A set $H$ of histories is is distributed linearizable w.r.t distributed specification $\mathit{spec}_d$, if each of its history is. 
\end{definition}




%!TEX root = draft.tex
%\newcommand{\seqPQ}{\mathsf{SeqPQ}}

\section{CRDT Implementation Semantics and Correctness}
\label{sec:CRDT implementation semantics and correctness}

In this section, we propose the semantics of a state-based CRDT or a operation-based CRDT. Then we shows how to extract histories from execution, and the correctness of histories of multiple objects. 



\subsection{Semantics of a Single Object}
\label{subsec:semantics of a single object}

Given a object $\mathit{obj}$ of a state-based CRDT, we define its semantics as a set of executions generated from an LTS $\llbracket \mathit{obj} \rrbracket_s = (\mathit{Config},\mathit{config}_0,\Sigma',\rightarrow)$ as in \figurename~\ref{fig:the semantics of a state-based CRDT object}.

\begin{figure}[ht]
$\mathit{RState} = \mathbb{R} \rightarrow \Sigma$

$\mathit{TState} = \mathbb{MID} \times \mathbb{MSG} \times \mathbb{R}$

$\mathit{Config} = \mathit{RState} \times \mathit{TState}$, $\mathit{config}_0 \in \mathit{Config}$. 

$\Sigma' = \mathit{do}(\mathbb{M} \times \mathbb{D} \times \mathbb{D} \times \mathbb{R} \times (\mathbb{MID} \cup \emptyset)) \cup \mathit{receive}(\mathbb{MID} \times \mathbb{R})$ 

\[
\begin{array}{l c}
\bigfrac{ R(r) = \sigma, r.\mathit{do}(\sigma,m,a) = (\sigma',b,\mathit{msg}), \mathit{msg} \neq \emptyset, \mathit{unique}(\mathit{mid}) }
{ (R,T) {\xrightarrow{\mathit{do}(m,a,b,r,\mathit{mid})}} (R[r:\sigma'],T \cup \{ (\mathit{mid},\mathit{msg},r) \}) }
\end{array}
\]

\[
\begin{array}{l c}
\bigfrac{ R(r) = \sigma, r.\mathit{do}(\sigma,m,a) = (\sigma',b,\emptyset) }
{ (R,T) {\xrightarrow{\mathit{do}(m,a,b,r)}} (R[r:\sigma'],T ) }
\end{array}
\]

\[
\begin{array}{l c}
\bigfrac{ R(r) = \sigma, r.\mathit{receive}(\sigma,\mathit{msg}) = \sigma',(\mathit{mid},\mathit{msg},r') \in T, r \neq r'}
{ (R,T) {\xrightarrow{\mathit{receive}(\mathit{mid},r)}} (R[r:\sigma'],T) }
\end{array}
\]
\caption{The definition of semantics of $\llbracket \mathit{obj} \rrbracket_s$}
\label{fig:the semantics of a state-based CRDT object}
\end{figure}

A configuration $(R,T)$ is a snapshot of distributed system and contains two parts: $R$ gives the local state of each replica, and $T$ gives the set of messages that has been generated. Let $\mathbb{MID}$ be the set of message identifiers of message content. A message is a tuple $(\mathit{mid},\mathit{msg},r)$, where $\mathit{msgId} \in \mathbb{MID}$ is the identifier, $\mathit{msg} \in \mathbb{MSG}$ is the message content, and $r$ is the original replica of message. $\mathit{config}_0$ is the initial configuration, which maps each replica into the initial local state, and has no message inside.

Each element of $\Sigma'$ is called an action. $\rightarrow \in \mathit{Config} \times \Sigma' \times \mathit{Config}$ is the transition relation and describe a single step of distributed systems. The first rule in \figurename~\ref{fig:the semantics of a state-based CRDT object} describes replica $r$ performs a update operation $m(a) \Rightarrow b$ and generates a message with message content $\mathit{msg}$. Here $\mathit{unique}$ is a function that ensures $\mathit{mid}$ is a fresh message identifier. The second rule describes replica $r$ performs a query operation $m(a) \Rightarrow b$ and thus does not generate message. The third rule describes delivery of a message to a replica $r$ other than its origin replica $r'$. 

A sequence $l$ of actions is an execution of $\llbracket \mathit{obj} \rrbracket_s = (\mathit{Config},\mathit{config}_0,\Sigma',\rightarrow)$, if there exists $(R,T) \in \mathit{Config}$, such that $\mathit{config}_0 {\xrightarrow{ l }} (R,T)$. The semantics of $\mathit{obj}$ is defined as the set of executions of $\llbracket \mathit{obj} \rrbracket_s$. Given an execution, when the context is clear, we can associate a unique operation identifier to each action. Or we can say, it is safe to assume each action is either $\mathit{do}(i,x,m,a,b,r,\mathit{mid})$ or $\mathit{receive}(i,x,\mathit{mid},r)$, where $i \in \mathbb{OID}$ is a unique operation identifier.

Note that the transition relation of $\llbracket \mathit{obj} \rrbracket_s$ does not make any assumption about message delivery: Messages can be delivered in any order; a message can be delivery to a replica multiple times; and a message can be never delivered to a replica. Although this is true for state-based CRDT, the operation-based CRDT assume the following two guarantees of message delivery:

\begin{itemize}
\setlength{\itemsep}{0.5pt}
\item[-] Each message is delivered to a replica at most once.
\item[-] Causal delivery: Given two update operations $o_1$ and $o_2$, and assume $o_1$ is visible to $o_2$. Or we can say, $o_1$ is related to $o_2$ by several times of replica order and message delivery. Then, for each replica, once it receives $o_2$, it must be the case that $o_1$ has been previously received. 
\end{itemize} 

To deal with operation-based CRDT, we additionally record a happen-before relation in each configuration. Our semantics of a operation-based CRDT object $\mathit{obj}$ is defined as $\llbracket \mathit{obj} \rrbracket_{\mathit{op}} = (\mathit{Config},\mathit{config}_0,\Sigma',\rightarrow)$ as in \figurename~\ref{fig:the semantics of a operation-based CRDT object}. 


\begin{figure}[ht]
$\mathit{RState} = \mathbb{R} \rightarrow \Sigma$

$\mathit{TState} = \mathbb{MID} \times \mathbb{MSG} \times \mathbb{R}$ 

$\mathit{MsgHB} \subseteq (\mathbb{MID} \times \mathbb{MID}) \cup (\mathbb{MID} \times \mathbb{R})$ 

$\mathit{Config} = \mathit{RState} \times \mathit{TState} \times \mathit{MsgHB}$, $\mathit{config}_0 \in \mathit{Config}$. 

\begin{itemize}
\setlength{\itemsep}{0.5pt}
\item[] $\begin{array}{l c}
   \bigfrac{
   \begin{array}{c}
     R(r) = \sigma, r.\mathit{do}(\sigma,m,a) = (\sigma',b,\mathit{msg}), \mathit{msg} \neq \emptyset, \mathit{unique}(\mathit{mid}), \\
     S_1 = \{ (\mathit{mid}',r) \in \mathit{MsgHB} \}, S_2 = \{ (\mathit{mid}',\mathit{mid}) \vert (\mathit{mid'},r) \in S_1 \}, S_3 = (\mathit{max}(\mathit{MsgHB}^*,r), \mathit{mid})
   \end{array}}
     {(R,T,\mathit{msgHB}) {\xrightarrow{\mathit{do}(m,a,b,r,\mathit{mid})}} (R[r:\sigma'], T \cup \{ (\mathit{mid},\mathit{msg},r) \}, \mathit{msgHB}',\mathit{MsgHB} \cup S_2 \cup S_3 \setminus S_1)}
\end{array}$ 

\item[] $\begin{array}{l c}
   \bigfrac{
   \begin{array}{c}
     R(r) = \sigma, r.\mathit{do}(\sigma,m,a) = (\sigma',b,\emptyset) 
   \end{array}}
     {(R,T,\mathit{msgHB}) {\xrightarrow{\mathit{do}(m,a,b,r)}} (R[r:\sigma'], T \cup \{ (\mathit{mid},\mathit{msg},r) \}, \mathit{msgHB}',\mathit{MsgHB})}
\end{array}$ 

\item[-] $\begin{array}{l c}
   \bigfrac{
   \begin{array}{c}
      R(r) = \sigma, r.\mathit{receive}(\sigma,\mathit{msg}) = \sigma', (\mathit{mid},\mathit{msg},r') \in T, r \neq r', \\
      (\mathit{mid},r) \notin \mathit{MsgHB}, \forall \ \mathit{mid}' \ of \ replica \ r, (\mathit{mid},\mathit{mid}') \notin \mathit{MsgHB}^*, \mathit{mid} \in \mathit{min}(\mathit{MsgHB}^*)
   \end{array}}
     {(R,T,\mathit{msgHB}) {\xrightarrow{\mathit{receive}(\mathit{mid},r)}} (R,T,\mathit{MsgHB} )}
\end{array}$
\end{itemize}
\caption{The definition of semantics of $\llbracket \mathit{obj} \rrbracket_{\mathit{op}}$}
\label{fig:the semantics of a operation-based CRDT object}
\end{figure}

A configuration $(R,T,\mathit{MsgHB})$ is a snapshot of distributed system. $\mathit{MsgHB}$ is used to record the happen-before relation between messages: two messages $(\mathit{mid}_1,\mathit{mid}_2) \in \mathit{MsgHB}$ represents that the operation of $\mathit{mid}_1$ happens before the operation of $\mathit{mid}_2$, while $(\mathit{mid}_1,r) \in \mathit{MsgHB}$ represents that message $\mathit{mid}_1$ has already been delivered to replica $r$ while there does not exist message $\mathit{mid}_2$ of replica $r$, such that the operation of $\mathit{mid}_1$ happens-before the operation of $\mathit{mid}_2$.

$\rightarrow \in \mathit{Config} \times \Sigma' \times \mathit{Config}$ is the transition relation and describe a single step of distributed systems. The first rule in \figurename~\ref{fig:the semantics of a operation-based CRDT object} describes replica $r$ performs a update operation $m(a) \Rightarrow b$ and generates a message with message content $\mathit{msg}$. This action also updates $\mathit{MsgHB}$ tuple. Here $\mathit{max}(\mathit{MsgHB}^*,r)$ returns a message identifier that is the maximal w.r.t $\mathit{MsgHB}^*$ and is generated by replica $r$. The update of $\mathit{MsgHB}$ keeps our intuition of $\mathit{MsgHB}$. The second rule describes replica $r$ performs a query operation $m(a) \Rightarrow b$ and thus does not generate message. Since no message is generated, the $\mathit{MsgHB}$ tuple remains the same. The third rule describes delivery of a message to a replica $r$ other than its origin replica $r'$. By $(\mathit{mid},r) \notin \mathit{MsgHB}$ and $(\mathit{mid},\mathit{mid}') \notin \mathit{MsgHB}^*$, we ensure that $\mathit{mid}$ being delivered to replica $r$ at most once. By $\mathit{mid} \in \mathit{min}(\mathit{MsgHB}^*)$, we always choose a minimal element w.r.t $\mathit{MsgHB}^*$ and ensures causal-delivery. 

The notion of executions of $\llbracket \mathit{obj} \rrbracket_{\mathit{op}}$ is similarly defined as that of $\llbracket \mathit{obj} \rrbracket_s$. 



\subsection{Correctness of a Single Object}
\label{subsec:correctness of a single object} 

Given an execution $l = \alpha_1 \cdot \ldots \cdot \alpha_k$ of $\llbracket \mathit{obj} \rrbracket_s$ of state-based CRDT $\mathit{obj}$, we can obtain a corresponding history $\mathit{history}(l) = (\mathit{Op},\mathit{ro},\mathit{vis})$, such that

\begin{itemize}
\setlength{\itemsep}{0.5pt}
\item[-] Each operation in $\mathit{Op}$ is a tuple $(\ell,i,\mathit{obj})$, such that $i$ is the operation identifier of a $\mathit{do}(m,a,b,r,\mathit{mid})$ action or a $\mathit{do}(m,a,b,r)$ action of $l$. 

\item[-] $(o_1,o_2) \in \mathit{ro}$, if they are of same replica, and the index of $o_1$ in $h$ is before that of $o_2$. 

\item[-] Let us defines a delivery relation $\mathit{del}$ as follows: $(o_1,o_2) \in \mathit{del}$, if there exists action $\alpha_k$, such that $\alpha_k$ receives the message generated by $o_1$, $\alpha_k$ and $o_2$ happens on the same replica, and $\alpha_k$ is before the action of $o_2$. 

\item[-] $\mathit{vis} = (\mathit{ro}+\mathit{del})^*$. 
\end{itemize} 

Intuitively, each local state can be considered as the consequence of all updates it receives. Since state-based CRDT sends the modified local state as message, the visibility relation is then the transitive closure of replica order and delivery relation. Let $\mathit{history}(\llbracket \mathit{obj} \rrbracket_s)$ be the set of histories of all executions of $\llbracket \mathit{obj} \rrbracket_s$. Similarly, given an execution $l$ of $\llbracket \mathit{obj} \rrbracket_{\mathit{op}}$ of operation-based CRDT $\mathit{obj}$, we can obtain a corresponding history $\mathit{history}(l) = (\mathit{Op},\mathit{ro},\mathit{vis})$. The only different is that, $\mathit{vis} = \mathit{del} \cup (\mathit{del} \cdot \mathit{ro})$. Since operation-based CRDT sends the operation as message, the visibility relation just contains delivery relation and (possibly) replica order. Let $\mathit{history}(\llbracket \mathit{obj} \rrbracket_{\mathit{op}})$ be the set of histories of all executions of $\llbracket \mathit{obj} \rrbracket_{\mathit{op}}$. Then, a object is distributed linearizable, if each of its history is, as stated by the following: 

\begin{definition}[Correctness of a CRDT Object]
\label{definition:correctness of a CRDT object} 
A state-based (resp., operation-based) CRDT object $\mathit{obj}$ is linearizable w.r.t a sequential specification $\mathit{spec}$, if each history of $\mathit{history}(\llbracket \mathit{obj} \rrbracket_s)$ (resp., $\mathit{history}(\llbracket \mathit{obj} \rrbracket_{\mathit{op}})$) is linearizable w.r.t $\mathit{spec}$. 
\end{definition}





















\forget
{
\subsection{Semantics}
\label{subsec:semantics}

Given a set $\mathit{Obj}$ of objects, we define its semantics as a set of executions generated from an LTS $\llbracket \mathit{Obj} \rrbracket = (\mathit{Config},\mathit{config}_0,\Sigma',\rightarrow)$ as in \figurename~\ref{fig:the semantics of multiple objects}.

\begin{figure}[ht]
$\mathit{RState} = \cup_{x \in \mathit{Obj}} (\mathbb{R} \rightarrow x.\Sigma)$

$\mathit{TState} = \mathbb{MID} \times \mathbb{MSG} \times \mathit{Obj} \times \mathbb{R}$

$\mathit{Config} = \mathit{RState} \times \mathit{TState}$, $\mathit{config}_0 \in \mathit{Config}$.

$\Sigma' = \mathit{do}(\mathit{Obj} \times \mathbb{M} \times \mathbb{D} \times \mathbb{D} \mathbb{MID}) \cup \mathit{receive}(\mathit{Obj} \times \mathbb{MID} \times \mathbb{R})$

\[
\begin{array}{l c}
\bigfrac{ R(x,r) = \sigma, x(r).\mathit{do}(\sigma,m,a) = (\sigma',b,\mathit{msg}), \mathit{msg} \neq \emptyset, \mathit{unique}(\mathit{mid}) }
{ (R,T) {\xrightarrow{\mathit{do}(x,m,a,b,r,\mathit{mid})}} (R[(x,r):\sigma'],T \cup \{ (\mathit{mid},\mathit{msg},x,r) \}) }
\end{array}
\]

\[
\begin{array}{l c}
\bigfrac{ R(x,r) = \sigma, x(r).\mathit{do}(\sigma,m,a) = (\sigma',b,\mathit{msg}), \mathit{msg} = \emptyset }
{ (R,T) {\xrightarrow{\mathit{do}(x,m,a,b,r)}} (R[(x,r):\sigma'],T ) }
\end{array}
\]

\[
\begin{array}{l c}
\bigfrac{ R(x,r) = \sigma, x(r).\mathit{receive}(\sigma,\mathit{msg}) = \sigma',(\mathit{mid},\mathit{msg},x,r') \in T, r \neq r'}
{ (R,T) {\xrightarrow{\mathit{receive}(x,\mathit{mid},r)}} (R[(x,r):\sigma'],T) }
\end{array}
\]
\caption{The definition of semantics of $\llbracket \mathit{Obj} \rrbracket$}
\label{fig:the semantics of multiple objects}
\end{figure}

A configuration $(R,T)$ is a snapshot of distributed system and contains two parts: $R$ gives the local state of each object at each replica, and $T$ gives the set of messages that has been generated. Let $\mathbb{MID}$ be the set of message identifiers of message content. A message is a tuple $(\mathit{mid},\mathit{msg},x,r)$, where $\mathit{msgId} \in \mathbb{MID}$ is the identifier, $\mathit{msg} \in \mathbb{MSG}$ is the message content, $x$ is the object this message pertains to, and $r$ is the original replica of message. $\mathit{config}_0$ is the initial configuration, which maps each object at each replica into its initial local state, and has no message inside.

Each element of $\Sigma'$ is called an action. $\rightarrow \in \mathit{Config} \times \Sigma' \times \mathit{Config}$ is the transition relation and describe a single step of distributed systems. The first rule in \figurename~\ref{fig:the semantics of multiple objects} describes replica $r$ performs a update operation $m(a) \Rightarrow b$ of object $x$ and generate a message with message content $\mathit{msg}$. Here $\mathit{unique}$ is a function that ensures $\mathit{mid}$ is a fresh message identifier. The first rule  describes replica $r$ performs a query operation $m(a) \Rightarrow b$ of object $x$ and thus does not generate message. The third rule describes delivery of a message of object $x$ to a replica $r$ other than its origin replica $r'$.

A sequence $l$ of actions is an execution of $\llbracket \mathit{Obj} \rrbracket = (\mathit{Config},\mathit{config}_0,\Sigma',\rightarrow)$, if there exists $(R,T) \in \mathit{Config}$, such that $\mathit{config}_0 {\xrightarrow{ l }} (R,T)$. The semantics of $\mathit{Obj}$ is defined as the set of executions of $\llbracket \mathit{Obj} \rrbracket$. Given an execution, when the context is clear, we can associate a unique operation identifier to each action. Or we can say, it is safe to assume each action is either $\mathit{do}(i,x,m,a,b,r,\mathit{mid})$ or $\mathit{receive}(i,x,\mathit{mid},r)$, where $i \mathbb{OID}$ is a unique operation identifier.





\subsection{Correctness of a CRDT Implementation}
\label{subsec:correctness of a CRDT implementation}

To state the correctness of a single CRDT implementation, we consider the semantics of a single CRDT object.



Note that the transition relation of $\llbracket \mathit{Obj} \rrbracket$ does not make any assumption about message delivery: Messages can be delivered in any order; a message can be delivery to a replica multiple times; and a message can be never delivered to a replica. However, although the state-based CRDT do not assume any guarantee about message delivery, the operation-based CRDT assume the following two guarantees about message delivery:

\begin{itemize}
\setlength{\itemsep}{0.5pt}
\item[-] Each message is delivered to a replica at most once.
\item[-] Causal delivery: Given two update operations $o_1$ and $o_2$ of a same object, and assume $o_1$ is visible to $o_2$. Or we can say, $o_1$ is related to $o_2$ by several times of replica order and message delivery (all intermediate action is of the same object of $o_1$ and $o_2$). Then, for each replica, once it receives $o_2$, it must be the case that $o_1$ has been previously received.
\end{itemize}

Given an execution $l = \alpha_1 \cdot \ldots \cdot \alpha_k$ of $\llbracket \mathit{Obj} \rrbracket$,
}

%%% Local Variables:
%%% mode: latex
%%% TeX-master: "draft"
%%% End:


%!TEX root = draft.tex
%\newcommand{\seqPQ}{\mathsf{SeqPQ}}

\section{Proving Distributed Linearizability}
\label{sec:proving distributed linearizability} 

The proof process of distributed linearizability is as follows: We construct a function $g$ called linearization function, which takes a configuration $c$ as argument, and returns a set. Each item of the set is a tuple $(e,l)$, where intuitively $e$ is an execution that generate $c$, and $l = (s,f)$ is in the distributed specification of $e$. Moreover, function $g$ is a simulation relation: whenever $c {\xrightarrow{\alpha}} c'$, there exists $e',l'$, such that $e'$ is generated from $e$ by adding a action according to $\alpha$, and $(e',l') \in g(c')$.

Let us take the RGA algorithm as example to show how to construct function $g$. Since each configuration contains all its previous messages as well as their happen-before and delivery relation, we should consider only correct configurations. A configuration $((N,\mathit{Tomb}),T,\mathit{msgHB},\mathit{msgDel})$ is correct, if 

\begin{itemize}
\setlength{\itemsep}{0.5pt}
\item[-] a
\end{itemize}



First, we propose a function $f$ from state to history, and a function $g$ from history to linearization. Such function $f$ and $g$ holds when distributed system proceed. Then, we prove that such linearization is in specification. We also need a invariant of states.




%%% Local Variables:
%%% mode: latex
%%% TeX-master: "draft"
%%% End:


%\section{CRDT-Linearizable Implementations}
%\label{sec:crdt-lin-imp}

%\gpn{Proofs of CRDT-Linearizable implementations}

%\begin{itemize}
%\item Counters
%  \begin{itemize}
%  \item Grow-Only Counter
%  \item PN-Counter
%  \end{itemize}
%\item Registers:
%  \begin{itemize}
%  \item LWW
%  \item MVR
%  \end{itemize}
%\item Sets:
%  \begin{itemize}
%  \item LAW (OR-Set)
%  \item Grow-Only Set
%  \end{itemize}
%\item Text-Editing (Graphs/Lists)
%  \begin{itemize}
%  \item RGA (addRight)
%  \item WOOT
%  \end{itemize}
%\end{itemize}

\section{Meta-properties of CRDT-Linearizability}
\label{sec:meta-prop-lincrdt}

\subsection{Compositionality}
\label{sec:compositionality}

\gpn{Here we explain the compositionality result a la Herlihy Wing}

\begin{itemize}
\item What are the conditions for compositionality (deterministic
  specs?, conflict?)
  \begin{itemize}
  \item Can we share the same timestamp for different data structures.
  \item Coherent conflict resolution.
  \end{itemize}
\item Should we change the implementations to change the
  conflict-resolution.
\end{itemize}

\subsection{Abstraction}
\label{sec:abstraction}

\gpn{Here we explain refinement}
\begin{itemize}
\item What is the power of the client? Messages, other calls to CRDTs?
\item Observational refinement?
\item Completeness?
\end{itemize}

\bibliographystyle{plainurl}
\bibliography{draft}

\begin{thebibliography}{50}
\bibitem{Burckhardt:2014POPL}
Burckhardt, S., Gotsman, A., Yang, H., Zawirski, M.:
Replicated data types: specification, verification, optimality.
\newblock In: Jagannathan, S., Sewell, P. (eds.) POPL 2014, pp. 271-284. ACM (2014)
\end{thebibliography}



\newpage

\appendix

\section{Definitions of Section \ref{sec:CRDT implementations and its semantics}}
\label{sec:appendix definitions of section CRDT implementations and its semantics}


\end{document}

%%% Local Variables:
%%% mode: latex
%%% TeX-master: t
%%% End:
