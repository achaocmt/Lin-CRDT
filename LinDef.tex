%!TEX root = draft.tex
%\newcommand{\seqPQ}{\mathsf{SeqPQ}}

\section{Distributed Linearizability}
\label{sec:distributed-lin}

Let us start our formation of executions, specifications and linearizations of CRDT.

We consider a finite set $\mathbb{M}$ of method names; and a possibly infinite set $\mathbb{D}$ of arguments and return values, the data domain. Without loss of generality we will consider that the methods in $\mathbb{M}$ can be separated in two disjoint sets of $\mathbb{Q}$ query methods, and $\mathbb{U}$ update methods. Some update operation also implicitly do query to some extent. For example, a $add(a,b)$ operation of distributed list intends to put $a$ immediately after $b$, which also ensures that $b$ is in list. When we say query operation, we mean purely query operation.
We consider replicated data types which are distributed across a set of replicas; the set of replica identifiers is denoted by $\mathbb{R}$. We assume that each replica contains a copy of the data type state. Finally we have a infinite set $\mathbb{O}$ of operation identifiers, corresponding to each individual operation performed on the CRDT throughout an execution. Let $\mathbb{T}$ be the set of data types, while each data type $t = (M,D)$ has a set $M \subseteq \mathbb{M}$ of methods and a data domain $D \subseteq \mathbb{D}$. 

Operation labels \mbox{$m(a)\Rightarrow b$} with $m \in \mathbb{M}$ and $a,b \in \mathbb{D}$, indicate that the operation is a call to method $m$ with argument $a$ and the result of the operation is the value $b$. When $m$ does not use the argument (resp., return value), we write $m()\Rightarrow b$ (resp., $m(a)$) instead. We define an operation $o$ to be a tuple $(\ell,i)$, where $\ell$ is an operation label and $i \in \mathbb{O}$ is a unique operation identifier. Let $\mathbb{OP}$ be the set of operations. 

\noindent {\bf Sequential Specification:} A sequential specification intends to give a sequential view of the data type, and is given as a set of sequential behaviors. However, some particular methods seem inherently non-sequential. For example, in OR-set data type, a $rem(a)$ operation only cancels the added items $a$ of the local replica, and it seems hard to give a completely sequential explanation for this. To deal with such operations, we introduce an additional tuple, which are the set of operations needed for this operation. Formally, let specification alphabet $\mathbb{A} = \mathbb{OP} \cup \{ (o,s) \vert o \in \mathbb{OP}, s \subseteq \mathbb{O} \}$. Then, a sequential specification of CRDT is given as a set of sequences over specification alphabets $\mathbb{A}$. From now on, we implicitly assume that for each sequences of sequential specification, each operation has a unique operation identifier. Given a data type $t$, let $\mathit{specMth}(t) \subseteq M$ be the set of methods, such that in its sequential specification, only operations of $\mathit{specMth}(t)$ uses specification alphabet of the form $(o,s)$. Or we can say, $\mathit{specMth}(t)$ is the set of methods that needs specially treated, like $rem$ of OR-set. 

\gp{We should add the specifications (even non-deterministic here.)}
\begin{definition}[Sequential Specification]
\label{definition:sequential specification}
A sequential specification $\mathit{spec}_s \subseteq \mathbb{A}^*$ is a set of strings over specification alphabet $\mathbb{A}$.
\end{definition} 

Normally, the sequential specification is defined by the pre-condition and post-conditions. For example, for $inc$ operation of counter, it is defined as $\{ state = i \}$ $inc$ $\{ state = i+1 \}$, while $state$ is a value of integer. Usually in sequential specification, update operations $(o,s)$ changes the state of pre-condition and ignores $s$, while query operations does not change the state of pre-condition and uses $s$. The detailed examples of sequential specifications are shown latter of this section.

\noindent {\bf Distributed Specification:} Given a sequence $s$ and a set $S$ of operations $S$, let $s \uparrow_{S}$ be the projection of $s$ over $S$. To give distributed specification, we need to give the linearization, as well as the visibility relation, and check that, for each query operation, its projection w.r.t the visibility relation is in the sequential specification. The notion of distributed specification is defined as follows:


%where $s \subseteq \mathbb{A}^*$ is a sequence of specification alphabet, and $v$ is a function that maps each item $a$ of $s$ into a subset of items before $a$ in $s$. Here $s$ essentially is the linearization of an execution, while $v$ is used to ensure each specification alphabet is correct in this sequence. %Given a sequence $s \subseteq \mathbb{A}^*$ of specification alphabets and a specification alphabets $a$, let $\mathit{itm}(s)$ be the set of specification alphabets of $s$, and $\mathit{itmBef}(s,a)$ be the set of specification alphabets of $s$ that appears before $a$, or $\emptyset$ otherwise.
%Given a sequence $s$ and a set of operations $S$, let $s \uparrow_{S}$ be the projection of $s$ over $S$. Given a sequence $s$ and an item $a$ of $s$, let $\mathit{bef}(s,a)$ contains the set of items of $s$ that appear before $a$ in $s$, as well as all the subsets of this set.

\gp{I don't really know what a Dist. Spec. is.}
\begin{definition}[Distributed Specification]
\label{definition:distributed specification} 
A distributed specification $\mathit{spec}_d$ w.r.t a sequential specification $\mathit{spec}_d$ is a set of tuples $(s,f)$, such that 

\begin{enumerate}[(i)]
\item $s \in \mathbb{OP}^*$ is a sequence of operations, 
\item $f$ is a function that maps each operation $o$ of $s$ into a subset of operations in prefix of $o$, and 
\item for each query operation $o$, $f(s) \uparrow_{f^{-1}(o)} \in \mathit{spec}_s$. 
\end{enumerate}

where $f(s)$ is the sequence obtained from $s$ by replacing each $\mathit{specMth}(t)$ $o$ with $f(o)$. 
\end{definition} 

Essentially, $s$ is the ilnearization of an execution, and $f$ is the visibility relation of the execution. This definition assumes that $\mathit{specMth}(t)$ operation choose operations visible to it as arguments, which fits the practice. It is easy to see that, as long as the sequential specification is given, its distributed specification is done. 

The examples of sequential specifications of typical CRDT are given below. 


\begin{example}[Counter]
\label{definition:sequential specification of counter}
The sequential specification $\mathit{Counter}_s$ of counter are given as follows: Let $state$ be a natural number.

\begin{itemize}
\setlength{\itemsep}{0.5pt}
\item[-] $\{ state = i \}$ $inc$ $\{ state = i+1 \}$.
\item[-] $\{ state = i \}$ $read() \Rightarrow i$ $\{ state = i+1 \}$.
\end{itemize}
\end{example}


\begin{example}[Set]
\label{definition:sequential specification of set}
The sequential specification $\mathit{Set}_s$ of set are given as follows: Here we assume that each item is put into the set only once. Let $state$ be a set and each its element $(a,flag)$ is a tuple of a data $a$ and a flag $flag \in \{ \mathit{true},\mathit{false} \}$.

\begin{itemize}
\setlength{\itemsep}{0.5pt}
\item[-] $\{ state = S \wedge a \notin S \}$ $add(a)$ $\{ state = S \cup \{ (a,\mathit{true}) \} \}$.
\item[-] $\{ state = S \wedge S' = \{a \vert (a,\mathit{true}) \in S \} \}$ $read() \Rightarrow S'$ $\{ state = S \}$.
\item[-] $\{ state = S \wedge (a,\_) \in S \}$ $rem(a)$ $\{ state = S \setminus \{ (a,\_) \} \cup \{ (a,\mathit{false}) \} \}$.
\end{itemize}
\end{example}



\begin{example}[OR-Set]
\label{definition:sequential specification of or-set}
The sequential specification $\mathit{OR-Set}_s$ of OR-set are given as follows: Let $state$ be a set and each its element $(a,id,flag)$ is a tuple of a data $a$, a operation identifier $id$, and a flag $flag \in \{ \mathit{true},\mathit{false} \}$.
\begin{itemize}
\setlength{\itemsep}{0.5pt}
\item[-] $\{ state = S  \}$ $((add(a),\mathit{id}),\emptyset)$ $\{ state = S \cup \{ (a,\mathit{id},\mathit{true}) \} \}$.
\item[-] $\{ state = S \wedge S' = \{ a \vert (a,\_,\mathit{true}) \in S \} \}$ $read() \Rightarrow S'$ $\{ state = S \}$.
\item[-] $\{ state = S  \wedge S_1 \subseteq \{a \vert (a,\_,\_) \in S\} \}$ $((rem(a),\_),S_1)$ $\{ state = S_2  \}$. Here $S_2$ is obtained from $S$ by marking each $S$ item with $\mathit{false}$ flag.
\end{itemize}
\end{example}


\begin{example}[Register]
\label{definition:sequential specification of register}
The sequential specification $\mathit{Reg}_s$ of register are given as follows: Let $state \in \mathbb{D}$ be a value.
\begin{itemize}
\setlength{\itemsep}{0.5pt}
\item[-] $\{ state = a  \}$ $write(b)$ $\{ state = b \}$.
\item[-] $\{ state = a \}$ $read() \Rightarrow a$ $\{ state = a \}$.
\end{itemize}
\end{example}


\begin{example}[Multi-value Register]
\label{definition:sequential specification of multi-value register}
The sequential specification $\mathit{MVReg}_s$ of multi-value register are given as follows: Let $state$ be a set and each its element $(a,id,flag)$ is a tuple of a data $a$, a operation identifier $id$, and a flag $flag \in \{ \mathit{true},\mathit{false} \}$.
\begin{itemize}
\setlength{\itemsep}{0.5pt}
\item[-] $\{ state = S \wedge \forall x \in S_1, (b,x,\mathit{true}) \in S_1 \vee (b,x,\mathit{false}) \in S_1 \}$ $((write(b),id),S_1)$ $\{ state = S_2 \}$. Here $S_2$ is obtained from $S$ by mark each $(b,x)$ with $\mathit{false}$, and then insert $(b,id,\mathit{true})$.
\item[-] $\{ state = S \wedge S' = \{ a \vert (a,\_,\mathit{true}) \in S \} \}$ $read() \Rightarrow S'$ $\{ state = S \}$.
\end{itemize}
\end{example}


\begin{example}[List with add-after interface]
\label{definition:sequential specification of list with add-after interface}
The sequential specification $\mathit{List}_s$ of list are given as follows: Let $state$ be a sequence, where each item is a tuple $(a,flag)$ with data $a$ and flag $flag \in \{ \mathit{true},\mathit{false} \}$.
\begin{itemize}
\setlength{\itemsep}{0.5pt}
\item[-] $\{ state = (a_1,\_) \cdot \ldots \cdot (a_n,\_) \wedge l \leq n \wedge a_k \notin \{ a_1, \ldots, a_n \} \}$ $add(a_k,a_l)$ $\{ state = (a_1,\_) \cdot \ldots \cdot (a_l,\_) \cdot (a_k,\mathit{true}) \cdot (a_{l+1},\_) \cdot \ldots \cdot (a_n,\_) \}$.
\item[-] $\{ state = (a_1,\_) \cdot \ldots \cdot (a_n,\_) \wedge S = \{ a \vert (a,\mathit{true}) \in state \} \wedge l = a_1 \cdot \ldots \cdot a_n \uparrow_{S} \}$ $read() \Rightarrow l$ $\{ state = (a_1,\_) \cdot \ldots \cdot (a_n,\_) \}$.
\end{itemize}
\end{example}



As customary, to capture the notion of client-observable effects of an execution over a CRDT, we will define the notion of \emph{history}. A history contains a set of operations, and the order in which they were effected in each replica. Formally, a history $h$ is a tuple of the form $h = (O,\mathit{lab},\mathit{ro})$ where $O$ is a set of operation identifiers, $\mathit{lab}$ is a function that maps each operation identifiers of $O$ into a operation label, and $\mathit{ro}$ is a union of transitive, irreflexive and total orders of $O$.

A history is distributed linearizable w.r.t a distributed specification, if we can find a visibility relation $\mathit{vis}$ of the history and a total sequence $s$ in distributed specification, such that $\mathit{vis}$ is consistent with $s$. Formally,

\begin{definition}[Distributed Linearizability]
\label{definition:distributed linearizability}
A history $h = (O,\mathit{ro})$ is distributed linearizable w.r.t distributed specification $\mathit{spec}_d$, if $\exists (s,v) \in \mathit{spec}_d$, $\exists \mathit{vis} \subseteq O \times O$ be a acyclic visibility relation, such that

\begin{enumerate}[(i)]
\item $\mathit{ro} \subseteq \mathit{vis}$,
\item $\mathit{vis} \subseteq s$,
\item For each operation identifiers $i$ of $h$, if $v(i) \neq \emptyset$, then $v(i) = \{ i \} \cup \{ a \vert (a,i) \in \mathit{vis} \}$.
\end{enumerate}

A set $H$ of histories is is distributed linearizable w.r.t distributed specification $\mathit{spec}_d$, if each of its history is.
\end{definition}


\gpn{Definition of CRDT-Linearizablity correctness.}


%%% Local Variables:
%%% mode: latex
%%% TeX-master: "draft"
%%% End:
