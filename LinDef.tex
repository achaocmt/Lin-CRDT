%!TEX root = draft.tex
%\newcommand{\seqPQ}{\mathsf{SeqPQ}} 


\section{Definition of Linearizability} 
\label{sec:definition of linearizability} 






Let us start our formal development by introducing the definitions
required to specify CRDTs.
%
We will consider a finite set $\mathbb{M}$ of method names; and a possibly
infinite set $\mathbb{D}$ of argument and return values, the data
domain.
%
We consider replicated data types which are distributed across a set
of replicas; the set of replica identifiers is denoted by $\mathbb{R}$.
We assume that each replica contains a copy of the data type state.
Finally we have a infinite set $\mathbb{O}$ of operation identifiers,
corresponding to each individual operation performed on the CRDT
throughout an execution.
%

Operation labels \mbox{$m(a)\Rightarrow b$} with $m \in \mathbb{M}$ and $a,b \in
\mathbb{D}$, indicate that the operation is a call to method $m$
with argument $a$ and the result of the operation is the value
$b$.
When $m$ does not use the argument (resp., return value), we write
$m()\Rightarrow b$ (resp., $m(a)$) instead.
We define an operation $o$ to be a tuple $(\ell,r,i)$, where $\ell$ is
an operation label, $r \in \mathbb{R}$ is the identifier of the
replica to which the operation is submitted by the client, and $i \in \mathbb{O}$ is a
unique operation identifier.
%This definition extends naturally to the case when there are more than one arguments.
We denote by $\mathit{lab}(o)$ the label $\ell$ whenever $o = (\ell,
r, i)$.
%
Without loss of generality we will consider that the methods in
$\mathbb{M}$ can be separated in two disjoint sets of $\mathbb{Q}$
query methods, and $\mathbb{U}$ update methods.

As customary, to capture the notion of client-observable effects of an
execution over a CRDT, we will define the notion of \emph{history}.
%
A history contains a set of operations, and the order in which
they were effected in each replica.
%
Formally, a history $h$ is a tuple of the form $h = (O,\mathit{ro})$
where $O$ is a set of operations, and $\mathit{ro}$ is the replica
order.
%
For each replica $r \in \mathbb{R}$, $\mathit{ro}|_{r}$ (that is the
projection of $ro$ to operations of replica $r$) is an irreflexive
total order over all the operations with replica identifier $r$.
%
We require $\mathit{ro}$ to not relate operations with different replica identifier.
%
We also require that for each operation $o \in O$,
$\mathit{ro}^{-1}(o)$ is finite, meaning that the past of any replica
is only finite.
An example of a history of a list CRDT is shown in
\figurename~\ref{fig:history, annotated history and operation context} (a).
The $\mathit{add}(a,1)$ operation adds element $a$
on position 1 of the list, and $\mathit{read}(b\cdot a \cdot c)$
represents the fact that a read operation obtained the snapshot
$b\cdot a \cdot c$ of the list, where we denote by $s \cdot s'$ the
composition of sequences $s$ and $s'$. These operations are submitted
to two replicas.

While histories provide the means to describe the behaviors of CRDTs,
they are insufficient to describe the effects of the delivery
policy among different replicas.
%
In other words, when does an operation become visible to another
replica, and therefore to all operations being issued at that replica?
%
This is an important notion, since specifications sometimes rely on a
certain delivery for their correctness.
%
To deal with this issue we will augment histories with additional
information, indicating when operations become visible to other
operations i.e. the replica in which they are generated, and we
shall call them \emph{annotated histories}.
%

An annotated history $ah$ is a tuple $ah = (O,\mathit{vis},\mathit{arb})$, where
$O$ is a set of operations; $\mathit{vis}$, the visibility relation, is
an irreflexive and acyclic relation over $O$ representing for each
operation the set of operations which can affect its result
(essentially for $o$, these are the operations that are in
$\mathit{vis}^{-1}(o)$); and $\mathit{arb}$, the arbitration order, is
an irreflexive total order over a subset of the update operations of
$O$.
An important aspect of CRDTs is that in the presence of conflicting
operations (that is operations which if executed in different order
change the resulting behavior), the convergence mechanism of the CRDT
will determine the order in which the operations should be applied by
all replicas, hence guaranteeing convergence.
The order $\mathit{arb}$ plays this role in the definition of
annotated histories.
% \fxnote{GP: Why only update at this point?. Add the cite to
%   Burckhardt.}
The annotated history for history corresponding to \figurename~\ref{fig:history, annotated history and operation context} (a) is given in \figurename~\ref{fig:history, annotated
  history and operation context} (b). Here we use the arbitration order $\mathit{add}(b,1) < \mathit{add}(a,1) < \mathit{add}(c,1) < \mathit{add}(d,4))$.
%

% {\color {red} Visibility relation tells how update operations are
%   delivered, and arbitration order represents some global orders that
%   are used to determine return values on each replica, such as list
%   order in \cite{Attiya:2016} or time stamp order.}
% We further require that for each operation $o \in O$, $\mathit{vis}^{-1}(o)$ is finite.

%\fxwarning{GP: \c@Chao explain or remove}
%{\color {red}The annotated history for history of \figurename~\ref{fig:history, annotated history and operation context} (a) is given in \figurename~\ref{fig:history, annotated
%history and operation context} (b). $(\mathit{add}(a,1),\mathit{add}(d,4))$ in visibility represents that $\mathit{add}(a,1)$ is visible when submitting $\mathit{add}(d,4))$.} % with arbitration order $\mathit{add}(b,1) \cdot \mathit{add}(c,2) \cdot \mathit{add}(a,1)$.}

%\begin{itemize}
%\setlength{\itemsep}{0.5pt}
%\item[-] $O$ is a set of operations.

%\item[-] $\mathit{vis}$ is a irreflexive and acyclic relation over $O$, and is called the visibility relation. We require that for each operation $o \in O$, $\mathit{vis}^{-1}(o)$ is finite.

%\item[-] $\mathit{arb}$ is a partial order over update operations of $O$ and is called the arbitration order.
%\item[-] $\mathit{arb}$ is a irreflexive total order over a subset of update operations of $O$ and is called the arbitration order.
%\end{itemize}


%\begin{figure}[t]
%  \centering
%  \includegraphics[width=1 \textwidth]{figures/PIC-his-anhis-context-1.pdf}
%\vspace{-10pt}
%  \caption{{A history of list, its annotated history, and an operation context of $\mathit{add}(d,4)$.}}

  %Here $+(a,1)$ represents $\mathit{add}(a,1)$, $r(b \cdot a \cdot c)$ represents $\mathit{read}(b \cdot a \cdot c)$; $+a_1$ and $-a_1$ represents $\mathit{add}(a)$ and $\mathit{rem}(a)$, respectively, while the subscript number is used to distinguish different $\mathit{add(a)}$; $(a,\surd)$ represents $\mathit{contains}(a,\mathit{true})$. In annotated history (b), we use arbitration order $\mathit{add}(b,1) \cdot \mathit{add}(a,1) \cdot \mathit{add}(c,1) \cdot \mathit{add}(d,4)$. Assume that the visibility relation contains the replica order.}
%  \label{fig:history, annotated history and operation context}
%\end{figure}


Following~\cite{Burckhardt:2014} we provide the formal specification of
CRDT by means of a specification function $\mathit{Spec}$ which takes
an operation name, a set of arguments corresponding to the arity and
the types of the operation, and a \emph{context} representing the
information available to the replica executing the operation.
Provided with these parameters, $\mathit{Spec}$ returns a single value
that results from executing the operation.
%
The context of an operation $o$ here is a triple of the form
$(O,<,\mathit{arb})$, where $O$ is a set of update operations, $<$ is
a pre-order a subset of $O$ such that $o \notin O$ and $\mathit{arb}$
is an irreflexive total order over a subset of the update
operations\footnote{Notice that unlike~\cite{Burckhardt:2014}, we
  consider that $o \notin O$. } of $O \cup \{o\}$.
%
% \fxnote{GP: Have no idea why!}
%
The intuitive meaning of the relation $<$ is that encodes the
happens-before relation~\cite{Lamport:1978} over the updates that are
visible to the operating through $\mathit{vis}$.

We can now formally define the specification function $\mathit{Spec}$
as in~\cite{Burckhardt:2014}.
$\mathit{Spec}$ takes an operation name $m \in \mathbb{B}$, arguments
$a$, and a context $(O, <, \mathit{arb})$, and it returns a value $b$
(written $\mathit{Spec}(m, a, (O, <, \mathit{arb})) = b$).
%
Moreover, assuming that the resulting operation from the call to
$m(a)$ will be given the operation identifier $o \in \mathbb{O}$, we
assume that $o\notin O$ (hence $lab(o) = m(a) \Rightarrow b$), and
that $o$ can occur in $\mathit{arb}$.
%
We shall in the rest of the paper overload the function name
$\mathit{Spec}$ parameterized by labels, and adopt the notation
\mbox{$Spec(m(a) \Rightarrow b)$} to denote the set of contexts that
generate that label, that is:
\[Spec(m(a) \Rightarrow b) \triangleq \{\ (O, <, \mathit{arb})\ |\
  Spec(m, a, (O, <, \mathit{arb})) = b\ \}\]
%
%We recall at this point that since specifications are deterministic,
%$\mathit{Spec}$ is a %%functi.

% As in~Burkhardt:2014} we require specifications to be deterministic.
% {\color {red}Similarly as in \cite{Burckhardt:2014}, in defining specification, we use the notion of operation context, which contains all information necessary to ensure the correctness of an operation.} A operation context of a operation $o$ is a tuple $(O,<,\mathit{arb})$, where $O$ is a set of update operations, $<$ is a relation over $O$, $o \notin O$, and $\mathit{arb}$ is a irreflexive total order over a subset of update operations of $O \cup \{ o \}$. {\color {red}Here we modified that of \cite{Burckhardt:2014} by making $o \notin O$ itself also contained in the arbitration order. This feature is used to deal with list specification.} A specification $\mathit{Spec}$ is a function that maps each operation label $\ell$ into a set of elements, while each of them is a operation context $(O,<,\mathit{arb})$ for some operation $o$ with $\mathit{lab}(o) = \ell$. {\color {red}


% A specification is deterministic, if there does not exists query
% method $m$ and tuple $(O,<,\mathit{arb})$, such that
% $(O,<,\mathit{arb})$ is in both $\mathit{Spec}(m(a) \Rightarrow b)$
% and $\mathit{Spec}(m(a') \Rightarrow b')$, and $a \neq a' \vee b \neq
% b'$.
% From now on, we consider only deterministic specifications.

Given an annotated history $ah = (O,\mathit{vis},\mathit{arb})$ and an
operation $o \in O$, we define the context extracting function $ctxt(ah, o)$
which returns a context $(O', <, \mathit{arb'})$ such that:
\begin{inparaenum}[(i)]
\item $O'$ is the set \emph{update} operations in $\mathit{vis}^{-1}(o)$,
\item $<$ is the projection of $\mathit{vis}$ over $O'$, and
\item $\mathit{arb'}$ is the projection of $\mathit{arb}$ over $O' \cup \{o\}$.
\end{inparaenum} Therefore, given an annotated history $ab =
(O,\mathit{vis},\mathit{arb})$ and $o_1,o_2 \in O$, if $o_1$ and $o_2$
see the same set of operations, i.e. $O_1 = O_2$, then $ctxt(o_1) = ctxt(o_2)$.
%
This is consistent with the commutativity requirement of CRDTs.
For example, \figurename~\ref{fig:history, annotated history and
  operation context}
(c) is the operation context of $o = \mathit{add}(d,4)$ in
\figurename~\ref{fig:history, annotated history and operation context}
(b).
Here the operations visible to $o$ are $\{
\mathit{add}(a,1), \mathit{add}(b,1), \mathit{add}(c,1)\}$, and the
arbitration order is the same as that of \figurename~\ref{fig:history, annotated history and operation context}
(b).
Similarly, this definition modifies that of \cite{Burckhardt:2014}
to deal with list.
%
To obtain the operation context from annotated history for OR-set
requires a little care as it will be explained in Remark \ref{remark:
  operation context of OR-set}.

%Note that the uniformly
%  method of obtaining operation context from annotated history is
%  suitable for most specifications.
%  For one special case, the OR-set, we shows how to deal with it in
%  Remark \ref{remark: operation context of OR-set}.}
  % To comply with the commutativity intuition of CRDT algorithm, we further require that, given an annotated history $(O,\mathit{vis},\mathit{arb})$ and $o_1,o_2 \in O$, if $\mathit{vis}^{-1}(o_1) = \mathit{vis}^{-1}(o_2)$, then $<_1 = <_2$, where $ctxt(o_1)=(O_1,<_1,\mathit{arb}_1)$ and $ctxt(o_2)=(O_2,<_2,\mathit{arb}_2)$.

% \fxfatal{GP: How is this different from RVal Consistency from~\cite{Burckhardt:2014}?}
The notion of Convergent Return-Value Consistency (CRVC, for short) is
similar to that of~\cite{Burckhardt:2014} and it requires the return
value of each operation to correspond to its specification, and it is
defined as follows:

\begin{definition}[Convergent Return-Value Consistency]
\label{definition:strong return value consistency}
An annotated history $ah = (O,\mathit{vis},\mathit{arb})$ is CRVC w.r.t specification $\mathit{Spec}$, if:
\begin{enumerate}[(i)]
\item $\mathit{vis}$ is acyclic, and
\item $\forall o \in O$, $\mathit{ctxt}(o) \in Spec(\mathit{lab}(o))$.
\end{enumerate}

Moreover, a history $h = (O,\mathit{ro})$ is CRVC w.r.t $\mathit{Spec}$, if
there exist $\mathit{vis}$ and $\mathit{arb}$, such that $\mathit{ro}
\subseteq \mathit{vis}$, and $(O,\mathit{vis},\mathit{arb})$ is CRVC
consistent w.r.t $\mathit{Spec}$.
\end{definition}


%\begin{figure}[t]
%  \centering
%  \includegraphics[width=1 \textwidth]{figures/PIC-his-anhis-context-2.pdf}
%\vspace{-10pt}
%  \caption{Generating operation context from annotated history for OR-set.}

  %Here $+(a,1)$ represents $\mathit{add}(a,1)$, $r(b \cdot a \cdot c)$ represents $\mathit{read}(b \cdot a \cdot c)$; $+a_1$ and $-a_1$ represents $\mathit{add}(a)$ and $\mathit{rem}(a)$, respectively, while the subscript number is used to distinguish different $\mathit{add(a)}$; $(a,\surd)$ represents $\mathit{contains}(a,\mathit{true})$. In annotated history (b), we use arbitration order $\mathit{add}(b,1) \cdot \mathit{add}(a,1) \cdot \mathit{add}(c,1) \cdot \mathit{add}(d,4)$. Assume that the visibility relation contains the replica order.}
%  \label{fig:generating operation context from annotated history for OR-set}
%\end{figure}




\noindent {\bf Example 1. OR-set}:
%
The Observed-Remove Set CRDT (OR-set for short)
\cite{Shapiro:2011,Bieniusa:2012} maintains the view of a set in
each replica. Each $\mathit{rem}(a)$ can only cancel an
$\mathit{add}(a)$ that has been made visible to it. Therefore, given
two current (not visible to each other) operations $\mathit{add}(a)$
and $\mathit{rem}(a)$, the $\mathit{add}(a)$ can always take
precedence.
OR-set contains four methods: (1)
$\mathit{add}(a)$ inserts $a$ into the set, (2) $\mathit{rem}(a)$
removes $a$ from set, (3) \mbox{$\mathit{lookup}(a)\Rightarrow
  \mathit{true}$} (resp., \mbox{$\mathit{lookup}(a)\Rightarrow
  \mathit{false}$}) represents the inclusion of the element $a$ in the
set (resp. its exclusion), and $\mathit{elements}() \Rightarrow S$
indicates that the contents of the set are $S$.

% \todo{Why do you need $\top$ as argument ??
%   If it's because you want to use operation labels for messages as
%   well, I don't like it.
%   Each part of the formalism should be as clear as possible, without
%   unnecessary junk.}
% Let $\Sigma_{\mathit{ORS}} = \{ add(a,\top),rem(\top,a) \vert a \in
% \mathbb{D} \}$ be the set of contents of update operations.
Given an annotated history $ah = (O, \textit{vis}, \emph{arb})$, the
specification of OR-set, $S_{\mathit{ORS}}$, is as follows:
\begin{inparaenum}[(1)]
\item $S_{\mathit{ORS}}(\mathit{add}(a))$ is the set of all tuples
  $(O,<,\emptyset)$,
\item $S_{\mathit{ORS}}(\mathit{lookup}(a) \Rightarrow \mathit{true})$
  is the set of all tuples $(O,<,\emptyset)$, such that the projection
  of $<$ over $\mathit{add}(a)$ and $\mathit{rem}(a)$ in $O$ contains a
  maximal element with operation label $\mathit{add}(a)$.
  A similar definition can be given for $\mathit{lookup}(a)
  \Rightarrow \mathit{false}$, where $O$, contains no maximal element
  with operation label $\mathit{add}(a)$,
\item $S_{\mathit{ORS}}(\mathit{rem}(a))$ is similar to
  $\mathit{lookup}(a) \Rightarrow \mathit{true}$, and
\item $S_{\mathit{ORS}}(\mathit{elements}() \Rightarrow S)$ are all
  the tuples $(O,<,\emptyset)$ that are in
  $S_{\mathit{ORS}}(\mathit{lookup}(a) \Rightarrow \mathit{true})$ for
  each $a \in S$, and are in $S_{\mathit{ORS}}(\mathit{lookup}(a)
  \Rightarrow \mathit{false})$ for each $a \notin S$.
 \end{inparaenum}
 Thus, we can observe that the specification $S_{\mathit{ORS}}$ makes
 $\mathit{rem}$ cancel only $\mathit{add}$ operations ``visible'' to
 them.
 For example, \figurename~\ref{fig:generating operation context from annotated history for OR-set} (b) is in $S_{\mathit{ORS}}(\mathit{lookup}(a)
 \Rightarrow \mathit{true})$, since there is an $\mathit{add}(a)$ that
 is maximal w.r.t $<$. Here the subscript $1$ of $\mathit{add}(a)_1$
 is used to distinguish several $\mathit{add}(a)$ operations.

\bigskip
\noindent {\bf Example 2. Distributed list}: The Distributed List CRDT~\cite{Attiya:2016} mains the view of a list in each replica. It contains three methods: (1)
$\mathit{add}(a,\mathit{pos})$ inserts value $a$ into position
$\mathit{pos} \in \mathbb{N}$ of the list, (2) $\mathit{rem}(a)$ removes the
value $a$ from the list, and (3) $\mathit{read}() \Rightarrow l$
returns the list contents.
%
We assume that each value appears at most once, and therefore we shall
call them identifiers.

%Let $\Sigma_{\mathit{list}} = \{ add(a,pos,\top),rem(\top,a) \vert a \in \mathbb{D},pos \in \mathbb{N} \}$ be the set of contents of update operations.
The specification $S_{\mathit{List}}$ of the distributed list is
defined as follows.
Given an annotated history $ah = (O,<,\mathit{arb})$, we say that $ ah
\in S_{\mathit{list}}(\ell)$, if
\begin{inparaenum}[(1)]
% \setlength{\itemsep}{0.5pt}
\item $arb$ is a total order over all of the $add$ operations in $O \cup \{ o \}$, where $o \notin O$, and
\item letting \mbox{$R = \{\ i\ |\ \exists b,
    (add(b),\_,i),(rem(b),\_,\_) \in O \}$} be the set of identifiers already removed identifiers and $\mathit{seq} =
  \mathit{arb}\!\!\!\uparrow_{( O \cup \{ o \} \setminus R)}$\footnote{We denote by $R \uparrow_{S}$ the projection $R$ on $S \times S$.} be the ``active'' list content
  then, either
    \begin{inparaenum}[(i)]
    \item $\ell = add(a,pos) \wedge \mathit{seq}[pos] = o$,
    \item $\ell = rem(a) \wedge (add(a),\_,\_) \in O$, or
    \item $\ell = read() \Rightarrow l$, and $l$ is obtained by using $\mathit{lab}$ on $\mathit{seq}$.
    \end{inparaenum}
  \end{inparaenum}
% \fxwarning{GP: Using lab on seq??}
  % \todo{Give a declarative specification of the list, like for OR-set. Descriptions which look like ``imperative programs'', e.g., "We can go through operations", "During this process".}
In a nutshell, this specification indicates that we use the list order
of strong list specification in \cite{Attiya:2016} as the arbitration
order.
%
For example, \figurename~\ref{fig:history, annotated history and
  operation context} (c) is in specification of $o =
\mathit{add}(d,4)$, where no identifiers are removed, and the active list content is $\mathit{seq} = \mathit{add}(b,1) \cdot \mathit{add}(a,1) \cdot \mathit{add}(c,1) \mathit{add}(d,4)$, and $\mathit{seq}[4]$ is $o$.



% \fxnote{Contains or lookup: gotta make up our mind.}
% \fxfatal{GP: I have no idea of what this means. Has to be rewritten.}
\begin{remark}
  \label{remark: operation context of OR-set} Extracting
  operation contexts of an annotated history for OR-set is more
  complicated, since some orders in the visibility relation need to
  be ignored. For instance, consider the annotated history
  \figurename~\ref{fig:generating operation context from annotated
    history for OR-set} (a). We remark that that the visibility
  relation on same replica (replica order) is transitive.
  Then what should be the operation context $(O,<,\emptyset)$ of
  operation $\mathit{lookup}(a) \Rightarrow \mathit{true}$?
  Obviously $O = \{ \mathit{add}(a)_1, \mathit{rem}(a)_2 \}$. If we
  choose $<$ to be the visibility relation, then we have
  $(\mathit{add}(a)_1, \mathit{rem}(a)_2) \in <$. However, this is
  wrong, since $\mathit{rem}(a)_2$ does not cancel
  $\mathit{add}(a)_1$, but cancels $\mathit{add}(a)_2$ instead. To
  deal with this, we let $<\ =\ \emptyset$ and hence we ignore the
  visibility relation between $\mathit{add}(a)_1$ and
  $\mathit{rem}(a)_2$. Formally, given $o_1$ with $\mathit{lab}(o_1)
  = \mathit{add}(a)$, let $\mathit{FstRem}(o_1)$ be the set of
  $\mathit{rem}(a)$ operations, say $o_2$, such that $o_1$ is visible to
  $o_2$, and such that there is no other $\mathit{rem}(a)$ operations $o_3$ that
  satisfies ($(o_1,o_3),(o_3,o_2) \in \mathit{vis}$).
  %
  When we
  extract $<$ from $\mathit{vis}$, such pairs $(o_1,o_2)$ will be
  ignored since $o_2$ does not cancel $o_1$ ($o_2 \notin
  \mathit{FstRem}(o_1)$), and some intermediate operations between
  $o_1$ and $o_2$ w.r.t $\mathit{vis}$ is not in $O$.
  % When we extract $<$ from $\mathit{vis}$, such pairs $(o_1,o_2)$ will be
  % ignored: $o_2 \notin \mathit{FstRem}(o_1)$, and there exists
  % operation $o_3$, such that $(o_1,o_3),(o_3,o_2) \in \mathit{vis}$,
  % while $o_3 \notin O$.
  In \figurename~\ref{fig:generating
    operation context from annotated history for OR-set} (a), one
  such ignored pair is $(\mathit{add}(a)_1,\mathit{rem}(a)_2)$, since
  $\mathit{rem}(a)_2 \notin \mathit{FstRem}(\mathit{add}(a)_1)$,
  $(\mathit{add}(a)_1,\mathit{rem}(a)_1),
  (\mathit{rem}(a)_1,\mathit{rem}(a)_2) \in \mathit{vis}$ and
  $\mathit{rem}(a_1)$ is not visible to $\mathit{lookup}(a)
  \Rightarrow \mathit{true}$.
\end{remark}

\forget{


is in $S_{\mathit{ORS}}(\mathit{lookup}(a)
 \Rightarrow \mathit{true})$


 For the OR-set CRDT, the
  operation context of an annotated history is more complicated.
  For instance, given the annotated history
  \figurename~\ref{fig:history, annotated history and operation context}
  (d), the operation context of $\mathit{lookup}(a,\textit{true})$
  should be the one depicted in \figurename~\ref{fig:history, annotated
    history and operation context} (e), since $-a_2$ does not cancel
  $+a_1$, and they should not be related.
  Formally, given an annotated history $ah =
  (O,\mathit{vis},\mathit{arb})$ and an operation $o$,
  $ctxt(o)=(O',<,\mathit{arb}')$, where $O'$ and $\mathit{vis}'$ is as
  before, and $<$ is defined as $\mathit{vis} \uparrow_{O'} - \{
  (o_1,o_2) \vert o_2 \notin \mathit{FstRem}(O,\mathit{vis},o_1),$
  $\exists o_3 \in O, (o_1,o_3), (o_3,o_2),(o_1,o_2) \in \mathit{vis},
  o_1,o_2 \in O', o_3 \notin O' \}$.
  {\color {red}Here $\mathit{FstRem}$ records matched $\mathit{add}$
    and $\mathit{rem}$ pairs like $(+a_1,-a_1)$,} and is defined as
  $\mathit{FstRem}(O,\mathit{vis},o) = \{o' \vert
  \mathit{lab}(o')=\mathit{rem}(a), (o,o') \in \mathit{vis}, \neg
  \exists o'', lab(o'') = \mathit{rem}(a) \wedge (o,o''),(o'',o') \in
  \mathit{vis} \}$ for $lab(o) = \mathit{add}(a)$.
}
%{\color {red}Remark: For OR-set, the operation context of annotated history is more complicated: Given an annotated history $(O,\mathit{vis},\mathit{arb})$ and an operation $o$, $ctxt(o)=(O',<,\mathit{arb}')$, where $O'$ and $\mathit{vis}'$ is as before, and $< = \mathit{vis} \uparrow_{(O' \times O')} - \{ (o_1,o_2) \vert o_2 \notin FstRem(O,\mathit{vis},o_1), \exists o_3 \in O, (o_1,o_3), (o_3,o_2),(o_1,o_2) \in \mathit{vis}, o_1,o_2 \in O', o_3 \notin O' \}$. Here $FstRem(O,\mathit{vis},o)$ is the set of first visible matching remove of $o$ w.r.t $\mathit{vis}$, and is defined as $FstRem(O,\mathit{vis},o) = \{o' \vert lab(o')=rem(a), (o,o') \in \mathit{vis}, \neg \exists o'',  lab(o'') = rem(a)  \wedge (o,o''),(o'',o') \in \mathit{vis} \}$ for $lab(o) = add(a)$.}


%Given an annotated history $(O,\mathit{vis},\mathit{arb})$ and an operation $o$, $ctxt(o)=(O',<,\mathit{arb}')$ of OR-set is defined as follows: $\mathit{arb}' = \emptyset$, and $< = \mathit{vis} \uparrow_{(O' \times O')} - \{ (o_1,o_2) \vert o_2 \notin Minus(O,\mathit{vis},o_1), \exists o_3 \in O, (o_1,o_3), (o_3,o_2),(o_1,o_2) \in \mathit{vis}, o_1,o_2 \in O', o_3 \notin O' \}$. Here $Minus(O,\mathit{vis},o)$ is the set of first visible matching remove of $o$ w.r.t $\mathit{vis}$, and is defined as $Minus(O,\mathit{vis},o) = \{o' \vert lab(o')=rem(a), (o,o') \in \mathit{vis}, \neg \exists o'',  lab(o'') = rem(a)  \wedge (o,o''),(o'',o') \in \mathit{vis} \}$ for $lab(o) = add(a)$.




%The specification $S_{\mathit{list}}$ is defined as follows: $(O,<,arb) \in S_{\mathit{list}}(\ell)$, if

%\begin{itemize}
%\setlength{\itemsep}{0.5pt}
%\item[-] $<^{-1}$ contains finite elements, $<$ is acyclic and $arb$ is a total order of $add$ operations in $O \cup \{ o \}$, where $o \notin O$ and is in domain of $arb$.

%\item[-] {\color {red}Function $f: O \cup \{ o \} \rightarrow P(O \cup \{ o \})$. For the case of $o' \in O$, let $S(o') = f(o_1) \cup \ldots \cup f(o_k)$, where $o_1,\ldots,o_k$ is the immediate predecessor of $o'$ w.r.t $<$. Then, $f(o')$ is recursively defined as

%    \begin{itemize}
%    \setlength{\itemsep}{0.5pt}
%    \item[-] $S(o')$, if $lab(o')=read()\Rightarrow list \wedge list = lab( arb \uparrow_{ (<^{-1}(o')-\{ x \vert (x,\_) \in S(o')\}) } )$.

%    \item[-] $S(o')$, if $lab(o')=add(a,pos) \wedge ( arb \uparrow_{ (<^{-1}(o')-\{ x \vert (x,\_) \in S(o')\}) } )[pos]=o'$.

%    \item[-] $S(o') \cup \{ (o_a,o') \}$, if $lab(o')=rem(pos)\Rightarrow a \wedge ( arb \uparrow_{ (<^{-1}(o')-\{ x \vert (x,\_) \in S(o')\}) } )[pos]=o_a \wedge lab(o_a)=add(a,\_)$.

%    \item[-] $\mathit{Undef}$, otherwise.
%    \end{itemize}

%    For the case of $o$, let $S(o)$ be the union of $f(o'')$ for each $o'' \in O$, and the other part is the same as above. We require that, for each $o' \in O \cup \{ o \}$, $f(o') \neq \mathit{Undef}$.}
%\end{itemize}

%\todo{Give a declarative specification of the list, like for OR-set. Descriptions which look like ``imperative programs'', e.g., "We can go through operations", "During this process".}

%In our definition of distributed list specification, the arbitration order works similarly as the list order of strong list specification in \cite{Attiya:2016}.














%\todo{Use $r$ instead of $rid$ and $i$ instead of $oid$. But be careful to not use $i$ in other contexts, for instance remove the $\forall i$ from the previous section. Keep $i$ and $j$ only for operation ids.}

%CRDT has two kinds of method: query methods and update methods: Operations of query methods take effect only in one replica, while operations of update methods will be delivered to other replicas.

%A specification $Spec$ is a function that maps each operation label $\ell$ into a set of tuples $(O,<,arb)$, where $O$ is a set of update operations, $<$ is a partial order over $O$, and $arb$ is a partial order over $O \cup \{ o \}$ called arbitration order, where $lab(o)=\ell \wedge o \notin O$.

%We require that, given $o_1,o_2 \in O$, if $\mathit{vis}^{-1}(o_1) = \mathit{vis}^{-1}(o_2)$, then $<_1 = <_2$, and $arb_1 \cup arb_2$ is acyclic, where $ctxt(o_1)=(O_1,<_1,arb_1)$ and $ctxt(o_2)=(O_2,<_2,arb_2)$.



%\todo{The rest of the paragraph should be a footnote. Uninteresting details.}
%Here we require that each operation in $O \cup \{ o \}$ has unique operation identifier. Such $(O,<,<_{\mathit{arb}},l)$ tuples are called ($\Sigma$-labeled) partial-ordered set (poset, for short), where $\Sigma$ is a set of update operation contents contains that of $O$. Two labeled posets are isomorphic if there exists a bijection of operations that preserve operation contents, labels and orders. Here we require $Spec$ to be isomorphic closed: if $x \in Spec$ and $x$ and $y$ are isomorphic, then $y \in Spec$. Since the labeling function of poset is fixed, we could ignore it when the context is clear.

%\todo{I would suggest to define specifications only for query operations. I guess that you need to include $o$ in $O$ for the updates like inserting in a list. But this is kind of ugly, so I would prefer that $O$ doesn't contain $o$}

%\todo{I guess $l$ is not needed. An operation is already a label (content in your terms) with ids}

%\todo{Use $\mathit{arb}$ instead of $<_{\mathit{arb}}$. I told you several times, don't try to minimize the space occupied by your notations. And don't use complicated indices or superscripts.}

%\todo{I don't see the "deterministic" condition: for a given tuple $(O,<,arb)$, the return value is unique.}

%\todo{I think that this part about plus-minus specifications is not useful here. Give standard examples and push this discussion/examples when needed.}





%\subsection{Consistencies}
%\label{subsec:consistencies}

%\todo{Define a history as $(O,\mathit{ro})$ (again, forget about long indices), then an annotated history as $(O,\mathit{ro},\mathit{vis},\mathit{arb})$ (dont use $\mathit{mathit}$).}

%A history is a tuple $(O,\mathit{ro})$, where

%\begin{itemize}
%\setlength{\itemsep}{0.5pt}
%\item[-] $O$ is a set of operations.

%\item[-] $\mathit{ro}$ is called the replica order. For each replica $r \in \mathbb{R}$, $\mathit{ro}$ is a irreflexive total order over operations with replica identifier $r$. $\mathit{ro}$ does not relate operations with different replica identifiers. We also require that for each operation $o \in O$, $\mathit{ro}^{-1}(o)$ is finite.
%\end{itemize}

%An annotated history is a tuple $(O,\mathit{ro},\mathit{vis},\mathit{arb})$, where

%\begin{itemize}
%\setlength{\itemsep}{0.5pt}
%\item[-] $(O,\mathit{ro})$ is a history.

%\item[-] $\mathit{vis}$ is irreflexive and acyclic, and is called the visibility order. We require that for each operation $o \in O$, $\mathit{vis}^{-1}(o)$ is finite.

%\item[-] $\mathit{arb}$ is the arbitration order over update operations of $O$.
%\end{itemize}

%\todo{Local interpretation meant something else in our previous paper. Use operation context for $(\mathit{vis}^{-1}(o),<,\mathit{arb}\downarrow (\mathit{vis}^{-1}(o)\times \mathit{vis}^{-1}(o)))$ where $<$ is defined as you say.}

%Given an annotated history $(O,\mathit{ro},\mathit{vis},\mathit{arb})$ and an operation $o \in O$, the operation context of $o$ is a tuple $ctxt(o)=(O_o,<,arb_o)$, where

%\begin{itemize}
%\setlength{\itemsep}{0.5pt}
%\item[-] $O_o$ is the set of update operations in $\mathit{vis}^{-1}(o)$,

%\item[-] $arb_o$ is the projection of $arb$ over update operations of $O_o \cup \{ o \}$.

%\item[-] $< \subseteq <_{\mathit{vis}} \uparrow_{(O_o \times O_o)}$ and it is irreflexive.

%We require that, given operations $o_1,o_2,o'_1,\ldots,o'_m \in O_o$, if $(o_1,o_2) \in \mathit{vis}$ via $o'_1,\ldots,o'_m$, then $(o_1,o_2) \in <$. We say that $(o_1,o_2) \in \mathit{vis}$ via $o'_1,\ldots,o'_m$, if $(o_1,o'_1),$ $(o'_1,o'_2), \ldots, (o'_{\mathit{m-1}},o'_m),(o'_m,o_2)$ are all in $\mathit{vis}$.

%\item[-] {\color {red}We require that, given $o_1,o_2 \in O$, if $\mathit{vis}^{-1}(o_1) = \mathit{vis}^{-1}(o_2)$, then $<_1 = <_2$, and $arb_1 \cup arb_2$ is acyclic, where $ctxt(o_1)=(O_1,<_1,arb_1)$ and $ctxt(o_2)=(O_2,<_2,arb_2)$.}
%\end{itemize}

%Let us define strong-return-value consistency (SRVC consistency, for short) as follows:

%\todo{Define ``an annotated history satisfying SRVC and then a history satisfying SRVC, i.e., there exists $\mathit{vis}$ and $\mathit{arb}$ such that the resulting annotated history satisfies SRVC.}

%\begin{definition}[Strong-return-value consistency]
%\label{definition:strong return value consistency}
%An annotated history $(O,\mathit{ro},\mathit{vis},\mathit{arb})$ is SRVC w.r.t specification $Spec$, if there exists function $ctxt$, such that,

%\begin{itemize}
%\setlength{\itemsep}{0.5pt}
%\item[-] $\mathit{ro} \subseteq \mathit{vis} \wedge \mathit{vis}$ is acyclic.

%\item[-] $\forall o \in O$, $ctxt(o) \in Spec(lab(o))$.
%\end{itemize}

%A history $(O,\mathit{ro})$ is SRVC w.r.t $Spec$, if there exists $\mathit{vis}$ and $\mathit{arb}$, such that $(O,\mathit{ro},\mathit{vis},\mathit{arb})$ is SRVC consistent w.r.t $Spec$.
%\end{definition}

%%% Local Variables:
%%% mode: latex
%%% TeX-master: "draft"
%%% End:
















\section{The Priority Queue ADT}
\label{sec:priority queue and data-independence}

We consider priority queues whose interface contains two methods $\textit{put}$ and $\textit{rm}$ for adding and respectively, removing a value. Each value is assigned with a priority when being added to the data structure (using a call to  $\textit{put}$) and the remove method $\textit{rm}$ removes a value with a minimal priority. For generality, we assume that the set of priorities is partially-ordered. Incomparable priorities can be removed in any order. In case multiple values are assigned with the same priority, $\textit{rm}$ returns the least recent value (according to a FIFO semantics). Also, when the set of values stored in the priority queue is empty, $\textit{rm}$ returns the distinguished value $\textit{empty}$. Concurrent implementations of the priority queue allow the methods $\textit{put}$ and $\textit{rm}$ to be called concurrently from different threads. However, every method invocation should give the illusion that it takes place instantaneously at some point between its invocation and its return. We formalize (concurrent) executions and implementations in Section~\ref{ssec:exec}, Section~\ref{ssec:semantic_prop} introduces a set of properties satisfied by all the  implementations we are aware of, and Section~\ref{ssec:lin} defines the standard correctness criterion for concurrent implementations of ADTs known as \emph{linearizability}~\cite{journals/toplas/HerlihyW90}.

\subsection{Executions}\label{ssec:exec}

We fix a (possibly infinite) set $\mathbb{D}$ of data values, a (possibly infinite) set $\mathbb{P}$ of priorities, a partial order $\prec$ among elements in $\mathbb{P}$, and an infinite set $\mathbb{O}$ of operation identifiers.
The latter are used to match call and return actions of the same invocation. Call actions $\textit{call}_o(\textit{put},a,p)$ and $\textit{call}_o(\textit{rm},a')$ with $a\in \mathbb{D}$, $a'\in \mathbb{D}\cup\{\textit{empty}\}$, $p \in \mathbb{P}$, and $o \in \mathbb{O}$, combine a method name and a set of arguments with an operation identifier. The return value of a remove is transformed to an argument value for uniformity~\footnote{Method return values are guessed nondeterministically, and validated at return points.
This can be handled using the {\tt assume} statements of typical formal specification languages, which only admit executions satisfying a given predicate.}.
The return actions are denoted in a similar way as $\textit{ret}_o(\textit{put},a,p)$ and respectively, $\textit{ret}_o(\textit{rm},a')$.

An \emph{execution} $e$ is a sequence of call and return actions which satisfy the following well-formedness properties: each return is preceded by a matching call (i.e., having the same operation identifier), and each operation identifier is used in at most one call/return. We assume every set of executions is closed under isomorphic renaming of operation identifiers. An execution is called \emph{sequential} when no two operations overlap, i.e., each call action is immediately followed by its matching return action, and \emph{concurrent} otherwise. To ease the reading, we write a sequential execution as a sequence of $\textit{put}(a,p)$ and $\textit{rm}(a)$ symbols representing a pair of actions $\textit{call}_o(\textit{put},a,p)\cdot \textit{ret}_o(\textit{put},a,p)$ and $\textit{call}_o(\textit{rm},a)\cdot \textit{ret}_o(\textit{rm},a)$, respectively (where $o\in\mathbb{O}$). For example, given two priorities $p_1 \prec p_2$, $\textit{put}(a,p_2) \cdot \textit{put}(b,p_1) \cdot \textit{rm}(b)$ is a sequential execution of the priority queue ($\textit{rm}$ returns $b$ because it has smaller priority).

We define seqPQ, the set of sequential priority queue executions, semantically via labelled transition system ($LTS$). An LTS is a tuple $A=(Q,\Sigma,\rightarrow,q_0)$, where $Q$ is a set of states, $\Sigma$ is an alphabet of transition labels, $\rightarrow\subseteq Q\times\Sigma\times Q$ is a transition relation and $q_0$ is the initial state. We model priority queue as an LTS $\textit{PQ}$ defined as follows: {\color {blue}each state of $\textit{PQ}$}{\color {red}each state of the LTS $\textit{PQ}$} is a mapping associating priorities in $\mathbb{P}$ with sequences of values in $\mathbb{D}$ representing a snapshot of the priority queue (for each priority, the values are ordered as they were inserted); the transition labels are $\textit{put}(a,p)$ and $\textit{rm}(a)$ operations; Each transition modifies the state as expected. For example, $q_1 \xrightarrow{\textit{rm}(\textit{empty})} q_2$ if $q_1 = q_2$, and $q_1$ and $q_2$ map each priority to the empty sequence $\epsilon$. Then, {\color{blue}seqPQ is the set of words accepted by $\textit{PQ}$}{\color {red}seqPQ is the set of traces of $\textit{PQ}$}. {\color {red}The detailed definition of $\textit{PQ}$ can be found in the long version \cite{CONCUR2017Ahmed}.}
%{The detailed definition of $\textit{PQ}$ can be found in Appendix \ref{sec:appendix definition of seqPQ and proof of Lemma EQP rules and semantics}.}


An implementation $\mathcal{I}$ is a set of executions. Implementations represent libraries whose methods are called by external programs. In the remainder of this work, we consider only completed executions, where each call action has a corresponding return action. This simplification is sound when implementation methods can always make progress in isolation: formally, for any execution $e$ with pending operations, there exists an execution $e'$ obtained by extending $e$ only with the return actions of the pending operations of $e$. Intuitively this means that methods can always return without any help from outside threads, avoiding deadlock.


%a finite set $\mathbb{M}$ of methods, and an infinite set $\mathbb{O}$ of operation (identifiers). Given $m \in \mathbb{M}$, $x \in \mathbb{D}$, and $o \in \mathbb{O}$, a call action $\textit{call}_o (m,x)$ represents an invocation to method $m$ with argument $x$, while a return action $\textit{ret}_o (m,x)$ represents a response from method $m$ with return value $x$. \footnote{Call actions with more than one arguments are similarly defined.} Here $o$ is used to match return actions to their call actions: $\textit{cal}_o (m,x)$ matches $\textit{ret}_{o'} (m',x')$, if $o=o' \wedge m=m'$. A sequential execution is a sequence of call and return actions, while each call action is immediately followed by its matching return action. Let $\cdot$ be the concatenation of sequences. To ease the reading, given a sequential execution $e=\textit{call}_{o_1}(m_1,a_1) \cdot \textit{rm}_{o_1}(m_1,b_1) \cdot \ldots \cdot \textit{call}_{o_n}(m_n,a_n) \cdot \textit{rm}_{o_n}(m_n,b_n)$, when the context is clear, we can write it as $e=m_1(a_1,b_1) \cdot \ldots \cdot m_n(a_n,b_n)$. We write $m(a,b)$ as $m(a)$ or $m(b)$, when there is no return value or there is no argument, respectively.



%A priority queue with partially-ordered priorities (priority queue for short) contains two method: $\textit{put}$ and $\textit{rm}$. A $\textit{put}$ method has two arguments, while the first argument is an item and the second argument is its priority. A $\textit{put}$ method is used to put an item into the priority queue with certain priority. Here the item is chosen from a specific (possibly infinite) data domain $\mathbb{D}$ and priority is chosen from a (possibly infinite) set $\mathbb{P}$. Moreover, there is a strict partial-order $\prec$ among elements in $\mathbb{P}$. A $\textit{rm}$ method intends to remove the item with minimal priority (w.r.t $\prec$) in priority queue and then returns it. It works as follows:

%\begin{itemize}
%\setlength{\itemsep}{0.5pt}
%\item[-] If the priority queue is empty, then $\textit{rm}$ returns $\textit{empty}$.
%
%\item[-] Else, $\textit{rm}$ choose one of minimal priority of items in priority queue, and returns the earliest putted item of this priority. Note that if there are more than one incomparable and minimal candidate priorities, then the chosen of priority is arbitrary.
%\end{itemize}


%Let us introduce the notion of sequential executions, which modelled the specification of concurrent libraries. We fix a (possibly infinite) set $\mathbb{D}$ of data values, a finite set $\mathbb{M}$ of methods, and an infinite set $\mathbb{O}$ of operation (identifiers). Given $m \in \mathbb{M}$, $x \in \mathbb{D}$, and $o \in \mathbb{O}$, a call action $\textit{call}_o (m,x)$ represents an invocation to method $m$ with argument $x$, while a return action $\textit{ret}_o (m,x)$ represents a response from method $m$ with return value $x$. \footnote{Call actions with more than one arguments are similarly defined.} Here $o$ is used to match return actions to their call actions: $\textit{cal}_o (m,x)$ matches $\textit{ret}_{o'} (m',x')$, if $o=o' \wedge m=m'$. A sequential execution is a sequence of call and return actions, while each call action is immediately followed by its matching return action. Let $\cdot$ be the concatenation of sequences. To ease the reading, given a sequential execution $e=\textit{call}_{o_1}(m_1,a_1) \cdot \textit{rm}_{o_1}(m_1,b_1) \cdot \ldots \cdot \textit{call}_{o_n}(m_n,a_n) \cdot \textit{rm}_{o_n}(m_n,b_n)$, when the context is clear, we can write it as $e=m_1(a_1,b_1) \cdot \ldots \cdot m_n(a_n,b_n)$. We write $m(a,b)$ as $m(a)$ or $m(b)$, when there is no return value or there is no argument, respectively.

\subsection{Semantic Properties of Priority Queues}\label{ssec:semantic_prop}

We define two properties which are satisfied by priority queue implementations and which are important for the results that follow: (1) \emph{data independence}~\cite{conf/popl/Wolper86,conf/tacas/AbdullaHHJR13} states that priority queue behaviors do not depend on the actual values which are added to the queue, and (2) \emph{closure under projection}~\cite{DBLP:conf/icalp/BouajjaniEEH15} states that intuitively, remove operations can return the same values no matter how many other different values are in the queue, assuming they don't have more important priorities.

%Data-independence \cite{Wolper:1986} can be used to effectively handle unbounded data domain. In this paper, we slightly modify the notion of data-independence in \cite{Wolper:1986} and propose data-independence for priority queues. Let $\_$ denote an element, of which the value is irrelevant.
An execution $e$ is \emph{data-differentiated} if every value is added at most once, i.e., for each $d \in \mathbb{D}$, $e$ contains at most one action $\textit{call}_o(\textit{put},d,p)$ with $o\in\mathbb{O}$ and $p\in \mathbb{P}$. Note that this property concerns only values, a data-differentiated execution $e$ may contain more than one value with the same priority. The subset of data-differentiated executions of a set of executions $E$ is denoted by $E_{\neq}$.

A renaming function $r$ is a function from $\mathbb{D}$ to $\mathbb{D}$. Given an execution $e$, we denote by $r(e)$ the execution obtained from $e$ by replacing every data value $x$ by $r(x)$. Note that $r$ renames only the values and keep the priorities unchanged. Intuitively, renaming values has no influence on the behavior of the priority queue, contrary to renaming priorities.

%\vspace{-6pt}
\begin{definition}\label{def:priority-value data-independence}
A set of executions $E$ is \emph{data independent} iff
\begin{itemize}
\setlength{\itemsep}{0.5pt}
\item[-] for all $e \in E$, there exists $e' \in E_{\neq}$ and a renaming function $r$, such that $e=r(e')$,

\item[-] for all $e \in E$ and for all renamings $r$, $r(e) \in E$.
\end{itemize}
\end{definition}

The following lemma is a direct consequence of definitions.

\begin{lemma}
seqPQ is data independent.
\end{lemma}

Beyond the sequential executions, every (concurrent) implementation of the priority queue that we are aware of is data-independent. Therefore, from now on, we consider only data-independent implementations. This assumption enables a reduction from checking the correctness of an implementation $\mathcal{I}$ to checking the correctness of only its data-differentiated executions in $\mathcal{I}_{\neq}$.

Besides data independence, the sequential behaviors of the priority queue satisfy the following closure property: a behavior remains valid when removing all the operations with an argument in some set of values $D \subseteq \mathbb{D}$ and any $\textit{rm}(\textit{empty})$ operation (since they are read-only and they don't affect the queue's state).
In order to distinguish between different $\textit{rm}(\textit{empty})$ operations while simplifying the technical exposition, we assume that they receive as argument a value, i.e., call actions are of the form $\textit{call}_o(\textit{rm},\textit{empty},a)$ for some $a\in \mathbb{D}$. We will make explicit this argument only when needed in our technical development. The projection $e \vert D$ of an execution $e$ to a set of values $D \subseteq \mathbb{D}$ is obtained from $e$ by erasing all call/return actions with an argument not in $D$. We write $e \setminus x$ for the projection $e \vert_{ \mathbb{D} \setminus \{ x \} }$. Let $\textit{proj}(e)$ be the set of all projections of $e$ to a set of values. {\color {red}The following lemma states that seqPQ is closed under projection.}

%The proof of the following lemma can be found in Appendix \ref{sec:appendix proofs in section priority queue and data-independence}.

%\begin{lemma}\label{lem:closure_proj}
%$\seqPQ$ is closed under projection, i.e., $\textit{proj}(e)\subseteq \seqPQ$ for each $e\in \seqPQ$.
%\end{lemma}
\begin{restatable}{lemma}{EPQClosedUnderProjection}
\label{lem:closure_proj}
seqPQ is closed under projection, i.e., $\textit{proj}(e)\subseteq seqPQ$ for each $e\in seqPQ$.
\end{restatable}


\subsection{Linearizability}\label{ssec:lin}

%A (concurrent) execution $e$ is a sequence of call and return actions which satisfy a well-formedness property: every return has a matching call action before it in $e$, and an operation $o$ can be used only twice in $e$, once in a call action, and once in a return action. The definition of data-differentiated, renaming function, data-independence and projection extends to (concurrent) executions. For instance, an execution is data-differentiated if, for all $d \in \mathbb{D}$, there is at most one $\textit{cal}_{\_}(\textit{put},d,\_)$.
%
%An implementation $\mathcal{I}$ is a set of concurrent executions. Implementations represent libraries whose methods are called by external programs. In the remainder of this work, we consider only completed executions, where each call action has a corresponding return action. This simplification is sound when implementation methods can always make progress in isolation \cite{Henzinger:2013}: formally, for any execution $e$ with pending operations, there exists an execution $e'$ obtained by extending $e$ only with the return actions of the pending operations of $e$. Intuitively this means that methods can always return without any help from outside threads, avoiding deadlock.
We recall the notion of \emph{linearizability}~\cite{journals/toplas/HerlihyW90} which is the \emph{de facto} standard correctness condition for concurrent data structures.
Given an execution $e$, the happen-before relation $<_{\textit{hb}}$ between operations~\footnote{In general, we refer to operations using their identifiers.} is defined as follows: $o_1 <_{\textit{hb}} o_2$, if the return action of $o_1$ occurs before the call action of $o_2$ in $e$. The happens-before relation is an interval order \cite{DBLP:conf/popl/BouajjaniEEH15}: for distinct $o_1,o_2,o_3,o_4$, if $o_1 <_{\textit{hb}} o_2$ and $o_3 <_{\textit{hb}} o_4$, then either $o_1 <_{\textit{hb}} o_4$, or $o_3 <_{\textit{hb}} o_2$. Intuitively, this comes from the fact that concurrent threads share a notion of global time.

Given a (concurrent) execution $e$ and a sequential execution $s$, we say that $e$ is linearizable w.r.t $s$, denoted $e \sqsubseteq s$, if there is a bijection $f: O_1 \rightarrow O_2$, where $O_1$ and $O_2$ are the set of operations of $e$ and $s$, respectively, such that (1) {\color{blue}$o$ and $f(o)$ have the same call and return actions}{\color {red}$o$ and $f(o)$ is the same operation}\footnote{An $m(a)$-operation in an execution $e$ is an operation identifier $o$ s.t. $e$ contains the actions $\textit{call}_o(m,a)$, $\textit{ret}_o(m,a)$.}, and (2) if $o_1 <_{\textit{hb}} o_2$, then $f(o_1) <_{\textit{hb}} f(o_2)$.
A (concurrent) execution $e$ is linearizable w.r.t a set $S$ of sequential executions, denoted $e \sqsubseteq S$, if there exists $s \in S$ such that $e \sqsubseteq s$. A set of concurrent executions $E$ is linearizable w.r.t $S$, denoted $E \sqsubseteq S$, if $e \sqsubseteq S$ for all $e \in E$.

The following lemma states that by data-independence, it is enough to consider only data-differentiated executions when checking linearizability. %(see Appendix \ref{sec:appendix proofs in section priority queue and data-independence}).
This is similar to that in \cite{conf/tacas/AbdullaHHJR13,DBLP:conf/icalp/BouajjaniEEH15}, where they use the notion of data-independence in \cite{conf/popl/Wolper86}.
Section~\ref{sec:checking inclusion by recursive procedure} will focus on characterizing linearizability for data-differentiated executions. %The proof of this lemma can be found in Appendix \ref{sec:appendix in section data-independence of EPQ}.

\begin{restatable}{lemma}{DataDifferentiatedisEnoughforPQ}
\label{lemma:data differentiated is enough for PQ}
A data-independent implementation $\mathcal{I}$ is linearizable w.r.t a data-independent set $S$ of sequential executions, if and only if $\mathcal{I}_{\neq}$ is linearizable w.r.t. $S_{\neq}$.
\end{restatable}


