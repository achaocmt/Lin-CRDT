%!TEX root = draft.tex

\section{CRDT Implementations}
\label{sec:CRDT implementations}

\begin{figure}
  \centering

  \(
  \begin{array}[t]{rcll}
    \aobj & \in  & \objs & \text{Objects} \\
    \arep & \in & \reps & \text{Replicas} \\
    \amethod & \in & \methods & \text{Methods}\\
    \adomain & \subseteq & \datadomain & \text{Data Domain} \\
    \acrdttyp & \in & \powerset{\methods} \times \powerset{\adomain} & \text{CRDT definition} \\
    \astate & \in & \states & \text{States} \\
    \amesg & \in & \messages & \text{Messages} \\
    \amesgset & \subseteq & \messages & \text{Message Set} \\
    \aop & \in & \ops \equiv \opids & \text{Operations}\ (ID) \\
    \alabellong[\amethod]{\argv}{\retv} \equiv \alabel  & \in & \methods \times \adomain \times \adomain & \text{Operation label} \\
    \arepord & \subseteq & \ops \times \ops & \text{Replica Order} \\
    \avisord & \subseteq & \ops \times \ops & \text{Visibility Order}
  \end{array}
  \)
  \caption{Summary of notations}
  \label{fig:notations}
\end{figure}


To formalize our CRDT correctness criterion we will introduce the
following semantic domains.
%
We let $\aobj \in \objs$ be an object in the set of objects $\objs$. Similarly,
$\arep \in \mathbb{R}$ is a replica in the set of replicas
$\reps$.
We assume that both objects and replicas are uniquely identified, and
therefore we will equate an object with its identifier, with the same
convention applying to replicas.

We will assume that the specification of a CRDT is given by a set of
method names $\amethod \in \methods$, and that each method has a
number of arguments sampled from a data domain $\datadomain$, and a
return value also from the data domain.
We ignore here the issues of typing which should be ensured by the
underlying programming language.
Hence, a data type $\acrdttyp = (\amethodset, \adomain)$ is given by a
set of method names $\amethodset \subseteq \methods$ and a data domain
$\adomain \subseteq \datadomain$.

%A distributed system contains multiple objects, and each objects is replicated on each replica. Each object has a type, which contains its method and data type. A client of a replica interact with the objects by calling the method and then obtaining the return value. Here we do not bound the number of replica identifiers and objects.

% Let $\objs$ be the set of objects and $\mathbb{R}$ be the set of replica identifiers. We consider a finite set $\mathbb{M}$ of method names; and a possibly infinite set $\mathbb{D}$ of arguments and return values, the data domain. Each data type $t = (M,D)$ has a set $M \subseteq \mathbb{M}$ of methods and a data domain $D \subseteq \mathbb{D}$.

Without loss of generality we will consider that the methods in $\mathbb{M}$ can be separated in two disjoint sets of methods: $\mathbb{Q}$ query methods that has no influence on the ``abstract state'' and normally returns an observation of the ``abstract state'' , and $\mathbb{U}$ update methods that has influence on the ``abstract state''. %Note that some update operation also need to read the ``abstract state''. For example, a $add(a,b)$ operation is an operation of distributed list which intends to put item $a$ immediately after item $b$. This operation implicitly requires that item $b$ is already in list.

$\mathit{Optimistic \ replication \ algorithms}$ is a type of distributed algorithms where each client contains a copy of data structure; a client operations takes effect instantly at its replica without any synchronization, and then broadcast to other replicas and got applied. Convergent or Commutative Replicated Data Types (CRDTs) is a typical kinds of optimistic replication algorithms. In this section, we will introduce CRDT algorithms and their formation \footnote{To be exact, there are two kinds of CRDT algorithms: state-based and operation based. The implementations we discussed in this section is operation-based. The state-based CRDT can be similarly formalized and verified, and we discuss state-based CRDT in the discussion section of this paper.}.

In CRDT, each query operation works locally, while each update operation will inform other replica about its update. We takes a algorithm, replicated growable array (RGA), as an example of operation-based CRDT and it is shown below.

\renewcommand{\algorithmcfname}{CRDT Implementation}
\noindent
%\begin{minipage}{.5\textwidth}
\noindent\begin{algorithm}[H]
$\mathit{payload}$ TI-tree N, set $\mathit{Tomb}$; \\
$\mathit{initial}$ $\emptyset$,$\emptyset$; \\

$add(a,b)$ \\
\ \ $\mathit{atSource}$: \\
\ \ \ \ $\mathit{pre}$: \ $b = \circ \vee$ %( b \neq \circ \wedge (b,\_,\_) \in N \wedge b \notin \mathit{Tomb})$ \\

% \ \ \ \ \If {$N = \emptyset$}
% { \ \ \ \ let \ $ts_a$ = (myID(),1); \\ }
% \ \ \ \ \Else
% {\ \ \ \ let \ $ts_a$ = (myID(),$\mathit{max}\{ c' \vert (\_,(\_,c'),\_) \in N \} +1$); \\ }

% \ \ \ \ \%If {$b = \circ$}
%    { \ \ \ \ let \ $ts_b$ = (0,0); \\ }
%\ \ \ \ \Else
%    { \ \ \ \ let \ $ts_b$ be time-stamp of $b$ in $N$; \\ }

\ \ \ \ let \ $ts_a$ = ($N = \emptyset$) ? (1,$\mathit{myID}$()) ! ($\mathit{max}\{ c' \vert (\_,(\_,c'),\_) \in N \} +1$,$\mathit{myID}$()); \\
\ \ \ \ let \ $ts_b$ = ($b = \circ$) ? (0,$r_0$) ! (the time-stamp of $b$ in $N$); \\

\ \ $\mathit{downstream}(a,ts_a,ts_b)$: \\
\ \ \ \ $\mathit{pre}$: \ $b = \circ \vee ( b \neq \circ \wedge (b,ts_b,\_) \in N)$ \\

\ \ \ \ $N = N \cup \{ (a,ts_a,ts_b) \}$.


$rem(a)$ \\
\ \ $\mathit{atSource}$: \\
\ \ \ \ $\mathit{pre}$: \ $a \neq \circ \wedge (a,\_,\_) \in N \wedge a \notin \mathit{Tomb}$ \\

\ \ $\mathit{downstream}(a)$: $\mathit{pre}$ \ $a \neq \circ \wedge (a,\_,\_) \in N)$

\ \ \ \ $\mathit{Tomb} = \mathit{Tomb} \cup \{ a \}$.

$read()$ \\
\ \ \ \ \KwRet $\mathit{trans}(N,\mathit{Tomb})$; \\

\caption{RGA}
\label{Method-rga}
\end{algorithm}

Keyword $\mathit{payload}$ indicate the local state of a replica, and keyword $\mathit{initial}$ specifies the initial value of local state. Each update operation is executed with two phases: Its first phase, marked $\mathit{atSource}$, is local to the current replica and does not modify the local state. It is enabled if its (optional) pre-condition, marked $\mathit{pre}$, is true currently in local state. It generates the information to be delivered, which is the argument of $\mathit{downstream}$. Its second phase, marked $\mathit{downstream}$, executed immediate after the current replica, and asynchronously at other replica when they receive the message of this operation. It is enabled if its (optional) pre-condition is true. The local state will be modified in this phase.

In RGA algorithm, a replica store the list as a timestamp insertion tree (TI-tree) $N$, and stores the deleted items in tombstone $\mathit{Tomb}$. A TI-tree $N$ is a set of tuples $(a,t,p)$, where $a$ is a item, $t$ is its unique time-stamp, and $p$ is the time-stamp of its ``parent'' node. Each time-stamp is a tuple $(c,r)$ with $c \in \mathbb{N}$ and $r \in \mathbb{R}$. A total order $<_{\mathit{ts}}$ between time-stamps is defined, such that $(c_1,r_1) <_{\mathit{ts}} (c_2,r_2)$, if $c_1 < c_2 \vee (c_1 = c_2 \wedge r_1 <_r r_2)$, where $<_r$ is a total-order over $\mathbb{R}$. There is a pre-existed item $\circ$ of TI-tree with time stamp $(0,r_0)$, which are considered as the root of the tree. Each element of $N$ should have unique item and time stamp, and the elements of $N$ are required to form a tree by following the parent field. The tombstone $\mathit{Tomb}$ is a set of items and records items been removed from the list.

Method $\mathit{add}(a,b)$ intends to add item $a$ into the list at a position immediately after that of a existing item $b$. Method $\mathit{rem}(a)$ removes $a$ from the list. Method $\mathit{read}$ returns the current list content. When the current replica does $\mathit{add}(a,b)$, it generate a tuple $(a,ts_a,ts_b)$ and put it into $N$. Here $ts_b$ is the time-stamp of $b$, and $ts_a$ is a new time-stamp that is larger than any time stamp in $N$. When the current replica does $\mathit{rem}(a)$, it put $a$ into tombstone. When the current replica does $\mathit{read}$, it uses function $\mathit{trans}(N,\mathit{Tomb})$ to return the list seen by the current replica, which is a sequences obtained by traversing $N$ in prefix order (children are visited in decreasing time-stamp order) and keeping only items that are not in $\mathit{Tomb}$.

Let $\mathbb{MSG}$ be the set of message contents, such as $(a,ts_a,ts_b)$ of RGA. Then, CRDT implementations are defined as follows, where operations and receiving messages are defined as functions.

\begin{definition}[CRDT implementations]
\label{definition:operation-based CRDT implementations}
A CRDT implementation for a type $t = (M,D)$ is a tuple $I(r) = (\Sigma, \Sigma_0, \mathit{Msg}, \mathit{do},\mathit{receive})$. Here $r \in \mathbb{R}$, $\Sigma_0 \subseteq \Sigma$, $\mathit{Msg} \subseteq \mathbb{MSG}$, $\mathit{do}:\Sigma \times M \times D \rightarrow \Sigma \times D \times (\mathit{Msg} \cup \{ \emptyset \} )$, and $\mathit{receive}: \Sigma \times \mathit{Msg} \rightarrow \Sigma$.
\end{definition}

Here $\Sigma$ is the set of local states and $\Sigma_0$ is the set of initial state. $r$ is the replica identifier of current replica, and some CRDT implementation requires current replica identifier to generate time-stamp. When the current local state is $\sigma$ and the client intends to perform a operation of method $m$ with argument $a$, a $\mathit{do}$ action is launched, which update the local states, returns a value, and generate message if $m$ is a update method. When this replica receives a message of other replica, a $\mathit{receive}$ action will be launched, which updates the current local states according to the message. If a operation has no arguments or return value, or does not generate message, then we can safely omit the corresponding tuples in $\mathit{do}$ action.
% The formal definition of RGA can be easily obtained from its algorithms. For example, when $(a,\_,\_) \in N$, we have $\mathit{do}((N,\mathit{Tomb}),\mathit{rem},a) = ((N,\mathit{Tomb} \cup \{ a \}),a)$. Here we use $\_$ in indicate a element whose value is irrelevant.
The formal definition of RGA, as well as more CRDT implementations, can be found in  Appendix \ref{sec:appendix definitions of section CRDT implementations}.

















\forget{
\section{CRDT Implementations}
\label{sec:CRDT implementations}

A distributed system contains multiple objects, and each objects is replicated on each replica. Each object has a type, which contains its method and data type. A client of a replica interact with the objects by calling the method and then obtaining the return value. Here we do not bound the number of replica identifiers and objects.

Let $\mathbb{OBJ}$ be the set of objects and $\mathbb{R}$ be the set of replica identifiers. We consider a finite set $\mathbb{M}$ of method names; and a possibly infinite set $\mathbb{D}$ of arguments and return values, the data domain. Each data type $t = (M,D)$ has a set $M \subseteq \mathbb{M}$ of methods and a data domain $D \subseteq \mathbb{D}$. Finally we have a infinite set $\mathbb{OID}$ of operation identifiers, corresponding to each individual operation performed on the CRDT throughout an execution.

Without loss of generality we will consider that the methods in $\mathbb{M}$ can be separated in two disjoint sets of methods: $\mathbb{Q}$ query methods that has no influence on the ``abstract state'' and normally returns an observation of the ``abstract state'' , and $\mathbb{U}$ update methods that has influence on the ``abstract state''. Note that some update operation also need to read the ``abstract state''. For example, a $add(a,b)$ operation is an operation of distributed list which intends to put item $a$ immediately after item $b$. This operation implicitly requires that item $b$ is already in list.

$\mathit{Optimistic \ replication \ algorithms}$ is a type of distributed algorithms where each client contains a copy of data structure; a client operations takes effect instantly at its replica without any synchronization, and then broadcast to other replicas and got applied. Convergent or Commutative Replicated Data Types (CRDTs) is a typical kinds of optimistic replication algorithms. In this section, we will introduce CRDT algorithms and their formation.

In practice, there are two kinds of CRDT implementations: state-based CRDT and operation-based CRDT. In state-based CRDT, a update operation take effects locally; in nondeterministic time, a replica may decide to send the (modified) local state into other replicas. The state-based PN-counter is an example of state-based CRDT algorithms and is shown below. Keyword $\mathit{payload}$ indicate the local state, and keyword $\mathit{initial}$ specifies the initial value of local state. Function $\mathit{myID}()$ returns the current replica identifier, and $\mathit{reps}()$ returns the number of replicas of the distributed system. Vector $P$ (resp., $N$) is a vector such that $P[i]$ (resp., $N[i]$) is the number of increase that is generated by replica $i$ and is observed by current replica. This algorithm assumes that the set of replica is already known and is fixed and finite.

Method $\mathit{inc}$ increase the counter by $1$, method $\mathit{dec}$ decrease the counter by $1$, and method $\mathit{read}$ returns the current counter value. Assume the replica identifier of current replica is $r$. When the current replica does $\mathit{inc}$, it modify $P[r]$ into $P[r]+1$. When the current replica does $\mathit{dec}$, it modify $N[r]$ into $N[r]+1$. When the current replica does $\mathit{read}$, it returns $\Sigma_{i}^{n} P[i] - \Sigma_{i}^{n} N[i]$. When the current replica receive a message of modified payload $Z$, it uses function $\mathit{merge}()$ to update the current local state. $\mathit{merge}$ takes the maximum of each replica in the vector.

\renewcommand{\algorithmcfname}{CRDT Implementation}
\noindent
%\begin{minipage}{.5\textwidth}
\noindent\begin{algorithm}[H]
$\mathit{payload}$ integer[$\mathit{reps}$()] P, integer[$\mathit{reps}$()] N; \\
$\mathit{initial}$ [0,\ldots,0],[0,\ldots,0]; \\

$\mathit{inc}()$ \\
%\ \ \ \ let \ g = myID();\\
\ \ \ \ P[$\mathit{myID}$()] = P[$\mathit{myID}$()] + 1; \\

$\mathit{dec}()$ \\
%\ \ \ \ let \ g = myID();\\
\ \ \ \ N[$\mathit{myID}$()] = N[$\mathit{myID}$()] + 1; \\

$\mathit{read}()$ \\
\ \ \ \ \KwRet $\Sigma_{i}^{n} P[i] - \Sigma_{i}^{n} N[i]$; \\

$\mathit{merge}(Z)$ \\
\ \ \ \ $\forall i$, $P[i] = \mathit{max}(P[i],Z.P[i])$; \\
\ \ \ \ $\forall i$, $N[i] = \mathit{max}(N[i],Z.N[i])$; \\
\caption{State-based PN-counter}
\label{Method1}
\end{algorithm}

In operation-based CRDT, an update operation not only updates its local state, but also sends a description of this operation into other replica. Here we take a more complex algorithm, replicated growable array (RGA), as an example of operation-based CRDT and it is shown below.

\renewcommand{\algorithmcfname}{CRDT Implementation}
\noindent
%\begin{minipage}{.5\textwidth}
\noindent\begin{algorithm}[H]
$\mathit{payload}$ TI-tree N, set $\mathit{Tomb}$; \\
$\mathit{initial}$ $\emptyset$,$\emptyset$; \\

$add(a,b)$ \\
\ \ $\mathit{atSource}$: \\
\ \ \ \ $\mathit{pre}$: \ $b = \circ \vee ( b \neq \circ \wedge (b,\_,\_) \in N \wedge b \notin \mathit{Tomb})$ \\

%\ \ \ \ \If {$N = \emptyset$}
%    { \ \ \ \ let \ $ts_a$ = (myID(),1); \\ }
%\ \ \ \ \Else
%    {\ \ \ \ let \ $ts_a$ = (myID(),$\mathit{max}\{ c' \vert (\_,(\_,c'),\_) \in N \} +1$); \\ }

%\ \ \ \ \If {$b = \circ$}
%    { \ \ \ \ let \ $ts_b$ = (0,0); \\ }
%\ \ \ \ \Else
%    { \ \ \ \ let \ $ts_b$ be time-stamp of $b$ in $N$; \\ }

\ \ \ \ let \ $ts_a$ = ($N = \emptyset$) ? (1,$\mathit{myID}$()) ! ($\mathit{max}\{ c' \vert (\_,(\_,c'),\_) \in N \} +1$,$\mathit{myID}$()); \\
\ \ \ \ let \ $ts_b$ = ($b = \circ$) ? (0,$r_0$) ! (the time-stamp of $b$ in $N$); \\

\ \ $\mathit{downstream}(a,ts_a,ts_b)$: \\
\ \ \ \ $\mathit{pre}$: \ $b = \circ \vee ( b \neq \circ \wedge (b,ts_b,\_) \in N)$ \\

\ \ \ \ $N = N \cup \{ (a,ts_a,ts_b) \}$.


$rem(a)$ \\
\ \ $\mathit{atSource}$: \\
\ \ \ \ $\mathit{pre}$: \ $a \neq \circ \wedge (a,\_,\_) \in N \wedge a \notin \mathit{Tomb}$ \\

\ \ $\mathit{downstream}(a)$: $\mathit{pre}$ \ $a \neq \circ \wedge (a,\_,\_) \in N)$

\ \ \ \ $\mathit{Tomb} = \mathit{Tomb} \cup \{ a \}$.

$read()$ \\
\ \ \ \ \KwRet $\mathit{trans}(N,\mathit{Tomb})$; \\

\caption{RGA}
\label{Method1}
\end{algorithm}

Each update operation of operation-based CRDT ie executed with two phases: Its first phase, marked $\mathit{atSource}$, is local to the current replica. It is enabled if its (optional) pre-condition, marked $\mathit{pre}$, is true currently in local state. It generates the information to be delivered, which is the argument of $\mathit{downstream}$. Note that this phase does not modify the local state. Its second phase, marked $\mathit{downstream}$, executed immediate after the current replica, and asynchronously at other replica when they receive the message of this operation. It is enabled if its (optional) pre-condition is true.

In RGA algorithm, a replica store the list as a timestamp insertion tree (TI-tree) $N$, and stores the deleted items in tombstone $\mathit{Tomb}$. A TI-tree $N$ is a set of tuples $(a,t,p)$, where $a$ is a item, $t$ is its unique time-stamp, and $p$ is the time-stamp of its ``parent'' node. Each time-stamp is a tuple $(c,r)$ with $c \in \mathbb{N}$ and $r \in \mathbb{R}$. A order $<_{\mathit{ts}}$ between time-stamps is defined, such that $(c_1,r_1) <_{\mathit{ts}} (c_2,r_2)$, if $c_1 < c_2 \vee (c_1 = c_2 \wedge r_1 <_r r_2)$, where $<_r$ is a total-order over $\mathbb{R}$. There is a pre-existed item $\circ$ of TI-tree with time stamp $(0,r_0)$, which are considered as the root of the tree. Each element of $N$ should have unique item and time stamp, and the elements of $N$ are required to form a tree by following the parent field. The tombstone $\mathit{Tomb}$ is a set of items and records items been removed from the list.

Method $\mathit{add}(a,b)$ intends to add item $a$ into the list immediately after a existing item $b$. Method $\mathit{rem}(a)$ removes $a$ from the list. Method $\mathit{read}$ returns the current list content. When the current replica does $\mathit{add}(a,b)$, it generate a tuple $(a,ts_a,ts_b)$ and put it into $N$. Here $ts_b$ is the time-stamp of $b$, and $ts_a$ is a new time-stamp that is larger than any time stamp in $N$. When the current replica does $\mathit{rem}(a)$, it put $a$ into tombstone. When the current replica does $\mathit{read}$, it uses function $\mathit{trans}(N,\mathit{Tomb})$ to return the list seen by the current replica, which is a sequences obtained by traversing $N$ in prefix order (children are visited in decreasing time-stamp order) and keeping only items that are not in $\mathit{Tomb}$.


Multi-value register is also a common-used data structures and its sequential specification is nondeterministic and thus different from that of the previous two examples, which are deterministic (seen in the next section). A state-based multi-value register algorithm is shown below.


\renewcommand{\algorithmcfname}{CRDT Implementation}
\noindent
%\begin{minipage}{.5\textwidth}
\noindent\begin{algorithm}[H]
$\mathit{payload}$ $S \subseteq D \times \mathbb{N}^{\mathit{reps}()}$; \\
$\mathit{initial}$ $\emptyset$; \\

$\mathit{write}(a)$ \\
\ \ \ \ let \ g = $\mathit{myID}$(); \\
\ \ \ \ let $\mathcal{V} = \{ V \vert \exists x, (x,V) \in S \}$; \\
\ \ \ \ let $V' = [ \mathit{max}_{V \in \mathcal{V}} V[j] ]_{j \neq g}$; \\
\ \ \ \ let $V'[g] = (\mathit{max}_{V \in \mathcal{V}} V[g]) + 1$; \\
\ \ \ \ $S = (a,V')$; \\

$\mathit{read}()$ \\
\ \ \ \ \KwRet $S' = \{ a \vert (a,\_) \in S \}$; \\

$\mathit{merge}(Z)$ \\
\ \ \ \ let $A' = \{ (x,V) \in S \vert \forall (x',V') \in Z.S, \exists i, V[i] \geq V'[i] \}$; \\
\ \ \ \ let $B' = \{ (x,V) \in Z.S \vert \forall (x',V') \in S, \exists i, V[i] \geq V'[i] \}$; \\
\ \ \ \ $S = A' \cup B'$; \\
\caption{state-based multi-value register}
\label{Method1}
\end{algorithm}

Each replica stores a set $S$ of items such that each item can not dominate other items. To do conflict resolution, we associate each item $a$ in $S$ with a version vector $V$. We say version vector $V$ dominates version vector $V'$, if $\forall i$, $V[i] > V'[i]$.

Method $\mathit{write}(a)$ intends to write $a$ into register. Method $\mathit{read}$ returns the current register content. When the current replica does $\mathit{write}(a)$, it generates a new version vector that dominates all previous ones in $S$. When the current replica does $\mathit{read}$, it returns the set of items in $S$. When the current replica receive a message of modified payload $Z$, we takes the union of every items in $S$ and $Z.S$ whose version vector is not dominated by that of an item in the other set. This algorithm assumes that the set of replica is already known and is fixed and finite.


To enable formally verification of CRDT algorithms, it is necessary to give formal definition of CRDT-algorithms. Let $\mathbb{MSG}$ be the set of message contents, such as $(P,N)$ of state-based PN-counter, or $(a,ts_a,ts_b)$ of RGA. Then, CRDT implementations are defined as follows, where operations and receiving messages are defined as functions.

\begin{definition}[operation-based CRDT implementations]
\label{definition:operation-based CRDT implementations}
A operation-based CRDT implementation for a type $t = (M,D)$ is a tuple $I_t(r) = (\Sigma, \Sigma_0, \mathit{Msg}, \mathit{do},\mathit{receive})$. Here $r \in \mathbb{R}$, $\Sigma_0 \subseteq \Sigma$, $\mathit{Msg} \subseteq \mathbb{MSG}$, $\mathit{do}:\Sigma \times M \times D \rightarrow \Sigma \times D \times (\mathit{Msg} \cup \{ \emptyset \} )$, and $\mathit{receive}: \Sigma \times \mathit{Msg} \rightarrow \Sigma$.
\end{definition}

Here $\Sigma$ is the set of local states and $\Sigma_0$ is the set of initial state. For example, in state-based PN-counter, since there are many possibility of total number of replicas, $\Sigma_0$ is a set of more than one elements. $r$ is the replica identifier of current replica. The reason of containing $r$ in the definition of CRDT implementations is that, some algorithms need the current replica identifier to generate time-stamp. When the current local state is $\sigma$ and the client intends to perform a operation of method $m$ with argument $a$, a $\mathit{do}$ action is launched, which update the local states, returns a value, and possibly generate messages. A $\mathit{do}$ action of update method will generate messages, while a $\mathit{do}$ action of query method will not generate message. When this replica receives a message of other replica, a $\mathit{receive}$ action will be launched, which updates the current local states according to the message. If a operation has no arguments or return value, or does not generate message, then we can safely omit the corresponding tuples in $\mathit{do}$ actions.

\begin{definition}[state-based CRDT implementations]
\label{definition:state-based CRDT implementations}
A state-based CRDT implementation for a type $t = (M,D)$ is a tuple $I_t(r) = (\Sigma, \Sigma_0, \mathit{Msg}, \mathit{do},\mathit{receive})$. Here $r \in \mathbb{R}$, $\Sigma_0 \subseteq \Sigma$, $\mathit{Msg} \subseteq \Sigma$, $\mathit{do}:\Sigma \times M \times D \rightarrow \Sigma \times D$, and $\mathit{receive}: \Sigma \times \Sigma \rightarrow \Sigma$.
\end{definition}

The state-based CRDT implementation is similarly defined. The difference is that, each operation does not send message, and the message content is fixed to be a local state. The following is an example of formal definition of state-based PN-counter. The formal definition of more CRDT implementations are given in Appendix \ref{sec:appendix definitions of section CRDT implementations}. Here we denote by $f[i:j]$ the function that has the same value as $f$ everywhere, except for $i$, where it has the value $j$. %Since each operation is executed without synchronization, it is not hard to obtain formal definition from informal algorithms.

\begin{example}[formal definition of state-based PN-counter]
\label{definition:formal definition of state-based PN-counter}
$I_t(r) = (\Sigma, \sigma_0, \mathit{Msg}, \mathit{do},\mathit{receive})$, where

\begin{itemize}
\setlength{\itemsep}{0.5pt}
\item[-] $\Sigma = \{ (P,N) \vert$, $P$ and $N$ are vector of integers with same length $\}$. $\Sigma_0 = \{ (P_0,N_0) \vert (P_0,N_0) \in \Sigma$, $P_0$ and $N_0$ maps each index into $0 \}$.

\item[-] $\mathit{Msg} \subseteq \Sigma$.

\item[-] $\mathit{do}((P,N),\mathit{inc}) = (P[r:P[r]+1],N)$,

\item[-] $\mathit{do}((P,N),\mathit{dec}) = (P,N[r:N[r]+1])$,

\item[-] $\mathit{receive}((P,N),(P',N')) = (\lambda s. \mathit{max}\{  P[s], N'[s] \}, \mathit{max}\{  N[s], N'[s] \},)$,
\end{itemize}
\end{example}
}































%%% Local Variables:
%%% mode: latex
%%% TeX-master: "draft"
%%% End:
