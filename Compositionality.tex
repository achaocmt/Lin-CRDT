%!TEX root = draft.tex
%\newcommand{\seqPQ}{\mathsf{SeqPQ}}

\section{Compositionality of Distributed Linearizability}
\label{sec:compositionality of distributed linearizability}


\begin{definition}[Compositionality]
\label{definition:compositionality}
A history $h$ of multiple objects is called compositional, if: $h \uparrow_{\mathit{obj}}$ is distributed linearizable for each object $\mathit{obj}$, if and only if, there exists a sequence $\mathit{lin}$, called linearization of $h$, such that

\begin{enumerate}[(i)]
\item The elements of $\mathit{lin}$ is generated from the operations of $h$: each operation $o = (m(a) \Rightarrow b,i,\mathit{obj})$ is transformed into $(m(a) \Rightarrow b,i,S)$ with $S$ set of identifiers of operations of visible to $o$ via $( h \uparrow_{\mathit{obj}}). \mathit{vis}$.
\item For each object $\mathit{obj}$, $\mathit{lin}$ is consistent with $( h \uparrow_{\mathit{obj}}). \mathit{vis}$.
\item For each object $\mathit{obj}$, $\mathit{lin} \uparrow_{ \mathit{obj} }$ is a linearization of $h \uparrow_{\mathit{obj}}$.
\end{enumerate}
\end{definition}

\noindent {\bf Composing Several t0}:

\cite{BurckhardtGYZ14}





%%% Local Variables:
%%% mode: latex
%%% TeX-master: "draft"
%%% End:
