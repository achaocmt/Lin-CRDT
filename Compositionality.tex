%!TEX root = draft.tex
%\newcommand{\seqPQ}{\mathsf{SeqPQ}}

\section{Compositionality of \CRDTLin{}}
\label{sec:compositionality}

%\textblue{
%This section should give three results of the form: if $o_1$ is linearizable w.r.t. $S_1$ and $o_2$ is linearizable w.r.t. $S_2$, and ???, then $o_1 \otimes o_2$ is linearizable w.r.t. $S_1\times S_2$ (the 2-object spec defined by interleavings). ??? may be an additional condition, while $\otimes$ is an "operator" for composing two CRDT implementations. Every result will have a different $\otimes$ operator.
%\begin{itemize}
%\item \gpnote*{}{Start with some counter-examples of compositionality (Check with the intro)}
%\item T0 + T0: ??? = $S_1$ and $S_2$ are $T0$-specs, and $\otimes$ is the trivial composition (the "unsynchronized" product)
%\item T1 + T1: ??? = $S_1$ and $S_2$ are $T1$-specs, and $\otimes$ is the "shared counter" composition (the "unsynchronized" product with a restriction on the set of generated histories)
%\item T0 + T1: ??? = $S_1$ is a $T_0$ spec, and $S_2$ is a $T1$-spec, and $\otimes$ is the "global causal delivery" composition (the "unsynchronized" product with a restriction on the set of generated histories)
%\end{itemize}
%Also, we should prove that composing two $T0$, resp., $T1$, specs results in a $T0$, resp., $T1$ spec, and composing a $T0$ with $T1$ results in a $T1$. With this, the extension to sets of objects is straightforward:  compose all $T0$ and independently, all $T1$, then compose the two resulting objects.}

%\gpnote*{}{Are there things that are neither T0 nor T1?: Not in our paper}

%\fxwarning[nomargin, inline]{introduce this section}

We investigate the issue of whether the composition of a set of objects satisfying RA-linearizability is also \crdtlinearizable{}. While we show that this is not true in general, we identify a set of correctness conditions along with requirements on the composition of objects which imply RA-linearizability of the objects and their composition. The correctness conditions are exactly the ones defined in \sectionautorefname~\ref{sec:proofs}, i.e., admitting execution-order or timestamp-order linearizations, and therefore, they come up as natural strengthenings while proving RA-linearizability of concrete CRDTs. The requirement on the object composition is related to sharing the same timestamp generator among different objects.

\subsection{Object Compositions and \CRDTLin{}}\label{ssec:comp_intro}

Given two objects $\aobj_1$ and $\aobj_2$, the semantics of their composition $\aobj_1\comp \aobj_2$ is the product of the LTSs corresponding to $\aobj_1$ and $\aobj_2$, respectively. Formally, given $\llbracket \aobj_1 \rrbracket =(\globalstates_1,\acts_1,\aglobalstate_0^1,\rightarrow_1)$ and $\llbracket \aobj_2 \rrbracket =(\globalstates_2,\acts_2,\aglobalstate_0^2,\rightarrow_2)$, we define $\llbracket \aobj_1\comp \aobj_2 \rrbracket =(\globalstates_1\times \globalstates_2,\acts_1\cup \acts_2,(\aglobalstate_0^1,\aglobalstate_0^2,\emptyset),\rightarrow_{1,2})$ where\\[2pt]
\centerline{
\(
\begin{array}{rcl}
  \rightarrow_{1,2} & = & \{ ((\aglobalstate_1,\aglobalstate_2),\aact_1,(\aglobalstate_1',\aglobalstate_2)): (\aglobalstate_1,\aact_1,\aglobalstate_1')\in\rightarrow_1\}\\
                    & \cup & \{ ((\aglobalstate_1,\aglobalstate_2),\aact_2,(\aglobalstate_1,\aglobalstate_2')): (\aglobalstate_2,\aact_2,\aglobalstate_2')\in\rightarrow_2\}
\end{array}
\)
}\\[2pt]
% \begin{align*}
% \rightarrow_{1,2}\ & =\ \{ ((\aglobalstate_1,\aglobalstate_2),\aact_1,(\aglobalstate_1',\aglobalstate_2)): (\aglobalstate_1,\aact_1,\aglobalstate_1')\in\rightarrow_1\} \\
% & \cup\ \{ ((\aglobalstate_1,\aglobalstate_2),\aact_2,(\aglobalstate_1,\aglobalstate_2')): (\aglobalstate_2,\aact_2,\aglobalstate_2')\in\rightarrow_2\}
% \end{align*}
 The history $\hist{\atrace}$ of a trace $\atrace$ of $\aobj_1\comp \aobj_2$ records a ``global'' visibility relation between the operations in the trace, i.e., which operations of $\aobj_1$ or $\aobj_2$ are visible when issuing an operation of $\aobj_1$, and similarly, for operations of $\aobj_2$. Formally, $\hist{\atrace}=(\alabelset, \avisord)$ where $\alabelset$ is the set of labels occurring in $\atrace$, and $(\alabel_1,\alabel_2)\in\avisord$ if there exists a replica $\arep$ such that $\dwn{\arep}{\alabel_1}$ occurs before $\src{\arep}{\alabel_2}$ in the trace $\atrace$. Note that $\avisord$ may not be a partial order in general, since the causal delivery assumption holds only among operations of the same object. The set of histories $\histories(\aobj_1\comp \aobj_2)$ of the composition $\aobj_1\comp \aobj_2$ is the set of histories $h$ of a trace $\atrace$ of $\aobj_1\comp \aobj_2$.

%\fxnote{Add new notations.}
Given two specifications $\Spec_1$ and $\Spec_2$ of two objects $\aobj_1$ and $\aobj_2$, respectively, their composition $\Spec_1\comp \Spec_2$ is defined as the set of sequences which are interleavings of sequences in $\Spec_1$ and $\Spec_2$, respectively. More precisely, $\Spec_1\comp \Spec_2$ is the set of sequences $(\alabelset,\aseqord)$ such that their projection on labels of $\aobj_1$, resp., $\aobj_2$, is admitted by $\Spec_1$, resp., $\Spec_2$. Given two objects  $\aobj_1$ and $\aobj_2$ with specifications $\Spec_1$ and $\Spec_2$, respectively, we say that their composition $\aobj_1\comp \aobj_2$ is \emph{\crdtlinearizable{}} if every history of $\aobj_1\comp \aobj_2$ is \crdtlinearizable{} w.r.t. $\Spec_1\comp \Spec_2$. The extension of implementation and specification composition to a set of objects is defined as usual.

\begin{wrapfigure}{l}{.45\linewidth}
  \vspace{-5pt}
  \centering
  \includegraphics[width=0.35 \textwidth]{figures/TwoSubLin-NotaGlobalLin.pdf}
\vspace{-5pt}
\caption{A history of two OR-Set objects showing that per-object \crdtlinearization{s} cannot be combined into a global \crdtlinearization{}. Every operation is visible only at the origin, hence visibility is defined by the horizontal lines (left to right).}
  \vspace{-9pt}
  \label{fig:negative_composition}
\end{wrapfigure}

Linearizability~\cite{HerlihyW90} ensures that the composition of a set of linearizable objects is also linearizable. More precisely, it ensures that for every history, any per-object linearizations, concerning the operations of a single object, can be combined into a global linearization, concerning all the operations in the history. By combining linearizations, we mean constructing a global linearization whose projections on the operations of a single object are exactly the per-object linearizations considered in the beginning.
However, this is not true for our notion of RA-linearizability. An example of a history showing this fact is given in \figureautorefname~\ref{fig:negative_composition}. This history contains operations of two OR-Set objects $\aobj_1$ and $\aobj_2$, the operations of $\aobj_1$ being represented using blank circles and the operations of $\aobj_2$ using filled circles. The operations of $\aobj_1$ can be linearized to $\aobj_1.\alabelshort[{\tt add}]{c}\cdot \aobj_1.\alabelshort[{\tt add}]{d}$ (this is a valid \crdtlinearization{} since any sequence of ${\tt add}$ operations is admitted by $\specOrSet$) while the operations of $\aobj_2$ can be linearized to $\aobj_2.\alabelshort[{\tt add}]{a}\cdot \aobj_2.\alabelshort[{\tt add}]{b}$. There is no \crdtlinearization{} of this history whose projections on each of the two objects correspond to these per-object linearizations: trying to construct a linearization where $\aobj_2.\alabelshort[{\tt add}]{a}$ occurs before $\aobj_2.\alabelshort[{\tt add}]{b}$ will imply that $\aobj_1.\alabelshort[{\tt add}]{d}$ must occur before $\aobj_1.\alabelshort[{\tt add}]{c}$ (since it must be consistent with the visibility relation), which contradicts the linearization of $\aobj_1$, and similarly for the other case, when trying to construct a global linearization consistent with the linearization of $\aobj_2$'s operations. A reader knowledgeable of the literature on linearizability may notice that this discrepancy between standard linearizability and RA-linearizability comes from the fact that the partial order defining a history in the case of standard linearizability is actually an interval order\footnote{A partial order $R$ is an interval-order if $\{(a,b), (c,d)\} \subseteq R$ implies that $(a,d) \in R$ or $(c,b) \in R$, for every $a,b,c,d$.}, while in the case of RA-linearizability it is an arbitrary partial order.
%\gpwarning[nomargin, inline]{I think this would be a place to talk about~\cite{JagadeesanR18}.}


\subsection{Composing Objects That Admit Execution-Order Linearizations}

Although not all per-object \crdtlinearization{s} can be combined into global \crdtlinearization{s}, this may still be true in some cases. For instance, let us consider again the history in \autoref{fig:negative_composition}. The operations of $\aobj_1$ can also be linearized to $\aobj_1.\alabelshort[{\tt add}]{d}\cdot \aobj_1.\alabelshort[{\tt add}]{c}$ which enables a global \crdtlinearization{} $\aobj_1.\alabelshort[{\tt add}]{d}\cdot \aobj_2.\alabelshort[{\tt add}]{a}\cdot \aobj_2.\alabelshort[{\tt add}]{b}\cdot \aobj_1.\alabelshort[{\tt add}]{c}$ whose projection on individual objects is consistent with the per-object linearizations (we consider the same linearization $\aobj_2.\alabelshort[{\tt add}]{a}\cdot \aobj_2.\alabelshort[{\tt add}]{b}$ for $\aobj_2$).

We show that in the case of \crdtlinearizable{} objects that admit execution-order linearizations, there always exist per-object \crdtlinearization{s} that can be combined into global \crdtlinearization{s}, which implies that their composition is \crdtlinearizable{}. Moreover, their composition also admits execution-order linearizations. A first preliminary result states that the order in which concurrent operations are executed at the origin replica can be permuted arbitrarily while still leading to a valid trace. More precisely, for every linearization $\alinord$ of the visibility in the history of a trace, there exists another valid trace where operations are executed at the origin replica in the order defined by $\alinord$.

\begin{lemma}\label{lem:trace_closure}
Let $\atrace$ be a trace of an object $\aobj$ and $\hist{\atrace}=(\alabelset,\avisord)$. Then, for every sequence $(\alabelset,\alinord)$ which is consistent with $\avisord$ (i.e., $\avisord
    \cup \alinord$ is acyclic), there exists a trace $\atrace'$ of $\aobj$ such that $\src{}{\alabel_1}$ occurs before $\src{}{\alabel_2}$ in $\atrace'$ iff $\alabel_1$ occurs before $\alabel_2$ in $\alinord$. Moreover, $\atrace'$ has the same history as $\atrace$.
\end{lemma}
\vspace{-10pt}
\begin{proof}(Skech)
We define a \emph{dependency} relation $\circledcirc$ between actions as follows: $\src{\arep_1}{\alabel_1}\circledcirc \dwn{\arep_2}{\alabel_2}$ iff $\arep_1=\arep_2$ or $\alabel_1=\alabel_2$, $\src{\arep_1}{\alabel_1}\circledcirc \src{\arep_2}{\alabel_2}$ iff $\arep_1=\arep_2$, $\dwn{\arep_1}{\alabel_1}\circledcirc \src{\arep_2}{\alabel_2}$ iff $\arep_1=\arep_2$, and $\dwn{\arep_1}{\alabel_1}\circledcirc \dwn{\arep_2}{\alabel_2}$ iff $\arep_1=\arep_2$. Given a trace $\atrace=\atrace_1\cdot \aact_1\cdot\aact_2\cdot\atrace_2$, we say that a trace $\atrace'=\atrace_1\cdot \aact_2\cdot\aact_1\cdot\atrace_2$ is derived from $\atrace$ by a \emph{$\circledcirc$-valid swap} iff $\aact_1$ and $\aact_2$ are \emph{not} related by $\circledcirc$. Using standard reasoning about traces of a concurrent system, it can be shown that if $\atrace$ is a trace of $\aobj$, then any trace $\atrace'$ derived through a sequence of $\circledcirc$-valid swaps is also a trace of $\aobj$. Moreover, $\atrace'$ has the same history as $\atrace$. Then, by the definition of the visibility relation $\avisord$ in the history of a trace and the causal delivery assumption, it can be shown that given a trace $\atrace$ of $\aobj$ containing two actions $\src{}{\alabel_1}$ and $\src{}{\alabel_2}$ such that $\alabel_1$ and $\alabel_2$ are concurrent and $\src{}{\alabel_1}$ occurs before $\src{}{\alabel_2}$, there exists another trace $\atrace'$ of $\aobj$ that is defined through a sequence of $\circledcirc$-valid swaps, where $\src{}{\alabel_2}$ occurs before $\src{}{\alabel_1}$. Thus, the order of concurrent generator actions can be permuted arbitrarily, concluding the lemma.
\end{proof}

Lemma~\ref{lem:trace_closure} implies that given an \crdtlinearizable{} object $\aobj$, if it admits execution-order linearizations, then every linearization of a history of $\aobj$ consistent with visibility is a valid \crdtlinearization{}.

\begin{lemma}\label{lem:t0_lin}
Let $\aobj$ be an object which is \crdtlinearizable{} w.r.t. a specification $\Spec$ and admits execution-order linearizations. Then, for every history $\ahist=(\alabelset,\avisord)$ of $\aobj$ and every sequence $(\alabelset,\alinord)$ which is consistent with $\avisord$, we have that $(\alabelset,\alinord)$ is an \crdtlinearization{} of $\ahist$ w.r.t. $\Spec$.
\end{lemma}
\vspace{-10pt}
\begin{proof}(Skech)
Let $(\alabelset,\alinord)$ be a sequence consistent with the visibility relation of $\ahist$. By Lemma~\ref{lem:trace_closure}, there exists a trace $\atrace$ of $\aobj$ such that  $\src{}{\alabel_1}$ occurs before $\src{}{\alabel_2}$ in $\atrace$ iff $\alabel_1$ occurs before $\alabel_2$ in $\alinord$. Definition~\ref{def:exec_order} implies that $(\alabelset,\alinord)$ is an RA-linearization of $\hist{\atrace}=\ahist$.
\end{proof}

Lemma~\ref{lem:t0_lin} is the essential ingredient for proving that the composition of a set of \crdtlinearizable{} objects that admit execution-order linearizations is also \crdtlinearizable{}. Intuitively, given a history $\ahist$ with multiple such objects, any linearization of $\ahist$ is a valid \crdtlinearization{} since each of its projections on the set of operations of a single object is a linearization of the per-object visibility relation~\footnote{By definition, the visibility relation in the history of an object composition $\aobj_1\comp\aobj_2$ projected on operations of $\aobj_1$, resp., $\aobj_2$, is exactly the visibility relation stored in the global configurations of $\aobj_1$, resp., $\aobj_2$ (at the end of the execution).}  and thus, a valid \crdtlinearization{} of that object.

\begin{theorem}\label{th:comp_execution_order}
The composition of a set of \crdtlinearizable{} objects that admit execution-order linearizations is \crdtlinearizable{}. Moreover, the composition admits execution-order linearizations.
\end{theorem}
\vspace{-\topsep}
%\begin{proof}(Skech)
%TODO
%\end{proof}

%\begin{definition}[\tzerolin{}]
%\label{definition:t0-linearizability}
%A single-object history $\ahis$ is \tzerolinearizable{} w.r.t a sequential specification \Spec{}, if $\ahis$ is \crdtlinearizable{} w.r.t \Spec{}, and each specification sequence $(\alabelset, \aseqord)$ consistent with visibility order is a \crdtlinearization{}.
%\end{definition}

\subsection{Composing Objects That Admit Timestamp-Order Linearizations}

%We say a specification sequence $(\alabelset, \aseqord)$ be consistent with timestamp order, if the time-stamp of operation $\aobj_1$ being less than the timestamp of $\aobj_2$ implies that $\aobj_1$ being before $\aobj_2$ in $\aseqord$, and if $(\aobj_3,\aobj_4) \in \avisord$, then the timestamp of $\aobj_3$ is less or equal to the timestamp of $\aobj_4$.
%
%\begin{definition}[\tonelin{}]
%\label{definition:t1-linearizability}
%A single-object history $\ahis$ is \tonelinearizable{} w.r.t a sequential specification \Spec{}, if $\ahis$ is \crdtlinearizable{} w.r.t \Spec{}, and each specification sequence $(\alabelset, \aseqord)$ consistent with visibility order and timestamp order is a \crdtlinearization{}.
%\end{definition}

\begin{figure}[t]
  \centering
  \includegraphics[width=0.7 \textwidth]{figures/LWWReg-LWWReg-NoSTS.pdf}
%\vspace{-10pt}
\vspace{-1mm}
  \caption{A history showing that the composition $\otimes$ of two RGA objects is not \crdtlinearizable{}.}
  \label{fig:negative_ts_composition}
  \vspace{-3mm}
\end{figure}


The positive result in Theorem~\ref{th:comp_execution_order} does not apply directly to objects that admit timestamp-order linearizations. The main reason is that the unrestricted object composition $\otimes$ defined in \sectionautorefname~\ref{ssec:comp_intro} allows different objects to generate timestamps independently, and in ``conflicting'' orders along some execution. For instance, \figureautorefname~\ref{fig:negative_ts_composition} shows a history with two RGA objects $\aobj_1$ and $\aobj_2$. We assume that $\ats_1 < \ats_2 < \ats_3$ and $\ats'_1 < \ats'_2$ (the orderings between unprimed and primed timestamps are not important). The operations of $\aobj_1$ can be linearized to\\[3pt]
\centerline{
\(
\aobj_1.\alabelshort[{\tt addAfter}]{\circ,a}\cdot \aobj_1.\alabelshort[{\tt addAfter}]{\circ,b}\cdot \aobj_1.\alabellongind[{\tt read}]{}{b\cdot a}{}
\)
}\\[3pt]
and the operations of $\aobj_2$ to\\[3pt]
\centerline{
\(
\aobj_2.\alabelshort[{\tt addAfter}]{\circ,c}\cdot \aobj_2.\alabelshort[{\tt addAfter}]{\circ,d}\cdot \aobj_2.\alabelshort[{\tt addAfter}]{\circ,e}\cdot \aobj_2.\alabellongind[{\tt read}]{}{e\cdot d\cdot c}{}
\)
}\\[3pt]
Note that these are the only \crdtlinearization{s} possible for these operations. It can be easily seen that there is no ``global'' linearization consistent with these per-object linearizations. Ordering $\alabelshort[{\tt addAfter}]{\circ,a}$ before $\alabelshort[{\tt addAfter}]{\circ,b}$ implies that $\alabelshort[{\tt addAfter}]{\circ,e}$ occurs before $\alabelshort[{\tt addAfter}]{\circ,d}$ which contradicts the second linearization above.
We solve this problem by requiring an additional constraint on the composition operator $\comp$ such that intuitively, all objects use a common timestamp generator. This in turn requires that the implementation of the objects share a common timestamp generator, and it must ensure that each new timestamp is bigger than the  timestamps used by operations delivered to a replica, independently of the object on which they are applied. For instance, the history of \figureautorefname~\ref{fig:negative_ts_composition} would not be admitted because $\ats_1'$ should be bigger than $\ats_3$ (since the operation that received $\ats_3$ from the timestamp generator originates from the same replica as the operation receiving $\ats_1'$ at a later time) and $\ats_2$ should be bigger than $\ats_2'$. These two constraints together with $\ats_1'<\ats_2'$ contradict $\ats_2 < \ats_3$.
%
We remark here that while this requires a modification of the algorithms, where the timestamp generator is a parameter, this has no algorithmic or run-time cost, and in fact a similar idea have been suggested in the systems literature (e.g.~\cite{EnesPB17}).

%Before going into the details of our solution, we present a formal definition of timestamp-order linearizations. For a history $\ahist=(\alabelset,\avisord)$, we define the timestamp $\tsof_\ahist(\alabel)$ of a label $\alabel$ in the context of the history $h$ to be $\tsof_\ahist(\alabel)=\tsof(\alabel)$ if $\tsof(\alabel)\neq\bot$ and $\tsof_\ahist(\alabel)=\mathsf{max}\, \{\tsof(\alabel'):(\alabel',\alabel)\in \avisord \}$, otherwise.
%
%\begin{definition}
%An object $\aobj$ which is \crdtlinearizable{} w.r.t. a specification $\Spec$ admits \emph{timestamp-order linearizations} if for every trace $\atrace\in\traces(\aobj)$ with $\hist{\atrace}=(\alabelset,\avisord)$, there exists an \crdtlinearization{} $(\alabelset,\alinord)$ of $\hist{\atrace}$ w.r.t. $\Spec$ such that $\alabel_1$ occurs before $\alabel_2$ in $\alinord$ iff $\tsof_\ahist(\alabel_1) < \tsof_\ahist(\alabel_2)$ or $\src{}{\alabel_1}$ occurs before $\src{}{\alabel_2}$ in $\atrace$, for every two labels $\alabel_1,\alabel_2\in\alabelset$.
%\end{definition}

Using again Lemma~\ref{lem:trace_closure}, which ensures that the generators of ``concurrent'' operations can be executed in any order, we get that for any \crdtlinearizable{} object $\aobj$ which admits timestamp-order linearizations, every linearization of a history of $\aobj$ which is also consistent with the order between timestamps is a valid \crdtlinearization{}.
For a history $\ahist=(\alabelset,\avisord)$, let $\atsord{\ahist}$ be an order between the labels in $\alabelset$ such that $\alabel_1\atsord{\ahist}\alabel_2$ iff $\tsof_\ahist(\alabel_1) < \tsof_\ahist(\alabel_2)$.
\vspace{-5pt}

\begin{lemma}
Let $\aobj$ be an object which is \crdtlinearizable{} w.r.t. a specification $\Spec$ and admits timestamp-order linearizations. Then, for every history $\ahist=(\alabelset,\avisord)$ of $\aobj$ and every sequence $(\alabelset,\alinord)$ which is consistent with $\avisord$ and $\atsord{\ahist}$, we have that $(\alabelset,\alinord)$ is an \crdtlinearization{} of $\ahist$ w.r.t. $\Spec$.
\end{lemma}
\vspace{-5pt}

\begin{figure}[t]
  \centering
  \footnotesize
\[
  \inferrule[\text{\sc Operation}]
  {\alabel = \aobj_k.\alabellongind{\argv}{\retv}{(i,\ats)}\mbox{ with } k\in \{1,2\} \\ (\gstates_k, \avisord_k, \downstreams_k) \xrightarrow{\src{\arep}{\alabel}}_{k} (\gstates_k', \avisord_k', \downstreams_k') \\
  (\gstates'_{k'}, \avisord'_{k'}, \downstreams'_{k'}) = (\gstates_{k'}, \avisord_{k'}, \downstreams_{k'})\mbox{ for $k'\neq k$} \\
  \gstates_1(\arep) = (\alabelset_1, \astate_1) \\ \gstates_2(\arep) = (\alabelset_2, \astate_2) \\
%  \ats\neq\bot\implies (\,\forall \alabel\in\alabelset_1\cup\alabelset_2.\ \tsof(\alabel)\neq\bot \implies \tsof(\alabel) < \ats\,) 
  \ats\neq\bot\implies (\,\forall \alabel'\in\alabelset_1\cup\alabelset_2.\ \tsof(\alabel') < \ats\,) \\
  \forall \alabel'\in \labeldom{\avisord_1\cup\avisord_2}.\ \tsof(\alabel') \neq \ats
  }
  {((\gstates_1, \avisord_1, \downstreams_1),(\gstates_2, \avisord_2, \downstreams_2)) \xrightarrow{\src{\arep}{\alabel}} (\gstates_1', \avisord_1', \downstreams_1'),(\gstates_2', \avisord_2', \downstreams_2')}
\]
\vspace{-5pt}
\caption{The transition rule for generators in the object composition operator $\otimes_{\tsof}$.}
  \label{fig:comp-ts}
\vspace{-10pt}
\end{figure}

We define a restriction $\otimes_{\tsof}$ of the object composition operator $\otimes$ such that the set of histories $h=(\alabelset,\avisord)$ in the composition $\aobj_1\otimes_{\tsof}\aobj_2$ satisfy the property that the order between timestamps (of all objects) is consistent with the visibility relation $\avisord$ (i.e., $\avisord\cup \atsord{\ahist}$ is acyclic). With respect to the ``unrestricted'' composition $\otimes$ defined in \sectionautorefname~\ref{ssec:comp_intro}, we only modify the transition rule corresponding to generators, as shown in \figureautorefname~\ref{fig:comp-ts}. The only addition consists in ensuring that when a new timestamp is generated, it is bigger than all the timestamps ``visible'' to the replica executing that generator (even if they were generated in the context of a different object). Its extension to a set of objects is defined as usual. The composition operator  $\otimes_{\tsof}$ is called \emph{shared timestamp generator composition}. From a practical point of view, if we were to consider the standard timestamp mechanism used in CRDTs, i.e., each replica maintains a counter which is increased monotonically with every new operation (originating at the replica or delivered from another replica) and timestamps are defined as pairs of replica identifiers and counter values, then $\otimes_{\tsof}$ can be implemented using a ``shared'' counter which increases monotonically with every new operation, independently of the object on which it is applied.

The following theorem shows that composing \crdtlinearizable{} objects that admit execution-order or timestamp-order linearizations, i.e., proved using the methodologies of \sectionautorefname~\ref{subsec:time order of execution as linearization} and \sectionautorefname~\ref{subsec:time-stamp order as linearizabtion}, is \crdtlinearizable{}, provided that all the objects share the same timestamp generator.

\begin{theorem}\label{th:comp_all}
The shared timestamp generator composition of a set of \crdtlinearizable{} objects that admit execution-order or timestamp-order linearizations is \crdtlinearizable{}.
\end{theorem}
\vspace{-5pt}
%\begin{proof}(Sketch)
%TODO
%\end{proof}

Based on the results in Theorem~\ref{th:execution_order_objects} and Theorem~\ref{th:ts_order_objects}, the theorem above shows that any shared timestamp generator composition of the objects mentioned in \figureautorefname~\ref{fig:crdt-implementaton of this paper, their correctness, and their interface} is RA-linearizable.

% \input{tables}

%%% Local Variables:
%%% mode: latex
%%% TeX-master: "draft"
%%% End:
