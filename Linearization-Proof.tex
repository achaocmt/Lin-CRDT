%!TEX root = draft.tex
%\newcommand{\seqPQ}{\mathsf{SeqPQ}}

\section{Proving Distributed Linearizability}
\label{sec:proving distributed linearizability}

In this section, we propose our approach for proving distributed linearizability for CRDT vis simulation. When the context is clear, we do not distinguish an operation and its unique identifier.


\subsection{Proof Approach for CRDT Implementations without Time-stamp}
\label{subsec:proof approach for CRDT implementations without time-stamp}

Let us consider CRDT implementations that do not use time-stamp. We construct a predicate $P(\mathit{config},h,\mathit{lin},\mathit{map})$ where $\mathit{config} = (R,\_,\mathit{MsgHB},\mathit{MsgDel})$ is a configuration of $\llbracket \mathit{obj} \rrbracket_{\mathit{op}}$, $h$ is a history, $\mathit{lin} \in \mathbb{A}^*$, and $\mathit{map} \subseteq \mathbb{MID} \times 2^{\mathbb{OID}}$. We require $P(\mathit{config},h,\mathit{lin},\mathit{map})$ to be a conjunction of the following statements:

\begin{itemize}
\setlength{\itemsep}{0.5pt}
\item[-] $C_1$ (linearizability): $h$ is distributed linearizable w.r.t $\mathit{spec}$ and $\mathit{lin}$ is a linearization.

\item[-] $C_2$ (correct $\mathit{map}$): $\mathit{map}$ is a bijection between messages of $\mathit{config}$ and operations of $h$.

    \begin{itemize}
    \setlength{\itemsep}{0.5pt}
    \item[-] For message visibility of $\mathit{config}$: $(o_1,o_2) \in h.\mathit{vis}$, if and only if $(\mathit{map}(o_1),\mathit{map}(o_2)) \in \mathit{config}.\mathit{MsgHB}$.

    \item[-] For message delivery of $\mathit{config}$: If $(\mathit{mid},r) \notin \mathit{config}.\mathit{MsgDel}$, then for each $\mathit{mid}'$ where $\mathit{map}(\mathit{mid}') = r$, $(\mathit{map}(\mathit{mid}),\mathit{map}(\mathit{mid}')) \notin h.\mathit{vis}$.
    \end{itemize}

\item[-] $C_3$ (causal delivery): $h.\mathit{vis}$ is transitive. Moreover, if $(o_1,o_2) \in h.\mathit{vis}$ and $(\mathit{map}(o_2),r) \in \mathit{config}.\mathit{MsgDel}$, then $(\mathit{map}(o_1),r) \in \mathit{config}.\mathit{MsgDel}$.
    
\item[-] $C_4$ (message content): A statement about message content of each message.   

\item[-] $C_5$ (sequential explanation): For each replica $r$, we have that $\mathit{config}.R(r) = \mathit{apply}(\mathit{lin},\mathit{vd}(h,\mathit{config},r))$. 
\end{itemize} 

Here $\mathit{vd}(h,\mathit{config},r) = \{ i \vert (i,j) \in h.\mathit{vis}, j$ is of replica $r \} \cup \{ i \vert (\mathit{map}(i)i,r) \in \mathit{config}.\mathit{MsgHB} \}$. $\mathit{vd}(h,\mathit{config},r)$ is the set of operations that are either visible to some operation of replica $r$, or has been delivered into replica $r$. The function $\mathit{apply}(\mathit{lin},S)$ returns a local state by applying messages of operations in $S$ according to total order $\mathit{lin}$. 


Then, we need to prove that $P$ is a simulation relation in a way shown below. Note that when increasing linearization, a new item is always put in the last position. 

\begin{itemize}
\setlength{\itemsep}{0.5pt}
\item[-] If $P(\mathit{config},h,\mathit{lin},\mathit{map})$ holds and $\mathit{config} {\xrightarrow{\mathit{do}(m,a,b,r,\mathit{mid})}} \mathit{config}'$, then $P(\mathit{config}', h' = h \otimes (m(a) \Rightarrow b,i,\mathit{obj}), \mathit{lin} \cdot (m(a) \Rightarrow b,i,h'.\mathit{vis}^{-1}(i)),\mathit{map} \cup \{ (\mathit{mid}, i) \})$ holds. Here $i$ is a unique operation identifier.

\item[-] If $P(\mathit{config},h,\mathit{lin},\mathit{map})$ holds and $\mathit{config} {\xrightarrow{\mathit{do}(m,a,b,r)}} \mathit{config}'$, then $P(\mathit{config}',h' = h \otimes (m(a) \Rightarrow b,i,\mathit{obj}), \mathit{lin} \cdot (m(a) \Rightarrow b,i,h'.\mathit{vis}^{-1}(i)),\mathit{map})$ holds.

\item[-] If $P(\mathit{config},h,\mathit{lin},\mathit{map})$ holds and $\mathit{config} {\xrightarrow{\mathit{receive}(\mathit{mid},r)}} \mathit{config}'$, then $P(\mathit{config}',h,\mathit{lin},\mathit{map})$ holds.
\end{itemize} 

Given history $h = (\mathit{Op},\mathit{ro},\mathit{vis})$ and operation $o = (m(a) \Rightarrow b,i,\mathit{obj})$, $h \otimes o$ returns a history $(\mathit{Op}',\mathit{ro}',\mathit{vis}')$, where $\mathit{Op}' = \mathit{Op} \cup \{ o \}$, $\mathit{ro}' = \mathit{ro} \cup \{ (o',o) \vert \mathit{map}(o')$ is of replica $r \}$, and $\mathit{vis}' = (\mathit{vis} \cup \{ (o',o) \vert (\mathit{map}(o'),r) \in \mathit{config}.\mathit{MsgDel} \} \cup \{ (o',o) \vert \mathit{map}(o')$ is of replica $r \})^*$. A predicate $P$ above requirements is called an invariant. The following lemma states that the existence of such $P$ implies distributed linearizability. Its proof can be easily proved by induction for the executions and is omitted here. 

\begin{lemma}
\label{lemma:invariant of operation-based CRDT implies distributed linearizability}
If there exists an invariant $P$ for a CRDT object $\mathit{obj}$ and sequential specification $\mathit{spec}$, then each history of $\mathit{history}(\llbracket \mathit{obj} \rrbracket_{\mathit{op}})$ is distributed linearizable w.r.t $\mathit{spec}$.
\end{lemma} 

In definition of invariant, only $C_4$ is specific to CRDT implementation. 

To give $\mathit{inv}$, it only remains to give the virtual messages. The virtual message of state-based PN-counter and state-based multi-value register as follows. The proof of them being invariants of state-based CRDT is given in Appendix \ref{subsec:appendix proof of state-based PN-counter} and Appendix \ref{subsec:appendix proof of state-based multi-value register}, respectively.

\begin{example}[virtual messages of state-based PN-counter]
\label{example:virtual messagess of state-based PN-counter}

For each update operation $o$, $\mathit{ds}(o) = (P,N)$, where

\begin{itemize}
\setlength{\itemsep}{0.5pt}
\item[-] $\forall r'$, $P[r'] = \vert \{ o' \vert o'$ is a $\mathit{inc}$ operation of replica $r'$, $o' = o \vee (o',o) \in h.\mathit{vis} \} \vert$.

\item[-] $\forall r'$, $N[r'] = \vert \{ o' \vert o'$ is a $\mathit{dec}$ operation of replica $r'$, $o' = o \vee (o',o) \in h.\mathit{vis} \} \vert$.
\end{itemize}
\end{example}

\begin{example}[virtual messages of state-based Multi-value Register]
\label{example:virtual messages of state-based multi-value register}

For each update operation $o = (\mathit{write}(a),\_,\_)$ of replica $r$, $\mathit{ds}(o) = (a,V)$, where

\begin{itemize}
\setlength{\itemsep}{0.5pt}
\item[-] $\forall r'$, $V[r'] = \vert \{ o' \vert o'$ is a $\mathit{write}$ operation of replica $r'$, $o' = o \vee (o',o) \in h.\mathit{vis} \} \vert$.
\end{itemize}
\end{example}









%%% Local Variables:
%%% mode: latex
%%% TeX-master: "draft"
%%% End:
