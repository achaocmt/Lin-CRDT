%!TEX root = draft.tex
%\newcommand{\seqPQ}{\mathsf{SeqPQ}}

\section{Proving Distributed Linearizability}
\label{sec:proving distributed linearizability}

In this section, we propose the definition of t0 and t1 specification, and then propose our approach for proving distributed linearizability for t0 and t1 specification, respectively. Our proof is done by simulation.



\subsection{T0-Specification and T1-Specification}
\label{subsec:t0 specification and t1 specification}



\begin{definition}[t0-specification]
\label{definition:t0-specification}
A sequential specification $\mathit{spec}$ is a t0-specification, if for each history $h=(\mathit{Op},\mathit{ro},\mathit{vis})$ where $h$ is distributed linearizable w.r.t $\mathit{spec}$, each such sequence $\mathit{lin}$ is a linearization of $h$: Each element $(\ell,i,\mathit{vis}^{-1}(i))$ of $\mathit{lin}$ is generated from an operation $(\ell,i,\_)$ of $h$; $\mathit{lin}$ is consistent with $\mathit{vis}$. 
%A sequential specification $\mathit{spec}$ is a t0-specification w.r.t a set $S$ of operation label pairs, if for each history $h=(\mathit{Op},\mathit{ro},\mathit{vis})$ where $h$ is distributed linearizable w.r.t $\mathit{spec}$, the following conditions hold:

%\begin{itemize}
%\setlength{\itemsep}{0.5pt}
%\item[-] Given operations $o_1,o_2 \in \mathit{Op}$, $(\ell_1,\ell_2) \in S$ implies that $(o_1,o_2) \in \mathit{vis}$. Here $\ell_1$ and $\ell_2$ is the operation labels of $o_1$ and $o_2$, respectively.

%\item[-] Each such sequence $\mathit{lin}$ is a linearization of $h$: Each element $(\ell,i,\mathit{vis}^{-1}(i))$ of $\mathit{lin}$ is generated from an operation $(\ell,i,\_)$ of $h$; $\mathit{lin}$ is consistent with $\mathit{vis}$.
%\end{itemize}
\end{definition}

For t0-specification, any sequences consistent with visibility relation is its linearization. The following lemma shows several t0-specifications. Its proof can be found in Appendix \ref{subsec:proof in subsection subsec t0 specification and t1 specification}

\begin{lemma}
\label{lemma:several t0-specifications}
$\mathit{counter}_s$, $\mathit{set}_s$ and $\mathit{OR}$-$\mathit{set}_s$ are t0-specification. 

%The following are t0-specifications:

%\begin{itemize}
%\setlength{\itemsep}{0.5pt}
%\item[-] $\mathit{counter}_s$ is a t0-specification w.r.t $\emptyset$.

%\item[-] $\mathit{set}_s^u$ is a t0-specification w.r.t $\{ (\mathit{add}(a),\mathit{rem}(a)) \vert a \in D \}$.

%\item[-] $\mathit{OR}$-$\mathit{set}_s$ is a t0-specification w.r.t $\emptyset$.
%\end{itemize}
\end{lemma}

For a history where time-stamp is used for conflict resolution, such as RGA and last-write-win register (LWW-register), we can assume that each operation also gives the information of time-stamp. Or we can say, we can implicitly assume that each operation is of the form $o = (\ell,i,\mathit{obj},\mathit{ts})$, where $\mathit{ts}$ is the ``time-stamp'' of this operation: %A set of method is selected and called special method.

\begin{itemize}
\setlength{\itemsep}{0.5pt}
\item[-] If $o$ generates a new unique time-stamp, then $\mathit{ts}$ is this new time-stamp.

\item[-] If $o$ does not generate new time-stamp, then $\mathit{ts}$ is the maximum of time-stamp among operations visible to $o$.

\item[-] Moreover, we require that given operations $o_1$ and $o_2$, if $(o_1,o_2) \in \mathit{vis}$, then the time-stamp of $o_1$ is less or equal than that of $o_2$.
\end{itemize}


\begin{definition}[t1-specification]
\label{definition:t1-specification}
A sequential specification $\mathit{spec}$ is a t1-specification, if for each distributed linearizable history $h=(\mathit{Op},\mathit{ro},\mathit{vis})$, a sequence $\mathit{lin}$ shown below is a linearization of $h$ w.r.t $\mathit{spec}$:

\begin{itemize}
\setlength{\itemsep}{0.5pt}
\item[-] Each element $(\ell,i,\mathit{vis}^{-1}(i))$ of $\mathit{lin}$ is generated from an operation $(\ell,i,\_,\mathit{ts})$ of $h$.

\item[-] $\mathit{lin}$ is consistent with $\mathit{vis}$.

\item[-] If the time-stamp of $o_1$ is less than that of $o_2$, then $o_1$ is before $o_2$ in $\mathit{lin}$.
\end{itemize}
\end{definition}








\subsection{Proof Approach for T0-Specifications}
\label{subsec:proof approach for t0-specifications}

When the context is clear, we do not distinguish an operation and its unique identifier.

For a t0-specification $\mathit{spec}$, we construct a predicate $P(\mathit{config},h,\mathit{lin},\mathit{map})$ where $\mathit{config} = (R,T,\mathit{MsgHB},\mathit{MsgDel})$ is a configuration of $\llbracket \mathit{obj} \rrbracket_{\mathit{op}}$, $h$ is a history, $\mathit{lin} \in \mathbb{A}^*$ is used for linearization, and $\mathit{map} \subseteq \mathbb{MID} \times \mathbb{OID}$ is a function used to associate messages of $\mathit{config}.T$ with operations of $h$. We require $P(\mathit{config},h,\mathit{lin},\mathit{map})$ to be a conjunction of the following statements:

\begin{itemize}
\setlength{\itemsep}{0.5pt}
\item[-] $C_1$ (linearizability): $h$ is distributed linearizable w.r.t $\mathit{spec}$ and $\mathit{lin}$ is a linearization.

\item[-] $C_2$ (correct $\mathit{map}$): $\mathit{map}$ is a bijection between messages of $\mathit{config}.T$ and operations of $h$.

    \begin{itemize}
    \setlength{\itemsep}{0.5pt}
    \item[-] For message visibility of $\mathit{config}$: $(o_1,o_2) \in h.\mathit{vis}$, if and only if $(\mathit{map}(o_1),\mathit{map}(o_2)) \in \mathit{config}.\mathit{MsgHB}$.

    \item[-] For message delivery of $\mathit{config}$: If $(\mathit{mid},r) \notin \mathit{config}.\mathit{MsgDel}$, then for each $\mathit{mid}'$ where the message of $\mathit{mid}'$ is of replica $r$, $(\mathit{map}(\mathit{mid}),\mathit{map}(\mathit{mid}')) \notin h.\mathit{vis}$.
    \end{itemize}

\item[-] $C_3$ (causal delivery): $h.\mathit{vis}$ is transitive. Moreover, if $(o_1,o_2) \in h.\mathit{vis}$ and $(\mathit{map}(o_2),r) \in \mathit{config}.\mathit{MsgDel}$, then $(\mathit{map}(o_1),r) \in \mathit{config}.\mathit{MsgDel}$.

\item[-] $C_4$ (message content): A statement about message content of each message in $\mathit{config}.T$.

\item[-] $C_5$ (sequential explanation): For each replica $r$, we have that $\mathit{config}.R(r) = \mathit{apply}(\mathit{lin},\mathit{vd}(h,\mathit{config},r))$. 
\end{itemize}

Here $\mathit{vd}(h,\mathit{config},r) = \{ o \vert (o,o') \in h.\mathit{vis}, \mathit{map}(o')$ is of replica $r \} \cup \{ o \vert (\mathit{map}(o),r) \in \mathit{config}.\mathit{MsgHB} \}$. $\mathit{vd}(h,\mathit{config},r)$ is the set of operations that are either visible to some operation of replica $r$, or has been delivered into replica $r$. The function $\mathit{apply}(\mathit{lin},S)$ returns a local state by applying messages of operations in $S$ according to total order $\mathit{lin}$.


Then, we need to prove that $P$ is a simulation relation in a way shown below. Note that since $\mathit{spec}$ is a t0-specification, we can freely choose many possible ways to locate linearization point of new operations. Among these possible ways, we choose to always insert a new operation after the tail of the old linearization. 

\begin{itemize}
\setlength{\itemsep}{0.5pt}
\item[-] $P(\mathit{config}_0,\epsilon,\epsilon,\emptyset)$ holds, where $\mathit{config}_0$ is the initial configuration. 

\item[-] If $P(\mathit{config},h,\mathit{lin},\mathit{map})$ holds and $\mathit{config} {\xrightarrow{\mathit{do}(m,a,b,r,\mathit{mid})}} \mathit{config}'$, then $P(\mathit{config}', h' = h \otimes (m(a) \Rightarrow b,i,\mathit{obj}), \mathit{lin} \cdot (m(a) \Rightarrow b,i,h'.\mathit{vis}^{-1}(i)),\mathit{map} \cup \{ (\mathit{mid}, i) \})$ holds. Here $i$ is a unique operation identifier.

\item[-] If $P(\mathit{config},h,\mathit{lin},\mathit{map})$ holds and $\mathit{config} {\xrightarrow{\mathit{do}(m,a,b,r)}} \mathit{config}'$, then $P(\mathit{config}',h' = h \otimes (m(a) \Rightarrow b,i,\mathit{obj}), \mathit{lin} \cdot (m(a) \Rightarrow b,i,h'.\mathit{vis}^{-1}(i)),\mathit{map})$ holds.

\item[-] If $P(\mathit{config},h,\mathit{lin},\mathit{map})$ holds and $\mathit{config} {\xrightarrow{\mathit{receive}(\mathit{mid},r)}} \mathit{config}'$, then $P(\mathit{config}',h,\mathit{lin},\mathit{map})$ holds.
\end{itemize}

Given history $h = (\mathit{Op},\mathit{ro},\mathit{vis})$ and operation $o = (m(a) \Rightarrow b,i,\mathit{obj})$, $h \otimes o$ returns a history $(\mathit{Op}',\mathit{ro}',\mathit{vis}')$, where $\mathit{Op}' = \mathit{Op} \cup \{ o \}$, $\mathit{ro}' = \mathit{ro} \cup \{ (o',o) \vert \mathit{map}(o')$ is a message of replica $r \}$, and $\mathit{vis}' = (\mathit{vis} \cup \{ (o',o) \vert (\mathit{map}(o'),r) \in \mathit{config}.\mathit{MsgDel} \} \cup \{ (o',o) \vert \mathit{map}(o')$ is a message of replica $r \})^*$. A predicate $P(\mathit{config},h,\mathit{lin},\mathit{map})$ satisfies conditions $C_1$ to $C_5$ is called an invariant.

The following lemma states that the existence of such invariant implies distributed linearizability, and each sequence consistent with visibility is a linearization. To prove this lemma, we first prove that by induction on length of execution, we obtain a linearization. Then by definition of t0-specification, we know that each sequence consistent with visibility is a linearization. 

\begin{lemma}
\label{lemma:invariant of operation-based CRDT implies distributed linearizability}
If there exists an invariant $P$ for a CRDT object $\mathit{obj}$ and a t0-specification $\mathit{spec}$, then each history of $\mathit{history}(\llbracket \mathit{obj} \rrbracket_{\mathit{op}})$ is distributed linearizable w.r.t $\mathit{spec}$, and each sequences that consistent with visibility is a linearization of $h$.
\end{lemma}

In definition of invariant, only $C_4$ is specific to CRDT implementation. The following is the $C_4$ for or-set implementation. The detailed of such $C_4$ defining a correct variant can be found in Appendix \ref{subsec:appendix proofs of or-set implementation}.

\begin{example}[$C_4$ for or-set implementation]
\label{example:c4 for or-set implementation}

Given history $h = (\mathit{Op},\mathit{ro},\mathit{vis})$ and a update operation $o$ of $h$, the message of $o$ is given as follows:

\begin{itemize}
\setlength{\itemsep}{0.5pt}
\item[-] If $o$ is a $\mathit{add}(a)$ operation of replica $r$:  Then the message of $o$ is $(a,(c+1,r))$, where $c = \mathit{max}\{ c_1 \vert \exists o' = (\_,\_,\_,\mathit{ts}), (o',o) \in \mathit{vis}, \mathit{ts} = (c,\_) \}$. 

\item[-] If $o$ is a $\mathit{rem}(a)$ operation: the message of $o$ is $S_1$, where $S_1 = \{ (a,\mathit{ts}') \vert (a,\mathit{ts}') \in S(o) \}$. We also require that $S_1 \neq \emptyset$.
\end{itemize}
Here $S(o) = \{ (b,\mathit{ts}') \vert b \in D, \exists o' = (\mathit{add}(b),\_,\_,\mathit{ts}')$, $(o',o) \in \mathit{vis}$, for each $\mathit{rem}(b)$ operation $o'' \neq o, (o'',o) \in \mathit{vis} \Rightarrow (o',o'') \neq \mathit{vis} \}$.
\end{example}












%%% Local Variables:
%%% mode: latex
%%% TeX-master: "draft"
%%% End:
