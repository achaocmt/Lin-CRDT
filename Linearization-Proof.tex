%!TEX root = draft.tex
%\newcommand{\seqPQ}{\mathsf{SeqPQ}}

\section{Proving Distributed Linearizability}
\label{sec:proving distributed linearizability} 

In this section, we propose our approach for proving distributed linearizability for CRDT implementations. Our proof is done by proving a invariant. 


\subsection{Proof Strategy of State-based CRDT}
\label{subsec:proof strategy of operation-based CRDT} 

Given a state-based CRDT object $\mathit{obj}$ and a sequential specification $\mathit{spec}$, we need to construct a invariant $\mathit{inv}((R, T), h, \mathit{lin})$, which is the conjunction of the following conditions:    

\begin{itemize}
\setlength{\itemsep}{0.5pt}
\item[-] $(R, T)$ is a configuration of $\llbracket \mathit{obj} \rrbracket_s$.
\item[-] $h$ is distributed linearizable w.r.t $\mathit{spec}$ and $\mathit{lin}$ is a linearization.  
\item[-] $\forall r$, $C(r) = \mathit{apply}(\mathit{lin},\mathit{vis}^{-1}(r),T)$.  
\end{itemize} 

Let the sequence $\alpha_1 \cdot \ldots \cdot \alpha_k$ be the projection of $\mathit{lin}$ into update operations that is either of replica $r$, or visible to some operation of replica $r$. Then, $\mathit{apply}(\mathit{lin},\mathit{vis}^{-1}(r),T)$ is a local state obtained from the initial local state as follows: for each $1 \leq i \leq k$, if $\alpha_i$ is a update operation, then apply the message generated by $\alpha_i$ in $T$; otherwise, do nothing. $\mathit{inv}((R, T),h,\mathit{lin})$ represents that $h$ is a history corresponds to $(R,T)$ and $\mathit{lin}$ is a linearization of $h$. 

Then, we need to prove the following properties of $\mathit{inv}$: 

\begin{itemize}
\setlength{\itemsep}{0.5pt}
\item[-] $\mathit{inv}$ holds initially: $\mathit{inv}((R_0,T_0),\epsilon,\emptyset)$ holds, where $(R_0, T_0)$ is the initial configuration of $\llbracket \mathit{obj} \rrbracket_s$. 

\item[-] $\mathit{inv}$ is a transition invariant: 

    \begin{itemize}
    \setlength{\itemsep}{0.5pt}
    \item[-] If $\mathit{inv}((R, T),h,\mathit{lin})$ and $(R, T) {\xrightarrow{\mathit{do}(m,a,b,r,\mathit{mid})}} (R', T')$ or $(R, T) {\xrightarrow{\mathit{do}(m,a,b,r)}} (R', T')$, then there exists $\mathit{lin}'$, such that $\mathit{inv}((R', T'), h \otimes (m(a) \Rightarrow b, i, \mathit{obj}), \mathit{lin}')$ holds.
        
        Here $i$ is the operation identifier of the newly generated $\mathit{do}$ action. $h \otimes (m(a) \Rightarrow b,i,\mathit{obj})$ extends $h$ with operation $i$ as follows: make $i$ the maximum w.r.t $\mathit{ro}$ among operations of replica $r$, and for any operation $o$ that is visible to some operation of replica $r$ in $h$, make $o$ visible to $i$.
    
    \item[-] If $(R, T) {\xrightarrow{\mathit{receive}(\mathit{mid},r)}} (R', T')$, then $\mathit{inv}((R', T'),h,\mathit{lin})$ holds. 
    \end{itemize}
\end{itemize} 

A invariant $\mathit{inv}$ satisfies above properties is called invariant of state-based CRDT. The following lemma states that the existence of such invariant implies distributed linearizability. This can be obviously checked by induction. 

Let us use this approach to prove the correctness of two examples: state-based PN-counter and state-based multi-value register. 





 



%%% Local Variables:
%%% mode: latex
%%% TeX-master: "draft"
%%% End:
