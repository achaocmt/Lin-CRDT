%!TEX root = draft.tex
%\newcommand{\seqPQ}{\mathsf{SeqPQ}}

\section{Proving Distributed Linearizability}
\label{sec:proving distributed linearizability} 

The proof process of distributed linearizability is as follows: We construct a function $g$ called linearization function, which takes a configuration $c$ as argument, and returns a set. Each item of the set is a tuple $(e,l)$, where intuitively $e$ is an execution that generate $c$, and $l = (s,f)$ is in the distributed specification of $e$. Moreover, function $g$ is a simulation relation: whenever $c {\xrightarrow{\alpha}} c'$, there exists $e',l'$, such that $e'$ is generated from $e$ by adding a action according to $\alpha$, and $(e',l') \in g(c')$.

Let us take the RGA algorithm as example to show how to construct function $g$. Since each configuration contains all its previous messages as well as their happen-before and delivery relation, we should consider only correct configurations. A configuration $((N,\mathit{Tomb}),T,\mathit{msgHB},\mathit{msgDel})$ is correct, if 

\begin{itemize}
\setlength{\itemsep}{0.5pt}
\item[-] a
\end{itemize}



First, we propose a function $f$ from state to history, and a function $g$ from history to linearization. Such function $f$ and $g$ holds when distributed system proceed. Then, we prove that such linearization is in specification. We also need a invariant of states.




%%% Local Variables:
%%% mode: latex
%%% TeX-master: "draft"
%%% End:
