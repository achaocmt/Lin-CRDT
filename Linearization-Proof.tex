%!TEX root = draft.tex
%\newcommand{\seqPQ}{\mathsf{SeqPQ}}

\section{Proving Distributed Linearizability}
\label{sec:proving distributed linearizability}

In this section, we propose our approach for proving distributed linearizability for CRDT vis simulation. 

Given a sequential specification $\mathit{spec}$, a configuration $\mathit{config}$ of $\llbracket \mathit{obj} \rrbracket_{\mathit{op}}$, a history $h$, and a sequence $\mathit{lin}$, we say that  





Given a state-based CRDT object $\mathit{obj}$ and a sequential specification $\mathit{spec}$, we need to construct a invariant $\mathit{inv}(\mathit{config},h,\mathit{lin},\mathit{del},\mathit{map})$, where

\begin{itemize}
\setlength{\itemsep}{0.5pt}
\item[-] $\mathit{config}$ is a configuration of $\llbracket \mathit{obj} \rrbracket_s$.

\item[-] $h$ is a history.

\item[-] $h$ is distributed linearizable w.r.t $\mathit{spec}$ and $\mathit{lin}$ is a linearization.

\item[-] $\mathit{del} \subseteq \mathbb{MID} \times \mathbb{R}$ is the message delivery relation.

\item[-] $\mathit{map} \subseteq \mathbb{MID} \times 2^{\mathbb{OID}}$ maps each message $\mathit{mid}$ to a set $S_1$ of operations. Intuitively, $S_1$ is the set of operations whose information are contained in $\mathit{mid}$.
\end{itemize}

$\mathit{inv}(\mathit{config},h,\mathit{lin},\mathit{del},\mathit{map})$ needs to satisfy the following properties:

\begin{itemize}
\setlength{\itemsep}{0.5pt}
\item[-] The visibility of $h$ is transitive.

\item[-] $\mathit{del}$ preserves causal delivery: If $(o_1,o_2) \in \mathit{vis}$ and $(o_2,r) \in \mathit{del}$, then $(o_1,r) \in \mathit{del}$.

\item[-] $\mathit{map}$ preserves causal delivery: Given $o_1,o_3 \in \mathit{map}(\mathit{mid})$, if $\exists o_2$, such that $(o_1,o_2),(o_2,o_3) \in \mathit{vis}$, then $o_2 \in \mathit{map}(\mathit{mid})$.

\item[-] $\mathit{inv}$ holds initially: $\mathit{inv}(\mathit{config}_0,\epsilon,\emptyset,\emptyset,\emptyset)$ holds, where $\mathit{config}_0$ is the initial configuration of $\llbracket \mathit{obj} \rrbracket_s$.

\item[-] $\mathit{inv}$ is a transition invariant:

    \begin{itemize}
    \setlength{\itemsep}{0.5pt}
    \item[-] If $\mathit{inv}(\mathit{config},h,\mathit{lin},\mathit{del},\mathit{map})$ holds and $\mathit{config} {\xrightarrow{\mathit{do}(m,a,b,r)}} \mathit{config}'$, then $\mathit{inv}(\mathit{config}', h \otimes i, \mathit{lin} \cdot i,\mathit{del},\mathit{map})$ holds. Note that here we always put a new operation in the last of linearization.

        Here $i$ is the identifier of the newly-generated $\mathit{do}$ action. Given $h = (\mathit{Op},\mathit{ro},\mathit{vis})$, then, $h \otimes i = (\mathit{Op}',\mathit{ro}',\mathit{vis}')$, where $\mathit{Op}' = \mathit{Op} \cup \{ (m(a) \Rightarrow b,i,\mathit{obj}) \}$, $\mathit{ro}' = \mathit{ro} \cup \{ (j,i) \vert j \in \mathit{Op}, j$ is of replica $r \}$, and $\mathit{vis}' = (\mathit{vis} \cup \{ (j,i) \vert j \in \mathit{Op},(j,r) \in \mathit{del} \} \cup \{ (j,i) \vert j \in \mathit{Op}, j$ is of replica $r \})^*$.

    \item[-] If $\mathit{inv}(\mathit{config},h,\mathit{lin},\mathit{del},\mathit{map})$ holds and $\mathit{config} {\xrightarrow{\mathit{send}(\mathit{mid},r)}} \mathit{config}'$, then $\mathit{inv}(\mathit{config}',h,\mathit{lin},\mathit{del},\mathit{map}')$ holds, where $\mathit{map}' = \mathit{map} \cup (\mathit{mid}, \mathit{vd}(h,\mathit{del},r))$.


    \item[-] If $\mathit{inv}(\mathit{config},h,\mathit{lin},\mathit{del},\mathit{map})$ holds and $\mathit{config} {\xrightarrow{\mathit{receive}(\mathit{mid},r)}} \mathit{config}'$, then $\mathit{inv}(\mathit{config}',h,\mathit{lin},\mathit{del}',\mathit{map})$ holds, where $\mathit{del}' = \mathit{del} \cup \{ (i,r) \vert i \in \mathit{map}(\mathit{mid}) \}$.
    \end{itemize}
\end{itemize}

Here $\mathit{vd}(h,\mathit{del},r) = \{ i \vert (i,j) \in h.\mathit{vis}, j$ is of replica $r \} \cup \{ i \vert (i,r) \in \mathit{del} \}$ is the set of operations that are either to some operation of replica $r$, or has been delivered into replica $r$. An invariant $\mathit{inv}$ satisfies above properties is called invariant of state-based CRDT. The following lemma states that the existence of such invariant implies distributed linearizability.

\begin{lemma}
\label{lemma:invariant of state-based CRDT implies distributed linearizability}
If there exists a invariant $\mathit{inv}$ of state-based CRDT for object $\mathit{obj}$ and sequential specification $\mathit{spec}$, then each history of $\mathit{history}(\llbracket \mathit{obj} \rrbracket_s)$ is distributed linearizable w.r.t $\mathit{spec}$.
\end{lemma}

\begin {proof}
Given an execution $l=\alpha_1 \cdot \ldots \cdot \alpha_n$, let $\mathit{config}_0 {\xrightarrow{\alpha_1}} \mathit{config}_1 \ldots {\xrightarrow{\alpha_n}} \mathit{config}_n$ be the transitions from initial configuration. We need to prove that, for each $1 \leq k \leq n$, we have $\mathit{inv}(\mathit{config}_k,h_k,\mathit{lin}_k,\mathit{del}_k,\mathit{map}_k)$ holds, where $h_k$ is the history of execution $l_k = \alpha_1 \cdot \ldots \cdot \alpha_k$, $\mathit{lin}_k$ is the linearization of $h_k$, $\mathit{del}_k$ records message delivery relation of $l_k$, and $\mathit{map}_k$ records the operations contained in each message in $l_k$.

Since $\mathit{inv}$ holds initially and is a transition invariant, it is easy to prove above requirements by induction on execution. This completes the proof of this lemma. $\qed$
\end {proof}


For many state-based CRDT implementations, $\mathit{inv}((R,T),h,\mathit{lin},\mathit{del},\mathit{map}) = C_1 \wedge C_2$, where

\begin{itemize}
\setlength{\itemsep}{0.5pt}
%\item[-] For each update operation $o$ of $h$, define $\mathit{ds}(o)$ which is a local state. %be the local state of replica $r$ at the time point immediately after $o$ is launched. Here $r$ is the replica of $o$.

\item[-] $C_1: \forall (\mathit{mid},\mathit{msg},\_) \in T$, $\mathit{msg} = \mathit{apply}(\mathit{lin},\mathit{map}(\mathit{mid}))$.

\item[-] $C_2: \forall r$, $R(r) = \mathit{apply}(\mathit{lin},\mathit{vd}(h,\mathit{del},r))$.
\end{itemize}

The function $\mathit{apply}(\mathit{lin},S)$ returns a local state by applying ``virtual messages'' of operations in $S$ according to total order $\mathit{lin}$. Here for each update operation $o$ of $h$, we need to define a local state $\mathit{ds}(o)$, which is the ``virtual messages'' of $o$. Note that state-based CRDT send message randomly, instead of each message for a update operation. This is the reason why we need to manually generate virtual message for each update operation.

To give $\mathit{inv}$, it only remains to give the virtual messages. The virtual message of state-based PN-counter and state-based multi-value register as follows. The proof of them being invariants of state-based CRDT is given in Appendix \ref{subsec:appendix proof of state-based PN-counter} and Appendix \ref{subsec:appendix proof of state-based multi-value register}, respectively.

\begin{example}[virtual messages of state-based PN-counter]
\label{example:virtual messagess of state-based PN-counter}

For each update operation $o$, $\mathit{ds}(o) = (P,N)$, where

\begin{itemize}
\setlength{\itemsep}{0.5pt}
\item[-] $\forall r'$, $P[r'] = \vert \{ o' \vert o'$ is a $\mathit{inc}$ operation of replica $r'$, $o' = o \vee (o',o) \in h.\mathit{vis} \} \vert$.

\item[-] $\forall r'$, $N[r'] = \vert \{ o' \vert o'$ is a $\mathit{dec}$ operation of replica $r'$, $o' = o \vee (o',o) \in h.\mathit{vis} \} \vert$.
\end{itemize}
\end{example}

\begin{example}[virtual messages of state-based Multi-value Register]
\label{example:virtual messages of state-based multi-value register}

For each update operation $o = (\mathit{write}(a),\_,\_)$ of replica $r$, $\mathit{ds}(o) = (a,V)$, where

\begin{itemize}
\setlength{\itemsep}{0.5pt}
\item[-] $\forall r'$, $V[r'] = \vert \{ o' \vert o'$ is a $\mathit{write}$ operation of replica $r'$, $o' = o \vee (o',o) \in h.\mathit{vis} \} \vert$.
\end{itemize}
\end{example}









%%% Local Variables:
%%% mode: latex
%%% TeX-master: "draft"
%%% End:
